\begin{topic}{module}{module}
    Let $R$ be a commutative ring. An \textbf{$R$-module} is an abelian group $M$ with an action of $R$, that is a map
    \[ R \times M \to M, \qquad (a, m) \mapsto a \cdot m, \]
    satisfying
    \begin{itemize}
        \item $a \cdot (m + m') = a \cdot m + a \cdot m'$,
        \item $(a + b) \cdot m = a \cdot m + b \cdot m$,
        \item $a \cdot (bm) = (ab) \cdot m$,
        \item $1 \cdot m = m$,
    \end{itemize}
    for all $a, b \in R$ and $m, m' \in M$.
\end{topic}

\begin{topic}{free-module}{free module}
    Let $R$ be a commutative ring. An $R$-module $M$ is \textbf{free} if it is isomorphic to
    \[ \bigoplus_{i \in I} R , \]
    for some indexing set $I$.
\end{topic}

\begin{topic}{cyclic-module}{cyclic module}
    Let $R$ be a commutative ring. An $R$-module $M$ is \textbf{cyclic} if it can be generated by one element, that is $M = Rm$ for some $m \in M$.
\end{topic}

\begin{topic}{projective-module}{projective module}
    Let $R$ be a commutative ring. An $R$-module $P$ is called \textbf{projective} if for every morphism $g : P \to M$ and surjective morphism $f : N \to M$ of $R$-modules, there exists a morphism $h : P \to N$ of $R$-modules such that $fh = g$. We do not require this map to be unique.
    \[ \begin{tikzcd} & N \arrow[twoheadrightarrow]{d}{f} \\ P \arrow[swap]{r}{g} \arrow[dashed]{ur}{\exists h} & M \end{tikzcd} \]
\end{topic}

\begin{topic}{injective-module}{injective module}
    Let $R$ be a commutative ring. An $R$-module $I$ is called \textbf{injective} if for every morphism $g : M \to I$ and injective morphism $f : M \to N$ of $R$-modules, there exists a morphism $h : N \to Q$ of $R$-modules such that $hf = g$. We do not require this map to be unique.
    \[ \begin{tikzcd} M \arrow[hookrightarrow]{r}{f} \arrow[swap]{d}{g} & N \arrow[dashed]{ld}{\exists h} \\ I & \end{tikzcd} \]
\end{topic}

\begin{topic}{flat-module}{flat module}
    Let $R$ be a commutative ring. An $R$-module $M$ is \textbf{flat} if $(-) \otimes_R M$ is exact. Since the tensor product is already right-exact, this is equivalent to saying $(-) \otimes_R M$ sends injective morphisms to injective morphisms.
\end{topic}

\begin{example}{flat-module}
    Any \tref{free-module}{free module} $\bigoplus_{i \in I} R$ is flat over $R$, because if $M \to N$ is injective, then so is $\bigoplus_{i \in I} M \to \bigoplus_{i \in I} N$.
    In particular, $k$-vector spaces are flat over $k$.
\end{example}

\begin{example}{flat-module}
    Take $R = \ZZ$ and $M = \ZZ/2\ZZ$. Then $M$ is not flat, because tensoring
    \[ 0 \rightarrow \ZZ \xrightarrow{\cdot 2} \ZZ \rightarrow \ZZ/2\ZZ \to 0 \]
    with $\ZZ/2\ZZ$ gives
    \[ 0 \rightarrow \ZZ/2\ZZ \xrightarrow{0} \ZZ/2\ZZ \xrightarrow{\id} \ZZ/2\ZZ \to 0 , \]
    which is not exact on the left.
\end{example}

\begin{example}{flat-module}
    Let $f : A \to B$ be a morphism of rings, and suppose that $x \in A$ is not a \tref{zero-divisor}{zero-divisor} while $f(x) \in B$ is. Then $B$ cannot be flat as an $A$-module, since multiplication by $x$ is an injective map $A \to A$, while multiplication by $f(x)$ is not an injective map $B \to B$.
\end{example}
