\begin{topic}{partial-order}{partial order}
    A \textbf{partial order} on a set $A$ is a \tref{binary-relation}{binary relation} $\le$ on $A$ such that
    \begin{itemize}
        \item (\textit{reflexivity}) $x \le x$ for all $x \in A$,
        \item (\textit{anti-symmetry}) if $x \le y$ and $y \le x$ then $x = y$ for all $x, y \in A$,
        \item (\textit{transitivity}) if $x \le y$ and $y \le z$ then $x \le z$ for all $x, y, z \in A$.
    \end{itemize}
    A \textbf{partially ordered set} is a set $A$ together with a partial order $\le$.
\end{topic}

\begin{topic}{total-order}{total order}
    A \textbf{total order} on a set $A$ is a \tref{partial-order}{partial order} $\le$ such that
    \begin{itemize}
        \item (\textit{total}) $x \le y$ or $y \le x$ for all $x, y \in A$.
    \end{itemize}
\end{topic}

\begin{topic}{well-order}{well-order}
    A \textbf{well-order} on a set $A$ is a \tref{total-order}{total order} $\le$ such that
    \begin{itemize}
        \item (\textit{least element}) for every non-empty subset $S \subset A$, there exists an element $s \in S$ such that $s \le x$ for all $x \in S$.
    \end{itemize}
\end{topic}

\begin{example}{well-order}
    \begin{itemize}
        \item The natural numbers $\NN$ with the usual order $\le$ is a well-order.
        \item The real numbers $\RR$ with the usual order $\le$ is not a well-order. Namely, the open interval $(0, 1) \subset \RR$ does not have a least element.
    \end{itemize}
\end{example}

\begin{topic}{lattice}{lattice}
    A \textbf{lattice} is a \tref{partial-order}{partially ordered set} $(P, \le)$ such that
    \begin{itemize}
        \item (\textit{least upper bound}) any two elements $x, y \in P$ have a least upper bound $x \vee y$,
        \item (\textit{greatest lower bound}) any two elements $x, y \in P$ have a greatest lower bound $x \wedge y$.
    \end{itemize}
\end{topic}

\begin{example}{lattice}
    Let $G$ be a \tref{GT:group}{group} and let $P$ be the set of \tref{GT:subgroup}{subgroups} of $G$. Then $(P, \subset)$ is a lattice, where $H \vee K$ is the subgroup $\langle H, K \rangle$ generated by $H$ and $K$, and $H \wedge K$ is the intersection $H \cap K$.
    % This lattice is bounded, the least element being the trivial subgroup, and the greatest element being the group $G$ itself. Moreover, this lattice is complete, the least upper bound being given by the subgroup generated by the union, and the greatest lower bound being given by the intersection.
\end{example}

\begin{topic}{bounded-lattice}{bounded lattice}
    A \textbf{bounded lattice} is a \tref{lattice}{lattice} for which
    \begin{itemize}
        \item (\textit{least element}) there exists a least element $\bot \in P$,
        \item (\textit{greatest element}) there exists a greatest element $\top \in P$.
    \end{itemize}
\end{topic}

\begin{topic}{complete-lattice}{complete lattice}
    A \textbf{complete lattice} is a \tref{lattice}{lattice} for which
    \begin{itemize}
        \item (\textit{least upper bound}) every subset $A \subset P$ has a least upper bound $\bigvee A$,
        \item (\textit{greatest lower bound}) every subset $A \subset P$ has a greatest lower bound $\bigwedge A$.
    \end{itemize}
\end{topic}

\begin{topic}{complemented-lattice}{complemented lattice}
    A \textbf{complemented lattice} is a \tref{bounded-lattice}{bounded lattice} $(P, \le)$ such that
    \begin{itemize}
        \item (\textit{complements}) for every $x \in P$ there exists an element $\neg x \in P$ such that $x \vee \neg x = \top$ and $x \wedge \neg x = \bot$.
    \end{itemize}
\end{topic}

\begin{topic}{boolean-algebra}{Boolean algebra}
    A \textbf{Boolean algebra} is a \tref{complemented-lattice}{complemented lattice} $(P, \le)$ such that
    \begin{itemize}
        \item (\textit{distributivity}) $x \wedge (y \vee z) = (x \wedge y) \vee (x \wedge z)$ for all $x, y, z \in P$.
    \end{itemize}
    % A Boolean algebra $(P, \le)$ is \textbf{complete} if every subset $A \subset P$ has a least upper bound and a greatest lower bound.
\end{topic}

\begin{example}{boolean-algebra}
    Let $X$ be a set, and let $P$ be the \tref{power-set}{power set} of $X$. Then $(P, \subset)$ is a Boolean algebra, where the complement $\neg A$ is given by the difference $X \setminus A$.
\end{example}

\begin{topic}{atom}{atom}
    Let $(P, \le)$ be a \tref{partial-order}{partially ordered set} with a least element $\bot \in P$. An \textbf{atom} in $P$ is an element $x \in P$ not equal to $\bot$ such that $y \le x$ implies $y = \bot$ or $y = x$ for all $y \in P$.
\end{topic}

\begin{topic}{heyting-algebra}{Heyting algebra}
    A \textbf{Heyting algebra} is a \tref{bounded-lattice}{bounded lattice} $(H, \le)$ together with a binary operation $\rightarrow$, called \textit{Heyting implication}, such that $x \wedge y \le z$ if and only if $x \le y \rightarrow z$ for all $x, y, z \in H$.
\end{topic}

\begin{topic}{binary-relation}{binary relation}
    A \textbf{binary relation} $R$ over two sets $X$ and $Y$ is a subset of the product $X \times Y$.
    
    Usually, one writes $x R y$ to mean $(x, y) \in R$.
\end{topic}

\begin{topic}{modular-lattice}{modular lattice}
    A \tref{lattice}{lattice} $(P, \le)$ is \textbf{modular} if
    \[ x \lor (y \land z) = (x \lor y) \land z \]
    for all $x, y, z \in P$ with $x \le z$.
\end{topic}

\begin{example}{modular-lattice}
    Let $R$ be a \tref{AA:ring}{ring} and $M$ an \tref{AA:module}{$R$-module}. Then the lattice of submodules of $M$ with respect to inclusions, sums and intersections, is a modular lattice. That is, for any submodules $A, B, C \subset M$ with $A \subset C$, we have
    \[ A + (B \cap C) = (A + B) \cap C . \]
    Indeed, for any $x \in A + (B \cap C)$, we can write $x = a + y$ with $a \in A$ and $y \in B \cap C$. Since $A \subset C$, we have $a \in A \cap C$, so $x \in (A + B) \cap C$. Conversely, for any $x \in (A + B) \cap C$, we can write $x = a + b$ with $a \in A$ and $b \in B$ and $a + b \in C$. Since $a \in A \subset C$, also $b = (a + b) - a \in C$, so $x \in A + (B \cap C)$.
\end{example}

\begin{topic}{filter}{(ultra)filter}
    Let $(P, \le)$ be a \tref{partial-order}{partially ordered set}. A \textbf{filter} on $P$ is a subset $F \subset P$ such that
    \begin{itemize}
        \item (\textit{non-empty}) $F$ is non-empty,
        \item (\textit{downward directed}) for every $x, y \in F$ there exists some $z \in F$ with $z \le x$ and $z \le y$,
        \item (\textit{upward-closed}) for all $x \in F$ and $y \in P$ with $x \le y$, also $y \in F$.
    \end{itemize}
    An \textbf{ultrafilter} on $P$ is a filter $F \subsetneq P$ such that there exists no filter $F'$ with $F \subsetneq F' \subsetneq P$.
\end{topic}

\begin{example}{filter}
    Let $X$ be a \tref{TO:topological-space}{topological space}, and $x \in X$ a point. The \textit{neighborhood filter} of $x$, denoted $\mathcal{N}_x$, is the filter consisting of all \tref{TO:neighborhood}{neighborhoods} of $x$. It is a filter on the set of all subsets of $X$, partially ordered by inclusion.
\end{example}

\begin{topic}{de-morgan-laws}{De Morgan's laws}
    Let $B$ be a \tref{boolean-algebra}{Boolean algebra}. \textbf{De Morgan's laws} state that
    \[ \neg (x \lor y) = \neg x \land \neg y \quad \textup{ and } \quad \neg (x \land y) = \neg x \lor \neg y \]
    for all $x, y \in B$.
\end{topic}

