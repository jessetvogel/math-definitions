\begin{topic}{quiver}{quiver}
    A \textbf{quiver} is a finite directed graph, where loops and multiple arrows between vertices are allowed.    
\end{topic}

\begin{topic}{quiver-representation}{quiver representation}
    Let $Q$ be a \tref{quiver}{quiver}. A \textbf{representation} of $Q$ is a collection of vector spaces $V_i$ for each vertex $i \in Q_0$ and linear maps $V(\alpha) : V_i \to V_j$ for each arrow $(\alpha : i \to j) \in Q_1$.
    
    A \textbf{morphism} of quiver representations $V \to W$ is a collection of maps $\varphi_i : V_i \to W_i$ for each vertex $i \in Q_0$ such that $\varphi_j V(\alpha) = W(\alpha) \varphi_i$.
    
    The sequence of dimensions $(\dim V_i)_{i \in Q_0} \in \NN Q_0$ is called the \textbf{dimension vector}.
\end{topic}

\begin{topic}{path-algebra}{(quiver) path algebra}
    Let $Q$ be a \tref{quiver}{quiver}. A \textbf{path} in $Q$ is a sequence $\alpha_n \alpha_{n - 1} \cdots \alpha_1$ of arrows such that the head of $\alpha_{i + 1}$ equals the tail of $\alpha_{i}$. The \textbf{path algebra} of $Q$ is defined as a \tref{LA:vector-space}{vector space} with the set of paths of $Q$ as a basis. Multiplication is given by concatenation of paths, and if two paths cannot be connected because their endpoints are different, their product is defined as zero.
\end{topic}

\begin{topic}{euler-form}{Euler form}
    Let $Q$ be a \tref{quiver}{quiver}. The \textbf{Euler form} of $Q$ is the bilinear form on $\ZZ^{Q_0}$ given by
    \[ \langle d, e \rangle_Q = \sum_{i \in Q_0} d_i e_i - \sum_{(\alpha : i \to j) \in Q_1} d_i e_j \]
    for any $d = (d_i)_{i \in Q_0}$ and $e = (e_i)_{i \in Q_0}$.
\end{topic}

\begin{topic}{tits-form}{Tits form}
    Let $Q$ be a \tref{quiver}{quiver}. The \textbf{Tits form} of $Q$ is the quadratic form on $\ZZ^{Q_0}$ given by
    \[ q_Q(d) = \langle d, d \rangle_Q = \sum_{i \in Q_0} d_i^2 - \sum_{(\alpha : i \to j) \in Q_1} d_i d_j \]
    for any $d = (d_i)_{i \in Q_0}$, where $\langle \cdot, \cdot \rangle_Q$ denotes the \tref{euler-form}{Euler form} of $Q$.
\end{topic}
