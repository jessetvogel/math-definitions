\begin{topic}{koszul-complex}{Koszul complex}
    Let $F$ be a \tref{AA:free-module}{free module} of finite rank $r$ over a commutative \tref{AA:ring}{ring} $R$. Then, given an $R$-linear map $s : F \to R$ the \textbf{Koszul complex} associated to $s$ is the chain complex of $R$-modules
    \[ K_\bdot(s) : \qquad 0 \to \wedge^r F \xrightarrow{d_r} \wedge^{r - 1} F \xrightarrow{d_{r - 1}} \cdots \xrightarrow{d_2} F \xrightarrow{d_1} R \to 0 \]
    with the differentials given by
    \[ d_k(e_1 \wedge e_2 \ldots \wedge e_k) = \sum_{i = 1}^{k} (-1)^{i + 1} s(e_i) e_1 \wedge \cdots \wedge \hat{e_i} \wedge \cdots \wedge e_k , \]
    where the hat indicates the term is missing. Note that $d_1 = s$.
\end{topic}

\begin{example}{koszul-complex}
    When $R = k[x_1, \ldots, x_n]$ and $E$ is of rank $n$ with $s = (x_1, \ldots, x_n)$, we obtain the complex 
    \[ 0 \to \wedge^n E = R \to \cdots \to \wedge^i E = R^{\binom{n}{i}} \to \cdots \to \wedge^1 E = R^n \to R \to 0 \]
    which is a free resolution of $k$ as an $R$-module.
\end{example}

\begin{topic}{pure-hodge-structure}{pure Hodge structure}
    A \textbf{pure Hodge structure of weight $k \in \ZZ$} consists of a $\ZZ$-module $H_\ZZ$ of finite rank, and a direct sum decomposition of the complexification
    \[ H_\CC := H_\ZZ \otimes_\ZZ \CC = \bigoplus_{p + q = k} H^{p, q} \quad \text{ with } \quad H^{p, q} = \overline{H^{q, p}} . \]
    One speaks of \textit{rational} or \textit{real} Hodge structures when replacing $H_\ZZ$ by a rational or real vector space.
    
    A \textbf{morphism of pure Hodge structures} is a morphism $f : H_\ZZ \to H'_\ZZ$ of $\ZZ$-modules such that its complexification $f_\CC$ preserves type, i.e. $f_\CC\left(H^{p, q}\right) \subset (H')^{p, q}$.
    
    The numbers $h^{p, q}(H) := \dim_\CC H^{p, q}$ are called the \textbf{Hodge numbers} of the Hodge structure.
\end{topic}

\begin{example}{pure-hodge-structure}
    For any $H_\ZZ$, taking $H^{k,k} = H_\CC$ and $H^{p, q} = 0$ when $(p, q) \ne (k, k)$ gives the \textit{trivial Hodge structure} of weight $2k$.
\end{example}

\begin{example}{pure-hodge-structure}
    Take $H_\ZZ = 2 \pi i \ZZ \subset \CC$ and $H_\CC = H^{-1, -1}$. This is a pure Hodge structure of weight $-2$. In fact, is the unique $1$-dimensional pure Hodge structure of weight $-2$ up to isomorphism, and it is called the \textit{Tate Hodge structure}, often denoted by $\ZZ(1)$.
\end{example}

\begin{topic}{hodge-filtration}{Hodge filtration}
    A \textbf{Hodge filtration} is an equivalent description of a \tref{pure-hodge-structure}{pure Hodge structure} of weight $k \in \ZZ$, given by a filtration
    \[ H_\CC \supset \cdots \supset F^p \supset F^{p + 1} \supset \cdots \quad \text{ with } \quad F^p \oplus \overline{F^q} = H_\CC \text{ for } p + q = k + 1 . \]
    A Hodge filtration can be obtained from a pure Hodge structure, and vice versa, via
    \[ F^p = \bigoplus_{r \ge p} H^{r, k - r} \quad \text{and} \quad H^{p, q} = F^p \cap \overline{F^q} . \]
    % Indeed $F^p$ gives a filtration, and for $p + q = k + 1$ one finds
    % \[ F^p = \bigoplus_{r \ge p} V^{r, k - r} \qquad \text{and} \qquad \overline{F^q} = \bigoplus_{s \ge k - p + 1} \overline{V^{s, k - s}} = \bigoplus_{r \le p - 1} V^{r, k - r}, \]
    % from which follows that $F^p \oplus \overline{F^q} = V_\CC$.
    % In the other direction
    % \[ \bigoplus_{p + q = k} H^{p, q} = \bigoplus_{p + q = k} F^p \cap \overline{F^q} = H_\CC , \]
    % because any $v \in H_\CC$ lies in $F^p \backslash F^{p + 1}$ for some unique $p$, and since $F^{p + 1} \oplus \overline{F^{k - p}} = H_\CC$, we have $v \in F^p \cap \overline{F^{k - p}}$. It is clear that $\overline{H^{p, q}} = H^{q, p}$.
\end{topic}

\begin{topic}{hodge-polynomial}{(mixed) Hodge polynomial}
    The \textbf{Hodge polynomial} of a \tref{pure-hodge-structure}{pure Hodge structure} $H$ is the polynomial,
    \[ P_\textup{hodge}(H) = \sum_{p, q \in \ZZ} h^{p, q}(H) u^p v^q , \]
    where $h^{p, q}(H) = \dim_\CC H^{p, q}$ denote the hodge numbers.
\end{topic}

\begin{topic}{mixed-hodge-structure}{mixed Hodge structure}
    A \textbf{mixed Hodge structure} on a $\ZZ$-module $H_\ZZ$ consists of an increasing $\ZZ$-filtration $W_\bdot$ on $H_\QQ = H_\ZZ \otimes \QQ$,
    \[ 0 \subset \cdots \subset W_i \subset W_{i + 1} \cdots \subset H_\QQ \]
    and a decreasing $\NN$-filtration $F^\bullet$ on $H_\CC = H_\ZZ \otimes \CC$,
    \[ H_\CC = F^0 \supset F^1 \supset \cdots \supset 0 \]
    such that the induced filtrations (obtained by intersections) of $F^\bdot$ on the graded pieces $\left(\text{Gr}^W_k H_\QQ\right) \otimes_\QQ \CC := \left(W_k H_\QQ / W_{k - 1} H_\QQ\right) \otimes_\QQ \CC$ are \tref{pure-hodge-structure}{pure (rational) Hodge structures} of weight $k$.
    
    A \textbf{morphism of mixed Hodge structures} is a morphism $f : H_\ZZ \to H'_\ZZ$ of $\ZZ$-modules compatible with the two filtrations $W_\bdot$ and $F^\bdot$.
    
    The numbers $h^{p, q}(H) := \dim_\CC \textup{Gr}_F^p \textup{Gr}^W_{p + q}(H_\CC)$ are called the \textbf{mixed Hodge numbers} of the mixed Hodge structure $H$.
\end{topic}

\begin{topic}{quiver}{quiver}
    A \textbf{quiver} is a finite directed graph, where loops and multiple arrows between vertices are allowed.    
\end{topic}

\begin{topic}{quiver-representation}{quiver representation}
    Let $Q$ be a \tref{quiver}{quiver}. A \textbf{representation} of $Q$ is a collection of vector spaces $V_i$ for each vertex $i \in Q_0$ and linear maps $V(\alpha) : V_i \to V_j$ for each arrow $(\alpha : i \to j) \in Q_1$.
    
    A \textbf{morphism} of quiver representations $V \to W$ is a collection of maps $\varphi_i : V_i \to W_i$ for each vertex $i \in Q_0$ such that $\varphi_j V(\alpha) = W(\alpha) \varphi_i$.
    
    The sequence of dimensions $(\dim V_i)_{i \in Q_0} \in \NN Q_0$ is called the \textbf{dimension vector}.
\end{topic}

\begin{topic}{path-algebra}{(quiver) path algebra}
    Let $Q$ be a \tref{quiver}{quiver}. A \textbf{path} in $Q$ is a sequence $\alpha_n \alpha_{n - 1} \cdots \alpha_1$ of arrows such that the head of $\alpha_{i + 1}$ equals the tail of $\alpha_{i}$. The \textbf{path algebra} of $Q$ is defined as a \tref{LA:vector-space}{vector space} with the set of paths of $Q$ as a basis. Multiplication is given by concatenation of paths, and if two paths cannot be connected because their endpoints are different, their product is defined as zero.
\end{topic}

\begin{topic}{euler-form}{Euler form}
    Let $Q$ be a \tref{quiver}{quiver}. The \textbf{Euler form} of $Q$ is the bilinear form on $\ZZ^{Q_0}$ given by
    \[ \langle d, e \rangle_Q = \sum_{i \in Q_0} d_i e_i - \sum_{(\alpha : i \to j) \in Q_1} d_i e_j \]
    for any $d = (d_i)_{i \in Q_0}$ and $e = (e_i)_{i \in Q_0}$.
\end{topic}

\begin{topic}{tits-form}{Tits form}
    Let $Q$ be a \tref{quiver}{quiver}. The \textbf{Tits form} of $Q$ is the quadratic form on $\ZZ^{Q_0}$ given by
    \[ q_Q(d) = \langle d, d \rangle_Q = \sum_{i \in Q_0} d_i^2 - \sum_{(\alpha : i \to j) \in Q_1} d_i d_j \]
    for any $d = (d_i)_{i \in Q_0}$, where $\langle \cdot, \cdot \rangle_Q$ denotes the \tref{euler-form}{Euler form} of $Q$.
\end{topic}

\begin{topic}{pontryagin-dual}{Pontryagin dual}
    Let $G$ be a \tref{TO:locally-compact-space}{locally compact} \tref{TO:topological-group}{topological} \tref{GT:abelian-group}{abelian group}. The \textbf{Pontryagin dual} of $G$, denoted $\widehat{G}$, is the group of \tref{TO:continuous-map}{continuous} \tref{GT:group-homomorphism}{group morphisms} from $G$ to the circle group $S^1$, that is,
    \[ \widehat{G} = \Hom_\textbf{TopGrp}(G, S^1) , \]
    equipped with the \tref{TO:mapping-space}{compact-open topology}.
\end{topic}

\begin{example}{pontryagin-dual}
    \begin{itemize}
        \item The Pontryagin dual of $\ZZ$ is $S^1$, and vice versa.
        \item The Pontryagin dual of $\RR$ is $\RR$ itself.
        \item The Pontryagin dual of $\ZZ/n\ZZ$ is $\ZZ/n\ZZ$ itself.
    \end{itemize}
\end{example}

\begin{example}{pontryagin-dual}
    The \textit{Pontryagin duality theorem} states there is a canonical isomorphism between locally compact abelian topological groups
    \[ G \xrightarrow{\sim} \widehat{\widehat{G}}, \quad g \mapsto (\chi \mapsto \chi(g)) . \]
\end{example}

\begin{topic}{fourier-transform-finite-groups}{Fourier transform finite groups}
    Let $G$ be a finite \tref{GT:group}{group} and $\widehat{G}$ the set of \tref{RT:character-representation}{characters} of \tref{RT:irreducible-representation}{irreducible representations} of $G$. Let $C(G)$ be the $\CC$-vector space of functions $f : G \to \CC$ which are constant on conjugacy classes, and $C(\widehat{G})$ the $\CC$-vector space of functions on $\widehat{G}$.
    
    The \textbf{Fourier transform} on $G$ is given by
    \[ \mathcal{F} : C(G) \to C(\widehat{G}), \quad \mathcal{F}(f)(\chi) = \sum_{g \in G} f(g) \frac{\chi(g)}{\chi(1)} , \]
    and the \textbf{inverse Fourier transform} by
    \[ \widehat{\mathcal{F}} : C(\widehat{G}) \to C(G), \quad \widehat{\mathcal{F}}(F)(g) = \sum_{\chi \in \widehat{G}} F(\chi) \chi(1) \overline{\chi(g)} . \]
    These operations are inverse to each other in the sense that
    \[ \widehat{\mathcal{F}} \circ \mathcal{F} = |G| \cdot \id_{C(G)} \quad \textup{ and } \quad \mathcal{F} \circ \widehat{\mathcal{F}} = |G| \cdot \id_{C(\widehat{G})} \]
\end{topic}

\begin{example}{fourier-transform-finite-groups}
    The \textit{convolution} of two functions $f_1, f_2 \in C(G)$ is given by
    \[ (f_1 * f_2)(g) = \sum_{xy = g} f_1(x) f_2(y) , \]
    and pointwise multiplication of $F_1, F_2 \in C(\widehat{G})$ by
    \[ (F_1 \cdot F_2)(\chi) = F_1(\chi) F_2(\chi) . \]
    It can be verified that $\mathcal{F}(f_1 * f_2) = \mathcal{F}(f_1) \cdot \mathcal{F}(f_2)$. Namely, the
    \[ \sum_{g \in G} \sum_{xy = g} f_1(x) f_2(y) \frac{\rho(g)}{\chi(1)} = \sum_{x, y \in G} f_1(x) f_2(y) \frac{\rho(xy)}{\chi(1)} = \frac{1}{\chi(1)} \left( \sum_{x \in G} f_1(x) \rho(x) \right) \left( \sum_{y \in G} f_2(y) \rho(y) \right) . \]
    By Schur's lemma, both factors are scalar matrices, so the trace of this quantity equals
    \[ \left( \sum_{x \in G} f_1(x) \frac{\chi(x)}{\chi(1)} \right) \left( \sum_{y \in G} f_2(y) \frac{\chi(y)}{\chi(1)} \right) = \mathcal{F}(f_1) \cdot \mathcal{F}(f_2) . \]
    
    
    % \quad \textup{ and } \quad \widehat{\mathcal{F}}(F_1 \cdot F_2) \cdot |G| = \widehat{\mathcal{F}}(F_1) * \widehat{\mathcal{F}}(F_2) . \]
\end{example}
