\begin{topic}{fourier-transform-finite-groups}{Fourier transform finite groups}
    Let $G$ be a finite \tref{GT:group}{group} and $\widehat{G}$ the set of \tref{RT:character-representation}{characters} of \tref{RT:irreducible-representation}{irreducible representations} of $G$. Let $C(G)$ be the $\CC$-vector space of functions $f : G \to \CC$ which are constant on conjugacy classes, and $C(\widehat{G})$ the $\CC$-vector space of functions on $\widehat{G}$.
    
    The \textbf{Fourier transform} on $G$ is given by
    \[ \mathcal{F} : C(G) \to C(\widehat{G}), \quad \mathcal{F}(f)(\chi) = \sum_{g \in G} f(g) \frac{\chi(g)}{\chi(1)} , \]
    and the \textbf{inverse Fourier transform} by
    \[ \widehat{\mathcal{F}} : C(\widehat{G}) \to C(G), \quad \widehat{\mathcal{F}}(F)(g) = \sum_{\chi \in \widehat{G}} F(\chi) \chi(1) \overline{\chi(g)} . \]
    These operations are inverse to each other in the sense that
    \[ \widehat{\mathcal{F}} \circ \mathcal{F} = |G| \cdot \id_{C(G)} \quad \textup{ and } \quad \mathcal{F} \circ \widehat{\mathcal{F}} = |G| \cdot \id_{C(\widehat{G})} \]
\end{topic}

\begin{example}{fourier-transform-finite-groups}
    The \textit{convolution} of two functions $f_1, f_2 \in C(G)$ is given by
    \[ (f_1 * f_2)(g) = \sum_{xy = g} f_1(x) f_2(y) , \]
    and pointwise multiplication of $F_1, F_2 \in C(\widehat{G})$ by
    \[ (F_1 \cdot F_2)(\chi) = F_1(\chi) F_2(\chi) . \]
    It can be verified that $\mathcal{F}(f_1 * f_2) = \mathcal{F}(f_1) \cdot \mathcal{F}(f_2)$. Namely, the
    \[ \sum_{g \in G} \sum_{xy = g} f_1(x) f_2(y) \frac{\rho(g)}{\chi(1)} = \sum_{x, y \in G} f_1(x) f_2(y) \frac{\rho(xy)}{\chi(1)} = \frac{1}{\chi(1)} \left( \sum_{x \in G} f_1(x) \rho(x) \right) \left( \sum_{y \in G} f_2(y) \rho(y) \right) . \]
    By Schur's lemma, both factors are scalar matrices, so the trace of this quantity equals
    \[ \left( \sum_{x \in G} f_1(x) \frac{\chi(x)}{\chi(1)} \right) \left( \sum_{y \in G} f_2(y) \frac{\chi(y)}{\chi(1)} \right) = \mathcal{F}(f_1) \cdot \mathcal{F}(f_2) . \]
\end{example}

\begin{topic}{topological-quantum-field-theory}{Topological Quantum Field Theory (TQFT)}
    Let $(\textbf{Bd}_n, \sqcup, \varnothing)$ be the \tref{CT:monoidal-category}{monoidal category} of (diffeomorphism classes of) $n$-dimensional (\tref{DG:orientable-manifold}{oriented}) \tref{DG:bordism}{bordisms}, and let $k$ be a \tref{AA:ring}{commutative ring}. An $n$-dimensional \textbf{Topological Quantum Field Theory (TQFT)} over $k$ is a \tref{CT:monoidal-functor}{monoidal functor}
    \[ Z : \textbf{Bd}_n \to k\textup{-}\textbf{Mod} , \]
    where the monoidal structure on $k\textup{-}\textbf{Mod}$ is given by the \tref{AA:tensor-product}{tensor product}.
\end{topic}

% \begin{example}{topological-quantum-field-theory}
%     Let $Z : \textbf{Bd}_1 \to k\textup{-}\textbf{Mod}$ be a $1$-dimensional TQFT. Since the objects of $\textbf{Bd}_1$ consist of finitely many points, the value of $Z$ on objects is completely determined by $Z(\star)$, as $Z(k \textup{ points}) = Z(\star)^{\otimes k}$. 
% \end{example}

\begin{example}{topological-quantum-field-theory}
    Let $Z : \textbf{Bd}_n \to k\textup{-}\textbf{Mod}$ be an $n$-dimensional TQFT over a field $k$. For any closed oriented $(n - 1)$-dimensional manifold $M$, let $\overline{M}$ be the manifold with reversed orientation. Consider the bordisms
    \[ U_M : M \sqcup \overline{M} \to \varnothing \quad \textup{ and } \quad U_M^\dag : \varnothing \to \overline{M} \sqcup M \]
    given by the cylinder $M \times [0, 1]$.
    Note that $Z(U_M)(1) = \sum_{i = 1}^{m} v_i \otimes \overline{v}_i$ for some $v_i \in Z(M)$ and $\overline{v}_i \in Z(\overline{M})$, and moreover we can pick such $v_i$ linearly independent and $\overline{v}_i$ linearly independent.
    Now since $(U_M \sqcup \id_M) \circ (\id_M \sqcup U_M^\dag) = \id_M$, it follows that
    \[ v = \sum_{i = 1}^{m} Z(U_M^\dag)(v \otimes \overline{v}_i) v_i \]
    for all $v \in Z(M)$. In particular, this implies $Z(M)$ is finite-dimensional and $\{ v_1, \ldots, v_m \}$ is a basis for $Z(M)$.
    Completely analogous, switching the roles of $M$ and $\overline{M}$, from the equality $(\id_{\overline{M}} \sqcup U_M) \circ (U_M^\dag \sqcup \id_{\overline{M}}) = \id_{\overline{M}}$ we find that
    \[ \overline{v} = \sum_{i = 1}^{m} Z(U_M^\dag)(v_i \otimes \overline{v}) \overline{v}_i \]
    for all $\overline{v} \in Z(\overline{M})$, so $Z(\overline{M})$ is finite-dimensional as well, and $\{ \overline{v}_1, \ldots, \overline{v}_m \}$ is a basis for $Z(\overline{M})$. Moreover, this shows that $Z(\overline{M})$ can be identified as the \tref{LA:dual-vector-space}{dual} to $Z(M)$, with $\{ \overline{v}_1, \ldots, \overline{v}_m \}$ as the dual basis of $\{ v_1, \ldots, v_m \}$ with respect to the non-degenerate pairing
    \[ Z(M) \otimes_k Z(\overline{M}) \to k, \quad v \otimes \overline{v} \mapsto Z(U_M^\dagger)(v \otimes \overline{v}) . \]
\end{example}
