\begin{topic}{simplex-category}{simplex category}
    The \textbf{simplex category}, often denoted by $\Delta$, is the \tref{CT:category}{category} whose objects are sets of the form
    \[ [n] = \{ 0, 1, 2, \ldots, n \}, \qquad \text{ with } n \ge 0 , \]
    and whose morphisms are order-preserving functions between these sets.
\end{topic}

\begin{topic}{simplicial-object}{simplicial object}
    A \textbf{simplicial object} in a \tref{CT:category}{category} $\mathcal{C}$ is a \tref{CT:functor}{functor}
    \[ X : \Delta^\text{op} \to \mathcal{C} , \]
    where $\Delta$ denotes the \tref{simplex-category}{simplex category}. Concretely, a simplicial object consists of a family of objects $X_n$ in $\mathcal{C}$ and a morphism $X_m \to X_n$ for each function $f : [n] \to [m]$, which compose nicely.
    
    The \textit{face maps} are the maps $d_{n, i} : X_n \to X_{n - 1}$ corresponding to the function $\delta^{n, i} : [n - 1] \to [n]$ that misses $i$.
    
    The \textit{degeneracy maps} are the maps $s_{n, i} : X_n \to X_{n + 1}$ corresponding to the function $\sigma^{n, i} : [n + 1] \to [n]$ that repeat $i$.
\end{topic}

\begin{example}{simplicial-object}
    Let $X$ be a topological space. The \textbf{singular simplicial set} of a topological space is the simplicial set $\mathcal{S}(X)_\bdot$ given by
    \[ \mathcal{S}(X)_n = \{ \text{cont. maps } \Delta^n \to X \} . \]
    Every $[n] \to [m]$ induces a map $\Delta^n \to \Delta^m$ by labelling the vertices, which in turn induces a map $\mathcal{S}(X)_m \to \mathcal{S}(X)_m$. Also see \tref{AT:singular-homology}{singular homology}.
\end{example}

\begin{topic}{kan-complex}{Kan complex}
    A \textbf{Kan complex} is a \tref{simplicial-object}{simplicial set} $X$ satisfying the \textit{Kan condition}: any map of simplicial sets $\Lambda^n_i \to X$ extends to a map of simplicial sets $\Delta^n \to X$. That is,
    \[ \begin{tikzcd} \Lambda^n_i \arrow{r}{f} \arrow{d} & X \\ \Delta^n \arrow[dashed]{ur} & \end{tikzcd} \]
\end{topic}

\begin{topic}{koszul-complex}{Koszul complex}
    Let $F$ be a \tref{CA:free-module}{free module} of finite rank $r$ over a commutative \tref{CA:ring}{ring} $R$. Then, given an $R$-linear map $s : F \to R$ the \textbf{Koszul complex} associated to $s$ is the chain complex of $R$-modules
    \[ K_\bdot(s) : \qquad 0 \to \wedge^r F \xrightarrow{d_r} \wedge^{r - 1} F \xrightarrow{d_{r - 1}} \cdots \xrightarrow{d_2} F \xrightarrow{d_1} R \to 0 \]
    with the differentials given by
    \[ d_k(e_1 \wedge e_2 \ldots \wedge e_k) = \sum_{i = 1}^{k} (-1)^{i + 1} s(e_i) e_1 \wedge \cdots \wedge \hat{e_i} \wedge \cdots \wedge e_k , \]
    where the hat indicates the term is missing. Note that $d_1 = s$.
\end{topic}


\begin{topic}{model-category}{model category}
    A \textbf{model category} is a \tref{CT:category}{category} $\mathcal{C}$ with three distinguished classes of maps: \textit{weak equivalences}, \textit{fibrations} and \textit{cofibrations}, each of which is closed under composition and contains all identity maps. A map which is both a fibration (resp. cofibration) and a weak equivalence is called an \textit{acyclic fibration} (resp. \textit{acyclic cofibration}). We require the following axioms.
    \begin{itemize}
        \item Finite limits and colimits exist in $\mathcal{C}$.
        \item For maps $f$ and $g$ in $\mathcal{C}$ such that $gf$ is defined, if two out of the three maps $f, g, gf$ are weak equivalences, then so is the third.
        \item If $f$ is a retract of $g$, and $g$ is a fibration, cofibration or a weak equivalence, then so is $f$.
        \item For any commutative diagram
        \[ \begin{tikzcd} A \arrow[swap]{d}{i} \arrow{r}{f} & X \arrow{d}{p} \\ B \arrow[swap]{r}{g} \arrow[dashed]{ur} & Y \end{tikzcd} \]
        such that either (i) $i$ is a cofibration and $p$ an acyclic fibration, or (ii) $i$ an acyclic cofibration and $p$ a fibration, there exists a lift $h : B \to X$ making the diagram commute.
        \item Any map $f$ can be factored in two ways: (i) $f = pi$ with $i$ a cofibration and $p$ an acyclic fibration, and (ii) $f = pi$ with $i$ an acyclic cofibration and $p$ a fibration.
    \end{itemize}
    
    By the first axiom, $\mathcal{C}$ has an \tref{CT:initial-object}{initial} object $\varnothing$ and a \tref{CT:terminal-object}{terminal} object $\star$. An object $A$ of $\mathcal{C}$ is called \textbf{cofibrant} if $\varnothing \to A$ is a cofibration, and \textbf{fibrant} if $A \to \star$ is a fibration.
\end{topic}

\begin{example}{model-category}
    The category $\textbf{Top}$ of topological spaces can be given the structure of a model category by defining a map $f : X \to Y$ to be a
    \begin{itemize}
        \item \textit{weak equivalence} if $f$ is a \tref{AT:weak-homotopy-equivalence}{weak-homotopy-equivalence},
        \item \textit{cofibrations} if $f$ is a retract of a map $X \to Y'$ in which $Y'$ is obtained from $X$ by attaching cells,
        \item \textit{fibration} if $f$ is a Serre fibration.
    \end{itemize}
    In this model category, all objects are fibrant and CW-complexes are cofibrant.
\end{example}

\begin{example}{model-category}
    The Kan--Quillen model structure on $\textbf{Set}_\Delta$, the category of \tref{UN:simplicial-object}{simplicial sets}. Take
    \begin{itemize}
        \item \textit{weak equivalences} = weak homotopy equivalences,
        \item \textit{fibrations} = Kan fibrations,
        \item \textit{cofibrations} = monomorphisms (degreewise injective maps).
    \end{itemize}
    All objects are cofibrant, and the fibrant objects are the Kan complexes.
\end{example}

\begin{example}{model-category}
    Let $R$ be a commutative \tref{CA:ring}{ring}, and let $\textbf{Ch}_R = \textbf{Ch}_{\ge 0}(\textbf{Mod}_R)$ be the category of chain complexes of $R$-modules, in non-negative degree. Take
    \begin{itemize}
        \item \textit{weak equivalences} = quasi-isomorphisms,
        \item \textit{fibrations} = degreewise surjective maps,
        \item \textit{cofibrations} = degreewise injective maps with projective cokernel.
    \end{itemize}
\end{example}

\begin{topic}{quillen-adjunction}{Quillen adjunction}
    Let $M, N$ be \tref{model-category}{model categories}. A \textbf{Quillen adjunction} is an \tref{CT:adjoint-functors}{adjunction}
    \[ F : M \to N, \qquad G : N \to M, \qquad F \dashv G , \]
    such that $F$ preserves cofibrations and $G$ preserves fibrations.
\end{topic}

\begin{topic}{quillen-equivalence}{Quillen equivalence}
    A \tref{quillen-adjunction}{Quillen adjunction} is called a \textbf{Quillen equivalence} if the derived adjunction between homotopy categories is an \tref{CT:equivalence-of-categories}{equivalence of categories}.
\end{topic}


% \begin{topic}{homotopy-pullback-pushout}{homotopy pullback/pushout}
    % Cofibrant replacements!
% \end{topic}

% \begin{topic}{dold-kan-correspondence}{Dold-Kan correspondence}
  
% \end{topic}


\begin{topic}{nerve}{nerve}
    The \textbf{nerve} of a \tref{CT:category}{category} $\mathcal{C}$ is the \tref{simplicial-object}{simplicial set} $\text{N}_\bdot(\mathcal{C})$, where $\text{N}_n(\mathcal{C})$ is the set of \tref{CT:functor}{functors} from $[n] = \{ 0, 1, \ldots, n \}$ (viewed as category: a unique morphism from $i$ to $j$ iff $i \le j$) to $\mathcal{C}$.
    Indeed, for any non-decreasing map $\alpha : [m] \to [n]$, precomposition with $\alpha$ gives a map $\text{N}_n(\mathcal{C}) \to \text{N}_m(\mathcal{C})$.
    
    It can be shown that the \textit{nerve functor}
    \[ \text{N}_\bdot : \textbf{Cat} \to \textbf{Set}_\Delta \]
    is \tref{CT:full-functor}{fully} \tref{CT:faithful-functor}{faithful}.
\end{topic}

\begin{topic}{homotopy-category}{homotopy category}
    The \textbf{homotopy category} of a \tref{simplicial-object}{simplicial set} $S_\bdot$ is the category $\text{h}S_\bdot$ defined as follows:
    \begin{itemize}
        \item The objects are the \textit{vertices} $x \in S_0$.
        \item Every \textit{edge} $e \in S_1$ determines a morphism $[e] : d_1(e) \to d_0(e)$. The collection of morphisms in $\text{h}S_\bdot$ is generated under composition by morphisms of the form $[e]$, subject to the relations
        \[ [s_0(x)] = \id_x \text{ for } x \in S_0, \qquad [d_1(\sigma)] = [d_0(\sigma)] \circ [d_2(\sigma)] \text{ for } \sigma \in S_2 . \]
    \end{itemize}
    
    The functor $S_\bdot \mapsto \text{h}S_\bdot$ is \tref{CT:adjoint-functors}{left adjoint} to the \tref{nerve}{nerve functor}.
\end{topic}

\begin{topic}{infinity-category}{infinity category}
    An \textbf{$\infty$-category} is a \tref{simplicial-object}{simplicial set} $\mathcal{C}$ such that every map of simplicial sets $\Lambda^n_i \to \mathcal{C}$ with $0 < i < n$ can be extended to a map $\Delta^n \to \mathcal{C}$.
    \[ \begin{tikzcd} \Lambda^n_i \arrow{r} \arrow{d} & \mathcal{C} \\ \Delta^n \arrow[dashed]{ur} & \end{tikzcd} \]
    
    One thinks of the vertices $\mathcal{C}_0$ as the objects, the edges $\mathcal{C}_1$ as the arrows ($\id_x = s_0(x)$ and $d_0(f) = \text{cod}(f)$ and $d_1(f) = \text{dom}(f)$), the triangles $\mathcal{C}_2$ as a specified homotopy from $g \circ f$ to $h$, etc.
    
    A \textit{functor} between $\infty$-categories is a map of simplicial sets.
\end{topic}

\begin{example}{infinity-category}
    Any \tref{kan-complex}{Kan complex} is an $\infty$-category.
    
    The \tref{nerve}{nerve} $N_\bdot(\mathcal{C})$ of a \tref{CT:category}{category} $\mathcal{C}$ is an $\infty$-category.
\end{example}

\begin{topic}{dold-kan-correspondence}{Dold--Kan correspondence}
    The \textbf{Dold--Kan correspondence} is an \tref{CT:equivalence-of-categories}{equivalence} between the category of \tref{simplicial-object}{simplicial} \tref{GT:abelian-group}{abelian groups} and (nonnegatively graded) \tref{HA:chain-complex}{chain complexes} of abelian groups,
    \[ \textbf{Ab}_\Delta \simeq \text{Ch}(\ZZ)_{\ge 0} . \]
    To any simplicial abelian group $A_\bdot$, one assigns the \textit{(normalized) Moore complex}
    \[ N(A_\bdot)_n := \bigcap_{i = 0}^{n - 1} \ker d_{n, i} \quad \text{ with differential } \quad \partial_n = d_{n, n}, \]
    where $d_{n, i} : A_n \to A_{n - 1}$ are the face maps of $A_\bdot$.
    
    Inversely, to any chain complex $C_\bdot$ one assigns the simplicial abelian group
    \[ \sigma(C_\bdot)_n := \bigoplus_{[n] \twoheadrightarrow [k]} C_k \]
    and the face and degeneracy maps are given by (todo).
    
    The statement can be generalized to any \tref{HA:abelian-category}{abelian category} $\mathcal{A}$: there is an equivalence
    \[ \mathcal{A}_\Delta \simeq \text{Ch}(\mathcal{A})_{\ge 0} . \]
\end{topic}

\begin{topic}{pure-hodge-structure}{pure Hodge structure}
    A \textbf{pure Hodge structure of weight $k \in \ZZ$} consists of a $\ZZ$-module $H_\ZZ$ of finite rank, and a direct sum decomposition of the complexification
    \[ H_\CC := H_\ZZ \otimes_\ZZ \CC = \bigoplus_{p + q = k} H^{p, q} \quad \text{ with } \quad H^{p, q} = \overline{H^{q, p}} . \]
    One speaks of \textit{rational} or \textit{real} Hodge structures when replacing $V_\ZZ$ by a rational of real vector space.
    
    A \textbf{morphism of Hodge structures} is a morphism $f : H_\ZZ \to H'_\ZZ$ of $\ZZ$-modules such that its complexification $f_\CC$ preserves type, i.e. $f_\CC\left(H^{p, q}\right) \subset (H')^{p, q}$.
    
    The numbers $h^{p, q}(H) := \dim_\CC H^{p, q}$ are called the \textbf{Hodge numbers} of the Hodge structure.
\end{topic}

\begin{example}{pure-hodge-structure}
    For any $H_\ZZ$, taking $H^{k,k} = H_\CC$ and $H^{p, q} = 0$ when $(p, q) \ne (k, k)$ gives the \textit{trivial Hodge structure} of weight $2k$.
\end{example}

\begin{example}{pure-hodge-structure}
    Take $H_\ZZ = 2 \pi i \ZZ \subset \CC$ and $H_\CC = H^{-1, -1}$. This is a pure Hodge structure of weight $-2$. In fact, is the unique $1$-dimensional pure Hodge structure of weight $-2$ up to isomorphism, and it is called the \textit{Tate Hodge structure}, often denoted by $\ZZ(1)$.
\end{example}

\begin{topic}{hodge-filtration}{Hodge filtration}
    A \textbf{Hodge filtration} is an equivalent description of a \tref{pure-hodge-structure}{pure Hodge structure} of weight $k \in \ZZ$, given by a filtration
    \[ H_\CC \supset \cdots \supset F^p \supset F^{p + 1} \supset \cdots \quad \text{ with } \quad F^p \oplus \overline{F^q} = H_\CC \text{ for } p + q = k + 1 . \]
    
    A Hodge filtration can be obtained from a pure Hodge structure, and vice versa, via
    \[ F^p = \bigoplus_{r \ge p} H^{r, k - r} \quad \text{and} \quad H^{p, q} = F^p \cap \overline{F^q} . \]
    % Indeed $F^p$ gives a filtration, and for $p + q = k + 1$ one finds
    % \[ F^p = \bigoplus_{r \ge p} V^{r, k - r} \qquad \text{and} \qquad \overline{F^q} = \bigoplus_{s \ge k - p + 1} \overline{V^{s, k - s}} = \bigoplus_{r \le p - 1} V^{r, k - r}, \]
    % from which follows that $F^p \oplus \overline{F^q} = V_\CC$.
    % In the other direction
    % \[ \bigoplus_{p + q = k} H^{p, q} = \bigoplus_{p + q = k} F^p \cap \overline{F^q} = H_\CC , \]
    % because any $v \in H_\CC$ lies in $F^p \backslash F^{p + 1}$ for some unique $p$, and since $F^{p + 1} \oplus \overline{F^{k - p}} = H_\CC$, we have $v \in F^p \cap \overline{F^{k - p}}$. It is clear that $\overline{H^{p, q}} = H^{q, p}$.
\end{topic}

\begin{topic}{hodge-polynomial}{Hodge polynomial}
    The \textbf{Hodge polynomial} of a \tref{pure-hodge-structure}{pure Hodge structure} $H$ is the polynomial,
    \[ P_\text{hodge}(H) = \sum_{p, q \in \ZZ} h^{p, q}(H) u^p v^q , \]
    where $h^{p, q}(H) = \dim_\CC H^{p, q}$ denote the hodge numbers.
\end{topic}

\begin{topic}{mixed-hodge-structure}{mixed Hodge structure}
    A \textbf{mixed Hodge structure} on a $\ZZ$-module $H_\ZZ$ consists of an increasing $\ZZ$-filtration $W_\bdot$ on $H_\QQ = H_\ZZ \otimes \QQ$,
    \[ 0 \subset \cdots \subset W_i \subset W_{i + 1} \cdots \subset H_\QQ \]
    and a decreasing $\NN$-filtration $F^\bullet$ on $H_\CC = H_\ZZ \otimes \CC$,
    \[ H_\CC = F^0 \supset F^1 \supset \cdots \supset 0 \]
    such that the induced filtrations (obtained by intersections) of $F^\bdot$ on the graded pieces $\text{Gr}^W_k H_\QQ := W_k H_\QQ / W_{k - 1} H_\QQ$ are \tref{pure-hodge-structure}{pure (rational) Hodge structures} of weight $k$.
    
    A \textbf{morphism of mixed Hodge structures} is a morphism $f : H_\ZZ \to H'_\ZZ$ of $\ZZ$-modules compatible with the two filtrations $W_\bdot$ and $F^\bdot$.
    
    The numbers $h^{p, q}(H) := \dim_\CC \textup{Gr}_F^p \textup{Gr}^W_{p + q}(H_\CC)$ are called the \textbf{mixed Hodge numbers} of the mixed Hodge structure $H$.
\end{topic}
