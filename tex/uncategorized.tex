\begin{topic}{simplex-category}{simplex category}
    The \textbf{simplex category}, often denoted by $\Delta$, is the \tref{CT:category}{category} whose objects are sets of the form
    \[ [n] = \{ 0, 1, 2, \ldots, n \}, \qquad \text{ with } n \ge 0 , \]
    and whose morphisms are order-preserving functions between these sets.
\end{topic}

\begin{topic}{simplicial-set}{simplicial set}
    A \textbf{simplicial set} is a \tref{CT:functor}{functor}
    \[ X : \Delta^\text{op} \to \textbf{Set} , \]
    where $\Delta$ denotes the \tref{simplex-category}{simplex category}. Concretely, a simplicial set consists of a family of sets $X_n$ and a map $X_m \to X_n$ for each function $f : [n] \to [m]$, which compose nicely.
    
    The \textit{face maps} are the maps $d_{n, i} : X_n \to X_{n - 1}$ corresponding to the function $\delta^{n, i} : [n - 1] \to [n]$ that misses $i$.
    
    The \textit{degeneracy maps} are the maps $s_{n, i} : X_n \to X_{n + 1}$ corresponding to the function $\sigma^{n, i} : [n + 1] \to [n]$ that repeat $i$.
\end{topic}

\begin{example}{simplicial-set}
    \[ \Delta^n = \Hom_\Delta(-, [n]) \]
    \[ \Lambda^n_i =  \]
\end{example}

\begin{example}{simplicial-set}
    Let $X$ be a topological space. The \textbf{singular set} of a topological space is the simplicial set $\mathcal{S}(X)$ given by
    \[ \mathcal{S}(X)_n = \{ \text{cont. maps } \Delta^n \to X \} . \]
    Every $[n] \to [m]$ induces a map $\Delta^n \to \Delta^m$ by labelling the vertices, which in turn induces a map $\mathcal{S}(X)_m \to \mathcal{S}(X)_m$. Also see \tref{AT:singular-homology}{singular homology}.
\end{example}

\begin{topic}{kan-complex}{Kan complex}
    A \textbf{Kan complex} is a \tref{simplicial-set}{simplicial set} $X$ satisfying the \textit{Kan condition}: any map of simplicial sets $\Lambda^n_i \to X$ extends to a map of simplicial sets $\Delta^n \to X$. That is,
    \[ \begin{tikzcd} \Lambda^n_i \arrow{r}{f} \arrow{d} & X \\ \Delta^n \arrow[dashed]{ur} & \end{tikzcd} \]
\end{topic}

\begin{topic}{koszul-complex}{Koszul complex}
    Let $F$ be a \tref{CA:free-module}{free module} of finite rank $r$ over a commutative \tref{CA:ring}{ring} $R$. Then, given an $R$-linear map $s : F \to R$ the \textbf{Koszul complex} associated to $s$ is the chain complex of $R$-modules
    \[ K_\bdot(s) : \qquad 0 \to \wedge^r F \xrightarrow{d_r} \wedge^{r - 1} F \xrightarrow{d_{r - 1}} \cdots \xrightarrow{d_2} F \xrightarrow{d_1} R \to 0 \]
    with the differentials given by
    \[ d_k(e_1 \wedge e_2 \ldots \wedge e_k) = \sum_{i = 1}^{k} (-1)^{i + 1} s(e_i) e_1 \wedge \cdots \wedge \hat{e_i} \wedge \cdots \wedge e_k , \]
    where the hat indicates the term is missing. Note that $d_1 = s$.
\end{topic}


\begin{topic}{model-category}{model category}
    A \textbf{model category} is a \tref{CT:category}{category} $\mathcal{C}$ with three distinguished classes of maps: \textit{weak equivalences}, \textit{fibrations} and \textit{cofibrations}, each of which is closed under composition and contains all identity maps. A map which is both a fibration (resp. cofibration) and a weak equivalence is called an \textit{acyclic fibration} (resp. \textit{acyclic cofibration}). We require the following axioms.
    \begin{itemize}
        \item Finite limits and colimits exist in $\mathcal{C}$.
        \item For maps $f$ and $g$ in $\mathcal{C}$ such that $gf$ is defined, if two out of the three maps $f, g, gf$ are weak equivalences, then so is the third.
        \item If $f$ is a retract of $g$, and $g$ is a fibration, cofibration or a weak equivalence, then so is $f$.
        \item For any commutative diagram
        \[ \begin{tikzcd} A \arrow[swap]{d}{i} \arrow{r}{f} & X \arrow{d}{p} \\ B \arrow[swap]{r}{g} \arrow[dashed]{ur} & Y \end{tikzcd} \]
        such that either (i) $i$ is a cofibration and $p$ an acyclic fibration, or (ii) $i$ an acyclic cofibration and $p$ a fibration, there exists a lift $h : B \to X$ making the diagram commute.
        \item Any map $f$ can be factored in two ways: (i) $f = pi$ with $i$ a cofibration and $p$ an acyclic fibration, and (ii) $f = pi$ with $i$ an acyclic cofibration and $p$ a fibration.
    \end{itemize}
    
    By the first axiom, $\mathcal{C}$ has an \tref{CT:initial-object}{initial} object $\varnothing$ and a \tref{CT:terminal-object}{terminal} object $\star$. An object $A$ of $\mathcal{C}$ is called \textbf{cofibrant} if $\varnothing \to A$ is a cofibration, and \textbf{fibrant} if $A \to \star$ is a fibration.
\end{topic}

\begin{example}{model-category}
    The category $\textbf{Top}$ of topological spaces can be given the structure of a model category by defining a map $f : X \to Y$ to be a
    \begin{itemize}
        \item \textit{weak equivalence} if $f$ is a \tref{AT:weak-homotopy-equivalence}{weak-homotopy-equivalence},
        \item \textit{cofibrations} if $f$ is a retract of a map $X \to Y'$ in which $Y'$ is obtained from $X$ by attaching cells,
        \item \textit{fibration} if $f$ is a Serre fibration.
    \end{itemize}
    In this model category, all objects are fibrant and CW-complexes are cofibrant.
\end{example}

\begin{example}{model-category}
    The Kan--Quillen model structure on $\textbf{sSet}$, the category of \tref{UN:simplicial-set}{simplicial sets}. Take
    \begin{itemize}
        \item \textit{weak equivalences} = weak homotopy equivalences,
        \item \textit{fibrations} = Kan fibrations,
        \item \textit{cofibrations} = monomorphisms (degreewise injective maps).
    \end{itemize}
    All objects are cofibrant, and the fibrant objects are the Kan complexes.
\end{example}

\begin{example}{model-category}
    Let $R$ be a commutative \tref{CA:ring}{ring}, and let $\textbf{Ch}_R = \textbf{Ch}_{\ge 0}(\textbf{Mod}_R)$ be the category of chain complexes of $R$-modules, in non-negative degree. Take
    \begin{itemize}
        \item \textit{weak equivalences} = quasi-isomorphisms,
        \item \textit{fibrations} = degreewise surjective maps,
        \item \textit{cofibrations} = degreewise injective maps with projective cokernel.
    \end{itemize}
\end{example}

\begin{topic}{quillen-adjunction}{Quillen adjunction}
    Let $M, N$ be \tref{model-category}{model categories}. A \textbf{Quillen adjunction} is an \tref{CT:adjoint-functors}{adjunction}
    \[ F : M \to N, \qquad G : N \to M, \qquad F \dashv G , \]
    such that $F$ preserves cofibrations and $G$ preserves fibrations.
\end{topic}

\begin{topic}{quillen-equivalence}{Quillen equivalence}
    A \tref{quillen-adjunction}{Quillen adjunction} is called a \textbf{Quillen equivalence} if the derived adjunction between homotopy categories is an \tref{CT:equivalence-of-categories}{equivalence of categories}.
\end{topic}


% \begin{topic}{homotopy-pullback-pushout}{homotopy pullback/pushout}
    % Cofibrant replacements!
% \end{topic}

% \begin{topic}{dold-kan-correspondence}{Dold-Kan correspondence}
  
% \end{topic}
