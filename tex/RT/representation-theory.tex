\begin{topic}{representation}{representation}
    A \textbf{representation} of a \tref{GT:group}{group} $G$ on a \tref{LA:vector-space}{vector space} $V$ over a \tref{CA:field}{field} $k$ is a \tref{GT:group-homomorphism}{group homomorphism}
    \[ \rho : G \to \textup{GL}(V) . \]
    The vector space $V$ is called the \textit{representation space} and $\dim_k V$ the \textit{dimension} of the representation.
\end{topic}

\begin{topic}{irreducible-representation}{(ir)reducible representation}
    A \tref{representation}{representation} $\rho : G \to \textup{GL}(V)$ is \textbf{reducible} if there exists a non-zero proper invariant subspace $W \subset V$, that is, $\rho(g) w \in W$ for all $w \in W$ and $g \in G$. If no such subspace exists, the representation is \textbf{irreducible}. If $\rho$ can be written as a direct sum $\rho_1 \oplus \rho_2$, then it is called \textbf{completely reducible}, or \textbf{semisimple}.
\end{topic}

\begin{topic}{faithful-representation}{faithful representation}
    A \tref{representation}{representation} $\rho : G \to \textup{GL}(V)$ is \textbf{faithful} if $\rho$ is injective.
\end{topic}

\begin{topic}{unitary-representation}{unitary representation}
    A \tref{representation}{representation} $\rho : G \to \textup{GL}(V)$ over $\CC$ is \textbf{unitary} if $\rho(g)$ is unitary for all $g \in G$, that is $\rho(g)^\dagger \rho(g) = \rho(g) \rho(g)^\dagger = \id_V$.
\end{topic}

\begin{topic}{character}{character}
    The \textbf{character} of a \tref{representation}{representation} $\rho : G \to \textup{GL}(V)$ is the function
    \[ \chi_\rho : G \to k, \quad g \mapsto \textup{tr}(\rho(g)) . \]
    Note that $\chi_\rho$ is constant on conjugacy classes.
\end{topic}

\begin{topic}{representation-ring}{representation ring}
    Given a \tref{GT:group}{group} $G$ and a \tref{CA:field}{field} $k$, the \textbf{representation ring} $R_k(G)$ is the \tref{GT:free-group}{free abelian group} on isomorphism classes of finite-dimensional \tref{representation}{$k$-representations} of $G$. For the ring structure, addition is given by the direct sum of representations, and multiplication by their tensor product over $k$.
    
    Equivalently, the representation ring $R_k(G)$ is the \tref{HA:grothendieck-group}{Grothendieck ring} of the category of finite-dimensional representations of $G$.
\end{topic}

\begin{example}{representation-ring}
    \begin{itemize}
        \item Any representation of $G = \textup{GL}_1(\CC) = \CC^*$ is a direct sum of $1$-dimensional representations of the form $\rho_n : \CC^* \to \CC^*, z \mapsto z^n$ for some character $n \in \ZZ$. Hence the representation ring of $G$ is $R_\CC(G) = \ZZ[t, t^{-1}]$, where $t$ corresponds to $\rho_1$.
        
        \item Any representation of the cyclic group $G = \ZZ/n\ZZ$ is a direct sum of $1$-dimensional representations which sends $1 \textup{ mod } n$ to an $n$-th root of unity. Hence the (complex) representation ring of $G$ is $R_\CC(G) = \ZZ[t]/(t^n - 1)$, where $t$ corresponds to the representation $1 \mapsto \zeta_n$, a primitive root of unity.
    \end{itemize}
\end{example}

\begin{topic}{equivalent-representations}{equivalent representations}
    Two \tref{representation}{representations} $\rho : G \to \textup{GL}(V)$ and $\rho' : G \to \textup{GL}(W)$ are called \textbf{equivalent}, or \textbf{isomorphic}, if there exists a linear isomorphism $A : V \to W$ such that $\rho'(g) = A \rho(g) A^{-1}$ for all $g \in G$.
\end{topic}

\begin{topic}{schur-lemma}{Schur's lemma}
    \textbf{Schur's lemma} states:
    \begin{enumerate}
        \item If $\rho : G \to \textup{GL}(V)$ and $\rho' : G \to \textup{GL}(W)$ are two \tref{equivalent-representations}{inequivalent} \tref{irreducible-representation}{irreducible} \tref{representation}{representations} over $\CC$, and $A : V \to W$ is a linear map such that $A \rho(g) = \rho'(g) A$ for all $g \in G$, then $A = 0$.
        \item If $\rho : G \to \textup{GL}(V)$ is an irreducible representation over $\CC$, and $A : V \to V$ is a linear map such that $A \rho(g) = \rho(g) A$ for all $g \in G$, then $A$ is a scalar multiple of the identity map.
    \end{enumerate}
\end{topic}

\begin{topic}{clebsch-gordan-series}{Clebsch--Gordan series}
    If $G$ is a finite \tref{GT:group}{group}, and $\rho_i, \rho_j$ two \tref{irreducible-representation}{irreducible} \tref{representation}{representations}, the tensor product $\rho_i \otimes \rho_j$ is generally reducible, and can be written as
    \[ \rho_i \otimes \rho_j = A \left( \bigoplus_k a^k_{ij} \rho_k \right) A^{-1} , \]
    for some invertible $A$ and coefficients $a^k_{ij}$, called the \textbf{Clebsch--Gordan series}. The matrix elements of $A$ in some standard basis are called the \textbf{Clebsch--Gordan coefficients}.
\end{topic}

\begin{topic}{projective-representation}{projective representation}
    A \textbf{projective representation} of a \tref{GT:group}{group} on a \tref{LA:vector-space}{vector space} $V$ over a field $k$ is a \tref{GT:group-homomorphism}{group homomorphism}
    \[ \rho : G \to \textup{PGL}(V) , \]
    where $\textup{PGL}(V) = \textup{GL}(V) / k^*$ is the \textit{projective linear group} of $V$.
\end{topic}

% \begin{example}{projective-representation}
%     Any representation $\rho : G \to \textup{GL}(V)$ can be composed with $\textup{GL}(V) \to \textup{PGL}(V)$ to give a projective representation, but the converse is not true. Consider $G = \textup{SO}(3)$ and $\rho : G \to \textup{PGL}_2(\CC)$ induced by
%     \[ \mathfrak{so}(3) \to \textup{GL}_2(\CC), \quad \sigma_1 \mapsto \begin{pmatrix}  \end{pmatrix}, \quad \sigma_2 \mapsto \begin{pmatrix} \end{pmatrix}, \quad \sigma_3 \mapsto \begin{pmatrix} \end{pmatrix} . \]
% \end{example}

\begin{topic}{dual-representation}{dual representation}
    The \textbf{dual representation} of a \tref{representation}{representation} $\rho : G \to \textup{GL}(V)$ is the representation on the \tref{LA:dual-vector-space}{dual vector space} $V^*$ given by
    \[ \rho^* : G \to \textup{GL}(V^*), \quad g \mapsto \rho(g^{-1})^T = (-) \circ \rho(g^{-1}) . \]
\end{topic}

\begin{topic}{first-orthogonality-theorem}{first orthogonality theorem}
    Let $G$ be a finite \tref{GT:group}{group} and let $\rho_1 : G \to \textup{GL}(V)$ and $\rho_2 : G \to \textup{GL}(W)$ be two \tref{irreducible-representation}{irreducible representations} over $\CC$ with corresponding \tref{character}{characters} $\chi_1$ and $\chi_2$. Then the \textbf{first orthogonality theorem} states that
    \[ \frac{1}{|G|} \sum_{g \in G} \chi_1(g) \overline{\chi_2(g)} = \left\{ \begin{array}{cl} 1 & \textup{ if } \rho_1 \simeq \rho_2 , \\ 0 & \textup{ if } \rho_1 \not\simeq \rho_2 , \end{array} \right. \]
    where $\simeq$ denotes \tref{equivalent-representations}{equivalence}.
\end{topic}

\begin{example}{first-orthogonality-theorem}
    \textbf{Proof}. Let $A : W \to V$ be an arbitrary linear map, and take
    \[ B = \sum_{g \in G} \rho_1(g) A \rho_2(g^{-1}) . \]
    Note that
    \[ \rho_1(h) B = \sum_{g \in G} \rho_1(hg) A \rho_2(g^{-1}) = \sum_{g' \in G} \rho_1(g') A \rho_2(g'^{-1} h) = B \rho_2(h) . \]
    If $\rho_1 \simeq \rho_2$, then \tref{schur-lemma}{Schur's first lemma} implies $B = \lambda I$ for some $\lambda \in \CC$, and if $\rho_1 \not\simeq \rho_2$, then \tref{schur-lemma}{Schur's second lemma} yields $B = 0$. Now taking $A_{\ell m} = \delta_{\ell r} \delta_{ms}$, in terms of some basis for $V$ and $W$, gives
    \[ \sum_{g \in G} (\rho_1(g))_{ir} (\rho_2(g^{-1}))_{sj} = \left\{ \begin{array}{cl} \lambda_{rs} \delta_{ij} & \textup{ if } \rho_1 \simeq \rho_2 , \\ 0 & \textup{ if } \rho_1 \not\simeq \rho_2 . \end{array} \right. \]
    Whenever $\rho_1 \simeq \rho_2$, we can set $i = j$ and sum over $i$ to obtain $|G| \delta_{rs} = \lambda_{rs} \dim_\CC(V)$. Substituting this into the above, setting $r = i$ and $s = j$ and summing over $i$ and $j$ now gives
    \[ \sum_{g \in G} \chi_1(g) \chi_2(g^{-1}) = \left\{ \begin{array}{cl} |G| & \textup{ if } \rho_1 \simeq \rho_2 , \\ 0 & \textup{ if } \rho_1 \not\simeq \rho_2 , \end{array} \right. \]
    as desired.
\end{example}

\begin{example}{first-orthogonality-theorem}
    Define an inner product on the characters of $G$,
    \[ \langle \chi_1, \chi_2 \rangle = \frac{1}{|G|} \sum_{g \in G} \chi_1(g) \chi_2(g^{-1}) . \]
    The first orthogonality theorem now says that $\langle \chi, \chi \rangle = 1$ for any character $\chi$, and that inequivalent characters are orthogonal. In particular, two irreducible representations $\rho_1$ and $\rho_2$ are equivalent if and only if their characters are equal. Namely, if $\rho_1 \not\simeq \rho_2$, then $\langle \chi_1, \chi_2 \rangle = 0$, and since $\langle \chi_1, \chi_1 \rangle = 1$, we must have $\chi_1 \ne \chi_2$.
\end{example}
