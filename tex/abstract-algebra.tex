\begin{topic}{monoid}{monoid}
    A \textbf{monoid} is a set $M$ with a binary operation $\cdot : M \times M \to M$ satisfying
    \begin{itemize}
        \item (\textit{associative}) $(a \cdot b) \cdot c = a \cdot (b \cdot c)$ for all $a, b, c \in M$,
        \item (\textit{identity element}) there exists an element $e \in M$ such that $e \cdot a = a = a \cdot e$ for all $a \in M$.
    \end{itemize}
\end{topic}

\begin{topic}{semigroup}{semigroup}
    A \textbf{semigroup} is a set $S$ with a binary operation $\cdot : S \times S \to S$ which is associative, that is, $(a \cdot b) \cdot c = a \cdot (b \cdot c)$ for all $a, b, c \in S$.
\end{topic}

% \begin{topic}{dg-algebra}{dg-algebra}
%     A \textbf{differential graded algebra} $A$ is a graded algebra with a map $d : A \to A$ having degree $1$ (cochain complex convention) or $-1$ (chain complex convention) satisfying
%     \begin{itemize}
%         \item (\textit{chain condition}) $d^2 = 0$, this gives $A$ the structure of a (co)chain complex,
%         \item (\textit{graded Leibniz rule}) $d(a \cdot b) = (da) \cdot b + (-1)^{\deg(a)} a \cdot (db)$.
%     \end{itemize}
% \end{topic}

% \begin{topic}{dg-category}{dg-category}
%     A \textbf{differential graded category} is a category whose morphism sets are \tref{chain-complex}{cochain complexes}, that is
%     \[ \Hom(A, B) = \bigoplus_{n \in \ZZ} \Hom_n(A, B) \]
%     with differential $d : \Hom_n(A, B) \to \Hom_{n + 1}(A, B)$ such that $d^2 = 0$. Furthermore, composition of morphisms
%     \[ \Hom(A, B) \otimes \Hom(B, C) \to \Hom(A, C) \]
%     must be a chain map and $d(\id_A) = 0$.
% \end{topic}

\begin{topic}{lie-algebra}{Lie algebra}
    A \textbf{Lie algebra} is is a \tref{LA:vector-space}{vector space} $\mathfrak{g}$ over a field $k$, together with an operation $\mathfrak{g} \times \mathfrak{g} \to \mathfrak{g}, (x, y) \mapsto [x, y]$ called the \textbf{Lie bracket}, satisfying
    \begin{itemize}
        \item (\textit{bilinearity}) $[ax + by, z] = a[x, z] + b[y, z]$ and $[z, ax + by] = a[z, x] + b[z, y]$ for all $x, y, z \in \mathfrak{g}$ and $a, b \in k$,
        \item (\textit{alternativity}) $[x, x] = 0$ for all $x \in \mathfrak{g}$,
        \item (\textit{Jacobi identity}) $[x, [y, z]] + [z, [x, y]] + [y, [z, x]] = 0$ for all $x, y, z \in \mathfrak{g}$.
    \end{itemize}
\end{topic}

\begin{example}{lie-algebra}
    Consider $\mathfrak{g} = \RR^3$ with bracket operation defined by the \textit{cross product} $[x, y] = x \times y$.
\end{example}

\begin{example}{lie-algebra}
    For any \tref{DG:lie-group}{Lie group} $G$, the tangent space at the identity $\mathfrak{g} = T_e G$ has a Lie algebra structure, called the \textit{Lie algebra} of $G$. Tangent vectors $v \in T_e G$ correspond one-to-one to \textit{left invariant vector fields} on $G$, i.e. vector fields $X$ satisfying $X_g = ((L_g)_* X)_e$ for all $g \in G$, where $L_g : G \to G$ denotes left multiplication by $g$. The Lie bracket on $\mathfrak{g}$ is then given by the \tref{lie-bracket-vector-fields}{Lie bracket of vector fields}. Note that the Lie bracket of left-invariant vector fields is indeed also left-invariant.
\end{example}
