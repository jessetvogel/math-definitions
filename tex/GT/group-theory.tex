\begin{topic}{group}{group}
    A \textbf{group} is a set $G$ together with an operation $G \times G \to G$ (the \textit{group law}) written as $(x, y) \mapsto xy$, and an element $1 \in G$ (the \textit{unit}), satisfying
    \begin{itemize}
        \item (\textit{associativity}) $(xy)z = x(yz)$ for all $x, y, z \in G$,
        \item (\textit{unit element}) $1 \cdot x = x \cdot 1 = x$ for all $x \in G$,
        \item (\textit{inverses}) for all $x \in G$ there exists an $x^{-1} \in G$ such that $x x^{-1} = x^{-1} x = 1$, called the \textit{inverse} of $x$.
    \end{itemize}
\end{topic}

\begin{topic}{subgroup}{subgroup}
    A \textbf{subgroup} $H$ of a \tref{group}{group} $G$ is a subset $H \subset G$ which, with the same group law and unit, is itself a group.
\end{topic}

\begin{topic}{abelian-group}{abelian group}
    A \tref{group}{group} $G$ is called \textbf{abelian} if $xy = yx$ for all $x, y \in G$.
\end{topic}

\begin{topic}{order}{order}
    The \textbf{order} of a \tref{group}{group} $G$ is its number of elements.
    
    The \textbf{order} of an element $x \in G$, denoted $\text{ord}(x)$, is the least positive integer $n$ such that $x^n = 1$. If no such $n$ exists, then $\text{ord}(x) = \infty$.
\end{topic}

\begin{topic}{cyclic-group}{cyclic group}
    A \textbf{cyclic group} is a \tref{group}{group} $G$ generated by a single element $x \in G$, that is $G = \{ x^n : x \in \ZZ \}$.
\end{topic}

\begin{topic}{group-homomorphism}{group homomorphism}
    Let $G$ and $H$ be two \tref{group}{groups}. A \textbf{homomorphism} from $G$ to $H$ is a map $f : G \to H$ satisfying $f(xy) = f(x) f(y)$ for all $x, y \in G$.
\end{topic}

\begin{topic}{kernel}{kernel}
    Let $f : G \to H$ be a \tref{group-homomorphism}{group homomorphism}. The \textbf{kernel} of $f$, denoted $\ker f$, is defined as
    \[ \ker f = \{ x \in G \mid f(x) = 0 \} . \]
    It is a \tref{normal-subgroup}{normal} \tref{subgroup}{subgroup} of $G$.
\end{topic}

\begin{topic}{group-center}{group center}
    The \textbf{center} of a \tref{group}{group} $G$ is the subgroup
    \[ Z(G) = \{ x \in G \mid xy = yx \text{ for all } y \in G \} . \]
\end{topic}

\begin{topic}{symmetric-group}{symmetric group}
    Let $\Sigma$ be a set. The \textbf{symmetric group} on $\Sigma$ is the \tref{group}{group} $S_\Sigma$ of all bijections $\Sigma \to \Sigma$, where multiplication is given by composition, and the unit is given by the identity map $\id_\Sigma$.
    
    Given an integer $n \ge 1$, one writes $S_n$ for the symmetric group on the set $\Sigma = \{ 1, 2, \ldots, n \}$. Elements of $S_n$ are called \textbf{permutations}.
\end{topic}

\begin{topic}{cayleys-theorem}{Cayley's theorem}
    \textbf{Cayley's theorem} states that every \tref{group}{group} $G$ is isomorphic to a \tref{subgroup}{subgroup} of the \tref{symmetric-group}{symmetric group} $S_G$. In particular, $G$ is isomorphic to the image of the morphism
    \[ \varphi : G \to S_G, \qquad x \mapsto (y \mapsto xy) . \]
\end{topic}

\begin{topic}{cyclic-permutation}{cyclic permutation}
    A \tref{symmetric-group}{permutation} $\sigma \in S_n$ is a \textbf{cyclic permutation}, or \textbf{cycle}, of length $k$ if there exist $k$ distinct integers $1 \le a_1, \ldots, a_k \le n$ with $\sigma(a_i) = a_{i + 1}$ for $1 \le i < k$ and $\sigma(a_k) = a_1$, and $\sigma(x) = x$ for $x \not\in \{ a_1, \ldots, a_k \}$. This is denoted by
    \[ \sigma = (a_1 \;\; a_2 \;\; \ldots \;\; a_k) . \]
    A cycle of length $2$ is also called a \textbf{transposition}.
\end{topic}

\begin{topic}{permutation-sign}{permutation sign}
    The \textbf{sign} of a \tref{symmetric-group}{permutation} $\sigma \in S_n$ is defined as
    \[ \text{sign}(\sigma) = \prod_{1 \le i < j \le n} \frac{\sigma(j) - \sigma(i)}{j - i} \in \{ +1, -1 \} . \]
    We call $\sigma$ \textit{even} if $\text{sign}(\sigma) = 1$ and \textit{odd} if $\text{sign}(\sigma) = -1$.
    
    This defines a \tref{group-homomorphism}{group homomorphism}
    \[ \text{sign} : S_n \to \{ +1, -1 \} . \]
\end{topic}

\begin{topic}{alternating-group}{alternating group}
    For $n \ge 1$, the \textbf{alternating group} $A_n$ is the subgroup of the \tref{symmetric-group}{symmetric group} $S_n$ of all \tref{permutation-sign}{even} permutations.
\end{topic}

\begin{topic}{coset}{coset}
    Let $G$ be a \tref{group}{group} and $H \subset G$ a \tref{subgroup}{subgroup}. A \textbf{left coset} of $H$ is a set of the form $gH = \{ gh : h \in H \}$ for some $g \in G$. A \textbf{right coset} of $H$ is a set of the form $Hg = \{ hg : h \in H \}$ for some $g \in G$.
    
    In particular, when $H$ is \tref{normal-subgroup}{normal}, left and right cosets coincide since $gH = Hg$ for all $g \in G$.
\end{topic}

\begin{topic}{index-subgroup}{index subgroup}
    The \textbf{index} of a \tref{subgroup}{subgroup} $H$ of a \tref{group}{group} $G$ is the number of \tref{coset}{left cosets} (or equivalently, right cosets) of $H$ in $G$, and is denoted $[G : H]$.
\end{topic}

\begin{topic}{normal-subgroup}{normal subgroup}
    A \tref{subgroup}{subgroup} $N$ of a \tref{group}{group} $G$ is called \textbf{normal} if $N = g N g^{-1}$ for all $g \in G$.
\end{topic}

\begin{example}{normal-subgroup}
    The \tref{kernel}{kernel} of a morphism $f : G \to H$ is always a normal subgroup. Indeed, for any $x \in \ker f$ and $g \in G$ we have
    \[ f(gxg^{-1}) = f(g) f(x) f(g)^{-1} = f(g) f(g)^{-1} = 1 , \]
    so $gxg^{-1} \in \ker f$, and thus $g (\ker f) g^{-1} \subset \ker f$. The other inclusion is shown completely similarly.
    
    Conversely, any normal subgroup $N \subset G$ is the kernel of the quotient map $\pi : G \to G / N$.
\end{example}

\begin{topic}{central-subgroup}{central subgroup}
    A \tref{subgroup}{subgroup} $H$ of a \tref{group}{group} $G$ is called \textbf{central} if $H$ lies in the \tref{group-center}{center} of $G$.
\end{topic}

\begin{topic}{quotient-group}{quotient group}
    Let $G$ be a \tref{group}{group} and $N$ a \tref{normal-subgroup}{normal subgroup}. The \textbf{quotient group} $G/N$ is the group of \tref{coset}{cosets}
    \[ G/N = \{ gN : g \in G \} \]
    with group law $(gN)(hN) = (gh)N$ and unit $N$.
\end{topic}

\begin{topic}{simple-group}{simple group}
    A \tref{group}{group} $G$ is called \textbf{simple} if it is non-trivial and its only \tref{normal-subgroup}{normal subgroups} are $\{ 1 \}$ and $G$.
\end{topic}

\begin{topic}{torsion-subgroup}{torsion subgroup}
    The \textbf{torsion subgroup} of a \tref{group}{group} $G$ is the subgroup of elements of finite \tref{order}{order}.
    \[ G_{\text{tor}} = \{ x \in G \mid \text{ord}(x) < \infty \} \]
\end{topic}

\begin{topic}{normalizer}{normalizer}
    Let $H$ be a \tref{subgroup}{subgroup} of a \tref{group}{group} $G$. The \textbf{normalizer} of $H$ is the subgroup
    \[ N_H = \{ x \in G \mid x H x^{-1} = H \} . \]
\end{topic}

\begin{topic}{group-action}{group action}
    Let $G$ be a \tref{group}{group} and $X$ a set. An \textbf{action} of $G$ on $X$ is map
    \[ G \times X \to X : (g, x) \mapsto g \cdot x \]
    satisfying $1 \cdot x = x$ and $g \cdot (h \cdot x) = (gh) \cdot x$ for all $g, h \in G$ and $x \in X$.
\end{topic}

\begin{topic}{free-group-action}{free group action}
    A \tref{group-action}{group action} of a \tref{group}{group} $G$ on a set $X$ is \textbf{free} if $g \cdot x = x$ implies $g = 1$, for all $x \in X$.
\end{topic}

\begin{topic}{transitive-group-action}{transitive group action}
    A \tref{group-action}{group action} of a \tref{group}{group} $G$ on a set $X$ is \textbf{transitive} if for all $x, y \in X$ there exists a $g \in G$ such that $y = g \cdot x$.
\end{topic}

\begin{topic}{faithful-group-action}{faithful group action}
    A \tref{group-action}{group action} of a \tref{group}{group} $G$ on a set $X$ is \textbf{faithful} if $g \cdot x = x$ for all $x \in X$, implies that $g = 1$.
\end{topic}

\begin{topic}{regular-group-action}{regular group action}
    A \tref{group-action}{group action} of a \tref{group}{group} $G$ on a set $X$ is \textbf{regular} if it is \tref{transitive-group-action}{transitive} and \tref{free-group-action}{free}.
\end{topic}

\begin{example}{regular-group-action}
    Any group $G$ act on itself by left multiplication, $G \times G \to G$, $(g, h) \mapsto gh$. This group action is transitive since $(hg^{-1}) g = h$ for all $g, h \in G$, and free because $gh = h$ implies $g = 1$ for all $h \in G$. Hence, this group action is regular.
\end{example}

\begin{topic}{centralizer}{centralizer}
    Let $G$ be a \tref{group}{group}. The \textbf{centralizer} of an element $g \in G$ is the subgroup
    \[ G_g = \{ h \in G \mid h g h^{-1} = g \} . \]
\end{topic}

\begin{topic}{stabilizer}{stabilizer}
    Let $G$ be a \tref{group}{group} \tref{group-action}{acting} on a set $X$. Then the \textbf{stabilizer} of $x \in X$ is the \tref{subgroup}{subgroup}
    \[ G_x = \{ g \in G \mid g \cdot x = x \} . \]
\end{topic}

\begin{topic}{orbit}{orbit}
    Let $G$ be a \tref{group}{group} \tref{group-action}{acting} on a set $X$. The \textbf{orbit} of an element $x \in X$ is the set
    \[ Gx = \{ g \cdot x \mid g \in G \} \subset X . \]
\end{topic}

\begin{topic}{solvable-group}{solvable group}
    A \tref{group}{group} $G$ is \textbf{solvable} if there exist \tref{subgroup}{subgroups} $H_0, H_1, \ldots, H_r$ of $G$ such that
    \[ G = H_0 \supset H_1 \supset \cdots \supset H_r = \{ 1 \} \]
    where $H_{i + 1}$ is \tref{normal-subgroup}{normal} in $H_i$ and $H_i/H_{i + 1}$ is \tref{abelian-group}{abelian}.
    
    Equivalently, $G$ is solvable if its \tref{derived-series}{derived series} terminates in the trivial subgroup. This is because the quotient $H_i/H_{i + 1}$ is abelian if and only if $H_{i + 1}$ contains the commutator subgroup $[H_i, H_i]$.
\end{topic}

\begin{example}{solvable-group}
    The group $S_3$ is solvable, since we have the sequence $S_3 \supset A_3 \supset \{ 1 \}$, and $S_3 / A_3 \simeq \ZZ/2\ZZ$ and $A_3/\{ 1 \} \simeq \ZZ/3\ZZ$.
    
    The group $S_n$ is not solvable for $n \ge 5$.
\end{example}

\begin{topic}{supersolvable-group}{supersolvable group}
    A \tref{group}{group} $G$ is \textbf{supersolvable} if there exist \tref{subgroup}{subgroups} $H_0, H_1, \ldots, H_r$ of $G$ such that
    \[ G = H_0 \supset H_1 \supset \cdots \supset H_r = \{ 1 \} \]
    where $H_{i + 1}$ is \tref{normal-subgroup}{normal} in $H_i$ and $H_i/H_{i + 1}$ is \tref{cyclic-group}{cyclic}.
\end{topic}

\begin{topic}{commutator-subgroup}{commutator subgroup}
    The \textbf{commutator subgroup} of a \tref{group}{group} $G$, denoted $[G, G]$, is the subgroup of $G$ generated by all elements of the form $ghg^{-1}h^{-1}$ with $g, h \in G$.
\end{topic}

\begin{topic}{conjugation}{conjugation}
    Let $G$ be a \tref{group}{group}. Two elements $x, y \in G$ are called \textbf{conjugate} if there exists some $g \in G$ such that $g x g^{-1} = y$.
\end{topic}

\begin{topic}{free-group}{free (abelian) group}
    Given a set $S$, the \textbf{free group} over $S$, denoted $F_S$, is the \tref{group}{group} consisting of all words that can be built from elements of $S$ and $\{ s^{-1} : s \in S \}$, where two words are different unless their equality follows from the group axioms. Composition is given by concatenation of words, and the unit element is the empty word. The members of $S$ are called \textit{generators} of $F_S$, and the number of generators (i.e. the size of $S$) is the \textit{rank} of $F_S$.

    The \textbf{free abelian group} over $S$ is the \tref{abelian-group}{abelian} group $\bigoplus_{s \in S} \ZZ$.
\end{topic}

\begin{topic}{dihedral-group}{dihedral group}
    The \textbf{dihedral group} $D_n$ is the group of symmetries of the regular $n$-gon in the plane. Its \tref{order}{order} is $2n$, and it has a presentation
    \[ D_n = \langle \rho, \sigma \mid \rho^n = 1, \sigma^2 = 1, \sigma \rho \sigma^{-1} = \rho^{-1} \rangle , \]
    where $\rho$ corresponds to a rotation of $2 \pi / n$ and $\sigma$ corresponds to a reflection.
\end{topic}

\begin{example}{dihedral-group}
    For $n = 3$, the multiplication table of $D_3$ is given by
    \[ \begin{array}{c||c|c|c|c|c|c} 
           & 1 & \rho & \rho^2 & \sigma & \sigma \rho & \sigma \rho^2 \\ \hline \hline
         1 & 1 & \rho & \rho^2 & \sigma & \sigma \rho & \sigma \rho^2 \\ \hline
         \rho & \rho & \rho^2 & 1 & \sigma \rho^2 & \sigma & \sigma \rho \\ \hline
         \rho^2 & \rho^2 & 1 & \rho & \sigma \rho & \sigma \rho^2 & \sigma \\ \hline
         \sigma & \sigma & \sigma \rho & \sigma \rho^2 & 1 & \rho & \rho^2 \\ \hline
         \sigma \rho & \sigma \rho & \sigma \rho^2 & \sigma & \rho^2 & 1 & \rho \\ \hline
         \sigma \rho^2 & \sigma \rho^2 & \sigma & \sigma \rho & \rho & \rho^2 & 1
    \end{array} \]
\end{example}

\begin{topic}{semidirect-product}{semidirect product}
    Given two \tref{group}{groups} $N, H$ and a \tref{group-homomorphism}{group homomorphism} $\varphi : H \to \text{Aut}(N)$, the \textbf{semidirect product} of $N$ and $H$ with respect to $\varphi$, denoted $N \rtimes_\varphi H$, is the group with underlying set $N \times H$ where products and inverses given by
    \[ (n, h) \cdot (n', h') = (n \varphi(h)(n'), h h') \quad \text{ and } \quad (n, h)^{-1} = (\varphi(h)^{-1}(n^{-1}), h^{-1}) . \]
    In particular, $N$ (viewed as $N \times 1 \subset N \rtimes_\varphi H$) is a \tref{normal-subgroup}{normal subgroup} and the \tref{quotient-group}{quotient} is isomorphic to $H$ via $\psi : (N \rtimes_\varphi H)/N \to H$ given by $(n, h) \mapsto h$.
    
    Conversely, if $G$ is a group with a normal subgroup $N$ such that the exact sequence
    \[ 1 \to N \hookrightarrow G \xrightarrow{\pi} G/N \to 1 \]
    splits, i.e. $\pi \circ \sigma = \id_{G/N}$ for some $\sigma : G/N \to G$, then $G = N \rtimes G/N$, with $G/N$ acting on $N$ by conjugation via $\sigma$.
\end{topic}

\begin{example}{semidirect-product}
    Let $G = D_{n}$ be the \tref{dihedral-group}{dihedral group}, and let $N = \langle \rho \rangle$ be the normal subgroup of rotations and $H = \langle \sigma \rangle$ the subgroup generated by any fixed reflection $\sigma$. Then $H$ acts on $N$ by conjugation, in particular $\sigma \rho^k \sigma^{-1} = \rho^{-k}$, and $G = N \rtimes H$.
\end{example}

\begin{topic}{proper-group-action}{proper group action}
    A continuous \tref{group-action}{group action} of a \tref{TO:topological-group}{topological group} $G$ on a \tref{TO:topological-space}{topological space} $X$ is \textbf{proper} if the map
    \[ G \times X \to X \times X, \quad (g, x) \mapsto (x, gx) \]
    is \tref{TO:proper-map}{proper map}.
\end{topic}

\begin{topic}{profinite-group}{profinite group}
    A \textbf{profinite group} $G$ is a \tref{TO:topological-group}{topological group} isomorphic to an \tref{CT:inverse-limit}{inverse limit} of \tref{TO:discrete-topology}{discrete} finite groups
    \[ G \simeq \varprojlim_I G_i = \left\{ (x_i)_{i \in I} \in \prod_{i \in I} G_i : f_{ij}(x_j) = x_i \text{ for all } i \le j \right\} . \]
    The topology on $G$ is the \tref{TO:subspace-topology}{subspace} \tref{TO:product-topology}{product topology.}
\end{topic}

\begin{example}{profinite-group}
    Let $G$ be an arbitrary group, and $I$ a collection of normal subgroups of finite index of $G$, partially ordered with respect to (reversed) inclusions: $N \le N' \iff N \supset N'$. This yields an inverse system $(G / N)_{N \in I}$ with transition maps $G/N' \to G/N$ for $N \le N'$, and the profinite group
    \[ \hat{G} = \varprojlim_{N \in I} G / N ,\]
    is called the \textit{profinite completion} of $G$.
\end{example}

\begin{topic}{finitely-generated-group}{finitely generated group}
    A \tref{group}{group} $G$ is \textbf{finitely generated} if there exists a finite set $S \subset G$ such that every element of $G$ can be written as a finite product of elements of $S$.
\end{topic}

\begin{topic}{wreath-product}{wreath product}
    Let $G, H$ be \tref{group}{groups} and $S$ a set with $H$ \tref{group-action}{acting} on $S$. Let $K$ be the direct product $K = \prod_{s \in S} G$. Then $H$ acts on $K$ by $h \cdot (g_s)_{s \in S} = (g_{h^{-1} s})_{s \in S}$. The \textbf{unrestricted wreath product} of $G$ and $H$, denoted $G \, \text{Wr}_S H$ or $G \overline{\wr}_S H$, is the \tref{semidirect-product}{semidirect product} $K \rtimes H$.
    
    The \textbf{restricted wreath product}, denoted $G \, \text{wr}_S H$ or $G \wr_S H$, is obtained by replacing $K$ with the direct sum $\bigoplus_{s \in S} G$.
\end{topic}

\begin{example}{wreath-product}
    Let $G$ be any group $H = S_n$ the \tref{symmetric-group}{symmetric group} and $S = \{ 1, 2, \ldots, n \}$, where $S^n$ acts naturally on $S$. Then the wreath product is
    \[ G_n = \left( \prod_{i = 1}^{n} G \right) \rtimes S_n . \]
\end{example}

\begin{topic}{torsion-group}{torsion group}
    A \textbf{torsion group} is a \tref{group}{group} in which each element has finite \tref{order}{order}.
\end{topic}

\begin{example}{torsion-group}
    Every finite group is a torsion group.
    
    Every abelian group $A$ has a torsion subgroup $T = \{ a \in A \mid a^n = 1 \text{ for some } n \ge 1 \}$.
\end{example}

\begin{topic}{central-series}{central series}
    A \textbf{central series} of a \tref{group}{group} $G$ is a sequence of \tref{subgroup}{subgroups}
    \[ \{ e \} = H_0 \subset H_1 \subset \cdots \subset H_n = G \]
    such that $H_i$ is \tref{normal-subgroup}{normal} in $H_{i + 1}$ and the quotient $H_{i + 1}/H_i$ lies in the \tref{group-center}{center} of $G/H_i$.
\end{topic}

\begin{topic}{lower-central-series}{lower central series}
    The \textbf{lower central series} of a \tref{group}{group} $G$ is the descending series of subgroups
    \[ G = G_1 \supset G_2 \supset \cdots \]
    where $G_{n + 1} = [G, G_n]$ is the subgroup of $G$ generated by all elements of the form $[g, h]$ with $g \in G$ and $h \in G_n$.
\end{topic}

\begin{topic}{upper-central-series}{upper central series}
    The \textbf{upper central series} of a \tref{group}{group} $G$ is the ascending series of subgroups
    \[ \{ e \} = Z_0 \subset Z_1 \subset \cdots \]
    where $Z_{i + 1} = \{ z \in G \mid [g, z] \in Z_i \text{ for all } g \in G \}$. That is, it is the largest subgroup such that $[G, Z_{i + 1}] \subset Z_i$.
\end{topic}

\begin{topic}{derived-series}{derived series}
    The \textbf{derived series} of a \tref{group}{group} $G$ is the descending series of subgroups
    \[ G = G^{(1)} \supset G^{(2)} \supset \cdots \]
    where $G^{(n + 1)} = [G^{(n)}, G^{(n)}]$ is the subgroup of $G$ generated by all elements of the form $[g, h]$ with $g, h \in G^{(n)}$.
\end{topic}

\begin{example}{derived-series}
    The derived series of the \tref{symmetric-group}{symmetric group} $G = S_3$ is
    \[ S_3 \supset \{ e, (1 \; 2 \; 3), (1 \; 3 \; 2) \} \supset \{ e \} . \]
    This shows $S_3$ is \tref{solvable-group}{solvable}.
\end{example}

\begin{topic}{braid-group}{braid group}
    The \textbf{braid group on $n$ strands} is the \tref{group}{group}
    \[ B_n = \left\langle \sigma_1, \ldots, \sigma_{n - 1} \;\bigg|\; \begin{array}{cc} \sigma_i \sigma_{i + 1} \sigma_i = \sigma_{i + 1} \sigma_i \sigma_{i + 1} \text{ for } 1 \le i \le n - 2 \\ \sigma_i \sigma_j = \sigma_j \sigma_i \text{ for } |i - j| > 1 \end{array} \right\rangle . \]
    Intuitively, its elements are ways to `braid` $n$ strands, where $\sigma_i$ interchanges strands $i$ and $i + 1$.
\end{topic}

\begin{example}{braid-group}
    Let $(\mathcal{C}, \otimes, \textbf{1}, \tau)$ be a \tref{CT:braided-monoidal-category}{braided monoidal category}, and $V$ an object of $\mathcal{C}$. Then there is a natural representation of $B_n$ on $V^{\otimes n}$ given by sending $\sigma_i$ to $\id_V^{\otimes i - 1} \otimes \tau_{V, V} \otimes \id^{n - i + 1}$.
\end{example}

\begin{topic}{modular-group}{modular group}
    The \textbf{modular group} is the \tref{group}{group}
    \[ \textup{PSL}_2(\ZZ) = \left\{ \begin{pmatrix} a & b \\ c & d \end{pmatrix} \mid ad - bc = 1 \right\} / \{ \pm 1 \} \]
    of $2 \times 2$ matrices with integer coefficients and determinant $1$, where any matrix $A$ is identified with $-A$.
\end{topic}

\begin{example}{modular-group}
    Let $\mathcal{H} = \{ z \in \CC \mid \operatorname{im}(z) > 0 \}$ be the upper half-plane. The modular group acts on $\mathcal{H}$ via
    \[ \begin{pmatrix} a & b \\ c & d \end{pmatrix} \cdot z = \frac{az + b}{cz + d} . \]
    Note that this gives a well-defined (left) \tref{group-action}{action}.
\end{example}

\begin{topic}{free-product}{free product}
    Given two \tref{group}{groups} $G$ and $H$, the \textbf{free product} $G * H$ is the group of words that can be built from elements of $G \sqcup H$, modulo the group relations from $G$ and $H$, together with the natural inclusion maps $i_G : G \to G * H$ and $i_H : H \to G * H$.
    
    More generally, if there is also given a group $F$ and morphisms $\varphi : F \to G$ and $\psi : F \to H$, then the \textbf{free product with amalgamation}, denoted $G *_F H$, is the group $G * H$ modulo the relations $\varphi(f) = \psi(f)^{-1}$.
    
    The free product is the \tref{CT:pushout}{pushout} in the \tref{CT:category}{category} of groups. That is, for any $g : G \to K$ and $h : H \to K$ with $g \circ \varphi = h \circ \psi$, there is a unique morphism $f : G *_F H \to K$ such that $h = f \circ i_H$ and $g = f \circ i_G$.
    \[ \begin{tikzcd} F \arrow{r}{\psi} \arrow{d}{\varphi} & H \arrow{d} \arrow[bend left=15]{ddr}{h} & \\ G \arrow{r} \arrow[bend right=15]{drr}{g} & G *_F H \arrow[dashed]{dr} & \\ & & K \end{tikzcd} \]
\end{topic}

\begin{example}{free-product}
    \begin{itemize}
        \item If $G = \langle x \mid x^m = 1 \rangle$ and $H = \langle y \mid y^n = 1 \rangle$, then $G * H = \langle x, y \mid x^m = y^n = 1 \rangle$.
        \item If $F_n$ is the \tref{free-group}{free group} on $n$ generators, then $F_m * F_n \simeq F_{m + n}$.
        \item The \tref{modular-group}{modular group} $\textup{PSL}_2(\ZZ)$ is isomorphic to the free product $(\ZZ/2\ZZ) * (\ZZ/3\ZZ)$.
    \end{itemize}
\end{example}

\begin{topic}{sylow-p-group}{Sylow p-group}
    Let $G$ be a finite \tref{group}{group} with $|G| = p^n \cdot m$ for a prime $p$ and $m \nmid p$ and $n \ge 1$. A \textbf{Sylow $p$-group} of $G$ is a \tref{subgroup}{subgroup} $H \subset G$ with $|H| = p^n$.
\end{topic}

\begin{topic}{sylow-theorem}{Sylow theorems}
    Let $G$ be a finite \tref{group}{group} with $|G| = p^n \cdot m$ for a prime $p$ and $m \nmid p$ and $n \ge 1$. The \textbf{Sylow theorems} state that
    \begin{itemize}
        \item $G$ contains a \tref{sylow-p-group}{Sylow $p$-group}.
        \item If $H$ and $H'$ are Sylow $p$-groups of $G$, then $H' = gHg^{-1}$ for some $g \in G$.
        \item The number of distinct Sylow $p$-groups of $G$ is $1 \mod p$.
        \item The number of distinct Sylow $p$-groups of $G$ divides $m$. 
    \end{itemize}
\end{topic}

\begin{example}{sylow-theorem}
    Suppose $G$ is a finite group of order $|G| = pq$, with $p$ and $q$ prime and $p \not\equiv 1 \mod q$ and $q \not\equiv 1 \mod p$. Then $G \simeq \ZZ/pq\ZZ$. Namely, from the Sylow theorems it follows that there is precisely one Sylow $p$-group and one Sylow $q$-group. Therefore, there are precisely $p - 1$ elements of \tref{order}{order} $p$, and $q - 1$ elements of order $q$. The only element of order $1$ is the group unit. Since $1 + (p - 1) + (q - 1) < pq = |G|$, it follows that there is at least one element $g \in G$ of order $\textup{ord}(g) = pq$, and hence $G \simeq \ZZ/pq\ZZ$.
\end{example}

\begin{example}{sylow-theorem}
    If $G$ is a finite group, and $p$ a prime number with $p$ dividing $|G|$, then there exists an element $g \in G$ with $\textup{ord}(g) = p$. Namely, writing $|G| = p^n \cdot m$ as before, by the Sylow theorems there exists a Sylow $p$-group $H$ of $G$. Take any $h \in H$ not equal to the group unit, then $\textup{ord}(h) | p^n$, so $\textup{ord}(h) = p^\ell$ for some $1 \le \ell \ell n$. In particular $h^{p^{\ell - 1}}$ has order $p$.
\end{example}

\begin{example}{sylow-theorem}
    \begin{proof}
    \begin{itemize}
        \item Let $N$ be the number of Sylow $p$-groups in $G$. Consider the set $\mathcal{P}$ of all subsets of $G$ with $p^n$ elements, and for any $V \in \mathcal{P}$ write $G_V = \{ x \in G \mid xV = V \}$. Partition $\mathcal{P} = \mathcal{P}_1 \sqcup \mathcal{P}_2$ with $\mathcal{P}_1 = \{ V \in \mathcal{P} \mid |G_V| = p^n \}$ and $\mathcal{P}_2 = \{ V \in \mathcal{P} \mid |G_V| \ne p^n \}$.
    
        Note that for any Sylow $p$-group $H$ and $g \in G$, the (right) coset $Hg$ lies in $\mathcal{P}_1$ since $G_{Hg} = H$ has $p^n$ elements. Conversely, for any $V \in \mathcal{P}_1$, we have that $G_V$ is a Sylow $p$-group, and $V = G_V v$ for any $v \in V$. Hence, $\mathcal{P}_1$ consists precisely of the (right) cosets of the Sylow $p$-groups, and thus $|\mathcal{P}_1| = N m$.
    
        Now take any $V \in \mathcal{P}_2$, and note that it can be written as a disjoint union of sets of the form $G_V v$ with $v \in V$. Therefore $|G_V|$ divides $|V| = p^n$, and since $|G_V| \ne p^n$, it follows that $|G_V| = p^k$ for some $k < n$. Also note that $xV \in \mathcal{P}_2$ for any $x \in G$, because $G_{xV} = \{ g \in G \mid gxV = xV \} = \{ g \in G \mid x^{-1} g x V = V \} = x G_V x^{-1}$, so $|G_{xV}| = |G_V| \ne p^n$. Now, $xV = yV$ for some $x, y \in G$ iff $y^{-1} x V = V$ iff $y^{-1} x \in G_V$ iff $x G_V = y G_V$, so there are $[G : G_V]$ distinct sets of the form $xV$ in $\mathcal{P}_2$. As $[G : G_V] = p^n m / p^k \equiv 0 \mod p$, we obtain $|\mathcal{P}_2| \equiv 0 \mod p$.
    
        Summarizing, we have
        \[ |\mathcal{P}| = \binom{p^n m}{p^n} = |\mathcal{P}_1| + |\mathcal{P}_2| \equiv Nm + 0 \mod p , \]
        and since $\gcd(p, m) = 1$, we have that $m \mod p$ is a unit in $\ZZ/p\ZZ$, so we find
        \[ N \mod p = (m \mod p)^{-1} \cdot \binom{p^n m}{p^n} \mod p . \]
        This shows that $N \mod p$ only depends on $p, n$ and $m$ and not on the actual group $G$. In particular, we can take $G = \ZZ/p^n \ZZ \times \ZZ/m\ZZ$ for which it is known that $N = 1 \equiv 1 \mod p$.
    
        \item Let $H$ and $H'$ be two Sylow $p$-groups in $G$. Partition the (left) cosets of $H$ as $\mathcal{H}_1 \sqcup \mathcal{H}_2$, where $\mathcal{H}_1$ contains all $gH$ with $hgH = gH$ for all $h \in H'$, and $\mathcal{H}_2$ contains all other cosets of $H$. For any coset $gH \in \mathcal{H}_2$ we have that $H'' = \{ h \in H' \mid hgH = gH \}$ is a proper subgroup of $H'$, so $p | [H' : H'']$. Note that all sets $hgH$ with $h \in H'$ are contained in $\mathcal{H}_2$, and for any $h_1, h_2 \in H'$ we have that $h_1 g H = h_2 g H$ iff $h_2^{-1} h_1 g H = g H$ iff $h_2^{-1} h_1 \in H''$ iff $h_1 H'' = h_2 H''$. So there are $[H' : H'']$ distinct such sets $hgH$ with $h \in H'$. This partitions $\mathcal{H}_2$ into disjoint subsets, each having a multiple of $p$ elements, since $p | [H' : H'']$. Therefore, $|\mathcal{H}_2| \equiv 0 \mod p$, and we find that
        \[ m = |\mathcal{H}_1| + |\mathcal{H}_2| \equiv |\mathcal{H}_1| \mod p , \]
        so in particular $\mathcal{H}_1$ is non-empty. Hence there exists some coset $gH$ such that $hgH = gH$ for all $h \in H'$, from which we find that $g^{-1}hg \in H$ for all $h \in H'$, and thus $H' = gHg^{-1}$.
        
        \item Take any Sylow $p$-group $H$ in $G$. There are $N$ such, and all are of the form $gHg^{-1}$ for some $g \in G$. Let $N_H = \{ g \in G \mid gHg^{-1} = H \}$ be the \tref{normalizer}{normalizer} of $H$, and choose $g_i \in G$ such that $G = \sqcup_i g_i N_H$. Then the Sylow $p$-groups are precisely the $g_i H g_i^{-1}$, and these are pairwise distinct. Hence $N = [G : N_H]$, and we find that
        \[ N = [G : N_H] = |G| / |N_H| = |G| / [N_H : H] / |H| , \]
        which divides $|G| / |H| = m$.
    \end{itemize}
    \end{proof}
\end{example}

\begin{topic}{nilpotent-group}{nilpotent group}
    A \tref{group}{group} $G$ is \textbf{nilpotent} if there exist \tref{subgroup}{subgroups} $H_1, H_2, \ldots, H_r$ such that
    \[ G = H_0 \supset H_1 \supset \cdots \supset H_r = \{ 1 \} \]
    where $H_{i + 1}$ is \tref{normal-subgroup}{normal} in $H_i$ and $H_i/H_{i + 1}$ is contained in the \tref{group-center}{center} of $G/H_{i + 1}$.
\end{topic}

\begin{example}{nilpotent-group}
    \begin{itemize}
        \item Every \tref{abelian-group}{abelian group} $G$ is nilpotent. Namely, the center $Z(G)$ equals $G$ and we have a series $G \supset \{ 1 \}$.
        \item The quaternion group $Q_8 = \langle -1, i, j, k \mid (-1)^2 = 1 \textup{ and } i^2 = j^2 = k^2 = ijk = -1 \rangle$ is nilpotent. Namely, its center is $Z(Q_8) = \{ -1, 1 \}$ and we have a series $Q_8 \supset \{ -1, 1 \} \supset \{ 1 \}$.
    \end{itemize}
\end{example}

\begin{topic}{abelianization}{abelianization}
    The \textbf{abelianization} of a \tref{group}{group} $G$ is the \tref{quotient-group}{quotient}
    \[ G/[G, G] \]
    where $[G, G]$ denotes the \tref{commutator-subgroup}{commutator subgroup}. Indeed, the abelianization of any group is \tref{abelian-group}{abelian}.
\end{topic}

\begin{example}{abelianization}
    Let $G = \textup{GL}_n(\RR)$ be the the \tref{LA:general-linear-group}{general linear group}. The \tref{LA:determinant}{determinant} $\det : G \to \RR^*$ has constant value $1$ on $[G, G]$, since the determinant of products of commutators is always $1$. Conversely, after showing that any matrix of determinant $1$ can be written as a product of commutators, it follows that
    \[ \det : G / [G, G] \xrightarrow{\sim} \RR^* \]
    is an isomorphism.
\end{example}

\begin{topic}{shuffle}{(un)shuffle}
    Given integers $p, q \ge 1$, a \textbf{$(p, q)$-(un)shuffle} is a \tref{symmetric-group}{permutation} $\sigma \in S_{p + q}$ such that
    \[ \sigma(1) < \sigma(2) < \cdots < \sigma(p) \quad \textup{and} \quad \sigma(p + 1) < \sigma(p + 2) < \cdots < \sigma(p + q) . \]
    % A \textbf{$(p, q)$-unshuffle} is a permutation $\sigma \in S_{p + q}$ such that $\sigma^{-1}$ is a $(p, q)$-shuffle.
\end{topic}

\begin{example}{shuffle}
    The permutation $\sigma \in S_6$ given by
    \[ (\sigma(1), \sigma(2), \sigma(3), \sigma(4), \sigma(5), \sigma(6)) = (2, 5, 1, 3, 4, 6) \]
    is a $(2, 4)$-shuffle.
\end{example}

\begin{topic}{exponent-group}{exponent of group}
    The \textbf{exponent} of a \tref{group}{group} $G$, is the smallest positive integer $n$ such that $g^n = 1$ for all $g \in G$.
\end{topic}

\begin{topic}{frattini-subgroup}{Frattini subgroup}
    Let $G$ be a \tref{group}{group}. The \textbf{Frattini subgroup} of $G$, denoted $\Phi(G)$, is the intersection of all proper maximal \tref{subgroup}{subgroups} of $G$.
\end{topic}

\begin{example}{frattini-subgroup}
    Let $G = \ZZ/p^k \ZZ$ for some prime number $p$ and integer $k \ge 1$. Then all subgroups of $G$ are given by $\langle p^\ell \rangle$ for $0 \le \ell \le k$. Hence, the only proper maximal subgroup of $G$ is $\langle p^{k - 1} \rangle$, which is therefore also its Frattini subgroup.
\end{example}

\begin{example}{frattini-subgroup}
    For any prime number $p$, consider the \textit{Prüfer $p$-group}
    \[ \ZZ(p^\infty) = \{ z \in \CC \mid z^{p^n} = 1 \textup{ for some } n \in \ZZ_{\ge 1} \} . \]
    As all subgroups of $\ZZ(p^\infty)$ are of the form $H_k = \{ \exp(2 \pi i \ell / p^k) \;:\; 0 \le \ell < p^k \}$ for some $k \ge 0$, there are no proper maximal subgroups in $\ZZ(p^\infty)$, and hence the Frattini subgroup of $\ZZ(p^\infty)$ is itself.
\end{example}

\begin{topic}{dicyclic-group}{dicyclic group}
    The \textbf{dicyclic group} $\textup{Dic}_n$ is the \tref{semidirect-product}{semidirect product} of the \tref{cyclic-group}{cyclic group} $C_2$ by the cyclic group $C_{2n}$, given by
    \[ \textup{Dic}_n = \langle a, x \mid a^{2n} = 1, \; x^2 = a^n, x^{-1} a x = a^{-1} \rangle . \]
    Its \tref{order}{order} is $4n$.
\end{topic}

\begin{topic}{complete-group}{complete group}
    A \tref{group}{group} $G$ is \textbf{complete} if the morphism
    \[ \phi : G \to \textup{Aut}(G), \quad \phi(g)(x) = g x g^{-1} \]
    is an isomorphism. That is, if the \tref{group-center}{center} $Z(G)$ is trivial and every automorphism of $G$ is an inner automorphism.
\end{topic}

\begin{example}{complete-group}
    \begin{itemize}
        \item The \tref{symmetric-group}{symmetric group} $S_n$ is complete for $n \ne 2, 6$. For $n = 2$, the group $S_2 \simeq \ZZ/2\ZZ$ has non-trivial center, and for $n = 6$, the group $S_6$ has an outer automorphism determined by
        \[ (1 \; 2) \mapsto (1 \; 2)(3 \; 4) (5 \; 6), \qquad (1 \; 2 \; 3 \; 4 \; 5 \; 6) \mapsto (2 \; 4) (3 \; 6 \; 5) . \]
    \end{itemize}
\end{example}

\begin{topic}{perfect-group}{perfect group}
    A \tref{group}{group} $G$ is \textbf{perfect} if its \tref{commutator-subgroup}{commutator subgroup} equals $G$.
\end{topic}

\begin{example}{perfect-group}
    \begin{itemize}
        \item Any non-abelian \tref{simple-group}{simple} group $G$ is perfect. Indeed, its commutator subgroup $[G, G]$ is \tref{normal-subgroup}{normal} and non-trivial, so must be equal to $G$.
        \item The direct product of perfect groups is perfect.
        \item Any quotient of a perfect group is perfect.
    \end{itemize}
\end{example}

\begin{topic}{grun-lemma}{Grün's lemma}
    Let $G$ be a \tref{perfect-group}{perfect} \tref{group}{group}. \textbf{Grün's lemma} states that the \tref{group-center}{center} of the \tref{quotient-group}{quotient} $G / Z(G)$, where $Z(G)$ is the center of $G$, is trivial.
\end{topic}

\begin{topic}{coxeter-group}{Coxeter group}
    A \textbf{Coxeter group} is a \tref{group}{group} with a presentation
    \[ W = \langle r_1, r_2, \ldots, r_n \mid (r_i r_j)^{m_{ij}} = 1 \rangle , \]
    where $m_{ii} = 1$ and $m_{ij} \in \{ 2, 3, \ldots \} \cup \{ \infty \}$ for $i \ne j$, where $m_{ij} = \infty$ means no relation of the form $(r_i r_j)^m = 1$ is imposed.
    
    A \textbf{Coxeter system} is a pair $(W, S)$, where $W$ is a Coxeter group, and $S = \{ r_1, r_2, \ldots, r_n \}$ a set of generators.
    
    The \textbf{Coxeter matrix} if a Coxeter system is the $n \times n$ symmetric matrix with entries $m_{ij}$.
\end{topic}

\begin{topic}{burnside-ring}{Burnside ring}
    Let $G$ be a finite \tref{group}{group}. The \textbf{Burnside ring} of $G$, denoted $A(G)$, is the \tref{free-group}{free abelian group} on isomorphism classes of finite $G$-sets, modulo the relations $[X \sqcup Y] = [X] + [Y]$ for every finite $G$-sets $X$ and $Y$. Multiplication is defined by the cartesian product
    \[ [X] \cdot [Y] = [X \times Y] , \]
    for all finite $G$-sets $X$ and $Y$, and extended linearly.
    
    The zero element in $A(G)$ is the empty $G$-set $[\varnothing]$, and the unit element is the singleton with trivial $G$-action $[\{ \star \}$.
\end{topic}

\begin{example}{burnside-ring}
    Every finite $G$-set is a disjoint union of its orbits, and any orbit is isomorphic to a $G$-set of cosets $G/H$, where $H$ is the stabilizer of an element in the orbit. Since $G/H \simeq G/H'$ as $G$-sets if and only if $H$ and $H'$ are conjugate in $G$, it follows that $A(G) \simeq \ZZ^c$ as abelian groups, where $c$ is the number of conjugacy classes of subgroups of $G$.
    
    Suppose that $G$ is a \tref{simple-group}{simple group}, so that the only subgroups of $G$ are $\{ 1 \}$ and $G$ itself. Since $[G/G] = 1 \in A(G)$, we find that $A(G)$ is generated by $[G / \{ 1 \}] = [G]$, where $G$ acts on itself by left multiplication. The isomorphism
    \[ G \times G \to G \times G, \quad (g, h) \mapsto (g, g^{-1} h) \]
    shows that $[G \times G] = |G| \cdot [G]$, and hence $A(G) = \ZZ[x] / (x^2 - |G| x)$.
\end{example}

\begin{example}{burnside-ring}
    Let $k$ be a field. Any finite $G$-set $X$ induces a \tref{RT:permutation-representation}{permutation representation} $\rho : G \to \textup{GL}\left(k^X\right)$ by permuting the basis vectors indexed by $X$. This induces a ring morphism
    \[ \varphi : A(G) \to R_k(G) \]
    to the \tref{RT:representation-ring}{representation ring} of $G$.
    
    This map need not be injective nor surjective. Namely, for $G = S_3$, we have that
    \[ \varphi(2 [S_3 / \langle (1, 2) \rangle] + [S_3 / \langle (1, 2, 3) \rangle]) = \varphi([S_3] + 2 [S_3 / S_3]) , \]
    and for general $G$, the image of $\varphi$ only contains rational representations.
\end{example}

\begin{topic}{orbit-stabilizer-theorem}{orbit-stabilizer theorem}
    Let $G$ be a \tref{group}{group} \tref{group-action}{acting} on a set $X$. The \textbf{orbit-stabilizer theorem} states that for any $x \in X$, the map
    \[ G/G_x \to Gx , \quad g G_x \mapsto g \cdot x , \]
    from the set of \tref{coset}{cosets} of the \tref{stabilizer}{stabilizer} $G_x$, to the \tref{orbit}{orbit} $Gx$ of $x$, is a bijection.
\end{topic}

\begin{topic}{lagrange-theorem}{Lagrange's theorem}
    Let $G$ be a \tref{group}{group} and $H \subset G$ a \tref{subgroup}{subgroup}. \textbf{Lagrange's theorem} states that
    \[ |G| = [G : H] \cdot |H| , \]
    where $[G : H]$ denotes the \tref{index-subgroup}{index} of $H$ in $G$.
\end{topic}

\begin{example}{lagrange-theorem}
    \begin{proof}
        The \tref{coset}{left cosets} of $H$ form a partition of $G$. Namely, if two cosets $g_1 H$ and $g_2 H$ intersect, that is, $g_1 h_1 = g_2 h_2$ for some $h_1, h_2 \in H$, then $g_1 H = g_2 H$ as $g_1 h = g_2 h_2 h_1^{-1} h \in g_2 H$ and $g_2 h = g_1 h_1 h_2^{-1} h \in g_1 H$ for any $h \in H$. Furthermore, all cosets $gH$ are in bijection with $H$ via $H \to gH, h \mapsto g h$ with inverse $x \mapsto g^{-1} x$. Therefore, after choosing a set $A$ of representatives for the cosets, we obtain a bijection of sets
        \[ A \times H \to G, \quad (g, h) \mapsto gh , \]
        with $|A| = [G : H]$, which proves the theorem.
    \end{proof}
\end{example}

\begin{topic}{metabelian-group}{metabelian group}
    A \tref{group}{group} $G$ is \textbf{metabelian} if its \tref{commutator-subgroup}{commutator subgroup} $[G, G]$ is \tref{abelian-group}{abelian}.
\end{topic}

\begin{topic}{characteristic-subgroup}{characteristic subgroup}
    Let $G$ be a \tref{group}{group}. A \tref{subgroup}{subgroup} $H \subset G$ is \textbf{characteristic} if $\varphi(H) = H$ for all automorphisms $\varphi : G \to G$.
\end{topic}

\begin{example}{characteristic-subgroup}
    Since conjugation $x \mapsto gxg^{-1}$, for any $g \in G$, defines an automorphism of $G$, it follows that characteristic subgroups are \tref{normal-subgroup}{normal}. However, the converse does not hold. Namely, if we consider the additive group $\QQ$ with normal subgroup $\ZZ \subset \QQ$, then $\ZZ$ is not preserved under the automorphism $\QQ \to \QQ$, $q \mapsto q/2$.
\end{example}

\begin{topic}{subnormal-subgroup}{subnormal subgroup}
    Let $G$ be a \tref{group}{group}. A \tref{subgroup}{subgroup} $H \subset G$ is \textbf{subnormal} if there exists a finite chain of subgroups
    \[ H = H_0 \subset H_1 \subset \cdots \subset H_n = G \]
    such that $H_{i - 1}$ is \tref{normal-subgroup}{normal} in $H_i$ for all $i = 1, \ldots, n$.
\end{topic}

\begin{example}{subnormal-subgroup}
    \begin{itemize}
        \item Every normal subgroup $N \subset G$ is subnormal by a chain of length $1$.
        \item From the chain of subgroups $\{ 1, (1, 2) (3, 4) \} \subset \{ 1, (1, 2) (3, 4), (1, 3) (2, 4), (1, 4) (2, 3) \} \subset S_4$ in the \tref{symmetric-group}{symmetric group} $S_4$ follows that $\{ 1, (1, 2)(3, 4) \}$ is subnormal in $S_4$, even though it is not normal in $S_4$.
    \end{itemize}
\end{example}

\begin{topic}{pontryagin-dual}{Pontryagin dual}
    Let $G$ be a \tref{TO:locally-compact-space}{locally compact} \tref{TO:topological-group}{topological} \tref{abelian-group}{abelian group}. The \textbf{Pontryagin dual} of $G$, denoted $\widehat{G}$, is the group of \tref{TO:continuous-map}{continuous} \tref{group-homomorphism}{group morphisms} from $G$ to the circle group $S^1$, that is,
    \[ \widehat{G} = \Hom_\textbf{TopGrp}(G, S^1) , \]
    equipped with the \tref{TO:mapping-space}{compact-open topology}.
\end{topic}

\begin{example}{pontryagin-dual}
    \begin{itemize}
        \item The Pontryagin dual of $\ZZ$ is $S^1$, and vice versa.
        \item The Pontryagin dual of $\RR$ is $\RR$ itself.
        \item The Pontryagin dual of $\ZZ/n\ZZ$ is $\ZZ/n\ZZ$ itself.
    \end{itemize}
\end{example}

\begin{example}{pontryagin-dual}
    The \textit{Pontryagin duality theorem} states there is a canonical isomorphism between locally compact abelian topological groups
    \[ G \xrightarrow{\sim} \widehat{\widehat{G}}, \quad g \mapsto (\chi \mapsto \chi(g)) . \]
\end{example}

\begin{topic}{burnside-theorem}{Burnside's theorem}
    Let $G$ be a finite \tref{group}{group} of \tref{order}{order} $p^m q^n$ for prime numbers $p$ and $q$, and non-negative integers $m$ and $n$. \textbf{Burnside's theorem} states that $G$ is \tref{solvable-group}{solvable}.
\end{topic}

\begin{topic}{elementary-group}{elementary group}
    Let $p$ be a prime number. A \tref{group}{group} $G$ is \textbf{$p$-elementary} if it is a direct product of a \tref{cyclic-group}{cyclic group} of order relatively prime to $p$, and a \tref{p-group}{$p$-group}.
\end{topic}

\begin{topic}{p-group}{p-group}
    Let $p$ be a prime number. A \textbf{$p$-group} is a \tref{group}{group} in which the \tref{order}{order} of every element is a power of $p$.
\end{topic}

\begin{example}{p-group}
    \begin{itemize}
        \item For any $k \ge 0$, the group $\ZZ/p^k \ZZ$ is a $p$-group.
        \item For $p = 2$, the \tref{dihedral-group}{dihedral group} $D_{2^k}$ is a $p$-group for any $k \ge 0$.
    \end{itemize}
\end{example}
