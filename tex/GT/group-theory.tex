\begin{topic}{group}{group}
    A \textbf{group} is a set $G$ together with an operation $G \times G \to G$ (the \textit{group law}) written as $(x, y) \mapsto xy$, and an element $1 \in G$ (the \textit{unit}), satisfying
    \begin{itemize}
        \item (\textit{associativity}) $(xy)z = x(yz)$ for all $x, y, z \in G$,
        \item (\textit{unit element}) $1 \cdot x = x \cdot 1 = x$ for all $x \in G$,
        \item (\textit{inverses}) for all $x \in G$ there exists an $x^{-1} \in G$ such that $x x^{-1} = x^{-1} x = 1$, called the \textit{inverse} of $x$.
    \end{itemize}
\end{topic}

\begin{topic}{subgroup}{subgroup}
    A \textbf{subgroup} $H$ of a \tref{group}{group} $G$ is a subset $H \subset G$ which, with the same group law and unit, is itself a group.
\end{topic}

\begin{topic}{abelian-group}{abelian group}
    A \tref{group}{group} $G$ is called \textbf{abelian} if $xy = yx$ for all $x, y \in G$.
\end{topic}

\begin{topic}{order}{order}
    The \textbf{order} of a \tref{group}{group} $G$ is its number of elements.
    
    The \textbf{order} of an element $x \in G$, denoted $\text{ord}(x)$, is the least positive integer $n$ such that $x^n = 1$. If no such $n$ exists, then $\text{ord}(x) = \infty$.
\end{topic}

\begin{topic}{cyclic-group}{cyclic group}
    A \textbf{cyclic group} is a \tref{group}{group} $G$ generated by a single element $x \in G$, that is $G = \{ x^n : x \in \ZZ \}$.
\end{topic}

\begin{topic}{group-homomorphism}{group homomorphism}
    Let $G$ and $H$ be two \tref{group}{groups}. A \textbf{homomorphism} from $G$ to $H$ is a map $f : G \to H$ satisfying $f(xy) = f(x) f(y)$ for all $x, y \in G$.
\end{topic}

\begin{topic}{kernel}{kernel}
    Let $f : G \to H$ be a \tref{group-homomorphism}{group homomorphism}. The \textbf{kernel} of $f$, denoted $\ker f$, is defined as
    \[ \ker f = \{ x \in G : f(x) = 0 \} . \]
    It is a \tref{normal-subgroup}{normal} \tref{subgroup}{subgroup} of $G$.
\end{topic}

\begin{topic}{group-center}{group center}
    The \textbf{center} of a \tref{group}{group} $G$ is the subgroup
    \[ Z(G) = \{ x \in G : xy = yx \text{ for all } y \in G \} . \]
\end{topic}

\begin{topic}{symmetric-group}{symmetric group}
    Let $\Sigma$ be a set. The \textbf{symmetric group} on $\Sigma$ is the \tref{group}{group} $S_\Sigma$ of all bijections $\Sigma \to \Sigma$. The group law is given by composition, and the unit is the identity map.
    
    When $\Sigma = \{ 1, 2, \ldots, n \}$ for some integer $n \ge 1$, one writes $S_n$ for the symmetry group. Its elements are called \textbf{permutations}.
\end{topic}

\begin{topic}{cayleys-theorem}{Cayley's theorem}
    \textbf{Cayley's theorem} states that every \tref{group}{group} $G$ is isomorphic to a \tref{subgroup}{subgroup} of the \tref{symmetric-group}{symmetric group} $S_G$. In particular, $G$ is isomorphic to the image of the morphism
    \[ \varphi : G \to S_G, \qquad x \mapsto (y \mapsto xy) . \]
\end{topic}

\begin{topic}{cyclic-permutation}{cyclic permutation}
    A \tref{symmetric-group}{permutation} $\sigma \in S_n$ is a \textbf{cyclic permutation}, or \textbf{cycle}, of length $k$ if there exist $k$ distinct integers $1 \le a_1, \ldots, a_k \le n$ with $\sigma(a_i) = a_{i + 1}$ for $1 \le i < k$ and $\sigma(a_k) = a_1$, and $\sigma(x) = x$ for $x \not\in \{ a_1, \ldots, a_k \}$. This is denoted by
    \[ \sigma = (a_1 \;\; a_2 \;\; \ldots \;\; a_k) . \]
    A cycle of length $2$ is also called a \textbf{transposition}.
\end{topic}

\begin{topic}{permutation-sign}{permutation sign}
    The \textbf{sign} of a \tref{symmetric-group}{permutation} $\sigma \in S_n$ is defined as
    \[ \text{sign}(\sigma) = \prod_{1 \le i < j \le n} \frac{\sigma(j) - \sigma(i)}{j - i} \in \{ +1, -1 \} . \]
    We call $\sigma$ \textit{even} if $\text{sign}(\sigma) = 1$ and \textit{odd} if $\text{sign}(\sigma) = -1$.
    
    This defines a \tref{group-homomorphism}{group homomorphism}
    \[ \text{sign} : S_n \to \{ +1, -1 \} . \]
\end{topic}

\begin{topic}{alternating-group}{alternating group}
    For $n \ge 1$, the \textbf{alternating group} $A_n$ is the subgroup of the \tref{symmetric-group}{symmetric group} $S_n$ of all \tref{permutation-sign}{even} permutations.
\end{topic}

\begin{topic}{coset}{coset}
    Let $G$ be a \tref{group}{group} and $H \subset G$ a \tref{subgroup}{subgroup}. A \textbf{left coset} of $H$ is a set of the form $gH = \{ gh : h \in H \}$ for some $g \in G$. A \textbf{right coset} of $H$ is a set of the form $Hg = \{ hg : h \in H \}$ for some $g \in G$.
    
    In particular, when $H$ is \tref{normal-subgroup}{normal}, left and right cosets coincide since $gH = Hg$ for all $g \in G$.
\end{topic}

\begin{topic}{index-subgroup}{index subgroup}
    The \textbf{index} of a \tref{subgroup}{subgroup} $H$ of a \tref{group}{group} $G$ is the number of \tref{coset}{left cosets} (or equivalently, right cosets) of $H$ in $G$, and is denoted $[G : H]$.
\end{topic}

\begin{topic}{normal-subgroup}{normal subgroup}
    A \tref{subgroup}{subgroup} $N$ of a \tref{group}{group} $G$ is called \textbf{normal} if $N = g N g^{-1}$ for all $g \in G$.
\end{topic}

\begin{example}{normal-subgroup}
    The \tref{kernel}{kernel} of a morphism $f : G \to H$ is always a normal subgroup. Indeed, for any $x \in \ker f$ and $g \in G$ we have
    \[ f(gxg^{-1}) = f(g) f(x) f(g)^{-1} = f(g) f(g)^{-1} = 1 , \]
    so $gxg^{-1} \in \ker f$, and thus $g (\ker f) g^{-1} \subset \ker f$. The other inclusion is shown completely similarly.
    
    Conversely, any normal subgroup $N \subset G$ is the kernel of the quotient map $\pi : G \to G / N$.
\end{example}

\begin{topic}{central-subgroup}{central subgroup}
    A \tref{subgroup}{subgroup} $H$ of a \tref{group}{group} $G$ is called \textbf{central} if $H$ lies in the \tref{group-center}{center} of $G$.
\end{topic}

\begin{topic}{quotient-group}{quotient group}
    Let $G$ be a \tref{group}{group} and $N$ a \tref{normal-subgroup}{normal subgroup}. The \textbf{quotient group} $G/N$ is the group of \tref{coset}{cosets}
    \[ G/N = \{ gN : g \in G \} \]
    with group law $(gN)(hN) = (gh)N$ and unit $N$.
\end{topic}

\begin{topic}{simple-group}{simple group}
    A \tref{group}{group} $G$ is called \textbf{simple} if its only \tref{normal-subgroup}{normal subgroups} are $\{ 1 \}$ and $G$.
\end{topic}

\begin{topic}{torsion-subgroup}{torsion subgroup}
    The \textbf{torsion subgroup} of a \tref{group}{group} $G$ is the subgroup of elements of finite \tref{order}{order}.
    \[ G_{\text{tor}} = \{ x \in G : \text{ord}(x) < \infty \} \]
\end{topic}

\begin{topic}{normalizer}{normalizer}
    Let $H$ be a \tref{subgroup}{subgroup} of a \tref{group}{group} $G$. The \textbf{normalizer} of $H$ is the subgroup
    \[ N_H = \{ x \in G : x H x^{-1} = H \} . \]
\end{topic}

\begin{topic}{group-action}{group action}
    Let $G$ be a \tref{group}{group} and $X$ a set. An \textbf{action} of $G$ on $X$ is map
    \[ G \times X \to X : (g, x) \mapsto g \cdot x \]
    satisfying $1 \cdot x = x$ and $g \cdot (h \cdot x) = (gh) \cdot x$ for all $g, h \in G$ and $x \in X$.
\end{topic}

\begin{topic}{free-group-action}{free group action}
    A \tref{group-action}{group action} of a \tref{group}{group} $G$ on a set $X$ is \textbf{free} if $g \cdot x = x$ implies $g = 1$, for all $x \in X$.
\end{topic}

\begin{topic}{transitive-group-action}{transitive group action}
    A \tref{group-action}{group action} of a \tref{group}{group} $G$ on a set $X$ is \textbf{transitive} if for all $x, y \in X$ there exists a $g \in G$ such that $y = g \cdot x$.
\end{topic}

\begin{topic}{faithful-group-action}{faithful group action}
    A \tref{group-action}{group action} of a \tref{group}{group} $G$ on a set $X$ is \textbf{faithful} if $g \cdot x = x$ for all $x \in X$, implies that $g = 1$.
\end{topic}

\begin{topic}{centralizer}{centralizer}
    Let $G$ be a \tref{group}{group}. The \textbf{centralizer} of an element $g \in G$ is the subgroup
    \[ G_g = \{ h \in G : h x h^{-1} = g \} . \]
\end{topic}

\begin{topic}{stabilizer}{stabilizer}
    Let $G$ be a \tref{group}{group} \tref{group-action}{acting} on a set $X$. Then the \textbf{stabilizer} of $x \in X$ is the \tref{subgroup}{subgroup}
    \[ G_x = \{ g \in G : g \cdot x = x \} . \]
\end{topic}

\begin{topic}{orbit}{orbit}
    Let $G$ be a \tref{group}{group} \tref{group-action}{acting} on a set $X$. The \textbf{orbit} of an element $x \in X$ is the set
    \[ Gx = \{ g \cdot x : g \in G \} \subset X . \]
\end{topic}

\begin{topic}{solvable-group}{solvable group}
    A \tref{group}{group} $G$ is \textbf{solvable} if there exist \tref{subgroup}{subgroups} $H_1, H_2, \ldots, H_r$
    \[ G = H_0 \supset H_1 \supset \cdots \supset H_r = \{ 1 \} \]
    where $H_{i + 1}$ is \tref{normal-subgroup}{normal} in $H_i$ and $H_i/H_{i + 1}$ is abelian.
    
    Equivalently, $G$ is solvable if its \tref{derived-series}{derived series} terminates in the trivial subgroup. This is because the quotient $H_i/H_{i + 1}$ is abelian if and only if $H_{i + 1}$ contains the commutator subgroup $[H_i, H_i]$.
\end{topic}

\begin{example}{solvable-group}
    The group $S_3$ is solvable, since we have the sequence $S_3 \supset A_3 \supset \{ 1 \}$, and $S_3 / A_3 \simeq \ZZ/2\ZZ$ and $A_3/\{ 1 \} \simeq \ZZ/3\ZZ$.
\end{example}

\begin{topic}{commutator-subgroup}{commutator subgroup}
    The \textbf{commutator subgroup} of a \tref{group}{group} $G$, denoted $[G, G]$, is the subgroup of $G$ generated by all elements of the form $ghg^{-1}h^{-1}$ with $g, h \in G$.
\end{topic}

\begin{topic}{conjugation}{conjugation}
    Let $G$ be a \tref{group}{group}. Two elements $x, y \in G$ are called \textbf{conjugate} if there exists some $g \in G$ such that $g x g^{-1} = y$.
\end{topic}

\begin{topic}{free-group}{free (abelian) group}
    Given a set $S$, the \textbf{free group} over $S$, denoted $F_S$, is the \tref{group}{group} consisting of all words that can be built from elements of $S$ and $\{ s^{-1} : s \in S \}$, where two words are different unless their equality follows from the group axioms. Composition is given by concatenation of words, and the unit element is the empty word. The members of $S$ are called \textit{generators} of $F_S$, and the number of generators (i.e. the size of $S$) is the \textit{rank} of $F_S$.

    The \textbf{free abelian group} over $S$ is the \tref{abelian-group}{abelian} group $\bigoplus_{s \in S} \ZZ$.
\end{topic}

\begin{topic}{dihedral-group}{dihedral group}
    The \textbf{dihedral group} $D_n$ is the group of symmetries of the regular $n$-gon in the plane. Its \tref{order}{order} is $2n$, and it has a presentation
    \[ D_n = \langle \rho, \sigma : \rho^n = 1, \sigma^2 = 1, \sigma \rho \sigma^{-1} = \rho^{-1} \rangle , \]
    where $\rho$ corresponds to a rotation of $2 \pi / n$ and $\sigma$ corresponds to a reflection.
\end{topic}

\begin{example}{dihedral-group}
    For $n = 3$, the multiplication table of $D_3$ is given by
    \[ \begin{array}{c||c|c|c|c|c|c} 
           & 1 & \rho & \rho^2 & \sigma & \sigma \rho & \sigma \rho^2 \\ \hline \hline
         1 & 1 & \rho & \rho^2 & \sigma & \sigma \rho & \sigma \rho^2 \\ \hline
         \rho & \rho & \rho^2 & 1 & \sigma \rho^2 & \sigma & \sigma \rho \\ \hline
         \rho^2 & \rho^2 & 1 & \rho & \sigma \rho & \sigma \rho^2 & \sigma \\ \hline
         \sigma & \sigma & \sigma \rho & \sigma \rho^2 & 1 & \rho & \rho^2 \\ \hline
         \sigma \rho & \sigma \rho & \sigma \rho^2 & \sigma & \rho^2 & 1 & \rho \\ \hline
         \sigma \rho^2 & \sigma \rho^2 & \sigma & \sigma \rho & \rho & \rho^2 & 1
    \end{array} \]
\end{example}

\begin{topic}{semidirect-product}{semidirect product}
    Given two \tref{group}{groups} $N, H$ and a \tref{group-homomorphism}{group homomorphism} $\varphi : H \to \text{Aut}(N)$, the \textbf{semidirect product} of $N$ and $H$ with respect to $\varphi$, denoted $N \rtimes_\varphi H$, is the group with underlying set $N \times H$ where products and inverses given by
    \[ (n, h) \cdot (n', h') = (n \varphi(h)(n'), h h') \quad \text{ and } \quad (n, h)^{-1} = (\varphi(h)^{-1}(n^{-1}), h^{-1}) . \]
    In particular, $N$ (viewed as $N \times 1 \subset N \rtimes_\varphi H$) is a \tref{normal-subgroup}{normal subgroup} and the \tref{quotient-group}{quotient} is isomorphic to $H$ via $\psi : (N \rtimes_\varphi H)/N \to H$ given by $(n, h) \mapsto h$.
    
    Conversely, if $G$ is a group with subgroup $H$ and normal subgroup $N$ such that
    \[ 1 \to N \hookrightarrow G \xrightarrow{\psi} H \to 1 \]
    is exact for some $\psi : G \to H$, then $G = N \rtimes H$, with $H$ acting on $N$ by conjugation.
\end{topic}

\begin{example}{semidirect-product}
    Let $G = D_{n}$ be the \tref{dihedral-group}{dihedral group}, and let $N = \langle \rho \rangle$ be the normal subgroup of rotations and $H = \langle \sigma \rangle$ the subgroup generated by any fixed reflection $\sigma$. Then $H$ acts on $N$ by conjugation (in particular $\sigma \rho^k \sigma^{-1} = \rho^{-k}$), and $G = N \rtimes H$.
\end{example}

% // Borel subgroup
% Dihedral group
% Subgroup index
% Lagrange's theorem
% Sylow p-group

\begin{topic}{proper-group-action}{proper group action}
    A continuous \tref{group-action}{group action} of a \tref{TO:topological-group}{topological group} $G$ on a \tref{TO:topological-space}{topological space} $X$ is \textbf{proper} if the map
    \[ G \times X \to X \times X, \quad (g, x) \mapsto (x, gx) \]
    is \tref{TO:proper-map}{proper map}.
\end{topic}

\begin{topic}{profinite-group}{profinite group}
    A \textbf{profinite group} $G$ is a \tref{TO:topological-group}{topological group} isomorphic to an \tref{CT:inverse-limit}{inverse limit} of \tref{TO:discrete-topology}{discrete} finite groups
    \[ G \simeq \varprojlim_I G_i = \left\{ (x_i)_{i \in I} \in \prod_{i \in I} G_i : f_{ij}(x_j) = x_i \text{ for all } i \le j \right\} . \]
    The topology on $G$ is the \tref{TO:subspace-topology}{subspace} \tref{TO:product-topology}{product topology.}
\end{topic}

\begin{example}{profinite-group}
    Let $G$ be an arbitrary group, and $I$ a collection of normal subgroups of finite index of $G$, partially ordered with respect to (reversed) inclusions: $N \le N' \iff N \supset N'$. This yields an inverse system $(G / N)_{N \in I}$ with transition maps $G/N' \to G/N$ for $N \le N'$, and the profinite group
    \[ \hat{G} = \varprojlim_{N \in I} G / N ,\]
    is called the \textit{profinite completion} of $G$.
\end{example}

\begin{topic}{finitely-generated-group}{finitely generated group}
    A \tref{group}{group} $G$ is \textbf{finitely generated} if there exists a finite set $S \subset G$ such that every element of $G$ can be written as a finite product of elements of $S$.
\end{topic}

\begin{topic}{wreath-product}{wreath product}
    Let $G, H$ be \tref{group}{groups} and $S$ a set with $H$ \tref{group-action}{acting} on $S$. Let $K$ be the direct product $K = \prod_{s \in S} G$. Then $H$ acts on $K$ by $h \cdot (g_s)_{s \in S} = (g_{h^{-1} s})_{s \in S}$. The \textbf{unrestricted wreath product} of $G$ and $H$, denoted $G \, \text{Wr}_S H$ or $G \overline{\wr}_S H$, is the \tref{semidirect-product}{semidirect product} $K \rtimes H$.
    
    The \textbf{restricted wreath product}, denoted $G \, \text{wr}_S H$ or $G \wr_S H$, is obtained by replacing $K$ with the direct sum $\bigoplus_{s \in S} G$.
\end{topic}

\begin{example}{wreath-product}
    Let $G$ be any group $H = S_n$ the \tref{symmetric-group}{symmetric group} and $S = \{ 1, 2, \ldots, n \}$, where $S^n$ acts naturally on $S$. Then the wreath product is
    \[ G_n = \left( \prod_{i = 1}^{n} G \right) \rtimes S_n . \]
\end{example}

\begin{topic}{torsion-group}{torsion group}
    A \textbf{torsion group} is a \tref{group}{group} in which each element has finite \tref{order}{order}.
\end{topic}

\begin{example}{torsion-group}
    Every finite group is a torsion group.
    
    Every abelian group $A$ has a torsion subgroup $T = \{ a \in A : a^n = 1 \text{ for some } n \ge 1 \}$.
\end{example}

\begin{topic}{central-series}{central series}
    A \textbf{central series} of a \tref{group}{group} $G$ is a sequence of \tref{subgroup}{subgroups}
    \[ \{ e \} = H_0 \subset H_1 \subset \cdots H_n = G \]
    such that $H_i$ is \tref{normal-subgroup}{normal} in $H_{i + 1}$ and the quotient $H_{i + 1}/H_i$ lies in the \tref{group-center}{center} of $G/H_i$.
\end{topic}

\begin{topic}{lower-central-series}{lower central series}
    The \textbf{lower central series} of a \tref{group}{group} $G$ is the descending series of subgroups
    \[ G = G_1 \supset G_2 \supset \cdots \]
    where $G_{n + 1} = [G, G_n]$ is the subgroup of $G$ generated by all elements of the form $[g, h]$ with $g \in G$ and $h \in G_n$.
\end{topic}

\begin{topic}{upper-central-series}{upper central series}
    The \textbf{upper central series} of a \tref{group}{group} $G$ is the ascending series of subgroups
    \[ \{ e \} = Z_0 \subset Z_1 \subset \cdots \]
    where $Z_{i + 1} = \{ z \in G : [g, z] \in Z_i \text{ for all } g \in G \}$. That is, it is the largest subgroup such that $[G, Z_{i + 1}] \subset Z_i$.
\end{topic}

\begin{topic}{derived-series}{derived series}
    The \textbf{derived series} of a \tref{group}{group} $G$ is the descending series of subgroups
    \[ G = G^{(1)} \supset G^{(2)} \supset \cdots \]
    where $G^{(n + 1)} = [G^{(n)}, G^{(n)}]$ is the subgroup of $G$ generated by all elements of the form $[g, h]$ with $g, h \in G^{(n)}$.
\end{topic}

\begin{example}{derived-series}
    The derived series of the \tref{symmetric-group}{symmetric group} $G = S_3$ is
    \[ S_3 \supset \{ e, (1 \; 2 \; 3), (1 \; 3 \; 2) \} \supset \{ e \} . \]
    This shows $S_3$ is \tref{solvable-group}{solvable}.
\end{example}

\begin{topic}{braid-group}{braid group}
    The \textbf{braid group on $n$ strands} is the \tref{group}{group}
    \[ B_n = \left\langle \sigma_1, \ldots, \sigma_{n - 1} \;\bigg|\; \begin{array}{cc} \sigma_i \sigma_{i + 1} \sigma_i = \sigma_{i + 1} \sigma_i \sigma_{i + 1} \text{ for } 1 \le i \le n - 2 \\ \sigma_i \sigma_j = \sigma_j \sigma_i \text{ for } |i - j| > 1 \end{array} \right\rangle . \]
    Intuitively, its elements are ways to `braid` $n$ strands, where $\sigma_i$ interchanges strands $i$ and $i + 1$.
\end{topic}

\begin{topic}{modular-group}{modular group}
    The \textbf{modular group} is the \tref{group}{group}
    \[ \textup{PSL}_2(\ZZ) = \left\{ \begin{pmatrix} a & b \\ c & d \end{pmatrix} : ad - bc = 1 \right\} / \{ \pm 1 \} \]
    of $2 \times 2$ matrices with integer coefficients and determinant $1$, where any matrix $A$ is identified with $-A$.
\end{topic}

\begin{example}{modular-group}
    Let $\mathcal{H} = \{ z \in \CC : \operatorname{im}(z) > 0 \}$ be the upper half-plane. The modular group acts on $\mathcal{H}$ via
    \[ \begin{pmatrix} a & b \\ c & d \end{pmatrix} \cdot z = \frac{az + b}{cz + d} . \]
    Note that this gives a well-defined (left) \tref{group-action}{action}.
\end{example}

\begin{topic}{free-product}{free product}
    Given two \tref{group}{groups} $G$ and $H$, the \textbf{free product} $G * H$ is the group of words that can be built from elements of $G \sqcup H$, modulo the group relations from $G$ and $H$, together with the natural inclusion maps $i_G : G \to G * H$ and $i_H : H \to G * H$.
    
    More generally, if there is also given a group $F$ and morphisms $\varphi : F \to G$ and $\psi : F \to H$, then the \textbf{free product with amalgamation}, denoted $G *_F H$, is the group $G * H$ modulo the relations $\varphi(f) = \psi(f)^{-1}$.
    
    The free product is the \tref{CT:pushout}{pushout} in the \tref{CT:category}{category} of groups. That is, for any $g : G \to K$ and $h : H \to K$ with $g \circ \varphi = h \circ \psi$, there is a unique morphism $f : G *_F H \to K$ such that $h = f \circ i_H$ and $g = f \circ i_G$.
    \[ \begin{tikzcd} F \arrow{r}{\psi} \arrow{d}{\varphi} & H \arrow{d} \arrow[bend left=15]{ddr}{h} & \\ G \arrow{r} \arrow[bend right=15]{drr}{g} & G *_F H \arrow[dashed]{dr} & \\ & & K \end{tikzcd} \]
\end{topic}

\begin{example}{free-product}
    \begin{itemize}
        \item If $G = \langle x \;|\; x^m = 1 \rangle$ and $H = \langle y \;|\; y^n = 1 \rangle$, then $G * H = \langle x, y \;|\; x^m = y^n = 1 \rangle$.
        \item If $F_n$ is the \tref{free-group}{free group} on $n$ generators, then $F_m * F_n \simeq F_{m + n}$.
        \item The \tref{modular-group}{modular group} $\textup{PSL}_2(\ZZ)$ is isomorphic to the free product $(\ZZ/2\ZZ) * (\ZZ/3\ZZ)$.
    \end{itemize}
\end{example}
