\begin{topic}{derived-category}{derived category}
    Let $\mathcal{A}$ be an \tref{abelian-category}{abelian category}. The \textbf{homotopy category} $\textbf{K}(\mathcal{A})$ is defined to be the category whose objects are \tref{chain-complex}{chain complexes} in $\mathcal{A}$, and morphisms are chain maps modulo \tref{chain-homotopy}{homotopy}.
    
    The \textbf{derived category} $\textbf{D}(\mathcal{A})$ is obtained from $\textbf{K}(\mathcal{A})$ by `pretending' any quasi-isomorphism is an isomorphism. Concretely, a morphism from $A^\bdot$ to $B^\bdot$ is given by a roof
    \[ \begin{tikzcd}[column sep = 0.5em] & \arrow[swap]{ld}{f} C^\bdot \arrow{rd}{g} & \\ A^\bdot \arrow[dashed]{rr} & & B^\bdot , \end{tikzcd} \]
    where $f$ is a quasi-isomorphism, and we think of this roof as $g \circ f^{-1}$ (even though $f^{-1}$ might not exist). Technically, one \textit{localizes} the category $\textbf{K}(\mathcal{A})$ w.r.t. quasi-isomorphisms.
    
    Both $\textbf{K}(\mathcal{A})$ and $\textbf{D}(\mathcal{A})$ naturally admit the structure of a \tref{triangulated-category}{triangulated category}. In both cases, a triangle is distinguished if it is isomorphic to one of the form
    \[ A^\bdot \xrightarrow{f} B^\bdot \to C(f) \xrightarrow{\pi} A^\bdot[1] , \]
    where $C(f)$ is the \textit{mapping cone} of $f$. That is,
    \[ C(f)^i = A^{i + 1} \oplus B^i \quad \text{and} \quad d_{C(f)}^i = \begin{pmatrix} -d_A^{i + 1} & 0 \\ f^{i + 1} & d_B^i \end{pmatrix} . \]
\end{topic}

% \begin{topic}{Grothendieck--Verdier-duality}{Grothendieck--Verdier duality}
    
% \end{topic}

\begin{topic}{perfect-complex}{perfect complex}
    A \textbf{perfect complex} of modules over a commutative ring $R$ is an object in the \tref{derived-category}{derived category} of $R$-modules that is quasi-isomorphic to a bounded complex of finite projective $R$-modules. A \textbf{perfect module} is a module which is perfect when seen as a complex concentrated at degree zero.
\end{topic}
