\begin{topic}{derived-category}{derived category}
    Let $\mathcal{A}$ be an \tref{abelian-category}{abelian category}. The \textbf{homotopy category} $\textbf{K}(\mathcal{A})$ is defined to be the category whose objects are \tref{chain-complex}{complexes} in $\mathcal{A}$, and morphisms are chain maps modulo \tref{homotopy}{homotopy}.
    
    The \textbf{derived category} $\textbf{D}(\mathcal{A})$ is obtained from $\textbf{K}(\mathcal{A})$ by `pretending' any quasi-isomorphism is an isomorphism. Concretely, a morphism from $A^\bdot$ to $B^\bdot$ is given by a roof
    \[ \begin{tikzcd}[column sep = 0.5em] & \arrow[swap]{ld}{f} C^\bdot \arrow{rd}{g} & \\ A^\bdot \arrow[dashed]{rr} & & B^\bdot , \end{tikzcd} \]
    where $f$ is a quasi-isomorphism, and we think of this roof as $g \circ f^{-1}$ (even though $f^{-1}$ might not exist). Technically, one \textit{localizes} the category $\textbf{K}(\mathcal{A})$ w.r.t. quasi-isomorphisms.
\end{topic}

% \begin{topic}{exact-triangle}{exact triangle}
%     Let $\mathcal{A}$ be an \tref{abelian-category}{abelian category}. An \textbf{exact triangle} in $\textbf{K}(\mathcal{A})$ is a diagram of complexes
%     \[ A^\bdot \overset{f}{\to} B^\bdot \to C^\bdot \to A^\bdot[1] , \]
%     which is isomorphic to one where $C^\bdot = \text{Cone}(f)$, that is $C^i = A^{i + 1} \oplus B^i$ and
%     \[ d_C^i = \begin{pmatrix} -d_A^{i + 1} & 0 \\ f^{i + 1} & d_B^i \end{pmatrix} . \]
% \end{topic}

% \begin{topic}{integral-transform}{integral transform}
%     Let $X$ and $Y$ be 
% \end{topic}

