\begin{topic}{minkowski-theorem}{Minkowski's theorem}
    Let $L$ be a lattice in $\RR^n$, and $S \subset \RR^n$ a convex subset, which is symmetric with respect to the origin. That is, for all $x \in S$ also $-x \in S$. Then \textbf{Minkowski's theorem} states that if the volume of $S$ is strictly larger than $2^n \det(L)$, then $S$ must contain at least one non-zero lattice point of $L$.
\end{topic}

\begin{example}{minkowski-theorem}
    The given bound is sharp. Namely, consider $L = \ZZ^n$ with $\det(L) = 1$, and $S$ the interior of $[-1, 1]^n$, whose volume is $2^n$. Then $S$ does not contain any non-zero lattice points.
\end{example}

\begin{example}{minkowski-theorem}
    \begin{proof}
        By assumption, the set $\tfrac{1}{2} S = \{ \tfrac{1}{2} x : x \in S \}$ has volume $\textup{vol}(\tfrac{1}{2} S) = 2^{-n} \textup{vol}(S) > \textup{vol}(\RR^n/L)$, so the map $\tfrac{1}{2} S \to \RR^n / L$ cannot be injective. Hence, there are distinct points $x_1, x_2 \in S$ such that $\tfrac{1}{2} x_1 - \tfrac{1}{2} x_2 \in L$. Since $-x_2 \in S$ and $S$ is convex, the linear combination $\tfrac{1}{2} x_1 - \tfrac{1}{2} x_2$ is a non-zero lattice point in $S$.
    \end{proof}
\end{example}

\begin{topic}{root-system}{root system}
    Let $V$ be a finite-dimensional real or complex \tref{vector-space}{vector space} with an \tref{FA:inner-product}{inner product} $(\cdot, \cdot)$. A \textbf{root system} $\Phi$ in $V$ is a finite set of non-zero vectors in $V$, called \textbf{roots}, such that
    \begin{itemize}
        \item (\textit{span}) $\Phi$ spans $V$,
        \item (\textit{multiples}) if $\alpha \in \Phi$, then $n \alpha \in \Phi \iff n = \pm 1$,
        \item (\textit{reflection}) $\beta - \frac{2 (\alpha, \beta)}{(\alpha, \alpha)} \alpha \in \Phi$ for all $\alpha, \beta \in \Phi$,
        \item (\textit{integrality}) $\frac{2 (\alpha, \beta)}{\alpha, \alpha}$ is an integer for all $\alpha, \beta \in \Phi$.
    \end{itemize}
\end{topic}

\begin{example}{root-system}
    Given a complex \tref{AA:semisimple-lie-algebra}{semisimple Lie algebra} $\mathfrak{g}$ and a \tref{AA:cartan-subalgebra}{Cartan subalgebra} $\mathfrak{h} \subset \mathfrak{g}$, one can construct a root system $\Phi$ in $\mathfrak{h}^*$, where a root is an element $\alpha \in \mathfrak{h}^*$ such that
    \[ \mathfrak{g}_\alpha = \{ x \in \mathfrak{g} : [h, x] = \alpha(h) x \textup{ for all } h \in \mathfrak{h} \} \]
    is non-empty.
\end{example}
