\begin{topic}{category}{category}
    A \textbf{category} $\mathcal{C}$ is given by a collection of \textit{objects} and \textit{morphisms}.
    \begin{itemize}
        \item Each morphism has a \textit{domain} and \textit{codomain}, which are objects. We write $f : X \to Y$ or $X \overset{f}{\to} Y$ if $X$ is the domain of $f$ and $Y$ the codomain. We also write $X = \text{dom}(X)$ and $Y = \text{cod}(Y)$.
        
        \item Given two morphisms $f$ and $g$ such that $\text{cod}(f) = \text{dom}(g)$, the \textbf{composition} of $f$ and $g$, written $gf$, is defined and has domain $\text{dom}(f)$ and codomain $\text{cod}(g)$.
        
        \item Composition is associative, i.e. $(fg)h = f(gh)$.
        
        \item For every object $X$ there is an \textit{identity} morphism $\id_X : X \to X$ satisfying $f \id_X = f$ and $\id_X g = g$ for all $f, g$.
    \end{itemize}
\end{topic}

\begin{topic}{full-subcategory}{full subcategory}
    A \textbf{full subcategory} $\mathcal{C}'$ of a \tref{category}{category} $\mathcal{C}$ is a subcategory which has all morphisms between its objects, that is
    \[ \Hom_{\mathcal{C}'}(X, Y) = \Hom_{\mathcal{C}}(X, Y) \]
    for all $X, Y$ in $\mathcal{C}'$.
\end{topic}

\begin{topic}{functor}{functor}
    Given two categories $\mathcal{C}$ and $\mathcal{D}$, a \textbf{functor} $F : \mathcal{C} \to \mathcal{D}$ assigns an object (resp. a morphism) in $\mathcal{D}$ to each object (resp. morphism) in $\mathcal{C}$, such that
    \begin{itemize}
        \item $F(f) : F(X) \to F(Y)$ for each $f : X \to Y$,
        \item $F(gf) = F(g) F(f)$,
        \item $F(\id_X) = \id_{F(X)}$.
    \end{itemize}
\end{topic}

\begin{topic}{monomorphism}{monomorphism}
    A morphism $f : X \to Y$ in a category $\mathcal{C}$ is a \textbf{monomorphism} if $f \circ g = f \circ h$ implies $g = h$.
    \[ \begin{tikzcd} W \arrow[shift left=0.25em]{r}{g} \arrow[shift right=0.25em, swap]{r}{h} & X \arrow{r}{f} & Y \end{tikzcd} \]
\end{topic}

\begin{topic}{epimorphism}{epimorphism}
    A morphism $f : X \to Y$ in a category $\mathcal{C}$ is a \textbf{epimorphism} if $g \circ f = h \circ f$ implies $g = h$.
    \[ \begin{tikzcd} X \arrow{r}{f} & Y \arrow[shift left=0.25em]{r}{g} \arrow[shift right=0.25em, swap]{r}{h} & Z \end{tikzcd} \]
\end{topic}

\begin{topic}{split-mono}{split monomorphism}
    A morphism $f : X \to Y$ is a \textbf{split monomorphism} if there exists a $g : Y \to X$ such that $gf = \id_X$. That is, it has a left-inverse. In particular, it is an \tref{monomorphism}{monomorphism}.
    \[ \begin{tikzcd} X \arrow[shift left=0.25em]{r}{f} & Y \arrow[shift left=0.25em]{l}{g} \end{tikzcd} \]
\end{topic}

\begin{topic}{split-epi}{split epimorphism}
    A morphism $f : X \to Y$ is a \textbf{split epimorphism} if there exists a $g : Y \to X$ such that $fg = \id_Y$. That is, it has a right-inverse. In particular, it is an \tref{epimorphism}{epimorphism}.
    \[ \begin{tikzcd} X \arrow[shift left=0.25em]{r}{f} & Y \arrow[shift left=0.25em]{l}{g} \end{tikzcd} \]
\end{topic}

\begin{topic}{isomorphism}{isomorphism}
    A morphism $f : X \to Y$ is an \textbf{isomorphism} if there exists a $g : Y \to X$ such that $fg = \id_Y$ and $gf = \id_X$. We call $g$ the \textbf{inverse} of $f$ and write $g = f^{-1}$. If it exists, it is unique.
\end{topic}

\begin{topic}{terminal-object}{terminal object}
    An object $X$ in a \tref{category}{category} $\mathcal{C}$ is called \textbf{terminal} if for any object $Y$ there is exactly one morphism $Y \to X$. Any two terminal objects are isomorphic.
\end{topic}

\begin{topic}{initial-object}{initial object}
    An object $X$ in a \tref{category}{category} $\mathcal{C}$ is called \textbf{initial} if for any object $Y$ there is exactly one morphism $X \to Y$. Any two initial objects are isomorphic.
\end{topic}

\begin{topic}{full-functor}{full functor}
    A \tref{functor}{functor} $F : \mathcal{C} \to \mathcal{D}$ is called \textbf{full} if for every two objects $X$ and $Y$ of $\mathcal{C}$, the map
    \[ F : \Hom_{\mathcal{C}}(X, Y) \to \Hom_{\mathcal{D}}(F(X), F(Y)) \]
    is surjective.
\end{topic}

\begin{topic}{faithful-functor}{faithful functor}
    A \tref{functor}{functor} $F : \mathcal{C} \to \mathcal{D}$ is called \textbf{full} if for every two objects $X$ and $Y$ of $\mathcal{C}$, the map
    \[ F : \Hom_{\mathcal{C}}(X, Y) \to \Hom_{\mathcal{D}}(F(X), F(Y)) \]
    is injective.
\end{topic}

\begin{topic}{natural-transformation}{natural transformation}
    A \textbf{natural transformation} between two functors $F, G : \mathcal{C} \to \mathcal{D}$ consists of a collection of morphisms $\mu_X : F(X) \to G(X)$ for each object $C$ in $\mathcal{C}$, such that for each $f : X \to Y$ the diagram
    \[ \begin{tikzcd} F(X) \arrow{r}{\mu_X} \arrow[swap]{d}{F(f)} & G(X) \arrow{d}{G(f)} \\ F(Y) \arrow[swap]{r}{\mu_Y} & G(Y) \end{tikzcd} \]
    commutes. We denote this by $\mu : F \Rightarrow G$.
\end{topic}

\begin{topic}{yoneda-embedding}{Yoneda embedding}
    Let $\mathcal{C}$ be a \tref{category}{category}. The \textbf{Yoneda embedding} is the functor
    \[ y_{(-)} : \mathcal{C} \to \text{Set}^{\mathcal{C}^\text{op}} \]
    given by $y_X(Y) = \Hom_{\mathcal{C}}(Y, X)$.
\end{topic}

\begin{topic}{yoneda-lemma}{Yoneda lemma}
    Let $\mathcal{C}$ be a \tref{category}{category}, and write $y_X = \Hom_{\mathcal{C}}(-, X)$. The \textbf{Yoneda lemma} states that for every $F : \mathcal{C}^\text{op} \to \text{Set}$ and $X$ in $\mathcal{C}$ there is a bijection
    \[ \Hom(y_X, F) \xrightarrow{\sim} F(X)  \]
    natural in $X$ and $F$. As a consequence, the \tref{yoneda-embedding}{Yoneda embedding}
    \[ y_{(-)} : \mathcal{C} \to \text{Set}^{\mathcal{C}^\text{op}} \]
    is \tref{full-functor}{full} and \tref{faithful-functor}{faithful}.
\end{topic}

\begin{topic}{groupoid}{groupoid}
    A \textbf{groupoid} is a category where every morphism is an isomorphism.
\end{topic}

\begin{topic}{equivalence-of-categories}{equivalence of categories}
    A \tref{functor}{functor} $F : \mathcal{C} \Rightarrow \mathcal{D}$ is said to be an \textbf{equivalence of categories} if there exists a functor $G : \mathcal{D} \Rightarrow \mathcal{C}$ and natural isomorphisms $\mu : F \Rightarrow G$ and $\nu : G \Rightarrow F$. In this case $F$ and $G$ are called \textit{pseudo-inverses} of each other.
\end{topic}

\begin{topic}{essentially-surjective-functor}{essentially surjective functor}
    A \tref{functor}{functor} $F : \mathcal{C} \to \mathcal{D}$ is \textbf{essentially surjective} if each object in $\mathcal{D}$ is isomorphic to $F(X)$ for some $X$ in $\mathcal{C}$.
\end{topic}

\begin{topic}{equalizer}{equalizer}
    Let $f, g : X \to Y$ be morphisms in a \tref{category}{category} $\mathcal{C}$. An object $E$ of $\mathcal{C}$ together with morphism $e : E \to X$ is said to be an \textbf{equalizer} of the pair $f, g$ if $fe = ge$ and for any other such $e' : E' \to X$ there exists a unique morphism $h : E' \to E$ such that $e' = eh$.
    \[ \begin{tikzcd} E \arrow{r}{e} & X \arrow[shift left=0.25em]{r}{f} \arrow[swap, shift right=0.25em]{r}{g} & Y \\ E' \arrow[dashed]{u}{\exists!} \arrow[swap]{ur}{e'} && \end{tikzcd} \]
\end{topic}

\begin{topic}{coequalizer}{coequalizer}
    Let $f, g : X \to Y$ be morphisms in a \tref{category}{category} $\mathcal{C}$. An object $Q$ of $\mathcal{C}$ together with morphism $q : Y \to Q$ is said to be a \textbf{coequalizer} of the pair $f, g$ if $qf = qg$ and for any other such $q' : Y \to Q'$ there exists a unique morphism $h : Q \to Q'$ such that $q' = hq$.
    \[ \begin{tikzcd} X \arrow[shift left=0.25em]{r}{f} \arrow[swap, shift right=0.25em]{r}{g} & Y \arrow{r}{q} \arrow[swap]{dr}{q'} & Q \arrow[dashed]{d}{\exists!} \\ && Q' \end{tikzcd} \]
\end{topic}

\begin{topic}{fiber-product}{fiber product}
    Let $f : X \to Z$ and $g : Y \to Z$ be morphisms in a \tref{category}{category} $\mathcal{C}$. An object $W$ of $\mathcal{C}$ together with morphisms $\pi_X: W \to X$ and $\pi_Y : W \to Y$ is said to be a \textbf{fiber product} of the pair $f, g$ if $f \pi_X = g \pi_Y$ and for any other such $W'$ with maps $\pi_X'$ and $\pi_Y'$ there exists a unique morphism $h : W' \to W$ such that $\pi_X' = \pi_X h$ and $\pi_Y' = \pi_Y h$.
    \[ \begin{tikzcd} W' \arrow[bend left=30]{rrd}{\pi_X'} \arrow[swap, bend right=30]{ddr}{\pi_Y'} \arrow[dashed]{rd}{\exists!} & & \\ & W \arrow{r}{\pi_X} \arrow[swap]{d}{\pi_Y} & X \arrow{d}{f} \\ & Y \arrow[swap]{r}{g} & Z \end{tikzcd} \]
\end{topic}

\begin{topic}{regular-monomorphism}{regular monomorphism}
    A \tref{monomorphism}{monomorphism} $f : X \to Y$ is called a \textbf{regular monomorphism} if fits into an equalizer diagram:
    \[ \begin{tikzcd} X \arrow{r}{f} & Y \arrow[shift left=0.25em]{r} \arrow[shift right=0.25em]{r} & Z \end{tikzcd} \]
\end{topic}

\begin{topic}{adjoint-functors}{adjoint functors}
    Let $F : \mathcal{C} \to \mathcal{D}$ and $G : \mathcal{D} \to \mathcal{C}$ be a pair of functors. We say that $F$ is \textbf{left adjoint} to $G$, or $G$ is \textbf{right adjoint} to $F$, denoted $F \dashv G$, if there is a natural bijection
    \[ \Hom_{\mathcal{D}}(F(C), D) \xrightarrow{\sim} \Hom_{\mathcal{D}}(C, G(D)) \]
    for each object $C$ in $\mathcal{C}$ and $D$ in $\mathcal{D}$. Naturality means that for all $f : C \to C'$ in $\mathcal{C}$ and $g : D' \to D$ in $\mathcal{D}$, the following square commutes.
    \[ \begin{tikzcd}[row sep=3em] \Hom_{\mathcal{D}}(F(C), D) \arrow{r} & \Hom_{\mathcal{C}}(C, G(D)) \\ \Hom_{\mathcal{D}}(F(C'), D') \arrow{r} \arrow{u}{g \circ (-) \circ F(f)} & \Hom_{\mathcal{C}}(C', G(D')) \arrow[swap]{u}{G(g) \circ (-) \circ f} \end{tikzcd} \]
    Two maps $f : F(C) \to D$ and $g : C \to G(D)$ which correspond to each other are called \textit{transposes}.
\end{topic}

\begin{topic}{retraction}{retraction}
    A \textbf{retraction} of a morphism $i : X \to Y$ is a morphism $r : Y \to X$ such that $r \circ i = \id_X$.
\end{topic}

\begin{topic}{section}{section}
    A \textbf{section} of a morphism $p : X \to Y$ is a morphism $s : Y \to X$ such that $p \circ s = \id_Y$.
\end{topic}

\begin{topic}{sieve}{sieve}
    Let $\mathcal{C}$ be a category, and $C$ an object of $\mathcal{C}$. A \textbf{sieve} on $C$ is a set of morphisms $S = \{ f : C' \to C \}$ such that if $f : C' \to C$ is in $S$ and $g : C'' \to C'$ is arbitrary, then $f \circ g$ is in $S$.
    
    Equivalently, it is a subpresheaf of $y_C$, the \tref{yoneda-embedding}{yoneda functor}.
\end{topic}

\begin{topic}{grothendieck-topology}{Grothendieck topology}
    Let $\mathcal{C}$ be a \tref{category}{category}. A \textbf{Grothendieck topology} on $\mathcal{C}$ consists of a family $\text{Cov}(C)$ of \tref{sieve}{sieves} on $C$ for every object $C$ of $\mathcal{C}$, called \textit{covering sieves}, such that
    \begin{itemize}
        \item the maximal sieve $\text{mac}(C)$ is in $\text{Cov}(C)$,
        \item if $R \in \text{Cov}(C)$ then for every $f : C' \to C$, $f^*(R) \in \text{Cov}(C')$,
        \item if $R$ is any sieve on $C$ and $S$ is a covering sieve on $C$, such that for every $f : C' \to C$ from $S$ we have $f^*(R) \in \text{Cov}(C')$, then $R \in \text{Cov}(C)$.
    \end{itemize}
\end{topic}

\begin{topic}{site}{site}
    A small category $\mathcal{C}$ together with a \tref{grothendieck-topology}{Grothendieck topology} on it is called a \textbf{site}.
\end{topic}

\begin{topic}{sheaf}{sheaf}
    A presheaf $\mathcal{F} : \mathcal{C}^\text{op} \to \textbf{Set}$ on a \tref{site}{site} $\mathcal{C}$ is a \textbf{sheaf} if for every covering sieve $\{ U_i \to U \}_{i \in I}$, the diagram
    \[ \mathcal{F}(U) \to \prod_{i \in I} \mathcal{F}(U_i) \rightrightarrows \prod_{i, j \in I} \mathcal{F}(U_i \times_U U_j) \]
    is an \tref{equalizer}{equalizer}.
\end{topic}

\begin{topic}{localization}{localization}
    Let $\mathcal{C}$ be a \tref{category}{category}, and $W$ a collection of morphisms of $\mathcal{C}$. A \textbf{localization} of $\mathcal{C}$ by $W$ is a category $\mathcal{C}\left[\frac{1}{W}\right]$ with a functor $Q : \mathcal{C} \to \mathcal{C}\left[\frac{1}{W}\right]$ such that
    \begin{itemize}
        \item $Q(f)$ is an isomorphism for all $f \in W$,
        \item for any category $\mathcal{D}$ and functor $F : \mathcal{C} \to \mathcal{D}$ such that $F(f)$ is an isomorphism for all $f \in W$, there exists a unique functor $G : \mathcal{C}\left[\frac{1}{W}\right] \to \mathcal{D}$ such that $F \simeq G \circ Q$.
        \[ \begin{tikzcd} \mathcal{C} \arrow{r}{Q} \arrow[swap]{rd}{F} & \mathcal{C}\left[\frac{1}{W}\right] \arrow[dashed]{d}{G} \\ & \mathcal{D} \end{tikzcd} \]
    \end{itemize}
\end{topic}

\begin{topic}{limit}{(co)limit}
    Given a \tref{functor}{functor} $F : \mathcal{C} \to \mathcal{D}$, a \textit{cone} for $F$ is an object $D$ of $\mathcal{D}$ together with a family of morphisms $\mu_C : D \to F(C)$ for all objects $C$ in $\mathcal{C}$, such that for all $f : C \to C'$ in $\mathcal{C}$,
    \[ \begin{tikzcd} & D \arrow[swap]{ld}{\mu_C} \arrow{rd}{\mu_{C'}} & \\ F(C) \arrow{rr}{F(f)} && F(C') \end{tikzcd} \]
    commutes in $\mathcal{D}$. A map of cones $(D, \mu) \to (D', \mu')$ is a map $g : D \to D'$ such that $\mu'_C g = \mu_C$ for all $C$ in $\mathcal{C}$. A \textit{limiting cone} or a \textbf{limit} for $F$ is a \tref{terminal-object}{terminal} object in the category of cones for $F$.
    
    Dually, one can define a \textit{cocone} for $F$ and \textit{colimiting cocones} to define a \textbf{colimit} for $F$.
\end{topic}

\begin{example}{limit}
    Let $\textbf{0}$ be the empty category. A limit for $! : \textbf{0} \to \mathcal{C}$ is a terminal object of $\mathcal{C}$, and a colimit is an initial object of $\mathcal{C}$.
\end{example}

\begin{example}{limit}
    Let $\textbf{2}$ be the discrete category with two objects. A functor $\textbf{2} \to \mathcal{C}$ is a pair $(A, B)$ of objects of $\mathcal{C}$, and a cone for this functor is an object $C$ with maps $C \to A$ and $C \to B$. Such a cone is a limiting cone iff for any $D$ with morphisms $D \to A$ and $D \to B$ there is a unique morphism $D \to C$ such that
    \[ \begin{tikzcd} & D \arrow[dashed]{d} \arrow{ld} \arrow{rd} & \\ A & \arrow{l} C \arrow{r} & B \end{tikzcd} \]
    commutes. In this case, $C$ is also known as the \textit{product} of $A$ and $B$, and denoted $A \times B$.
\end{example}
