\begin{topic}{inner-product}{inner product}
    An \textbf{inner product} on a \tref{LA:vector-space}{vector space} $V$ over $k = \RR$ or $\CC$ is a map $\langle \cdot, \cdot \rangle : V \times V \to k$ satisfying
    \begin{itemize}
        \item (\textit{linearity}) $\langle \alpha x + \beta y, z \rangle = \alpha \langle x, z \rangle + \beta \langle y, z \rangle$ for all scalars $\alpha$ and $x, y, z \in V$,
        \item \textit{(conjugate symmetry)} $\langle x, y \rangle = \overline{\langle y, x \rangle}$ for all $x, y \in V$,
        \item (\textit{positive definiteness}) $\langle x, x \rangle > 0$ for all $x \ne 0$ in $V$.
    \end{itemize}
    A vector space together with an inner product is called an \textbf{inner product space}.
\end{topic}

\begin{topic}{norm}{norm}
    A \textbf{norm} on a \tref{LA:vector-space}{vector space} $V$ over $\RR$ or $\CC$ is a function $\norm{\cdot} : V \to \RR_{\ge 0}$ satisfying
    \begin{itemize}
        \item (\textit{positive definite}) $\norm{x} \ge 0$ for all $x \in V$ with equality if and only if $x = 0$,
        \item (\textit{absolute homogeneity}) $\norm{\lambda x} = |\lambda| x$ for all scalars $\lambda$ and $x \in V$,
        \item (\textit{triangle inequality}) $\norm{x + y} \le \norm{x} + \norm{y}$ for all $x, y \in V$.
    \end{itemize}
\end{topic}

\begin{example}{norm}
    Every \tref{inner-product}{inner product} $\langle \cdot, \cdot \rangle$ induces a norm $\norm{x} = \langle x, x \rangle$.
\end{example}

\begin{example}{norm}
    Possible norms on $\RR^n$ or $\CC^n$ are
    \[ \norm{x}_p = \left( \sum_{i = 1}^n |x_i|^p \right)^{1/p} \quad \text{with} \quad 1 \le p < \infty \]
    and
    \[ \norm{x}_\infty = \max_{i} |x_i| . \]
    When $p = 2$, the above norm is equal to the \textit{Euclidean norm}.
\end{example}

\begin{example}{norm}
    Let $\mathcal{C}([a, b], k)$ with $k = \RR$ or $\CC$ be the vector space of continuous functions $[a, b] \to k$. Possible norms are
    \[ \norm{f}_p = \left(\int_a^b |f(x)|^p dx \right)^{1/p} \quad \text{with} \quad 1 \le p < \infty . \]
\end{example}

\begin{topic}{equivalent-norms}{equivalent norms}
    Two \tref{norm}{norms} $\norm{\cdot}_1$ and $\norm{\cdot}_2$ on a \tref{LA:vector-space}{vector space} $V$ are \textbf{equivalent} if there exists numbers $m, M > 0$ such that
    \[ m \norm{x}_1 \le \norm{x}_2 \le M \norm{x}_1 \]
    for all $x \in V$.
\end{topic}

\begin{example}{equivalent-norms}
    On a finite-dimensional vector space $V = \RR^n$ or $\CC^n$, all norms are equivalent. Namely, since the unit sphere is compact, both norms $\norm{\cdot}_1$ and $\norm{\cdot}_2$ attain minima $m_1, m_2$ and maxima $M_1, M_2$ on the sphere. Now it follows that
    \[ \frac{m_2}{M_1} \norm{x}_1 \le \norm{x}_2 \le \frac{M_2}{m_1} \norm{x}_1 \]
    for all $x \in V$ by linearity.
\end{example}

\begin{topic}{banach-space}{Banach space}
    A \textbf{Banach space} is a \tref{norm}{normed} \tref{LA:vector-space}{vector space} which is \tref{TO:complete-metric-space}{complete} with respect to the metric $d(x, y) = \norm{x - y}$ induced by the norm.
\end{topic}

\begin{topic}{banach-algebra}{Banach algebra}
    A \textbf{Banach algebra} is an \tref{AA:algebra}{algebra} $A$ over $\RR$ or $\CC$ which is also a \tref{banach-space}{Banach space}, satisfying
    \[ \norm{xy} \le \norm{x} \norm{y} \]
    for all $x, y \in A$.
\end{topic}

\begin{topic}{star-algebra}{*-algebra}
    A \textbf{*-algebra} is a \tref{banach-algebra}{Banach algebra} $A$ over $\CC$ together with a map $(-)^* : A \to A, x \mapsto x^*$ satisfying
    \begin{itemize}
        \item (\textit{involution}) $(x^*)^* = x$ for all $x \in A$,
        \item (\textit{antiautomorphism}) $(x + y)^* = x^* + y^*$ and $(xy)^* = y^* x^*$ for all $x, y \in A$,
        \item (\textit{$\CC$-antilinear}) $(\lambda x)^* = \overline{\lambda} x^*$ for all $\lambda \in \CC$ and $x \in A$.
    \end{itemize}
\end{topic}

\begin{topic}{c-star-algebra}{C*-algebra}
    A \textbf{C*-algebra} is a \tref{banach-algebra}{Banach algebra} $A$ over $\CC$ together with a map $A \to A, x \mapsto x^*$ satisfying
    \begin{itemize}
        \item (\textit{involution}) $(x^*)^* = x$ for all $x \in A$,
        \item (\textit{antiautomorphism}) $(x + y)^* = x^* + y^*$ and $(xy)^* = y^* x^*$ for all $x, y \in A$,
        \item (\textit{$\CC$-antilinear}) $(\lambda x)^* = \overline{\lambda} x^*$ for all $\lambda \in \CC$ and $x \in A$,
        \item (\textit{C*-identity}) $\norm{x^* x} = \norm{x}^2$ for all $x \in A$.
    \end{itemize}
    The first three conditions equivalent to saying $A$ is a \tref{star-algebra}{*-algebra}.
\end{topic}

\begin{example}{c-star-algebra}
    The algebra $\text{Mat}_n(\CC)$ of $n \times n$ matrices over $\CC$ becomes a C*-algebra if we use \textit{operator norm}
    \[ \norm{A} = \sup \{ \norm{A v} : v \in \CC^n \textup{ with } \norm{v} = 1 \} . \]
\end{example}

\begin{example}{c-star-algebra}
    Let $X$ be a \tref{TO:topological-space}{topological space} which is \tref{TO:compact-space}{compact} and \tref{TO:hausdorff-space}{Haudorff}. The algebra of complex-valued functions
    \[ C(X) = \{ f : X \to \CC \textup{ with $f$ continuous} \} \]
    with norm
    \[ \norm{f} = \sup_{x \in X} |f(x)| \]
    and involution $f^*(x) = \overline{f(x)}$, is a C*-algebra.
\end{example}

\begin{topic}{gelfand-duality}{Gelfand duality}
    \textbf{Gelfand duality} states that the \tref{CT:category}{category} of \tref{TO:compact-space}{compact} \tref{TO:hausdorff-space}{Hausdorff} \tref{TO:topological-space}{spaces} is \tref{CT:equivalence-of-categories}{equivalent} to the (\tref{CT:opposite-category}{opposite}) category of commutative \tref{c-star-algebra}{C*-algebras}. It is given by the functors
    \[ \textbf{Top}_\textup{cpt} \stackrel{\overset{C}{\longrightarrow}}{\underset{\textup{sp}}{\longleftarrow}} \textbf{C*Alg}_\textup{comm}^\textup{op} , \]
    where $C$ assigns to a space $X$ the C*-algebra of continuous functions
    \[ C(X) = \{ f : X \to \CC \textup{ with $f$ continuous} \} , \]
    and $\textup{sp}$ is given by the \tref{gelfand-spectrum}{Gelfand spectrum}.
\end{topic}

\begin{topic}{gelfand-spectrum}{Gelfand spectrum}
    Let $A$ be a commutative \tref{c-star-algebra}{C*-algebra}. The \textbf{Gelfand spectrum} of $A$ is the set of \textit{characters}
    \[ \textup{sp}(A) = \{ \phi : A \to \CC \mid \phi \ne 0 \textup{ is an algebra morphism} \} , \]
    with the \textit{weak *-topology}: the \tref{TO:coarser}{coarsest} \tref{TO:topological-space}{topology} such that for all $a \in A$ the functions $\hat{a} : \textup{sp}(A) \to \CC$ given by $\hat{a}(\chi) = \chi(a)$ are continuous.
    
    It can be shown that $\textup{sp}(A)$ is \tref{TO:compact-space}{compact} and \tref{TO:hausdorff-space}{Hausdorff}.
\end{topic}

\begin{topic}{gelfand-naimark-segal-theorem}{Gelfand--Naimark--Segal theorem}
    The \textbf{Gelfand--Naimark--Segal theorem} states that any \tref{c-star-algebra}{C*-algebra} is isometrically *-isomorphic to a C*-subalgebra of \tref{bounded-operator}{bounded operators} on a \tref{hilbert-space}{Hilbert space}. Such a Hilbert space and embedding can be constructed using the \textit{GNS construction}.
\end{topic}

\begin{topic}{hilbert-space}{Hilbert space}
    A \textbf{Hilbert space} is an \tref{inner-product}{inner product space} which is \tref{TO:complete-metric-space}{complete} with respect to the \tref{TO:metric-space}{metric} $d(x, y) = \langle x - y, x - y \rangle$ induced by the inner product.
\end{topic}

\begin{topic}{bounded-operator}{bounded operator}
    An operator $T : X \to Y$ between \tref{norm}{normed} \tref{LA:vector-space}{vector spaces} is \textbf{bounded} if there exists a number $K > 0$ such that $\norm{Tx} \le K \norm{x}$ for all $x \in X$.
\end{topic}

\begin{topic}{invertible-operator}{invertible operator}
    A \tref{bounded-operator}{bounded operator} $T : X \to Y$ between \tref{norm}{normed} \tref{LA:vector-space}{vector spaces} is \textbf{invertible} if there exists a bounded operator $S : Y \to X$ such that $ST = \id_X$ and $TS = \id_Y$.
\end{topic}

\begin{topic}{compact-operator}{compact operator}
    An operator $T : X \to Y$ between \tref{norm}{normed} \tref{LA:vector-space}{vector spaces} is \textbf{compact} if for every bounded subset $V \subset X$, the image $T(V)$ is relatively compact, i.e. the closure $\overline{T(V)} \subset Y$ is compact.
\end{topic}

\begin{topic}{cauchy-schwarz-inequality}{Cauchy--Schwarz inequality}
    The \textbf{Cauchy--Schwarz inequality} states that for any vectors $v, w$ in an \tref{inner-product}{inner-product-space},
    \[ \langle v, w \rangle^2 \le \langle v, v \rangle \cdot \langle w, w \rangle . \]
\end{topic}

\begin{topic}{bolzano-weierstrass-theorem}{Bolzano--Weierstrass theorem}
    The \textbf{Bolzano--Weierstrass theorem} states that any bounded sequence $(x_n)_{n \in \NN}$ in $\RR^n$ has a \tref{TO:convergent-sequence}{convergent} subsequence.
\end{topic}

\begin{topic}{monotone-convergence-theorem}{monotone convergence theorem}
    The \textbf{monotone convergence theorem} states that any monotone sequence $(a_n)_{n \in \NN}$ of $\RR$, that is $a_{n + 1} \ge a_n$ for all $n \in \NN$ or $a_{n + 1} \le a_n$ for all $n \in \NN$, \tref{TO:convergent-sequence}{converges} if and only if it is bounded.
\end{topic}

\begin{topic}{heine-borel-theorem}{Heine--Borel theorem}
    The \textbf{Heine--Borel theorem} states that a subset $S \subset \RR^n$ is \tref{TO:compact-space}{compact} if and only if it is closed and bounded.
\end{topic}

\begin{topic}{hilbert-schmidt-operator}{Hilbert--Schmidt operator}
    Let $H$ be a \tref{hilbert-space}{Hilbert space}. A \textbf{Hilbert--Schmidt operator} is an operator $A : H \to H$ with finite \textit{Hilbert--Schmidt norm}
    \[ \|A\|_\textup{HS}^2 = \sum_{i \in I} \|A e_i\|^2 < \infty , \]
    where $\{ e_i : i \in I \}$ is an orthonormal basis.
\end{topic}

\begin{topic}{seminorm}{seminorm}
    A \textbf{seminorm} on a \tref{LA:vector-space}{vector space} $V$ over $\RR$ or $\CC$ is a function $\norm{\cdot} : V \to \RR_{\ge 0}$ satisfying
    \begin{itemize}
        \item (\textit{absolute homogeneity}) $\norm{\lambda x} = |\lambda| x$ for all scalars $\lambda$ and $x \in V$,
        \item (\textit{triangle inequality}) $\norm{x + y} \le \norm{x} + \norm{y}$ for all $x, y \in V$.
    \end{itemize}
    That is, a seminorm is a \tref{norm}{norm} without the condition that it is positive definite.
\end{topic}

\begin{example}{seminorm}
    The following are examples of seminorms which are not a norm.
    \begin{itemize}
        \item The constant zero map on $V$ is a seminorm.
        \item On $V = \RR^2$, the map $(x, y) \mapsto x^2$ is a semi-norm.
        \item For any linear map $f : V \to \RR$, the map $v \mapsto |f(v)|$ is a seminorm.
    \end{itemize}
\end{example}

\begin{example}{seminorm}
    Any seminorm $\norm{\cdot}$ is non-negative, that is, $\norm{v} \ge 0$ for all $v \in V$. Namely, from absolute homogenity follows that $\norm{0} = 0$, and then
    \[ 0 = \norm{0} \le \norm{v} + \norm{-v} = 2 \cdot \norm{v} . \]
\end{example}

\begin{topic}{schwartz-space}{Schwartz space}
    The \textbf{Schwartz space} is the space of \textit{rapidly decreasing functions} on $\RR^n$, given by
    \[ \mathcal{S}(\RR^n) = \left\{ f : \RR^n \to \CC \mid \textup{for all } \alpha, \beta \in \NN^n , \sup_{x \in \RR^n} |x^\alpha \partial_x^\beta f| < \infty \right\} , \]
    where $x^\alpha$ denotes $x_1^{\alpha_1} x_2^{\alpha_2} \cdots x_n^{\alpha_n}$ and $\partial_x^\beta = \partial_{x_1}^{\beta_1} \partial_{x_2}^{\beta_2} \cdots \partial_{x_n}^{\beta_n}$.
    % The Schwartz space is a real vector space with \tref{seminorm}{seminorm} given by
    % \[ \norm{f}_{\alpha, \beta} = \sup_{x \in \RR^n} |x^\alpha \partial_x^\beta f| . \]
\end{topic}

\begin{topic}{lp-space}{Lp space}
    Let $(X, \mu)$ be a measurable space, $0 < p < \infty$ and $\mathbb{K}$ equal to $\RR$ or $\CC$. The space of \textit{$p$-integrable functions} on $X$ is
    \[ \mathcal{L}^p(X) = \left\{ f : X \to \mathbb{K} : \norm{f}_p < \infty \right\} , \]
    where
    \[ \norm{f}_p = \left(\int_X |f|^p \; d\mu \right)^{1/p} . \]
    Now $L^p(X)$ is defined as the quotient
    \[ L^p(X) = \mathcal{L}^p(X) / \mathcal{N} , \quad \text{ where } \mathcal{N} = \{ f \in L^p(X) : \norm{f}_p = 0 \} . \]
    
    Let $I$ be an indexing set, and $1 \le p < \infty$. Then
    \[ \ell^p(I) = \left\{ (x_i)_{i \in I} \in \mathbb{K}^I : \norm{(x_i)_{i \in I}}_p < \infty \right\} , \]
    where
    \[ \norm{(x_i)_{i \in I}}_p = \left(\sum_{i \in I} |x_i|^p \right)^{1/p} . \]
\end{topic}

\begin{topic}{fredholm-operator}{Fredholm operator}
    Let $X$ and $Y$ be \tref{banach-space}{Banach spaces}. A \tref{bounded-operator}{bounded operator} $T : X \to Y$ is \textbf{Fredholm} if $\ker T$ and $\operatorname{coker} T$ are both finite-dimensional subspaces.
    
    The \textit{index} of a Fredholm operator $T$ is defined as $\operatorname{Ind}(T) = \dim \ker T - \dim \operatorname{coker} T$.
\end{topic}

\begin{example}{fredholm-operator}
    Let $H$ be a Hilbert space with orthonormal basis $(e_n)_{n \in \NN}$. The right shift operator $S$ on $H$ defined by
    \[ S(e_n) = e_{n + 1}, \quad \textup{ for } n \ge 0 , \]
    has $\ker S = \{ 0 \}$ and $\operatorname{coker} S \simeq \langle e_0 \rangle$, so $S$ is a Fredholm operator with index $-1$. The powers $S^k$ for $k \ge 0$ are also Fredholm with index $-k$.
    
    Similarly, the adjoint $S^*$, the left shift operator,
    \[ S^*(e_0) = 0 \text{ and } S^*(e_{n + 1}) = e_n, \quad \textup { for } n \ge 0 , \]
    is Fredholm with index $1$.
\end{example}

\begin{example}{fredholm-operator}
    Sometimes Fredholm operators are required to have a closed range, but this additional condition redundant. Namely, since the cokernel of $T$ is finite-dimensional, the image $\im T$ has a closed complement $C$. The map
    \[ S : (X / \ker T) \oplus C \to Y, \quad (x, c) \mapsto T(x) + c \]
    is a bounded linear isomorphism, and thus by the \tref{open-mapping-theorem}{open mapping theorem} $S$ is a topological isomorphism. Therefore, $\im T = S((X / \ker T) \oplus \{ 0 \})$ is closed.
\end{example}

\begin{topic}{open-mapping-theorem}{open mapping theorem}
    Let $X$ and $Y$ be \tref{banach-space}{Banach spaces} and $T : X \to Y$ a surjective continuous linear operator. The \textbf{open mapping theorem} states that $T$ is an open map. That is, for all open $U \subset X$, the image  $T(U) \subset Y$ is open as well.
\end{topic}

\begin{topic}{calkin-algebra}{Calkin algebra}
    Let $H$ be a \tref{TO:separable-space}{separable} infinite-dimensional \tref{hilbert-space}{Hilbert space}. The \textbf{Calkin algebra} is the \tref{AA:quotient-ring}{quotient} of the ring of \tref{bounded-operator}{bounded operators} $B(H)$ by the \tref{AA:ideal}{ideal} $K(H)$ of \tref{compact-operator}{compact operators}.
\end{topic}

\begin{topic}{k-theory-c-star-algebra}{K-theory C*-algebra}
    Let $A$ be a unital \tref{c-star-algebra}{C*-algebra}. A \textit{projection over $A$} is an element $p \in \textup{Mat}_n(A)$ such that $p = p^* = p^2$, for some $n \ge 1$. Two such projections are \textit{homotopic} if there is a continuous path of projections connecting them. Define $\textup{K}_0(A)$ to be the \tref{GT:free-group}{free abelian group} of homotopy classes of projections over $A$, modulo the relations
    \begin{itemize}
        \item $[0_{n \times n}] = 0$ for all $n \ge 1$,
        \item $[p] + [q] = [p \oplus q]$ for all projections $p$ and $q$ over $A$.
    \end{itemize}
    The higher K-theory groups are defined by
    \[ K_n(A) = K_0(S^n A) , \]
    where
    \[ SA = \{ f : [0, 1] \to A \mid f \textup{ continuous and } f(0) = f(1) = 0 \} \]
    is the \textit{suspension} of $A$.
\end{topic}

\begin{example}{k-theory-c-star-algebra}
    Take $A = \CC$, and let $p$ and $q$ be two projections over $\CC$ of the same rank. Then there exists a unitary change of basis: $q = upu^*$ for some unitary matrix $u$. Now, the path
    \[ t \mapsto \begin{pmatrix} u \cos t & - \sin t \\ \sin t & u^* \cos t \end{pmatrix} \]
    defines a homotopy between $u \oplus u^*$ and $\left(\begin{smallmatrix} 0 & -1 \\ 1 & 0 \end{smallmatrix}\right)$, and in particular all $u \oplus u^*$ with $u$ unitary are homotopic. Hence,
    \[ [p] = \left[ \begin{pmatrix} p & \\ & 0 \end{pmatrix} \right] = \left[ \begin{pmatrix} u & \\ & u^* \end{pmatrix} \begin{pmatrix} p & \\ & 0 \end{pmatrix} \begin{pmatrix} u^* & \\ & u \end{pmatrix} \right] = \left[ \begin{pmatrix} q & \\ & 0 \end{pmatrix} \right] = [q] , \]
    which shows that
    \[ \textup{K}_0(\CC) \simeq \ZZ , \]
    where the isomorphism is given by the rank.
\end{example}

\begin{topic}{bott-periodicity}{Bott periodicity}
    Let $\textup{K}_n : \textbf{C*Alg} \to \textbf{Ab}$ denote the \tref{k-theory-c-star-algebra}{K-theory} of \tref{c-star-algebra}{C*-algebras}. \textbf{Bott periodicity} states that there are isomorphisms $\textup{K}_n \simeq \textup{K}_{n + 2}$ for all $n$. In particular, for any short exact sequence of $C^*$-algebras $0 \to A \xrightarrow{i} B \xrightarrow{q} C \to 0$, the long exact sequence of $K$-theory collapses to a hexagon:
    \[ \begin{tikzcd}
        \textup{K}_0(A) \arrow{r}{i_*} & \textup{K}_0(B) \arrow{r}{q_*} & \textup{K}_0(C) \arrow{d}{\exp} \\ \textup{K}_1(C) \arrow{u}{\textup{ind}} & \textup{K}_1(B) \arrow{l}{q_*} & \textup{K}_1(A) \arrow{l}{i_*}
    \end{tikzcd} \]
\end{topic}

% \begin{topic}{c-star-category}{C*-category}
%     A \textbf{C*-category} is a \tref{CT:category}{category} $\mathcal{C}$ \tref{CT:enriched-category}{enriched} over the category of complex \tref{banach-space}{Banach spaces}, together with an antilinear involutive contravariant \tref{CT:functor}{functor} $(-)^* : \mathcal{C} \to \mathcal{C}$, such that
%     \begin{itemize}
%         \item (\textit{composition}) $\norm{f \circ g} \le \norm{f} \norm{g}$ for all composable morphisms $f, g$ in $\mathcal{C}$,
%         \item (\textit{C*-identity}) $\norm{f^* \circ f} = \norm{f}^2$ for all morphisms $f$ in $\mathcal{C}$.
%     \end{itemize}
% \end{topic}

% \begin{example}{c-star-category}
%     A C*-category with a single object is a \tref{c-star-algebra}{C*-algebra} with a unit.
% \end{example}
