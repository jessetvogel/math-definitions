\begin{topic}{kan-complex}{Kan complex}
    A \textbf{Kan complex} is a \tref{CT:simplicial-object}{simplicial set} $X$ satisfying the \textit{Kan condition}: any map of simplicial sets $f : \Lambda^n_i \to X$ extends to a map of simplicial sets $\Delta^n \to X$. That is,
    \[ \begin{tikzcd} \Lambda^n_i \arrow{r}{f} \arrow{d} & X \\ \Delta^n \arrow[dashed]{ur} & \end{tikzcd} \]
\end{topic}

\begin{example}{kan-complex}
    For any \tref{TO:topological-space}{topological space} $X$, the singular simplicial set $\text{Sing}_\bdot(X)$ is a Kan-complex.
\end{example}

\begin{topic}{kan-fibration}{Kan fibration}
    A \textbf{Kan fibration} is a map of \tref{CT:simplicial-object}{simplicial sets} $f : X \to Y$ such that for any $1 \le i \le n$ and for any maps $s : \Lambda^n_i \to X$ and $y : \Delta^n \to Y$ with $f \circ s = y \circ i$ (where $i : \Lambda^n_i \to \Delta^n$ is the inclusion) there exists a map $x : \Delta^n \to X$ such that $s = x \circ i$.
    \[ \begin{tikzcd} \Lambda^n_i \arrow{r}{s} \arrow[swap]{d}{i} & X \arrow{d}{f} \\ \Delta^n \arrow{r}{y} \arrow[dashed]{ur}{x} & Y \end{tikzcd} \]
\end{topic}

\begin{topic}{model-category}{model category}
    A \textbf{model category} is a \tref{CT:category}{category} $\mathcal{C}$ with three distinguished classes of maps: \textit{weak equivalences}, \textit{fibrations} and \textit{cofibrations}, each of which is closed under composition and contains all identity maps. A map which is both a fibration (resp. cofibration) and a weak equivalence is called an \textit{acyclic fibration} (resp. \textit{acyclic cofibration}). The following axioms are required.
    \begin{itemize}
        \item Finite limits and colimits exist in $\mathcal{C}$.
        \item For maps $f$ and $g$ in $\mathcal{C}$ such that $gf$ is defined, if two out of the three maps $f, g$ and $gf$ are weak equivalences, then so is the third.
        \item If $f$ is a \tref{CT:retract}{retract} of $g$, and $g$ is a fibration, cofibration or a weak equivalence, then so is $f$.
        \item For any commutative diagram
        \[ \begin{tikzcd} A \arrow[swap]{d}{i} \arrow{r}{f} & X \arrow{d}{p} \\ B \arrow[swap]{r}{g} \arrow[dashed]{ur}{h} & Y \end{tikzcd} \]
        such that either (i) $i$ is a cofibration and $p$ an acyclic fibration, or (ii) $i$ an acyclic cofibration and $p$ a fibration, there exists a lift $h : B \to X$ making the diagram commute.
        \item Any map $f$ can be factored in two ways: (i) $f = pi$ with $i$ a cofibration and $p$ an acyclic fibration, and (ii) $f = pi$ with $i$ an acyclic cofibration and $p$ a fibration.
    \end{itemize}
    By the first axiom, $\mathcal{C}$ has an \tref{CT:initial-object}{initial} object $\varnothing$ and a \tref{CT:terminal-object}{terminal} object $\star$. An object $A$ of $\mathcal{C}$ is called \textbf{cofibrant} if $\varnothing \to A$ is a cofibration, and \textbf{fibrant} if $A \to \star$ is a fibration.
\end{topic}

\begin{example}{model-category}
    The category $\textbf{Top}$ of topological spaces can be given the structure of a model category by defining a map $f : X \to Y$ to be a
    \begin{itemize}
        \item \textit{weak equivalence} if $f$ is a \tref{AT:weak-homotopy-equivalence}{weak homotopy equivalence},
        \item \textit{cofibrations} if $f$ is a retract of a map $X \to Y'$ in which $Y'$ is obtained from $X$ by attaching cells,
        \item \textit{fibration} if $f$ is a \tref{AT:serre-fibration}{Serre fibration}.
    \end{itemize}
    In this model category, all objects are fibrant and \tref{AT:cw-complex}{CW-complexes} are cofibrant.
\end{example}

\begin{example}{model-category}
    The Kan--Quillen model structure on $\textbf{Set}_\Delta$, the category of \tref{CT:simplicial-object}{simplicial sets}. Take
    \begin{itemize}
        \item \textit{weak equivalences} = weak homotopy equivalences,
        \item \textit{fibrations} = \tref{kan-fibration}{Kan fibrations},
        \item \textit{cofibrations} = monomorphisms (degreewise injective maps).
    \end{itemize}
    All objects are cofibrant, and the fibrant objects are the \tref{HT:kan-complex}{Kan complexes}.
\end{example}

\begin{example}{model-category}
    Let $R$ be a commutative \tref{AA:ring}{ring}, and let $\textbf{Ch}_R = \textbf{Ch}_{\ge 0}(\textbf{Mod}_R)$ be the category of chain complexes of $R$-modules, in non-negative degree. Take
    \begin{itemize}
        \item \textit{weak equivalences} = quasi-isomorphisms,
        \item \textit{fibrations} = degreewise surjective maps,
        \item \textit{cofibrations} = degreewise injective maps with projective cokernel.
    \end{itemize}
    All objects are fibrant, and the cofibrant objects are the complexes which has in each degree a projective $R$-module.
\end{example}

\begin{example}{model-category}
    In any model category, the fibrations are precisely the maps satisfying the right lifting property (RLP) w.r.t. acyclic cofibrations. Namely, suppose that $f : X \to Y$ satisfies this RLP, and factor $f$ as $X \xrightarrow{i} Z \xrightarrow{p} Y$ with $i$ an acyclic cofibration and $p$ a fibration. Then the RLP yields a lift $h : Z \to X$ such that
    \[ \begin{tikzcd} X \arrow{r}{\id} \arrow[swap]{d}{i} & X \arrow{d}{f} \\ Z \arrow[dashed]{ur}{h} \arrow[swap]{r}{p} & Y \end{tikzcd} \]
    commutes. Now it follows that $f$ is a retract of $p$, because of the diagram
    \[ \begin{tikzcd} X \arrow{r}{i} \arrow{d}{f} & Z \arrow{d}{p} \arrow{r}{h} & X \arrow{d}{f} \\ Y \arrow{r}{\id} & Y \arrow{r}{\id} & Y \end{tikzcd} \]
    Similarly, the cofibrations are precisely the maps satisfying the left lifting property (LLP) w.r.t. acyclic fibrations.
\end{example}

\begin{topic}{proper-model-category}{proper model category}
    A \tref{model-category}{model category} $\mathcal{M}$ is called \textbf{right proper} if weak equivalences are preserved by pullback along fibrations. That is, for every weak equivalence $f : X \to Y$ and fibration $g : Z \to Y$, the pullback $g^* f : X \times_Y Z \to Z$ is a weak equivalence.
    
    Similarly, $\mathcal{M}$ is \textbf{left proper} if weak equivalences are preserved by pushout along cofibrations.
    Finally, $\mathcal{M}$ is \textbf{proper} if it is both left and right proper.
\end{topic}

\begin{example}{proper-model-category}
    \begin{itemize}
        \item All model categories in which all objects are fibrant are right proper. For example, the categories $\textbf{Top}$ of topological spaces and $\textbf{Ch}_R = \textbf{Ch}_{\ge 0}(\textbf{Mod}_R)$ of chain complexes of $R$-modules, both with their classical model structure, are right proper.
        \item All model categories in which all objects are cofibrant are left proper. For example, the category $\textbf{Set}_\Delta$ of simplicial sets with the classical model structure is left proper. 
        \item The category $\textbf{Set}_\Delta$ is also right proper, even though not all objects are fibrant. Similarly, $\textbf{Top}$ and $\textbf{Ch}_R$ are also left proper, even though not all their objects are cofibrant.
    \end{itemize}
\end{example}

\begin{topic}{quillen-adjunction}{Quillen adjunction}
    Let $\mathcal{C}, \mathcal{D}$ be \tref{model-category}{model categories}. A \textbf{Quillen adjunction} is an \tref{CT:adjunction}{CT:adjunction}
    \[ F : \mathcal{C} \to \mathcal{D}, \qquad G : \mathcal{D} \to \mathcal{C}, \qquad F \dashv G , \]
    such that $F$ preserves cofibrations and $G$ preserves fibrations.
    
    In particular, a Quillen adjunction descends to an adjunction on the \tref{homotopy-category}{homotopy categories}
    \[ \textbf{L} F : \textup{Ho}(\mathcal{C}) \to \textup{Ho}(\mathcal{D}), \qquad \textbf{R} G : \textup{Ho}(\mathcal{D}) \to \textup{Ho}(\mathcal{C}), \qquad \textbf{L} F \dashv \textbf{R} G . \]
\end{topic}

\begin{topic}{quillen-equivalence}{Quillen equivalence}
    A \tref{quillen-adjunction}{Quillen adjunction} is called a \textbf{Quillen equivalence} if the derived adjunction between \tref{homotopy-category}{homotopy categories} is an \tref{CT:equivalence-of-categories}{equivalence of categories}.
\end{topic}

% \begin{topic}{homotopy-category}{homotopy category}
%     The \textbf{homotopy category} of a \tref{CT:simplicial-object}{simplicial set} $S_\bdot$ is the category $\text{h}S_\bdot$ defined as follows:
%     \begin{itemize}
%         \item The objects are the \textit{vertices} $x \in S_0$.
%         \item Every \textit{edge} $e \in S_1$ determines a morphism $[e] : d_1(e) \to d_0(e)$. The collection of morphisms in $\text{h}S_\bdot$ is generated under composition by morphisms of the form $[e]$, subject to the relations
%         \[ [s_0(x)] = \id_x \text{ for } x \in S_0, \qquad [d_1(\sigma)] = [d_0(\sigma)] \circ [d_2(\sigma)] \text{ for } \sigma \in S_2 . \]
%     \end{itemize}
    
%     The functor $S_\bdot \mapsto \text{h}S_\bdot$ is \tref{CT:adjunction}{left adjoint} to the \tref{CT:nerve}{nerve functor}.
% \end{topic}

\begin{topic}{dold-kan-correspondence}{Dold--Kan correspondence}
    The \textbf{Dold--Kan correspondence} is an \tref{CT:equivalence-of-categories}{equivalence} between the category of \tref{CT:simplicial-object}{simplicial} \tref{GT:abelian-group}{abelian groups} and non-negatively graded \tref{HA:chain-complex}{chain complexes} of abelian groups,
    \[ \textbf{Ab}_\Delta \simeq \text{Ch}(\ZZ)_{\ge 0} . \]
    To any simplicial abelian group $A_\bdot$, one assigns the \textit{(normalized) Moore complex}
    \[ N(A_\bdot)_n := \bigcap_{i = 0}^{n - 1} \ker d_{n, i} \quad \text{ with differential } \quad \partial_n = d_{n, n}, \]
    where $d_{n, i} : A_n \to A_{n - 1}$ are the face maps of $A_\bdot$.
    Inversely, to any chain complex $C_\bdot$ one assigns the simplicial abelian group
    \[ \sigma(C_\bdot)_n := \bigoplus_{[n] \twoheadrightarrow [k]} C_k . \]
    For any $\nu : [m] \to [n]$ and $\tau : [n] \twoheadrightarrow [k]$, the composition $\tau\nu : [m] \to [k]$ uniquely factors as $[m] \overset{\sigma}{\twoheadrightarrow} [j] \overset{\iota}{\hookrightarrow} [k]$ for some $j \le m, k$, which defines a map
    \[ (C_k \to C_j) = \left\{ \begin{array}{cl} \id_{C_k} & \textup{ if } j = k , \\ (-1)^k \partial_k & \textup{ if } j = k - 1 \textup{ and } \iota = d_{k,k} , \\ 0 & \textup{ otherwise}. \end{array} \right. \]
    The universal property of the direct sum yields morphisms $\sigma(C_\bdot)_\nu : \sigma(C_\bdot)_n \to \sigma(C_\bdot)_m$.
    
    The statement can be generalized to any \tref{HA:abelian-category}{abelian category} $\mathcal{A}$: there is an equivalence
    \[ \mathcal{A}_\Delta \simeq \text{Ch}(\mathcal{A})_{\ge 0} . \]
\end{topic}

\begin{topic}{geometric-realization}{geometric realization}
    The \textbf{geometric realization} of a \tref{CT:simplicial-object}{simplicial set} $X : \Delta^\textup{op} \to \textbf{Set}$ is the \tref{TO:topological-space}{topological space} $|X|$ defined as follows. The geometric realization of the standard $n$-simplex $\Delta^n$ is given by
    \[ |\Delta^n| = \{ (x_0, \ldots, x_n) \in \RR^{n + 1} : 0 \le x_i \le 1, \sum_i x_i = 1 \} , \]
    and is naturally extended to
    \[ |X| = \underset{\Delta^n \to X}{\textup{colim}} |\Delta^n| , \]
    where the \tref{CT:limit}{colimit} is taken over the $n$-simplex category of $X$.
    
    This construction gives a functor
    \[ |\cdot| : \textbf{Set}_\Delta \to \textbf{Top} , \]
    which is \tref{CT:adjunction}{left adjoint} to the \tref{AT:singular-homology}{singular functor} $\textbf{Sing} : \textbf{Top} \to \textbf{Set}_\Delta$.
\end{topic}

\begin{topic}{homotopy-limit}{homotopy (co)limit}
    Let $\mathcal{C}$ be a \tref{model-category}{model category} and $\mathcal{I}$ a \textit{very small category}. Put a model structure on $\mathcal{C}^\mathcal{I}$ by saying $\mu : X \Rightarrow Y$ is a 
    \begin{itemize}
        \item weak equivalence if $\mu_C : X(C) \to Y(C)$ is a weak equivalence for all $C$ in $\mathcal{C}$,
        \item cofibration if $\mu_C : X(C) \to Y(C)$ is a cofibration for all $C$ in $\mathcal{C}$,
        \item fibration if (?).
    \end{itemize}
    Then the functors $\lim : \mathcal{C}^\mathcal{I} \to \mathcal{C}$ and $\Delta : \mathcal{C} \to \mathcal{C}^\mathcal{I}$ form a \tref{quillen-adjunction}{Quillen adjunction}, yielding an \tref{CT:adjunction}{CT:adjunction} on the \tref{homotopy-category}{homotopy categories}
    \[ \textbf{L} \Delta : \textup{Ho}(\mathcal{C}) \to \textup{Ho}(\mathcal{C}^\mathcal{I}), \qquad \textbf{R}\lim : \textup{Ho}(\mathcal{C}^\mathcal{I}) \to \textup{Ho}(\mathcal{C}), \qquad \textbf{L} \Delta \dashv \textbf{R} \lim . \]
     Now, for any functor $F : \mathcal{I} \to \mathcal{C}$, the \textbf{homotopy limit} of $F$ is $\textbf{R}\lim(F)$. In particular, $\textup{R}\lim(F)$ is isomorphic to $\lim(F)$ if $F$ is a fibrant object of $\mathcal{C}^\mathcal{I}$.
     
     Similarly, put another another model structure on $\mathcal{C}^\mathcal{I}$ by saying $\mu : X \Rightarrow Y$ is a
     \begin{itemize}
        \item weak equivalence if $\mu_C : X(C) \to Y(C)$ is a weak equivalence for all $C$ in $\mathcal{C}$,
        \item fibration if $\mu_C : X(C) \to Y(C)$ is a fibration for all $C$ in $\mathcal{C}$,
        \item cofibration if (?).
    \end{itemize}
    Again, the functors $\operatorname{colim} : \mathcal{C}^\mathcal{I} \to \mathcal{C}$ and $\Delta : \mathcal{C} \to \mathcal{C}^\mathcal{I}$ form a Quillen adjunction, yielding an adjunction on the homotopy categories
    \[ \textbf{L}\operatorname{colim} : \textup{Ho}(\mathcal{C}^\mathcal{I}) \to \textup{Ho}(\mathcal{C}), \qquad \textbf{R} \Delta : \textup{Ho}(\mathcal{C}) \to \textup{Ho}(\mathcal{C}^\mathcal{I}), \qquad \textbf{L} \operatorname{colim} \dashv \textbf{R} \Delta . \]
    Now, for any functor $F : \mathcal{I} \to \mathcal{C}$, the \textbf{homotopy colimit} of $F$ is $\textbf{L}\operatorname{colim}(F)$. In particular, $\textup{L}\operatorname{colim}(F)$ is isomorphic to $\operatorname{colim}(F)$ if $F$ is a cofibrant object of $\mathcal{C}^\mathcal{I}$.
\end{topic}

\begin{example}{homotopy-limit}
    Note that the usual (co)limits in $\mathcal{C}$ may not behave well with respect to the model structure. For example, take $\mathcal{C} = \textbf{Top}$ with the usual model structure, and consider the diagram
    \[ \begin{tikzcd} D^n \arrow{d} & S^{n - 1} \arrow[swap]{l}{i} \arrow{r}{i} \arrow[equals]{d} & D^n \arrow{d} \\ * & S^{n - 1} \arrow{l} \arrow{r} & * \end{tikzcd} \]
    with $i : S^{n - 1} \to D^n$ the inclusion of the boundary, and where all vertical maps are homotopy equivalences. The usual pushout of the top row is homeomorphic to $S^n$, while the usual pushout of the bottom row is a point, and the induced map $S^n \to *$ is not a homotopy equivalence. The `correct' pushout (the homotopy pushout) is given by the top row, since $i$ is a cofibration while $S^{n - 1} \to *$ is not.
\end{example}

\begin{example}{homotopy-limit}
    The \tref{AT:loop-space}{loop space} $\Omega_x X$ of a \tref{TO:topological-space}{topological space} $X$ with basepoint $x \in X$ can be seen as a homotopy pullback:
    \[ \begin{tikzcd} \Omega_x X \arrow{d} \arrow{r} & * \arrow{d}{x} \\ * \arrow[swap]{r}{x} & X \end{tikzcd} \]
    Namely, a fibrant replacement for $* \xrightarrow{x} X \xleftarrow{x} *$ is given by $P_x X \rightarrow X \leftarrow P_x X$, where $P_x X$ denotes the \tref{AT:path-space}{path space} of $X$ with respect to $x$. These diagrams are homotopy equivalent since $x \to P_x X$, the map to the constant path, is a homotopy equivalence.
\end{example}

\begin{example}{homotopy-limit}
    Let $k$ be a field, let $\textbf{Ch}(k)$ the category of chain complexes over $k$, and suppose we want to compute the homotopy fiber product $k \times_{k \oplus k[n]} k$. First note that $k \to k \oplus k[n]$ factors as $k \xrightarrow{i} k \oplus C(\id_{k[n]}) \xrightarrow{p} k \oplus k[n]$, where $C(\id_{k[n]}) = k[n] \oplus k[n - 1]$ is the \tref{HA:mapping-cone}{mapping cone} of $\id_{k[n]}$, and $i$ is an acyclic cofibration, and $p$ a fibration. Hence, after a fibrant replacement, we can compute the fiber product as usual:
    \[ \begin{tikzcd} k \oplus k[n] \oplus k[n - 1]^2 \arrow{d} \arrow{r} & k \oplus k[n] \oplus k[n - 1] \arrow{d} \\ k \oplus k[n] \oplus k[n - 1] \arrow{r} & k \oplus k[n] \end{tikzcd} \]
    Finally, since $[ \cdots \to k \xrightarrow{\Delta} k^2 \to \cdots ]$ is quasi-isomorphic to $[ \cdots \to 0 \to k \to \cdots ]$, we can also write the homotopy fiber product as $k \oplus k[n - 1]$.
\end{example}

\begin{topic}{homotopy-category}{homotopy category}
    Let $\mathcal{C}$ be a \tref{model-category}{model category}. The \textbf{homotopy category} $\textup{Ho}(\mathcal{C})$ of $\mathcal{C}$ is the \tref{CT:category}{category} whose
    \begin{itemize}
        \item objects are the objects of $\mathcal{C}$ which are both fibrant and cofibrant,
        \item morphisms are \tref{homotopy}{homotopy classes} of maps.
    \end{itemize}
    There is a functor $\mathcal{C} \to \textup{Ho}(\mathcal{C})$, sending an object $X$ to its fibrant-cofibrant replacement.
    
    Equivalently, the homotopy category is given as the \tref{CT:localization}{localization} of $\mathcal{C}$ by its class of weak equivalences.
\end{topic}

\begin{example}{homotopy-category}
    For $\mathcal{C} = \textbf{Ch}_{\ge 0}(\textbf{Mod}_R)$ the category of chain complexes of $R$-modules in non-negative degree, with the natural model structure, the homotopy category $\textup{Ho}(\mathcal{C})$ is the \tref{HA:derived-category}{derived category} $\textbf{D}(\mathcal{C})$.
\end{example}

\begin{topic}{cylinder-object}{cylinder object}
    Let $\mathcal{C}$ be a \tref{model-category}{model category}, and $X$ an object of $\mathcal{C}$. A \textbf{cylinder object} for $X$ is an object $\textup{Cyl}(X)$ of $\mathcal{C}$ together with a factorization
    \[ X \coprod X \to \textup{Cyl}(X) \xrightarrow{\sim} X , \]
    of the natural map $X \coprod X \to X$, where the second arrow is a weak equivalence.
\end{topic}

\begin{example}{cylinder-object}
    In the category of \tref{TO:topological-space}{topological spaces} with the natural model structure, a cylinder object of a topological space $X$ is the product $X \times [0, 1]$ with the factorization
    \[ X \coprod X = X \times \{ 0, 1 \} \xrightarrow{i} X \times [0, 1] \xrightarrow{\pi_X} X , \]
    where $i$ is the inclusion, and $\pi_X(x, t) = x$. Indeed $\pi_X$ is a \tref{AT:weak-homotopy-equivalence}{weak homotopy equivalence}.
\end{example}

\begin{topic}{path-object}{path object}
    Let $\mathcal{C}$ be a \tref{model-category}{model category}, and $X$ an object of $\mathcal{C}$. A \textbf{path object} for $X$ is an object $\textup{Path}(X)$ of $\mathcal{C}$ together with a factorization
    \[ X \xrightarrow{\sim} \textup{Path}(X) \to X \times X , \]
    of the diagonal $X \to X \times X$, where the first arrow is a weak equivalence.
\end{topic}

\begin{example}{path-object}
    In the category of \tref{TO:topological-space}{topological spaces} with the natural model structure, a path object of a topological space $X$ is the \tref{TO:mapping-space}{mapping space} $\textup{Map}([0, 1], X)$ with the factorization
    \[ X \xrightarrow{\textup{const}} \textup{Map}([0, 1], X) \xrightarrow{(d_0, d_1)} X \times X , \]
    where $\textup{const}(x)$ is the constant path at $x$, and $d_0(\gamma) = \gamma(0)$ and $d_1(\gamma) = \gamma(1)$. Indeed $\textup{const}$ is a \tref{AT:weak-homotopy-equivalence}{weak homotopy equivalence}.
\end{example}

\begin{topic}{homotopy}{homotopy}
    Let $\mathcal{C}$ be a \tref{model-category}{model category}, and let $f, g : X \to Y$ be two morphisms in $\mathcal{C}$.
    \begin{itemize}
        \item A \textbf{left homotopy} from $f$ to $g$ is a morphism $H : \textup{Cyl}(X) \to Y$, where $\textup{Cyl}(X)$ is a \tref{cylinder-object}{cylinder object} of $X$, such that
        \[ \begin{tikzcd} X \arrow{r} \arrow[swap]{dr}{f} & \textup{Cyl}(X) \arrow{d}{H} & X \arrow{l} \arrow{ld}{g} \\ & Y & \end{tikzcd} \]
        commutes. If such a left homotopy exists, then $f$ and $g$ are called \textbf{left homotopic}.
        \item A \textbf{right homotopy} from $f$ to $g$ is a morphism $H : X \to \textup{Path}(Y)$, where $\textup{Path}(Y)$ is a \tref{path-object}{path object} of $Y$, such that
        \[ \begin{tikzcd} & X \arrow{d}{H} \arrow[swap]{dl}{f} \arrow{dr}{g} & \\ Y & \textup{Path}(Y) \arrow{l} \arrow{r} & Y \end{tikzcd} \]
        commutes. If such a right homotopy exists, then $f$ and $g$ are called \textbf{right homotopic}.
    \end{itemize}
    If $X$ is cofibrant and $Y$ is fibrant, then being left homotopic is equivalent to being right homotopic.
\end{topic}

\begin{topic}{en-algebra}{En-algebra}
    Let $(\mathcal{C}, \otimes, \textbf{1})$ be a \tref{CT:symmetric-monoidal-category}{symmetric monoidal} \tref{CT:infinity-category}{$\infty$-category}, and $n \ge 0$ an integer. An \textbf{$E_n$-algebra} $\mathcal{E}$ of $\mathcal{C}$ consists of
    \begin{itemize}
        \item an object $\mathcal{E}(U) \in \mathcal{C}$ for each open set $U \subset \RR^n$ homeomorphic to a disk,
        \item for every embedding $U_1 \sqcup \cdots \sqcup U_m \subset V$ of disjoint disks into another disk, a multiplication map
        \[ \mu : \bigotimes_{i = 1}^{m} \mathcal{E}(U_i) \to \mathcal{E}(V) . \]
    \end{itemize}
    It is required that the multiplication maps $\mu$ are compatible with composition, and that $\mu$ is an equivalence for $m = 1$.
\end{topic}

\begin{example}{en-algebra}
    Let $X$ be a \tref{TO:topological-space}{topological space} and $\Omega_x X = \{ p : [0, 1] \to X : p(0) = p(1) = x \}$ its \tref{AT:loop-space}{loop space} with respect to some basepoint $x \in X$. One can concatenate paths $p, q : [0, 1] \to X$ as
    \[ (p \star q)(t) = \left\{ \begin{array}{cl}
         p(2t) & 0 \le t \le \tfrac{1}{2} , \\
         q(2t - 1) & \tfrac{1}{2} \le t \le 1 ,
    \end{array} \right. \]
    but this operation is not strictly associative, since loops $(p \star q) \star r$ and $p \star (q \star r)$ differ by their parametrization. However, concatenation is associative `up to coherent homotopy', and this can be made precise by viewing the loop space $\Omega_x X$ as an $E_1$-algebra.
    
    Let $\mathcal{C}$ the $\infty$-category of spaces. To every disk $U \subset \RR^1$ assign the space
    \[ \mathcal{E}(U) = \textup{Map}((U^+, \textup{pt}), (X, x)) , \]
    where $U^+$ denotes the one-point compactification of $U$. In particular, $U$ is homeomorphic to $S^1$, so $\mathcal{E}(U)$ is (non-canonically) homeomorphic to $\Omega_x X$. Furthermore, an embedding of disjoint disks $U_1 \sqcup \cdots \sqcup U_m \subset V$ induces a `collapse map' $V^+ \to U_1^+ \vee \cdots \vee U_m^+$ to the \tref{TO:wedge-sum}{wedge sum}, which yields a multiplication map
    \[ \mu : \mathcal{E}(U_1) \times \cdots \mathcal{E}(U_m) \to \mathcal{E}(V) . \]
    This makes $\mathcal{E}$ into an $E_1$-algebra.
    
    Note that the original concatenation $\star$ corresponds to the multiplication map $\mu$ for the embedding $(0, \tfrac{1}{2}) \sqcup (\tfrac{1}{2}, 1) \subset (0, 1)$.
\end{example}

\begin{example}{en-algebra}
    Suppose that $\mathcal{C}$ is an ordinary category, i.e. all $k$-morphisms for $k \ge 2$ are trivial, and let $\mathcal{E}$ be an $E_n$-algebra in $\mathcal{C}$. Note that for $n \ge 2$, all embeddings $(D^n)^{\sqcup k} \to D^n$ are isotopic, so all corresponding maps $\mathcal{E}^{\otimes k} \to \mathcal{E}$ are equal. This shows that $\mathcal{E}$ is precisely given by a commutative algebra in $\mathcal{C}$. For $n = 1$, all embeddings $(D^1)^{\sqcup k} \to D^1$ are isotopic up to a permutation of the factors. This shows that an $E_1$-algebra $\mathcal{E}$ in $\mathcal{C}$ is precisely an associative algebra in $\mathcal{C}$.
\end{example}
