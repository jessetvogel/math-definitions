\begin{topic}{enriched-category}{enriched category}
    Let $\mathcal{A}$ be a \tref{monoidal-category}{monoidal category}. An \textbf{$\mathcal{A}$-enriched category} $\mathcal{C}$ consists of
    \begin{itemize}
        \item (\textit{objects}) a collection $\textup{Ob}(\mathcal{C})$, whose elements are called \textit{objects} of $\mathcal{C}$,
        \item (\textit{hom-objects}) for every pair of objects $X, Y$ an object $\iHom_\mathcal{C}(X, Y)$ of $\mathcal{A}$,
        \item (\textit{composition law}) for every triple of objects $X, Y, Z$ a morphism
        \[ c_{Z, Y, X} \colon \iHom_\mathcal{C}(Y, Z) \otimes \iHom_\mathcal{C}(X, Y) \to \iHom_\mathcal{C}(X, Z) \]
        in $\mathcal{A}$,
        \item (\textit{identities}) for every object $X$ a morphism $e_X \colon \textbf{1} \to \iHom_\mathcal{C}(X, X)$ in $\mathcal{A}$,
    \end{itemize}
    such that for every quadruple of objects $W, X, Y, Z$ the diagram
    \[ \svg \begin{tikzcd}
        & \iHom_\mathcal{C}(Y, Z) \otimes \iHom_\mathcal{C}(W, Y) \arrow{dd}{c_{Z, Y, W}} \\ \iHom_\mathcal{C}(Y, Z) \otimes \left(\iHom_\mathcal{C}(X, Y) \otimes \iHom_\mathcal{C}(W, X)\right) \arrow{ur}{\id \otimes c_{Y, X, W}} \arrow{dd}{\alpha} & \\ & \iHom_\mathcal{C}(W, Z) \arrow{dd}{c_{Z, X, W}} \\ \left(\iHom_\mathcal{C}(Y, Z) \otimes \iHom_\mathcal{C}(X, Y)\right) \otimes \iHom_\mathcal{C}(W, X) \arrow{dr}{c_{Z, Y, X} \otimes \id} & \\ & \iHom_\mathcal{C}(X, Z) \otimes \iHom_\mathcal{C}(W, X)
    \end{tikzcd} \]
    commutes, and for every pair of objects $X, Y$, the diagrams
    \[ \svg \begin{tikzcd}
        \textbf{1} \otimes \iHom_\mathcal{C}(X, Y) \arrow{rr}{e_Y \otimes \id} \arrow{dr}{\lambda} && \iHom_\mathcal{C}(Y, Y) \otimes \iHom_\mathcal{C}(X, Y) \arrow{dl}{c_{Y, Y, X}} \\ & \iHom_\mathcal{C}(X, Y) & \\
        \iHom_\mathcal{C}(X, Y) \otimes \textbf{1} \arrow{rr}{\id \otimes e_X} \arrow{dr}{\rho} && \iHom_\mathcal{C}(X, Y) \otimes \iHom_\mathcal{C}(X, X) \arrow{dl}{c_{Y, X, X}} \\ & \iHom_\mathcal{C}(X, Y) &       
    \end{tikzcd} \]
    commute.
\end{topic}

\begin{example}{enriched-category}
    \begin{itemize}
        \item For $\mathcal{A} = \textbf{Set}$ the category of sets, with monoidal structure given by the cartesian product, an $\mathcal{A}$-enriched category is the same as an ordinary \tref{category}{category}.
        \item For $\mathcal{A} = \textbf{Cat}$ the category of (small) categories, with monoidal structure given by the product of categories, an $\mathcal{A}$-enriched category is the same as a 2-category.
        \item For $\mathcal{A} = \textbf{Ab}$ the category of abelian groups, with monoidal structure given by the product of groups, an $\mathcal{A}$-enriched category is the same as a \textit{preadditive category}.
        \item For $\mathcal{A} = \textbf{Vect}_k$ the category of vector spaces over a field $k$, with monoidal structure given by the tensor product over $k$, an $\mathcal{A}$-enriched category is also known as a \textit{$k$-linear category}.
    \end{itemize}
\end{example}

\begin{topic}{monoid-object}{monoid object}
    Let $(\mathcal{C}, \otimes, \textbf{1})$ be a \tref{monoidal-category}{monoidal category}. A \textbf{monoid object} in $\mathcal{C}$ is an object $M$ together with morphisms $\mu \colon M \otimes M \to M$ (the \textit{multiplication map}) and $\eta \colon \textbf{1} \to M$ (the \textit{unit}), such that
    \begin{itemize}
        \item (\textit{associativity}) the diagram
        \[ \svg \begin{tikzcd} (M \otimes M) \otimes M \arrow[swap]{d}{\mu \otimes \id} \arrow{r}{\alpha} & M \otimes (M \otimes M) \arrow{r}{\id \otimes \mu} & M \otimes M \arrow{d}{\mu} \\ M \otimes M \arrow{rr}{\mu} && M \end{tikzcd} \]
        commutes, where $\alpha$ is the \textit{associator}.
        \item (\textit{unit}) the diagram
        \[ \svg \begin{tikzcd} \textbf{1} \otimes M \arrow{r}{\eta \otimes \id} \arrow[swap]{rd}{\lambda} & M \otimes M \arrow{d}{\mu} & M \otimes \textbf{1} \arrow[swap]{l}{\id \otimes \eta} \arrow{dl}{\rho} \\ & M & \end{tikzcd} \]
        commutes, where $\lambda$ and $\rho$ are the left and right unitor, respectively.
    \end{itemize}
\end{topic}

\begin{example}{monoid-object}
    \begin{itemize}
        \item A monoid object in the category of sets $(\textbf{Set}, \times, \{ \star \})$ is a \tref{AA:monoid}{monoid}.
        \item A monoid object in the category of \tref{GT:abelian-group}{abelian groups} $(\textbf{Ab}, \otimes_\ZZ, \ZZ)$ is a \tref{AA:ring}{ring}.
        \item For a \tref{AA:ring}{commutative ring} $R$, a monoid object in the category of \tref{AA:module}{$R$-modules} $(\textbf{Mod}_R, \otimes_R, R)$ is an \tref{AA:algebra}{$R$-algebra}.
        \item For any category $\mathcal{C}$, a monoid in the category of endofunctors of $\mathcal{C}$, $(\Hom_\textbf{Cat}(\mathcal{C}, \mathcal{C}), \circ, \id_\mathcal{C})$, is a \tref{monad}{monad} on $\mathcal{C}$.
    \end{itemize}
\end{example}

\begin{topic}{weighted-limit}{weighted (co)limit}
    Let $\mathcal{V}$ be a \tref{monoidal-category}{monoidal category}, and let $F \colon \mathcal{I} \to \mathcal{C}$ be a \tref{functor}{functor} with $\mathcal{C}$ a category \tref{enriched-category}{enriced} in $\mathcal{V}$. A \textbf{weighted limit} for $F$ with respect to a \textit{weight functor} $W \colon \mathcal{I} \to \mathcal{V}$ is an object ${\lim}^W F$ in $\mathcal{C}$ represented by
    \[ \Hom_\mathcal{C}(C, {\lim}^W F) \isom \Hom_{\mathcal{V}^\mathcal{I}}(W, \Hom_\mathcal{C}(C, F(-))) . \]
    A \textbf{weighted colimit} for $F$ with respect to $W \colon \mathcal{I}^\textup{op} \to \mathcal{V}$ is an object ${\colim}_W F$ in $\mathcal{C}$ represented by
    \[ \Hom_\mathcal{C}({\colim}_W F, C) \isom \Hom_{\mathcal{V}^{\mathcal{I}^\textup{op}}}(W, \Hom_\mathcal{C}(F(-), C)) . \]
\end{topic}

\begin{example}{weighted-limit}
    Note that an ordinary \tref{limit}{limit} of a functor $F \colon \mathcal{I} \to \mathcal{C}$, if it exists, can be computed via the \tref{yoneda-embedding}{Yoneda embedding} on the level of presheaves:
    \[ \begin{aligned} \Hom_\mathcal{C}(C, \lim F) &\isom \lim (\Hom_\mathcal{C}(C, F(-))) \\ &\isom \Hom_\textbf{Set}(\textup{pt}, \lim (\Hom_\mathcal{C}(C, F(-)))) \\ &\isom \Hom_{\textbf{Set}^\mathcal{I}}(\Delta_{\textup{pt}}, \Hom_\mathcal{C}(C, F(-))) . \end{aligned} \]
    In particular, the weighted limit reduces to the ordinary limit when $W \colon \mathcal{I} \to \textbf{Set}$ is the constant functor $\Delta_\textup{pt}$.
\end{example}

\begin{topic}{operad}{operad}
    Let $(\mathcal{C}, \otimes, \textbf{1})$ be a \tref{CT:symmetric-monoidal-category}{symmetric} \tref{CT:monoidal-category}{monoidal category}. A \textbf{non-symmetric operad} in $\mathcal{C}$ consists of
    \begin{itemize}
        \item (\textit{$n$-ary operations}) an object $\mathcal{O}(n)$ for each integer $n \ge 0$
        \item (\textit{unit}) a morphism $e \colon \textbf{1} \to \mathcal{O}(1)$,
        \item (\textit{composition}) for all $n \ge 0$ and $k_1, \ldots, k_n \ge 0$,
        \[ \circ \colon \mathcal{O}(n) \otimes \mathcal{O}(k_1) \otimes \cdots \otimes \mathcal{O}(k_n) \to \mathcal{O}(k_1 + \cdots + k_n) , \]
    \end{itemize}
    satisfying
    \begin{itemize}
        \item (\textit{unit}) $e \circ \theta = \theta = \theta \circ (e \otimes \cdots \otimes e)$ for all $\theta \in \mathcal{O}(n)$,
        \item (\textit{associativity}) $\theta \circ (\theta_1 \circ (\theta_{1,1} \otimes \cdots \otimes \theta_{1,k_1}) \otimes \cdots \otimes \theta_n \circ (\theta_{n,1} \otimes \cdots \otimes \theta_{n,k_n}))$ $= (\theta \circ (\theta_1 \otimes \cdots \otimes \theta_n)) \circ (\theta_{1, 1} \otimes \cdots \otimes \theta_{1,k_1} \otimes \cdots \otimes \theta_{n,1} \otimes \cdots \otimes \theta_{n,k_n})$.
    \end{itemize}
    A \textbf{symmetric operad}, or just \textbf{operad}, consists of the above, together with an action of \tref{GT:symmetric-group}{$S_n$} on $\mathcal{O}(n)$, for each $n \ge 0$, such that
    \begin{itemize}
        \item (\textit{equivariance 1}) $\sigma(\theta) \circ (\theta_{\sigma(1)} \otimes \cdots \otimes \theta_{\sigma(n)}) = \sigma'(\theta \circ (\theta_1 \otimes \cdots \otimes \theta_n))$, for all $\sigma \in S_n$, where $\sigma' \in S_{k_1 + \cdots + k_n}$ acts on $\{ 1, 2, \ldots, k_1 + \cdots + k_n \}$ by permuting the blocks of size $k_1, k_2, \ldots, k_n$.
        \item (\textit{equivariance 2}) $\theta \circ (\sigma_1(\theta_1) \otimes \cdots \otimes \sigma_n(\theta_n)) = (\sigma_1, \ldots, \sigma_n)(\theta \circ (\theta_1 \otimes \cdots \otimes \theta_n))$ for all $\sigma_i \in S_{k_i}$.
    \end{itemize}
\end{topic}

\begin{topic}{derivator}{derivator}
    A \textbf{prederivator} is a strict $2$-functor $\mathbb{D} \colon \textbf{Cat}^\textup{op} \to \textbf{CAT}$ from the $2$-category of \tref{small-category}{small categories} to the $2$-category of all categories (ignoring set-theoretic issues). For any $u \colon I \to J$ in $\textbf{Cat}$, the induced functor $\mathbb{D}(u) \colon \mathbb{D}(J) \to \mathbb{D}(I)$ is usually denoted by $u^*$.

    A \textbf{derivator} is a prederivator $\mathbb{D}$ such that
    \begin{itemize}
        \item $\mathbb{D}(\varnothing)$ is not the empty category,
        \item the natural functor $\mathbb{D}(I \sqcup J) \to \mathbb{D}(I) \times \mathbb{D}(J)$ is an \tref{equivalence-of-categories}{equivalence of categories} for all $I, J \in \textbf{Cat}$,
        \item for all $I \in \textbf{Cat}$, any morphism $f \colon X \to Y$ in $\mathbb{D}(I)$ is an \tref{isomorphism}{isomorphism} if and only if $f_i \colon X_i \to Y_i$ is an isomorphism in $\mathbb{D}(\{ i \})$ for all $i \in I$.
        \item for every $u \colon I \to J$ in $\textbf{Cat}$, the functor $u^*$ has a \tref{adjunction}{left adjoint} $u_!$ and right adjoint $u^*$,
        \item for every $u \colon I \to J$ in $\textbf{Cat}$ and $j \in J$, the morphisms
        \[ \operatorname*{hocolim}_{u \downarrow j} \textup{pr}_1^*(X) \xrightarrow{\alpha_!} u_!(X)_j \quad \textup{ and } \quad u_*(X)_j \xrightarrow{\alpha_!} \operatorname*{hocolim}_{j \downarrow u} \textup{pr}_2^*(X) \]
        are isomorphisms for all $X \in \mathbb{D}(I)$, where $\textup{pr}_1 \colon u \downarrow j \to I$ and $\textup{pr}_2 \colon j \downarrow u \to I$ denote the natural projections.
    \end{itemize}
\end{topic}

\begin{example}{derivator}
    \begin{itemize}
        \item Given a category $\mathcal{C}$, the functor $y_\mathcal{C} : I \mapsto \mathcal{C}^I$ is a prederivator. It is a derivator if and only if $\mathcal{C}$ is \tref{complete-category}{complete} and cocomplete.
        \item Given an \tref{HA:abelian-category}{abelian category} $\mathcal{A}$, the functor $\mathbb{D}_\mathcal{A}$ which sends $I$ to the \tref{HA:derived-category}{derived category} $D(\mathcal{A}^I)$ is a prederivator.
        \item Given a cofibrantly generated \tref{HT:model-category}{model category} $\mathcal{C}$, one can endow $\mathcal{C}^I$ with the projective model structure for any $I \in \textbf{Cat}$. The functor $\mathbb{D}_\mathcal{C}$ which sends $I$ to the \tref{HT:homotopy-category}{homotopy category} $\operatorname{Ho}(\mathcal{C}^I)$ is a prederivator.
    \end{itemize}
\end{example}

\begin{topic}{six-functor-formalism}{six-functor formalism}
    Let $\mathcal{C}$ be a \tref{CT:category}{category} which has all finite \tref{CT:limit}{limits}, and let $E$ be a class of morphisms of $\mathcal{C}$ stable under pullback and composition and containing all isomorphisms.
    Let $\operatorname{Corr}(\mathcal{C}, E)$ be the \tref{CT:symmetric-monoidal-category}{symmetric monoidal} \tref{CT:infinity-category}{$\infty$-category} whose objects are the objects of $\mathcal{C}$ and whose morphisms are \tref{CT:correspondence}{correspondences} $X \xleftarrow{f} W \xrightarrow{g} Y$ with $g \in E$. The symmetric monoidal structure on $\operatorname{Corr}(\mathcal{C}, E)$ is the cartesian symmetric monoidal structure of $\mathcal{C}$.

    A \textbf{three-functor formalism} is a \tref{CT:monoidal-functor}{lax symmetric monoidal functor}
    \[ D \colon \operatorname{Corr}(\mathcal{C}, E) \to \textbf{Cat}_\infty . \]
    Such a functor encodes the three functors $\otimes, f^*$ and $f_!$ and their relations:
    \begin{itemize}
        \item On objects, $D$ defines an association $X \mapsto D(X)$.
        \item For any morphism $f \colon X \to Y$ in $\mathcal{C}$, the correspondence $Y \xleftarrow{f} X \xrightarrow{\id_X} X$ defines the pullback functor $f^* \colon D(Y) \to D(X)$.
        \item For any morphism $f \colon X \to Y$ in $E$, the correspondence $X \xleftarrow{\id_X} X \xrightarrow{f} Y$ defines the functor $f_! \colon D(X) \to D(Y)$.
        \item As $D$ is lax monoidal, there is a natural morphism $D(X) \otimes D(X) \to D(X \times X)$ which, when composed with pullback along the diagonal $\Delta_X^* \colon D(X \times X) \to D(X)$, defines the tensor product $\otimes$ on $D(X)$.
    \end{itemize}
    A \textbf{six-functor formalism} is a three-functor formalism for which the functors $(-) \otimes A, f^*$ and $f_!$ admit \tref{CT:adjunction}{right adjoints}.
\end{topic}

\begin{example}{six-functor-formalism}
    Let $f \colon X \to Y$ be a morphism in $E$. Let us prove the \textit{projection formula}:
    \[ f_! A \otimes B \cong f_! (A \otimes f^* B) \]
    in $D(Y)$ for any $A \in D(X)$ and $B \in D(Y)$.
    \begin{proof}
        The correspondence $X \times Y \xleftarrow{(\id_X, f)} X \xrightarrow{f} Y$ can be seen as a composition of correspondences in two different ways:
        \[ \svg \begin{tikzcd}[row sep=0.5em, column sep=1.5em] & & X \arrow[swap]{dl}{(\id_X, f)} \arrow{dr}{f} & & \\ & X \times Y \arrow[equals]{dl} \arrow{dr}{f \times \id_Y} & & Y \arrow[swap]{dl}{\Delta} \arrow[equals]{dr} & \\ X \times Y & & Y \times Y & & Y \end{tikzcd} \textup{ or } \begin{tikzcd}[row sep=0.5em, column sep=1.5em] & & X \arrow[equals]{dl} \arrow[equals]{dr} & & \\ & X \arrow[swap]{dl}{(\id_X, f)} \arrow[equals]{dr} & & X \arrow[equals]{dl} \arrow{dr}{f} & \\ X \times Y & & X & & Y \end{tikzcd} \]
        For any $A \in D(X)$ and $B \in D(Y)$, the first composition maps $(A, B) \mapsto (f_! A, B) \mapsto f_! A \otimes B$, while the latter maps $(A, B) \mapsto A \otimes f^* B \mapsto f_!(A \otimes f^* B)$.
    \end{proof}
\end{example}
