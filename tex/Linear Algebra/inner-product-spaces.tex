\begin{topic}{inner-product}{inner product}
    An \textbf{inner product} on a \tref{LA:vector-space}{vector space} $V$ over $k = \RR$ or $\CC$ is a map $\langle \cdot, \cdot \rangle \colon V \times V \to k$ satisfying
    \begin{itemize}
        \item (\textit{linearity}) $\langle \alpha x + \beta y, z \rangle = \alpha \langle x, z \rangle + \beta \langle y, z \rangle$ for all scalars $\alpha$ and $x, y, z \in V$,
        \item \textit{(conjugate symmetry)} $\langle x, y \rangle = \overline{\langle y, x \rangle}$ for all $x, y \in V$,
        \item (\textit{positive definiteness}) $\langle x, x \rangle > 0$ for all $x \ne 0$ in $V$.
    \end{itemize}
    A vector space together with an inner product is called an \textbf{inner product space}.
\end{topic}

\begin{topic}{cauchy-schwarz-inequality}{Cauchy--Schwarz inequality}
    The \textbf{Cauchy--Schwarz inequality} states that for any vectors $v, w$ in an \tref{inner-product}{inner-product-space},
    \[ \langle v, w \rangle^2 \le \langle v, v \rangle \cdot \langle w, w \rangle . \]
\end{topic}

\begin{topic}{polarization-identity}{polarization identity}
    Let $(V, \langle \cdot, \cdot \rangle)$ be an \tref{inner-product}{inner product space}. The polarization identity states that
    \[ \operatorname{Re} \langle x, y \rangle = \frac{1}{2} ( \norm{x + y}^2 - \norm{x}^2 - \norm{y}^2 ) , \]
    where $\norm{x}^2 = \langle x, x \rangle$.
\end{topic}

\begin{topic}{johnson-lindenstrauss-lemma}{Johnson--Lindenstrauss lemma}
    Consider a non-empty finite set $X \subset \RR^n$ for some $n \ge 0$. The \textbf{Johnson--Lindenstrauss lemma} states that for all $0 < \varepsilon < 1$ and integers $k \ge (8 \ln |X|) / \varepsilon^2$, there exists a \tref{linear-map}{linear map} $f \colon \RR^n \to \RR^k$ such that
    \[ (1 - \varepsilon) \norm{x - y}^2 \le \norm{f(x) - f(y)}^2 \le (1 + \varepsilon) \norm{x - y}^2 \]
    for all $x, y \in X$.
\end{topic}

\begin{example}{johnson-lindenstrauss-lemma}
    Consider the question of how many vectors $v_1, \ldots, v_n \in \RR^k$ could exist that are `nearly orthogonal', that is, such that
    \[ |\langle v_i, v_i \rangle - 1| \le \delta \quad \textup{ and } \quad |\langle v_i, v_j \rangle| \le \delta \]
    for some $0 < \delta < 1$, for all $i \ne j$.
    
    If $n$ were to be small enough such that $k \ge 8 \ln (n + 1) / \varepsilon^2$, we could apply the Johnson--Lindenstrauss lemma to $X = \{ 0, e_1, e_2, \ldots, e_n \} \subset \RR^n$ with $\varepsilon = \delta / 2$ to obtain a linear map $f \colon \RR^n \to \RR^k$. Defining $v_i = f(e_i)$, we would find that
    \[ (1 - \varepsilon) \le \norm{v_i}^2 \le (1 + \varepsilon) \quad \textup{ and hence } \quad |\langle v_i, v_i \rangle| \le \varepsilon < \delta \]
    for all $i$, and also, using the \tref{polarization-identity}{polarization identity}, that
    \[ |\langle v_i, v_j \rangle| = \left| \tfrac{1}{2} \left( \norm{v_i}^2 + \norm{v_j}^2 - \norm{v_i - v_j}^2 \right) \right| \le 2 \varepsilon = \delta \]
    for all $i \ne j$, since $\norm{v_i}^2, \norm{v_j}^2 \in [1 - \varepsilon, 1 + \varepsilon]$ and $\norm{v_i - v_j}^2 \in [2 - 2 \varepsilon, 2 + 2 \varepsilon]$. Therefore, this is possible for
    \[ n \le \exp(k \varepsilon^2 / 8) - 1 . \]
\end{example}
