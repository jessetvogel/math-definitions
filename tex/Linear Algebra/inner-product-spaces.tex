\begin{topic}{inner-product}{inner product}
    An \textbf{inner product} on a \tref{LA:vector-space}{vector space} $V$ over $k = \RR$ or $\CC$ is a map $\langle \cdot, \cdot \rangle \colon V \times V \to k$ satisfying
    \begin{itemize}
        \item (\textit{linearity}) $\langle \alpha x + \beta y, z \rangle = \alpha \langle x, z \rangle + \beta \langle y, z \rangle$ for all scalars $\alpha$ and $x, y, z \in V$,
        \item \textit{(conjugate symmetry)} $\langle x, y \rangle = \overline{\langle y, x \rangle}$ for all $x, y \in V$,
        \item (\textit{positive definiteness}) $\langle x, x \rangle > 0$ for all $x \ne 0$ in $V$.
    \end{itemize}
    A vector space together with an inner product is called an \textbf{inner product space}.
\end{topic}

\begin{topic}{cauchy-schwarz-inequality}{Cauchy--Schwarz inequality}
    The \textbf{Cauchy--Schwarz inequality} states that for any vectors $v, w$ in an \tref{inner-product}{inner-product-space},
    \[ \langle v, w \rangle^2 \le \langle v, v \rangle \cdot \langle w, w \rangle . \]
\end{topic}

\begin{topic}{polarization-identity}{polarization identity}
    Let $(V, \langle \cdot, \cdot \rangle)$ be an \tref{inner-product}{inner product space}. The polarization identity states that
    \[ \operatorname{Re} \langle x, y \rangle = \frac{1}{2} ( \norm{x + y} - \norm{x} - \norm{y} ) , \]
    where $\norm{x} = \langle x, x \rangle$.
\end{topic}
