\begin{topic}{symplectic-manifold}{symplectic manifold}
    Let $M$ be a \tref{smooth-manifold}{smooth manifold}. A \textbf{symplectic form} on $M$ is a \tref{closed-form}{closed} and non-degenerate \tref{differential-form}{$2$-form} $\omega$ on $M$. The pair $(M, \omega)$ is called a \textbf{symplectic manifold}.
\end{topic}

\begin{example}{symplectic-manifold}
    Let $M = \RR^{2n}$ with standard coordinates $q^1, \ldots, q^n, p_1, \ldots, p_n$. Then the $2$-form
    \[ \omega = \sum_{i = 1}^{n} d q^i \wedge d p_i , \]
    is a symplectic form, the \textit{standard symplectic form}.
\end{example}

\begin{topic}{symplectomorphism}{symplectomorphism}
    A \textbf{symplectomorphism} is a \tref{diffeomorphism}{diffeomorphism} $\varphi : M \to M'$ between \tref{symplectic-manifold}{symplectic manifolds} $(M, \omega)$ and $(M', \omega')$ satisfying $\varphi^* \omega' = \omega$.
\end{topic}

\begin{topic}{tautological-one-form}{tautological 1-form}
    Let $Q$ be a \tref{smooth-manifold}{smooth manifold}. The \textbf{tautological $1$-form} is the \tref{differential-form}{$1$-form} $\theta$ on the \tref{cotangent-bundle}{cotangent bundle} $T^*Q$ given by
    \[ \theta_x = p(d \pi_x) \quad \text{ for all } x = (q, p) \in T^*Q , \]
    where $\pi : T^*Q \to Q$ denotes the projection.
\end{topic}

\begin{example}{tautological-one-form}
    Take $Q = \RR^n$ so that $T^* Q \simeq \RR^{2n}$ with standard coordinates $q^1, p_1, \ldots, q^n, p_n$. Then the tautological $1$-form is given by
    \[ \theta = \sum_{i = 1}^{n} p_i dq^i . \]
\end{example}

\begin{topic}{canonical-symplectic-form}{canonical symplectic form}
    Let $Q$ be a \tref{smooth-manifold}{smooth manifold}. The \textbf{canonical symplectic form} is the \tref{symplectic-manifold}{symplectic form} $\omega$ on the \tref{cotangent-bundle}{cotangent bundle} $T^* Q$ given by
    \[ \omega = -d \theta , \]
    where $\theta$ is the \tref{tautological-one-form}{tautological $1$-form}.
\end{topic}

\begin{example}{canonical-symplectic-form}
    Take $Q = \RR^n$ so that $T^* Q \simeq \RR^{2n}$ with standard coordinates $q^1, p_1, \ldots, q^n, p_n$. Then the tautological $1$-form is given by
    \[ \theta = \sum_{i = 1}^{n} p_i dq^i , \]
    and thus the canonical symplectic form is given by
    \[ \omega = -d \theta = \sum_{i = 1}^{n} d q^i \wedge d p_i. \]
\end{example}

\begin{topic}{symplectic-vector-field}{symplectic vector field}
    Let $(M, \omega)$ be a \tref{symplectic-manifold}{symplectic manifold}. A \tref{vector-field}{vector field} $X$ on $M$ is called \textbf{symplectic} if the \tref{interior-product}{interior product}
    \[ \iota_X \omega = \omega(X, -) \]
    is \tref{closed-form}{closed}.
\end{topic}

\begin{example}{symplectic-vector-field}
    Every \tref{hamiltonian-vector-field}{Hamiltonian vector field} is symplectic, since every exact form is closed. The converse is however not true. Let $M$ be the $2n$-torus $\RR^{2n}/\ZZ^{2n}$ with the standard symplectic form
    \[ \omega = \sum_{i = 1}^{n} dp_i \wedge dq^i . \]
    Then the vector field $X$ on $M$ given by $X(x) = v$ for some constant non-zero $v \in \RR^{2n}$ (identifying $T_x M$ canonically with $\RR^{2n}$) is symplectic, but not Hamiltonian.
\end{example}

\begin{topic}{hamiltonian-vector-field}{Hamiltonian vector field}
    Let $(M, \omega)$ be a \tref{symplectic-manifold}{symplectic manifold}. Every smooth function $H : M \to \RR$ uniquely determines a \tref{vector-field}{vector field} $X_H$ satisfying
    \[ \omega(X_H, -) = dH . \]
    Such a vector field is called a \textbf{Hamiltonian vector field} with \textbf{Hamiltonian $H$}.
\end{topic}

\begin{example}{hamiltonian-vector-field}
    Let $M = \RR^{2n}$ with the standard symplectic form
    \[ \omega = \sum_{i = 1}^{n} dp_i \wedge dq^i . \]
    Then the Hamiltonian vector field with Hamiltonian $H$ has the form
    \[ X_H = \left(\frac{\partial H}{\partial p_i}, -\frac{\partial H}{\partial q^i} \right) . \]
\end{example}

\begin{topic}{symplectic-action}{symplectic action}
    Let $(M, \omega)$ be a \tref{symplectic-manifold}{symplectic manifold} and $G$ a \tref{lie-group}{Lie group} acting on $M$. The action is called \textbf{symplectic} if every element of $G$ acts by a \tref{symplectomorphism}{symplectomorphism}.
\end{topic}

\begin{topic}{fundamental-vector-field}{fundamental vector field}
    Let $M$ be a \tref{smooth-manifold}{smooth manifold}, $G$ a \tref{lie-group}{Lie group} acting on $M$ via $\varphi : G \times M \to M$. For any $\xi \in \mathfrak{g}$ in the Lie algebra of $G$, the \textbf{fundamental vector field} of $\xi$ is the \tref{vector-field}{vector field} on $M$ given by
    \[ X_\xi(p) = d(\varphi(-, p))(e) \xi \in T_p M . \]
    Alternatively, we have
    \[ X_\xi(p) = \frac{d}{d t}\Big|_{t = 0} \varphi(\exp(t \xi), p) . \]
\end{topic}

\begin{topic}{moment-map}{moment map}
    Let $(M, \omega)$ be a \tref{symplectic-manifold}{symplectic manifold} and $G$ a \tref{lie-group}{Lie group} with a \tref{symplectic-action}{symplectic action} on $M$. A \textbf{moment map} for the action is a smooth map $\mu : M \to \mathfrak{g}^*$ such that
    \[ d \langle \mu, \xi \rangle = \omega(X_\xi, -) \]
    for all $\xi \in \mathfrak{g}$, and such that $\mu$ is $G$-equivariant with respect to the \tref{adjoint-representation}{co-adjoint representation} of $G$, that is,
    \[ \mu(g \cdot p) = \text{Ad}_G^* (g) \mu(p) \]
    for all $p \in M$ and $g \in G$.
\end{topic}

\begin{topic}{hamiltonian-manifold}{Hamiltonian manifold}
    Let $(M, \omega)$ be a \tref{symplectic-manifold}{symplectic manifold} and $G$ a \tref{lie-group}{Lie group} with a \tref{symplectic-action}{symplectic action} $\varphi : G \times M \to M$ on $M$. The action is \textbf{Hamiltonian} if it admits a \tref{moment-map}{moment map}.
    
    The collection $(M, \omega, G, \varphi, \mu)$ is called a \textbf{Hamiltonian manifold}.
\end{topic}

\begin{topic}{marsden-weinstein-quotient}{Marsden--Weinstein quotient}
    Let $(M, \omega, G, \varphi, \mu)$ be a \tref{hamiltonian-manifold}{Hamiltonian manifold}, and assume $G$ acts \tref{GT:free-group-action}{freely} and \tref{GT:proper-group-action}{properly} on the zero-level set of the \tref{moment-map}{moment map} $\mu^{-1}(0) \subset M$. The \textbf{Marsden--Weinstein theorem} states that the orbit space
    \[ M \sslash G := \mu^{-1}(0) / G \]
    is a \tref{symplectic-manifold}{symplectic manifold} in a canonical way, called the \textbf{Marsden--Weinstein quotient} of $M$ by $G$. That is, there exists a unique symplectic form $\overline{\omega}$ on $M \sslash G$ which pulls back to the restriction of $\omega$ on $\mu^{-1}(0) / G$.
    
    Furthermore,
    \[ \dim(M \sslash G) = \dim M - 2 \dim G . \]
\end{topic}

\begin{topic}{poisson-manifold}{Poisson manifold}
    Let $M$ be a \tref{smooth-manifold}{smooth manifold}. A \textbf{Poisson structure} on $M$ is a bracket $\{ \cdot, \cdot \}$ on the algebra $C^\infty(M)$ of smooth functions making it into a \tref{AA:poisson-algebra}{Poisson algebra}. A \textbf{Poisson manifold} is a manifold with a Poisson structure.
\end{topic}

\begin{example}{poisson-manifold}
    Any \tref{symplectic-manifold}{symplectic manifold} $(M, \omega)$ gives rise to a Poisson manifold with bracket given by
    \[ \{ F, G \} = \omega(X_F, X_G) \]
    where $X_F$ is the \tref{hamiltonian-vector-field}{Hamiltonian vector field} of $F$.
    Indeed the Leibniz rule is satisfied as
    \[ \{ FG, H \} = d(FG) X_H = (F d G + G d F) X_H = F (d G X_H) + G (d F X_H) = F \{ G, H \} + G \{ F, H \} . \]
    Furthermore, the Jacobi identity can be shown from
    \[ \{ \{ F, G \}, H \} + \{ \{ G, H \}, F \} + \{ \{ H, F \}, G \} = -d \omega(X_F, X_G, X_H) \]
    and that $\omega$ is closed.
\end{example}
