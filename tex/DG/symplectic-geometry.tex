\begin{topic}{symplectic-manifold}{symplectic manifold}
    Let $M$ be a \tref{smooth-manifold}{smooth manifold}. A \textbf{symplectic form} on $M$ is a \tref{closed-form}{closed} and non-degenerate \tref{differential-form}{$2$-form} $\omega$ on $M$. The pair $(M, \omega)$ is called a \textbf{symplectic manifold}.
\end{topic}

\begin{example}{symplectic-manifold}
    Let $M = \RR^{2n}$ with standard coordinates $q^1, \ldots, q^n, p_1, \ldots, p_n$. Then the $2$-form
    \[ \omega = \sum_{i = 1}^{n} d q^i \wedge d p_i , \]
    is a symplectic form, the \textit{standard symplectic form}.
    
    In fact, \tref{darboux-theorem}{Darboux's theorem} states that any symplectic manifold locally has this form.
\end{example}

\begin{topic}{symplectomorphism}{symplectomorphism}
    A \textbf{symplectomorphism} is a \tref{diffeomorphism}{diffeomorphism} $\varphi : M \to M'$ between \tref{symplectic-manifold}{symplectic manifolds} $(M, \omega)$ and $(M', \omega')$ satisfying $\varphi^* \omega' = \omega$.
\end{topic}

\begin{topic}{tautological-one-form}{tautological 1-form}
    Let $Q$ be a \tref{smooth-manifold}{smooth manifold}. The \textbf{tautological $1$-form} is the \tref{differential-form}{$1$-form} $\theta$ on the \tref{cotangent-bundle}{cotangent bundle} $T^*Q$ given by
    \[ \theta_x = p(d \pi_x) \quad \text{ for all } x = (q, p) \in T^*Q , \]
    where $\pi : T^*Q \to Q$ denotes the projection.
\end{topic}

\begin{example}{tautological-one-form}
    Take $Q = \RR^n$ so that $T^* Q \simeq \RR^{2n}$ with standard coordinates $q^1, p_1, \ldots, q^n, p_n$. Then the tautological $1$-form is given by
    \[ \theta = \sum_{i = 1}^{n} p_i dq^i . \]
\end{example}

\begin{topic}{canonical-symplectic-form}{canonical symplectic form}
    Let $Q$ be a \tref{smooth-manifold}{smooth manifold}. The \textbf{canonical symplectic form} is the \tref{symplectic-manifold}{symplectic form} $\omega$ on the \tref{cotangent-bundle}{cotangent bundle} $T^* Q$ given by
    \[ \omega = -d \theta , \]
    where $\theta$ is the \tref{tautological-one-form}{tautological $1$-form}.
\end{topic}

\begin{example}{canonical-symplectic-form}
    Take $Q = \RR^n$ so that $T^* Q \simeq \RR^{2n}$ with standard coordinates $q^1, p_1, \ldots, q^n, p_n$. Then the tautological $1$-form is given by
    \[ \theta = \sum_{i = 1}^{n} p_i dq^i , \]
    and thus the canonical symplectic form is given by
    \[ \omega = -d \theta = \sum_{i = 1}^{n} d q^i \wedge d p_i. \]
\end{example}

\begin{topic}{symplectic-vector-field}{symplectic vector field}
    Let $(M, \omega)$ be a \tref{symplectic-manifold}{symplectic manifold}. A \tref{vector-field}{vector field} $X$ on $M$ is called \textbf{symplectic} if the \tref{interior-product}{interior product}
    \[ \iota_X \omega = \omega(X, -) \]
    is \tref{closed-form}{closed}.
\end{topic}

\begin{example}{symplectic-vector-field}
    Every \tref{hamiltonian-vector-field}{Hamiltonian vector field} is symplectic, since every exact form is closed. The converse is however not true. Let $M$ be the $2n$-torus $\RR^{2n}/\ZZ^{2n}$ with the standard symplectic form
    \[ \omega = \sum_{i = 1}^{n} dp_i \wedge dq^i . \]
    Then the vector field $X$ on $M$ given by $X(x) = v$ for some constant non-zero $v \in \RR^{2n}$ (identifying $T_x M$ canonically with $\RR^{2n}$) is symplectic, but not Hamiltonian.
\end{example}

\begin{topic}{hamiltonian-vector-field}{Hamiltonian vector field}
    Let $(M, \omega)$ be a \tref{symplectic-manifold}{symplectic manifold}. Every smooth function $H : M \to \RR$ uniquely determines a \tref{vector-field}{vector field} $X_H$ satisfying
    \[ \omega(X_H, -) = dH . \]
    Such a vector field is called a \textbf{Hamiltonian vector field} with \textbf{Hamiltonian $H$}.
\end{topic}

\begin{example}{hamiltonian-vector-field}
    Let $M = \RR^{2n}$ with the standard symplectic form
    \[ \omega = \sum_{i = 1}^{n} dp_i \wedge dq^i . \]
    Then the Hamiltonian vector field with Hamiltonian $H$ has the form
    \[ X_H = \left(\frac{\partial H}{\partial p_i}, -\frac{\partial H}{\partial q^i} \right) . \]
\end{example}

\begin{topic}{symplectic-action}{symplectic action}
    Let $(M, \omega)$ be a \tref{symplectic-manifold}{symplectic manifold} and $G$ a \tref{lie-group}{Lie group} acting on $M$. The action is called \textbf{symplectic} if every element of $G$ acts by a \tref{symplectomorphism}{symplectomorphism}.
\end{topic}

\begin{topic}{fundamental-vector-field}{fundamental vector field}
    Let $M$ be a \tref{smooth-manifold}{smooth manifold}, $G$ a \tref{lie-group}{Lie group} acting on $M$ via $\varphi : G \times M \to M$. For any $\xi \in \mathfrak{g}$ in the Lie algebra of $G$, the \textbf{fundamental vector field} of $\xi$ is the \tref{vector-field}{vector field} on $M$ given by
    \[ X_\xi(p) = d(\varphi(-, p))(e) \xi \in T_p M . \]
    Alternatively, we have
    \[ X_\xi(p) = \frac{d}{d t}\Big|_{t = 0} \varphi(\exp(t \xi), p) . \]
\end{topic}

\begin{topic}{moment-map}{moment map}
    Let $(M, \omega)$ be a \tref{symplectic-manifold}{symplectic manifold} and $G$ a \tref{lie-group}{Lie group} with a \tref{symplectic-action}{symplectic action} on $M$. A \textbf{moment map} for the action is a smooth map $\mu : M \to \mathfrak{g}^*$ such that
    \[ d \langle \mu, \xi \rangle = \omega(X_\xi, -) \]
    for all $\xi \in \mathfrak{g}$, and such that $\mu$ is $G$-equivariant with respect to the \tref{adjoint-representation}{co-adjoint representation} of $G$, that is,
    \[ \mu(g \cdot p) = \text{Ad}_G^* (g) \mu(p) \]
    for all $p \in M$ and $g \in G$.
\end{topic}

\begin{topic}{hamiltonian-manifold}{Hamiltonian manifold}
    Let $(M, \omega)$ be a \tref{symplectic-manifold}{symplectic manifold} and $G$ a \tref{lie-group}{Lie group} with a \tref{symplectic-action}{symplectic action} $\varphi : G \times M \to M$ on $M$. The action is \textbf{Hamiltonian} if it admits a \tref{moment-map}{moment map}.
    
    The collection $(M, \omega, G, \varphi, \mu)$ is called a \textbf{Hamiltonian manifold}.
\end{topic}

\begin{topic}{marsden-weinstein-quotient}{Marsden--Weinstein quotient}
    Let $(M, \omega, G, \varphi, \mu)$ be a \tref{hamiltonian-manifold}{Hamiltonian manifold}, and assume $G$ acts \tref{GT:free-group-action}{freely} and \tref{GT:proper-group-action}{properly} on the zero-level set of the \tref{moment-map}{moment map} $\mu^{-1}(0) \subset M$. The \textbf{Marsden--Weinstein theorem} states that the orbit space
    \[ M \sslash G := \mu^{-1}(0) / G \]
    is a \tref{symplectic-manifold}{symplectic manifold} in a canonical way, called the \textbf{Marsden--Weinstein quotient} of $M$ by $G$. That is, there exists a unique symplectic form $\overline{\omega}$ on $M \sslash G$ which pulls back to the restriction of $\omega$ on $\mu^{-1}(0) / G$.
    
    Furthermore,
    \[ \dim(M \sslash G) = \dim M - 2 \dim G . \]
\end{topic}

\begin{topic}{poisson-manifold}{Poisson manifold}
    Let $M$ be a \tref{smooth-manifold}{smooth manifold}. A \textbf{Poisson structure} on $M$ is a bracket $\{ \cdot, \cdot \}$ on the algebra $C^\infty(M)$ of smooth functions making it into a \tref{AA:poisson-algebra}{Poisson algebra}. A \textbf{Poisson manifold} is a manifold with a Poisson structure.
\end{topic}

\begin{example}{poisson-manifold}
    Any \tref{symplectic-manifold}{symplectic manifold} $(M, \omega)$ gives rise to a Poisson manifold with bracket given by
    \[ \{ F, G \} = \omega(X_F, X_G) \]
    where $X_F$ is the \tref{hamiltonian-vector-field}{Hamiltonian vector field} of $F$.
    Indeed the Leibniz rule is satisfied as
    \[ \{ FG, H \} = d(FG) X_H = (F d G + G d F) X_H = F (d G X_H) + G (d F X_H) = F \{ G, H \} + G \{ F, H \} . \]
    Furthermore, the Jacobi identity can be shown from
    \[ \{ \{ F, G \}, H \} + \{ \{ G, H \}, F \} + \{ \{ H, F \}, G \} = -d \omega(X_F, X_G, X_H) \]
    and that $\omega$ is closed.
\end{example}

\begin{topic}{poisson-map}{Poisson map}
    A \textbf{Poisson map} is a \tref{smooth-map}{smooth map} $\phi : M \to N$ between \tref{poisson-manifold}{Poisson manifolds} such that the pullback
    \[ \phi^* : C^\infty(N) \to C^\infty(M), \quad f \mapsto f \circ \phi \]
    is a morphism of Poisson algebras.
\end{topic}

\begin{topic}{darboux-theorem}{Darboux's theorem}
    Let $(M, \omega)$ be a \tref{symplectic-manifold}{symplectic manifold}. \textbf{Darboux's theorem} states that around any point $x \in M$, there are local coordinates $q^1, \ldots, q^n, p_1, \ldots, p_n$ such that
    \[ \omega = dq^i \wedge dp_i . \]
\end{topic}

\begin{example}{darboux-theorem}
    \textbf{Proof}. Locally around $x$, we can write $\omega = \frac{1}{2} \omega_{ij} dx^i \wedge dx^j$. Consider $\omega_0 = \frac{1}{2} \omega_{ij}(x) dx^i \wedge dx^1$, which is also a symplectic form in the same neighborhood of $x$. Moreover, for a sufficiently small neighborhood around $x$, we have a family of symplectic forms
    \[ \omega(t) = (1 - t) \omega_0 + t \omega , \quad t \in [0, 1] . \]
    Since $\omega - \omega_0$ is closed, using \tref{poincare-lemma}{Poincaré's lemma} we can locally around $x$ write $\dot{\omega}(t) = \omega - \omega_0 = - d \lambda$ for some $1$-form $\lambda$. Since $\omega - \omega_0$ vanishes at $x$, we may choose $\lambda$ to vanish at $x$ up to second order. Now let $X(t)$ be the vector field defined by $\iota_{X(t)} \omega(t) = \lambda$, and let $\varphi_t$ be the flow of $X(t)$ around $x_0$. Since $\varphi_t(x) = x$ for all $t \in [0, 1]$, the flow $\varphi_t$ exists on the whole interval $t \in [0, 1]$ sufficiently close to $x$. Now,
    \[ \frac{d}{dt} \varphi_t^* \omega(t) = \varphi_t^* \left(\frac{\partial \omega(t)}{\partial t} + \mathcal{L}_{X(t)} \omega(t) \right) = \varphi_t^* \left(\frac{\partial \omega(t)}{\partial t} + d(\iota_{X(t)} \omega(t)) \right) = \varphi_t^* \left( -d \lambda + d \lambda \right) = 0 , \]
    using the \tref{cartan-formula}{Cartan formula}. This implies that $\varphi_1^* \omega = \varphi_1^* \omega(1) = \varphi_0^* \omega(0) = \omega_0$, so $\varphi_1$ is the desired diffeomorphism. Finally, by a linear change of variables the form $\omega_0$ can be reduced to the canonical form.
\end{example}

\begin{topic}{moser-stability}{Moser's stability}
    \textbf{Moser's stability} states that for a \tref{closed-manifold}{closed} \tref{smooth-manifold}{smooth manifold} $M$ and a family of \tref{symplectic-manifold}{symplectic forms} $(\omega_t)_{t \in [0, 1]}$ on $M$ such that the \tref{de-rham-complex}{cohomology classes} $[\omega_t] \in H^2_{\textup{dR}}(M)$ are constant, there exists an isotopy $\varphi : M \times [0, 1] \to M$ such that $\varphi_t^* \omega_0 = \omega_t$ for all $t \in [0, 1]$.
\end{topic}

\begin{example}{moser-stability}
    The proof is known as \textbf{Moser's trick}. Suppose there exists such an isotopy $\varphi$, and define a vector field $X_t = \frac{d \varphi}{d t} \circ \varphi_t^{-1}$. Then by \tref{cartan-formula}{Cartan's formula} and properties of the \tref{lie-derivative}{Lie derivative},
    \[ 0 = \frac{d}{dt} \left(\varphi_t^* \omega_t\right) = \varphi_t^* \left(\mathcal{L}_{X_t} \omega_t + \frac{d}{dt} \omega_t \right) = \varphi_t^* \left(d \iota_{X_t} \omega_t + \frac{d}{dt} \omega_t \right) \]
    so we obtain
    \[ d \iota_{X_t} \omega_t + \frac{d \omega_t}{dt} = d(\iota_{X_t} \omega_t + \alpha_t) = 0 , \]
    where $\frac{d \omega_t}{d t} = d \alpha_t$ for a family of 1-forms $\alpha_t$ since $[\omega_t]$ is constant. Now since $\omega_t$ is non-degenerate, we can integrate $\iota_{X_t} \omega_t + \alpha_t$ to obtain $X_t$, and let $\varphi_t$ be the flow of $X_t$.
\end{example}

\begin{topic}{casimir-function}{Casimir function}
    Let $(P, \{ \cdot, \cdot \})$ be a \tref{poisson-manifold}{Poisson manifold}. A function $f \in C^\infty(P)$ is called \textbf{Casimir} if the \tref{hamiltonian-vector-field}{Hamiltonian vector field} $X_f = \{ f, - \}$ is zero.
\end{topic}

\begin{topic}{symplectic-connection}{symplectic connection}
    Let $(M, \omega)$ be a \tref{symplectic-manifold}{symplectic manifold}. A \textbf{symplectic connection} on $M$ is a \tref{affine-connection}{affine connection} $\nabla$ which is \tref{torsion-connection}{torsion-free} and preserves the symplectic form, that is
    \[ (\nabla_Z \omega)(X, Y) = d(w(X, Y)) - \omega(\nabla_Z X, Y) - \omega(X, \nabla_Z Y) = 0 \]
    for any \tref{vector-field}{vector fields} $X, Y$ and $Z$.
\end{topic}

\begin{topic}{fubini-study-form}{Fubini--Study form}
    The \textbf{Fubini--study form} on $\CC \PP^n$ is the unique $2$-form $\omega_\textup{FS}$ on $\CC \PP^n$ such that
    \[ \pi^* \omega_\textup{FS} = \frac{i}{2} \partial \bar{\partial} \log (|\cdot|^2) , \]
    where $\pi : \CC^n \setminus \{ 0 \} \to \CC \PP^n$ is the projection map, $\partial$ and $\bar{\partial}$ are the \tref{dolbeault-operators}{Dolbeault operators}, and $|\cdot| : \CC^n \to \RR$ is the Euclidean norm. It is a \tref{symplectic-manifold}{symplectic form}.
\end{topic}

\begin{topic}{compatible-triple}{compatible triple}
    Let $M$ be a \tref{smooth-manifold}{smooth manifold}. A \textbf{compatible triple} on $M$ is a triple $(\omega, J, g)$, where $\omega$ is a \tref{symplectic-manifold}{symplectic form} on $M$, $J$ is an \tref{complex-manifold}{almost complex structure} on $M$, and $g$ is a \tref{riemannian-manifold}{Riemannian metric} on $M$, such that
    \[ g(X, Y) = \omega(X, J(Y)) \]
    for all \tref{vector-field}{vector fields} $X$ and $Y$ on $M$.
    % Given two out of the three structures of a compatible triple, the third structure can be obtained.
\end{topic}

\begin{example}{compatible-triple}
    Let $M = \CC^n$ with the standard Hermitian metric $h(-, -)$. Decomposing in real and imaginary parts, we have
    \[ h(u, v) = g(u, v) + i \omega(u, v) \]
    for some bilinear forms $g$ and $\omega$. Since $h(v, u) = \overline{h(u, v)}$, it follows that $g$ is symmetric and $\omega$ is skew-symmetric. Moreover, $g$ and $\omega$ are non-degenerate since $h$ is. In particular, $g$ is a Riemannian metric, and $\omega$ a symplectic form. Furthermore, multiplication by $i$ defines an almost complex structure $J$. Note that
    \[ g(u, Jv) + i \omega(u, Jv) = h(u, iv) = i h(u, v) = i g(u, v) - \omega(u, v) , \]
    from which follows that
    \[ g(u, v) = \omega(u, Jv) , \]
    that is, $(\omega, J, g)$ is a compatible triple.
\end{example}

\begin{topic}{weyl-bundle}{Weyl bundle}
    Let $(M, \omega)$ be a \tref{symplectic-manifold}{symplectic manifold}. The \textbf{Weyl bundle} on $M$ is the \tref{vector-bundle}{vector bundle}
    \[ W = \widehat{\textup{Sym}}(T^*M) , \]
    that is, the \tref{CA:completion}{completed} \tref{CA:symmetric-algebra}{symmetric algebra} of the \tref{cotangent-bundle}{cotangent bundle}.
    
    The \textbf{formal Weyl algebra} is the vector bundle
    \[ W_\hbar = \widehat{\textup{Sym}}(T^*M) \llbracket \hbar \rrbracket , \]
    where $\hbar$ is a formal parameter.
\end{topic}

\begin{topic}{moyal-weyl-product}{Moyal--Weyl product}
    Let $(M, \omega)$ be a \tref{symplectic-manifold}{symplectic manifold}. The \textbf{Moyal--Weyl product} is a non-commutative product on the sections of the \tref{weyl-bundle}{formal Weyl bundle} $W_\hbar$. In local coordinates, if $y^1, \ldots, y^{2n}$ is a basis for $T^*M$, the Moyal--Weyl product is given by
    \[ \begin{aligned}
        a \circ b
            &= \left. \exp \left(-\frac{i\hbar}{2} \omega^{ij} \frac{\partial}{\partial y^i} \frac{\partial}{\partial z^j}\right) a(y) b(z) \right|_{z = y} \\
            &= \sum_{k = 0}^{\infty} \left(-\frac{i\hbar}{2}\right)^k \frac{1}{k!} \omega^{i_1 j_1} \cdots \omega^{i_k j_k} \frac{\partial^k a}{\partial y^{i_1} \cdots \partial y^{i_k}} \frac{\partial^k b}{\partial y^{j_1} \cdots \partial y^{j_k}} ,
    \end{aligned} \]
    for sections $a, b \in \Gamma(W_\hbar)$.
\end{topic}

\begin{example}{moyal-weyl-product}
    The center of $\Gamma(W_\hbar)$ with respect to the Weyl product is $C^\infty(M)\llbracket \hbar \rrbracket$. Namely, if $a$ is in the center of $\Gamma(W_\hbar)$ and $b = y^k$ for some $k$, then
    \[ a \circ b = a y^k - \frac{i\hbar}{2} \omega^{ik} \frac{\partial a}{\partial y^i} \qquad \text{ and } \qquad b \circ a = a y^k - \frac{i\hbar}{2} \omega^{kj} \frac{\partial a}{\partial y^j} , \]
    so
    \[ 0 = a \circ b - b \circ a = - i \hbar \omega^{ik} \frac{\partial a}{\partial y^i} . \]
    Varying $k$, we find that $\frac{\partial a}{\partial y^i} = 0$ for all $i$, so $a \in C^\infty(M)\llbracket \hbar \rrbracket$. The converse inclusion is straightforward.
\end{example}

\begin{topic}{fedosov-manifold}{Fedosov manifold}
    A \textbf{Fedosov manifold} is a triple $(M, \omega, \nabla)$, where $(M, \omega)$ is a \tref{symplectic-manifold}{symplectic manifold} and $\nabla$ is a \tref{torsion-connection}{torsion-free} \tref{symplectic-connection}{symplectic connection} on $M$.
\end{topic}

\begin{example}{fedosov-manifold}
    Any symplectic manifold $(M, \omega)$ admits a torsion-free symplectic connection. Namely, by \tref{darboux-theorem}{Darboux's theorem}, there exists a covering of $M$ by \textit{Darboux charts}, which are open subsets of the standard symplectic space $\RR^{2n}$ with coordinates $q^1, \ldots, q^n, p_1, \ldots, p_n$ and symplectic form $\omega = \sum_{i = 1}^{n} dq^i \wedge dp_i$. On such a standard symplectic space, the \tref{exterior-derivative}{exterior derivative} $d$ is a torsion-free symplectic connection, and gluing these using a \tref{partition-of-unity}{partition of unity}, we obtain a global torsion-free symplectic connection on $M$.
\end{example}
