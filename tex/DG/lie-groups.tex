\begin{topic}{lie-group}{Lie group}
    A \textbf{Lie group} is a \tref{GT:group}{group} $G$ which is also a finite-dimensional \tref{smooth-manifold}{smooth manifold}, such that the multiplication map $G \times G \to G : (x, y) \mapsto xy$ and the inversion map $G \to G : x \mapsto x^{-1}$ are \tref{smooth-map}{smooth}.
\end{topic}

\begin{topic}{adjoint-representation}{(co)adjoint representation}
    The \textbf{adjoint representation} of a \tref{lie-group}{Lie group} $G$ is a \tref{RT:representation}{representation} onto its Lie algebra $\mathfrak{g} = T_e G$ given by
    \[ \text{Ad}_G : G \to \text{GL}(\mathfrak{g}), \quad g \mapsto \text{Ad}_g := d c_g(e) , \]
    where $c_g : G \to G, h \mapsto g h g^{-1}$ denotes the \tref{GT:conjugation}{conjugation} map.
    
    The \textbf{co-adjoint representation} $\text{Ad}_G^* : G \to \text{GL}(\mathfrak{g}^*)$ is the \tref{RT:dual-representation}{dual representation} of $\text{Ad}_G$.
    
    The \textbf{adjoint representation} of a \tref{AA:lie-algebra}{Lie algebra} $\mathfrak{g}$ is a representation onto itself, given by
    \[ \text{ad}_{\mathfrak{g}}: \mathfrak{g} \to \mathfrak{gl}(\mathfrak{g}), \quad X \mapsto [X, -] . \]
\end{topic}

\begin{topic}{killing-form}{Killing form}
    Let $\mathfrak{g}$ be a \tref{AA:lie-algebra}{Lie algebra} of finite dimension over a field $k$. The \textbf{Killing form} of $\mathfrak{g}$ is the symmetric bilinear form
    \[ B : \mathfrak{g} \times \mathfrak{g} \to k, \quad B(x, y) = \textup{tr}(\textup{ad}_{\mathfrak{g}}(x) \circ \textup{ad}_{\mathfrak{g}}(y)) \]
    where $\text{ad}_\mathfrak{g} : \mathfrak{g} \to \mathfrak{gl}(\mathfrak{g})$ is the \tref{adjoint-representation}{adjoint representation} of $\mathfrak{g}$.
\end{topic}

\begin{topic}{coadjoint-orbit}{coadjoint orbit}
    Let $G$ be a \tref{lie-group}{Lie group} and $\mathfrak{g}$ its corresponding \tref{AA:lie-algebra}{Lie algebra}. The \textbf{coadjoint orbit} of some $\xi \in \mathfrak{g}^*$ is the \tref{GT:orbit}{orbit} of the \tref{adjoint-representation}{coadjoint representation} $\textup{Ad}^*_G : G \to \textup{GL}(\mathfrak{g}^*)$, that is
    \[ \mathcal{O}_\xi = \{ \textup{Ad}_g^*(\xi) : g \in G \} . \]
    Equivalently, it can be described as the quotient $G/G_\xi$, where $G_\xi$ is the \tref{GT:stabilizer}{stabilizer} of $\xi$, under the isomorphism
    \[ G/G_\xi \xrightarrow{\sim} \mathcal{O}_\xi, \quad g \mapsto \textup{Ad}_g^*(\mu) . \]
\end{topic}

\begin{topic}{kirillov-form}{Kirillov form}
    Let $G$ be a \tref{lie-group}{Lie group}, $\mathfrak{g}$ its corresponding \tref{AA:lie-algebra}{Lie algebra}, and $\mathcal{O}_\xi$ a \tref{coadjoint-orbit}{coadjoint orbit} for some $\xi \in \mathfrak{g}^*$. The \textbf{Kirillov form} on $\mathcal{O}_\xi$ is the \tref{symplectic-manifold}{symplectic form} $\omega \in \Omega^2(\mathcal{O}_\xi)$ given by
    \[ \omega_\eta(\textup{ad}_X^* \eta, \textup{ad}_Y^* \eta) = \eta([X, Y]) \textup{ for all } \eta \in \mathcal{O}_\mu \textup{ and } X, Y \in \mathfrak{g} , \]
    where the tangent space of $\mathcal{O}_\xi$ at $\eta$ is identified as
    \[ \mathfrak{g}/\mathfrak{g}_\eta \xrightarrow{\sim} T_\eta \mathcal{O}_\xi, \quad X \mapsto \textup{ad}_X^* \eta . \]
    Note that $\textup{ad}_X^* \eta = \eta([X, -]) = 0$ if and only if $X \in \mathfrak{g}_\eta$, which makes $\omega$ well-defined and non-degenerate. Also $\omega$ is \tref{closed-form}{closed}.
\end{topic}

\begin{topic}{maurer-cartan-form}{Maurer--Cartan form}
    Let $G$ be a \tref{lie-group}{Lie group} and $\mathfrak{g}$ its corresponding \tref{AA:lie-algebra}{Lie algebra}. The \textbf{Maurer--Cartan form} $\Theta \in \Omega^1(G; \mathfrak{g})$ is the $\mathfrak{g}$-valued $1$-form on $G$ given by
    \[ \Theta_g(\xi) = (L_{g^{-1}})_* \xi \quad \textup{ for } g \in G \textup{ and } \xi \in T_g G, \]
    where $L_g : G \to G$ denotes left multiplication by $g$, that is $L_g(h) = gh$.
\end{topic}

\begin{topic}{lie-theorems}{Lie's theorems}
    \textbf{Lie's theorems} are three theorems about the relation between \tref{lie-group}{Lie groups} and finite-dimensional \tref{AA:lie-algebra}{Lie algebras}.
    \begin{enumerate}[I]
        \item If $G$ and $H$ are locally isomorphic Lie groups, then their Lie algebras $\mathfrak{g}$ and $\mathfrak{h}$ are isomorphic.
        \item If $G$ and $H$ are Lie groups with Lie algebras $\mathfrak{g}$ and $\mathfrak{h}$, and $G$ is \tref{TO:simply-connected-space}{simply connected}, then any morphism of Lie algebras $f : \mathfrak{g} \to \mathfrak{h}$ lifts uniquely to a morphism of Lie groups $F : G \to H$, that is, $dF_e = f$.
        \item Every Lie algebra is isomorphic to the Lie algebra of a (simply connected) Lie group.
    \end{enumerate}
\end{topic}

\begin{topic}{lie-algebra-cohomology}{Lie algebra cohomology}
    Let $G$ be a \tref{lie-group}{Lie group} and $\mathfrak{g}$ its corresponding \tref{AA:lie-algebra}{Lie algebra}. The \textit{left-invariant differential forms}, i.e. \tref{differential-form}{differential forms} $\omega \in \Omega^k(G)$ with $\omega_g = (L_{g^{-1}})^* \omega_e$ for all $g \in G$, where $L_g : G \to G$ denotes left multiplication $h \mapsto gh$, are completely determined by their value at the identity $e$. Hence there is a natural identification
    \[ \wedge^\bdot \mathfrak{g}^* \xrightarrow{\sim} \Omega^\bdot_\textup{inv}(G), \]
    where $(\Omega^\bdot_\textup{inv}(G), d) \subset (\Omega^\bdot(G), d)$ is the subcomplex of left-invariant differential forms. In this way, the \tref{exterior-derivative}{exterior} corresponds to the differential
    \[ d_\mathfrak{g} : \wedge^k \mathfrak{g}^* \to \wedge^{k + 1} \mathfrak{g}^*, \quad d_\mathfrak{g} \omega(v_0, \ldots, v_k) = \sum_{0 \le i < j \le k} (-1)^{i + j} \omega([v_i, v_j], v_0, \ldots, \hat{v}_i, \ldots, v_k) . \]
    The complex $(\wedge^\bdot \mathfrak{g}^*, d_\mathfrak{g})$ is called the \textbf{Chevalley--Eilenberg complex}, and the cohomology groups $H^k(\mathfrak{g})$ are the \textbf{Lie algebra cohomology groups} of $\mathfrak{g}$.
    
    If $\rho : \mathfrak{g} \to \mathfrak{gl}(V)$ is a Lie algebra representation of $\mathfrak{g}$, then the \textbf{Lie algebra cohomology of $\mathfrak{g}$ with coefficients in $V$}, denoted $H^k(\mathfrak{g}, V)$, is the cohomology of the complex $(\wedge^\bdot \mathfrak{g}^* \otimes V, d_\mathfrak{g})$ where
    \[ \begin{aligned}
        d_\mathfrak{g} \omega(v_0, \ldots, v_k) &= \sum_{i = 0}^{k} (-1)^i \rho(v_i) \cdot \omega(v_0, \ldots, \hat{v}_i, \ldots, v_k) \\ &+ \sum_{0 \le i < j \le k} (-1)^{i + j} \omega([v_i, v_j], v_0, \ldots, \hat{v}_i, \ldots, v_k) .
    \end{aligned} \]
\end{topic}
