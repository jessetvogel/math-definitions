\begin{topic}{module}{module}
    Let $R$ be a \tref{ring}{ring}. A \textbf{left $R$-module} is an \tref{GT:abelian-group}{abelian group} $M$ with a left action of $R$, that is, a map
    \[ R \times M \to M, \qquad (r, m) \mapsto r \cdot m, \]
    satisfying
    \begin{itemize}
        \item $r \cdot (m + m') = r \cdot m + r \cdot m'$,
        \item $(r + s) \cdot m = r \cdot m + s \cdot m$,
        \item $r \cdot (s \cdot m) = (rs) \cdot m$,
        \item $1 \cdot m = m$,
    \end{itemize}
    for all $r, s \in R$ and $m, m' \in M$.
    
    Similarly, a \textbf{right $R$-module} is an abelian group $M$ with a right action of $R$,
    \[ M \times R \to M, \qquad (m, r) \mapsto m \cdot r , \]
    satisfying the above conditions in reverse.
    
    When $R$ is commutative, left and right $R$-modules coincide, and are simply called \textbf{$R$-modules}.
\end{topic}

\begin{topic}{free-module}{free module}
    Let $R$ be a \tref{ring}{commutative ring}. An \tref{module}{$R$-module} $M$ is \textbf{free} if it is isomorphic to
    \[ \bigoplus_{i \in I} R , \]
    for some indexing set $I$.
\end{topic}

\begin{topic}{cyclic-module}{cyclic module}
    Let $R$ be a \tref{ring}{commutative ring}. An \tref{module}{$R$-module} $M$ is \textbf{cyclic} if it can be generated by one element, that is $M = Rm$ for some $m \in M$.
\end{topic}

\begin{topic}{projective-module}{projective module}
    Let $R$ be a \tref{ring}{commutative ring}. An \tref{module}{$R$-module} $P$ is called \textbf{projective} if for every morphism $g \colon P \to M$ and surjective morphism $f \colon N \to M$ of $R$-modules, there exists a morphism $h \colon P \to N$ of $R$-modules such that $fh = g$. We do not require this map to be unique.
    \[ \svg \begin{tikzcd} & N \arrow[twoheadrightarrow]{d}{f} \\ P \arrow[swap]{r}{g} \arrow[dashed]{ur}{\exists h} & M \end{tikzcd} \]
\end{topic}

\begin{example}{projective-module}
    \tref{free-module}{Free modules} are projective.
    
    More generally, a \tref{finitely-generated-module}{finitely generated} $R$-module $P$ is projective if and only if $P$ is finitely presented and the localization $P_\mathfrak{m}$ is a free $R_\mathfrak{m}$-module for all \tref{maximal-ideal}{maximal ideals} $\mathfrak{m}$ of $R$.
\end{example}

\begin{example}{projective-module}
    An $R$-module $P$ is projective if and only if there exists another $R$-module $Q$ such that $P \oplus Q$ is free. Namely, if such $Q$ exists, then for any $f \colon N \to M$ surjective and $g \colon P \to N$, we can extend $g$ to $\tilde{g} \colon F \to N$ with $\tilde{g}(p, q) = g(p)$. Since free modules are projective, there exists $\tilde{h} \colon F \to M$ with $f \tilde{h} = \tilde{g}$, so restricting $\tilde{h}$ to $P$ gives $h \colon P \to M$ with $fh = g$.
    
    Conversely, if $P$ is projective, consider the free module $F = \oplus_{p \in P} R$ and the surjective $R$-module morphism
    \[ f \colon F \to P, \quad (r_p)_{p \in P} \mapsto \sum_{p \in P} r_p \cdot p . \]
    Since $P$ is projective, $f$ splits, i.e. there exists $g \colon P \to F$ with $fg = \id_P$. Then it follows that $F \isom P \oplus \ker f$. Note that if $P$ is finitely generated, $F$ can be taken of finite rank.
\end{example}

\begin{topic}{injective-module}{injective module}
    Let $R$ be a \tref{ring}{commutative ring}. An \tref{module}{$R$-module} $I$ is called \textbf{injective} if for every morphism $g \colon M \to I$ and injective morphism $f \colon M \to N$ of $R$-modules, there exists a morphism $h \colon N \to Q$ of $R$-modules such that $hf = g$. We do not require this map to be unique.
    \[ \svg \begin{tikzcd} M \arrow[hookrightarrow]{r}{f} \arrow[swap]{d}{g} & N \arrow[dashed]{ld}{\exists h} \\ I & \end{tikzcd} \]
\end{topic}

\begin{topic}{flat-module}{flat module}
    Let $R$ be a \tref{ring}{commutative ring}. An \tref{module}{$R$-module} $M$ is \textbf{flat} if $(-) \otimes_R M$ is exact. Since the tensor product is already right-exact, this is equivalent to saying $(-) \otimes_R M$ sends injective morphisms to injective morphisms.
\end{topic}

\begin{example}{flat-module}
    \begin{itemize}
        \item Any \tref{free-module}{free module} $\bigoplus_{i \in I} R$ is flat over $R$, because if $M \to N$ is injective, then so is $\bigoplus_{i \in I} M \to \bigoplus_{i \in I} N$.
        \item In particular, $k$-vector spaces are flat over $k$.
        \item More generally, since direct summands of flat modules are again flat, it follows that \tref{projective-module}{projective modules} are flat, since a projective module is a direct summand of a free module.
        \item Using \tref{HA:tor-functors}{Tor functors}, an $R$-module $M$ is flat if and only if $\textup{Tor}_1(M, N) = 0$ for all $R$-modules $N$. Hence, any extension of flat modules is exact. Namely, if $0 \to M' \to M \to M'' \to 0$ is a short exact sequence of modules with $M'$ and $M''$ flat, then from the long exact sequence
        \[ \cdots \to \textup{Tor}_1(M', N) \to \textup{Tor}_1(M, N) \to \textup{Tor}_1(M'', N) \to \cdots \]
        follows that $M$ is flat as well.
        \item For any multiplicative subset $S \subset R$, the \tref{localization}{localization} $S^{-1} R$ is a flat $R$-module.
    \end{itemize}
\end{example}

\begin{example}{flat-module}
    Take $R = \ZZ$ and $M = \ZZ/2\ZZ$. Then $M$ is not flat, because tensoring
    \[ 0 \rightarrow \ZZ \xrightarrow{\cdot 2} \ZZ \rightarrow \ZZ/2\ZZ \to 0 \]
    with $\ZZ/2\ZZ$ gives
    \[ 0 \rightarrow \ZZ/2\ZZ \xrightarrow{0} \ZZ/2\ZZ \xrightarrow{\id} \ZZ/2\ZZ \to 0 , \]
    which is not exact on the left.
\end{example}

\begin{example}{flat-module}
    Let $f \colon A \to B$ be a \tref{ring-morphism}{morphism of rings}, and suppose that $x \in A$ is not a \tref{zero-divisor}{zero-divisor} while $f(x) \in B$ is. Then $B$ cannot be flat as an $A$-module, since multiplication by $x$ is an injective map $A \to A$, while multiplication by $f(x)$ is not an injective map $B \to B$.
\end{example}

\begin{topic}{faithful-module}{faithful module}
     Let $R$ be a \tref{ring}{commutative ring}. An \tref{module}{$R$-module} $M$ is \textbf{faithful} if the \tref{annihilator}{annihilator} $\operatorname{Ann}_R(M) = 0$.
\end{topic}

\begin{topic}{finitely-generated-module}{finitely generated module}
    Let $R$ be a \tref{ring}{commutative ring}. An \tref{module}{$R$-module} $M$ is \textbf{finitely generated} if there exists a finite set $m_1, m_2, \ldots, m_n \in M$ (\textit{generators}) such that
    \[ M = R m_1 + R m_2 + \cdots + R m_n . \]
\end{topic}

\begin{topic}{nakayamas-lemma}{Nakayama's lemma}
    \textbf{Nakayama's lemma} states that if $M$ is a \tref{finitely-generated-module}{finitely generated module} over $R$, and $I \subset R$ an \tref{ideal}{ideal} such that $IM = M$, then there exists an $r \in R$ with $r \equiv 1 \mod I$ such that $rM = 0$.
    
    The following corollary is also known as Nakayama's lemma: if $M$ is a finitely generated module over $R$, and $I \subset R$ an ideal contained in the \tref{jacobson-radical}{Jacobson radical} $\mathfrak{J}_R$ of $R$ such that $IM = M$, then $M = 0$.
    
    (This follows from the above by observing that $r \equiv 1 \mod \mathfrak{J}_R$ must be a unit, so $M = r^{-1} r M = 0$.)
\end{topic}

\begin{topic}{faithfully-flat-module}{faithfully flat module}
    Let $R$ be a \tref{ring}{commutative ring}. An \tref{module}{$R$-module} $M$ is \textbf{faithfully flat} if for all sequences
    \[ 0 \to A \to B \to C \to 0 \]
    of $R$-modules, the sequence is exact if and only if
    \[ 0 \to A \otimes_R M \to B \otimes_R M \to C \otimes_R M \to 0 \]
    is exact.
\end{topic}

\begin{topic}{simple-module}{(semi)simple module}
    Let $R$ be a \tref{ring}{ring}. A \tref{module}{$R$-module} $M$ is \textbf{simple} if it is non-zero and has no proper submodules. It is \textbf{semisimple} if it is isomorphic to the direct sum of simple modules.
\end{topic}

\begin{topic}{graded-module}{graded module}
    A \textbf{graded module} $M$ over a \tref{graded-ring}{graded ring} $R$ is an \tref{module}{$R$-module}
    \[ M = \bigoplus_{i \ge 0} M_i \]
    such that $R_i M_j \subset M_{i + j}$ for all $i, j \ge 0$.
\end{topic}

\begin{topic}{length-module}{length of a module}
    The \textbf{length} of a \tref{module}{module} $M$ over a \tref{ring}{ring} $R$ is the maximum length $n$ of chains of submodules
    \[ M_0 \subsetneq M_1 \subsetneq \cdots \subsetneq M_n = M . \]
    It may be infinite.
\end{topic}

\begin{example}{length-module}
    Given a short exact sequence of $R$-modules
    \[ 0 \to L \xrightarrow{f} M \xrightarrow{g} N \to 0 , \]
    we have $\textup{length}_R(M) = \textup{length}_R(L) + \textup{length}_R(N)$. Namely, given two chains of submodules
    \[ L_0 \subsetneq L_1 \subsetneq \cdots \subsetneq L_\ell = L \quad \textup{ and } \quad N_0 \subsetneq N_1 \subsetneq \cdots \subsetneq N_n = N , \]
    we can construct a chain of submodules
    \[ f(L_0) \subsetneq f(L_1) \subsetneq \cdots \subsetneq f(L_\ell) \subset g^{-1}(N_0) \subsetneq g^{-1}(N_1) \subsetneq \cdots \subsetneq g^{-1}(N_n) = M , \]
    which shows that $\textup{length}_R(M) \ge \textup{length}_R(L) + \textup{length}_R(N)$. Conversely, from a chain of submodules
    \[ M_0 \subsetneq M_1 \subsetneq \cdots \subsetneq M_m = M , \]
    we can construct chains of submodules $L_i = M_i \cap L$ and $N_i = \im(M_i \to N)$. Note that if $L_i = L_{i + 1}$ and $N_i = N_{i + 1}$ for some $i$, then $M_i = M_{i + 1}$ as well. Hence, we must have $\textup{length}_R(M) \le \textup{length}_R(L) + \textup{length}_R(N)$.
\end{example}

\begin{topic}{depth-module}{depth of a module}
    Let $R$ be a \tref{ring}{commutative ring}, $I \subset R$ an \tref{ideal}{ideal} and $M$ a finitely generated \tref{module}{$R$-module} such that $IM \subsetneq M$. Then the $I$-depth of $M$ is defined as
    \[ \textup{depth}_I(M) = \min \{ i \in \ZZ : \textup{Ext}^i(R/I, M) \ne 0 \} . \]
    When $R$ is a \tref{local-ring}{local ring}, one usually takes $I$ equal to be the maximal ideal $\mathfrak{m}$.
\end{topic}

\begin{topic}{bimodule}{bimodule}
    Let $R$ and $S$ be two \tref{ring}{rings}. An \textbf{$R$-$S$-bimodule} is an \tref{GT:abelian-group}{abelian group} $M$ such that $M$ is both a left $R$-module and a right $S$-module, and such that $(rm)s = r(ms)$ for all $r \in R$, $s \in S$ and $m \in M$.
    
    An $R$-bimodule is an $R$-$R$-bimodule.
\end{topic}

\begin{example}{bimodule}
    Given any field $k$, the ring of $m \times n$ matrices $\textup{Mat}_{m \times n}(k)$ is an $R$-$S$-bimodule where $R = \textup{Mat}_{m \times m}(k)$ and $S = \textup{Mat}_{n \times n}(k)$.
\end{example}

\begin{topic}{torsionless-module}{torsionless module}
    A \tref{module}{module} $M$ over a \tref{ring}{ring} $R$ is \textbf{torsionless} if for every $m \in M$ there exists some $R$-module morphism $f \colon M \to R$ with $f(m) \ne 0$.
\end{topic}

\begin{example}{torsionless-module}
    Any torsionless module $M$ over $R$ is \tref{torsion-free-module}{torsion-free}. Namely, if $m \in M$ exists with $rm = 0$ for some non zero-divisor $r \in R$, then for any $f \colon M \to R$ we have $0 = f(rm) = rf(m)$, implying $f(m) = 0$. However, the converse is not true. The $\ZZ$-module $\QQ$ is clearly torsion-free, but not torsionless since the only $\ZZ$-module morphisms $\QQ \to \ZZ$ is zero.
\end{example}

\begin{topic}{torsion-free-module}{torsion-free module}
    A \tref{module}{module} $M$ over a \tref{ring}{ring} $R$ is \textbf{torsion-free} if $rm = 0$ implies $m = 0$ or that $r$ is a \tref{zero-divisor}{zero-divisor}, for all $r \in R$ and $m \in M$.
\end{topic}

\begin{topic}{reflexive-module}{reflexive module}
    A \tref{module}{module} $M$ over a \tref{ring}{ring} $R$ is \textbf{reflexive} if the map
    \[ M \to \Hom_R(\Hom_R(M, R), R), \quad m \mapsto (f \mapsto f(m)) \]
    is a bijection.
\end{topic}

\begin{topic}{external-tensor-product}{external tensor product}
    Let $R$ and $S$ be \tref{algebra}{algebras} over a field $k$. If $M$ is an \tref{module}{$R$-module} and $N$ an $S$-module, then the \textbf{external tensor product} of $M$ and $N$ is the $R \otimes_k S$-module
    \[ M \boxtimes N = M|_k \otimes_k N|_k . \]
\end{topic}

\begin{topic}{indecomposable-module}{indecomposable module}
    A non-zero \tref{module}{module} $M$ over a \tref{ring}{ring} $R$ is \textbf{indecomposable} if it cannot be written as the direct sum of two non-zero submodules.
\end{topic}

\begin{example}{indecomposable-module}
    Every \tref{simple-module}{simple module} is indecomposable, but not every indecomposable module is simple. For example, let
    \[ R = \left\{ \begin{pmatrix} \alpha & \beta \\ 0 & \gamma \end{pmatrix} : \alpha, \beta, \gamma \in k \textup{ with } \alpha, \gamma \ne 0 \right\} \]
    be the ring of upper triangular $2 \times 2$ matrices over a field $k$, and $M = k^2$ the $R$-module with
    \[ \begin{pmatrix} \alpha & \beta \\ 0 & \gamma \end{pmatrix} \cdot \begin{pmatrix} x \\ y \end{pmatrix} = \begin{pmatrix} \alpha x + \beta y \\ \gamma y \end{pmatrix} . \]
    Then $M$ is not simple, as $N = k \times \{ 0 \}$ is a non-zero proper submodule of $M$. However, $M$ is indecomposable since $M \not\isom N \oplus (M / N)$.
\end{example}

\begin{topic}{noetherian-module}{noetherian module}
    A \tref{module}{module} $M$ over a \tref{ring}{ring} $R$ is \textbf{noetherian} if it satisfies the \textit{ascending chain condition}: for any increasing sequence of submodules of $M$,
    \[ M_1 \subset M_2 \subset M_3 \subset \cdots \]
    there exists some $N \in \NN$ such that $M_n = M_N$ for all $n \ge N$.
    
    Equivalently, $M$ is noetherian if all its submodules are finitely generated.
\end{topic}

\begin{topic}{artinian-module}{artinian module}
    A \tref{module}{module} $M$ over a \tref{ring}{ring} $R$ is \textbf{artinian} if it satisfies the \textit{descending chain condition}: for any decreasing sequence of submodules
    \[ M_1 \supset M_2 \supset M_3 \supset \cdots \]
    there exists some $N \in \NN$ such that $M_n = M_N$ for any $n \ge N$.
\end{topic}

\begin{topic}{finitely-presented-module}{finitely presented module}
    A \tref{module}{module} $M$ over a \tref{ring}{ring} $R$ is \textbf{finitely presented} if there exists a surjection
    \[ R^n \to M \]
    whose kernel is \tref{finitely-generated-module}{finitely generated}. That is, $M$ can be generated using finitely many generators, and finitely many relations.
\end{topic}

\begin{topic}{coherent-module}{coherent module}
    A \tref{module}{module} $M$ over a \tref{ring}{ring} $R$ is \textbf{coherent} if it is \tref{finitely-generated-module}{finitely generated} and every finitely generated submodule of $M$ is \tref{finitely-presented-module}{finitely presented}.
\end{topic}

\begin{topic}{dual-module}{dual module}
    Let $R$ be a \tref{ring}{commutative ring} and $M$ an \tref{module}{$R$-module}. The \textbf{dual} of $M$ is the $R$-module
    \[ M^* = \Hom_R(M, R) , \]
    whose $R$-module structure is given by $(r \cdot f)(m) = r \cdot f(m)$ for all $r \in R$, $f \in M^*$ and $m \in M$.
\end{topic}

\begin{topic}{lazard-theorem}{Lazard's theorem}
    Let $R$ be a \tref{ring}{commutative ring} and $M$ an \tref{module}{$R$-module}. Then \textbf{Lazard's theorem} states that $M$ is \tref{flat-module}{flat} if and only if it is the \tref{CT:direct-limit}{direct limit} of \tref{free-module}{free} $R$-modules of finite rank.
\end{topic}

\begin{example}{lazard-theorem}
    The condition `directed' is necessary, since the colimit of the diagram
    \[ \ZZ \xleftarrow{\cdot 2} \ZZ \xrightarrow{\cdot 2} \ZZ \]
    is $\ZZ^3 / ((1, 0, -2), (1, -2, 0))$, which has torsion (e.g. $(1, -1, -1)$), so is not flat.
\end{example}

\begin{topic}{baers-criterion}{Baer's criterion}
    Let $R$ be a \tref{ring}{commutative ring} and $Q$ an \tref{module}{$R$-module}. \textbf{Baer's criterion} states that $Q$ is \tref{injective-module}{injective} if and only if for every \tref{ideal}{ideal} $I \subset R$, regarded as $R$-module, and $R$-module morphisms $g \colon I \to Q$, there exists a unique extension $h \colon R \to Q$ along the inclusion $I \hookrightarrow R$.
    \[ \svg \begin{tikzcd} I \arrow[hook]{r} \arrow[swap]{d}{g} & R \arrow[dashed]{dl}{h} \\ Q \end{tikzcd} \]
\end{topic}

\begin{example}{baers-criterion}
    \begin{proof}
        Let $i \colon M \to N$ and $f \colon M \to Q$ be $R$-module morphisms, with $i$ injective. Consider the partially ordered set of pairs $(M', f')$, with $M \subset M' \subset N$ and $f'$ extending $f$, where $(M', f') \le (M'', f'')$ if $M' \subset M''$ and $f''$ extends $f'$. By \tref{GM:zorns-lemma}{Zorn's lemma}, there exists a maximal element $(M', f')$. If $M' = N$, we are done, and otherwise there exists some $x \in N \setminus M'$. Consider the ideal $I = \{ r \in R \mid rx \in M' \}$ and define $g \colon I \to Q$ by $g(r) = f'(rx)$. Then by assumption $g$ extends to some $h \colon R \to Q$. Now let $M'' = M' + \langle x \rangle$ and define
        \[ f'' \colon M'' \to Q, \quad y + rx \mapsto f'(y) + h(r) , \]
        for all $y \in M'$ and $r \in R$. Then $f''$ extends $f'$, contradicting the maximality of $M'$.
    \end{proof}
\end{example}

\begin{topic}{stably-free-module}{stably free module}
    Let $R$ be a \tref{ring}{ring}. An $R$-module $M$ is \textbf{stably free} if $P \bigoplus R^m \isom R^n$ for some $m$ and $n$.
\end{topic}

\begin{example}{stably-free-module}
    Clearly any \tref{free-module}{free module} is stably free, but the converse is not true. Namely, let $R = \RR[x, y, z] / (x^2 + y^2 + z^2 - 1)$ and consider the morphism
    \[ \sigma \colon R^3 \to R, \quad (a, b, c) \mapsto ax + by + cz . \]
    As $\sigma$ is surjective, we have $R^3 = \ker(\sigma) \oplus R$ so that $P = \ker(\sigma)$ is stably free. However, note that geometrically any element $(a, b, c) \in P$ defines a vector field $X = a \partial_x + b \partial_y + c \partial_z$ on the $2$-sphere $S^2$. So if $P$ were free, there would exist a basis of $P$, which would yield two vector fields $X$ and $Y$ that are linear independent at every point of $S^2$. However, this is impossible by the \tref{DG:hairy-ball-theorem}{hairy ball theorem}, so $P$ cannot be free.
\end{example}

\begin{topic}{schur-functor}{Schur functor}
    Let $k \supset \QQ$ be a \tref{ring}{commutative ring}. Note that for any \tref{module}{$k$-module} $M$, the \tref{GT:symmetric-group}{symmetric group} $S_n$ acts naturally on the $n$-fold \tref{tensor-product}{tensor product} $M^{\otimes n}$ by permuting the factors, turning it into a $k[S_n]$-module. Now, for any \tref{GM:integer-partition}{partition} $\lambda$ of an integer $n \ge 0$, the \textbf{Schur functor} corresponding to $\lambda$ is the \tref{CT:functor}{functor}
    \[ \mathbb{S}_\lambda \colon \textbf{Mod}_k \to \textbf{Mod}_k, \quad M \mapsto V_\lambda \otimes_{\QQ[S_n]} M^{\otimes n} , \]
    where $V_\lambda$ is the \tref{RT:irreducible-representation}{irreducible} \tref{RT:representation}{representation} of $S_n$ corresponding to $\lambda$.
\end{topic}

\begin{example}{schur-functor}
    \begin{itemize}
        \item The partition $\lambda = (n)$, corresponding to the trivial representation of $S_n$, yields the Schur functor
        \[ \mathbb{S}_{(n)}(M) = \operatorname{Sym}^n M . \]
        \item The partition $\lambda = (1, \ldots, 1)$, corresponding to the \tref{RT:alternating-representation}{alternating representation} of $S_n$, yields the Schur functor
        \[ \mathbb{S}_{(1, \ldots, 1)}(M) = \wedge^n M . \]
    \end{itemize}    
\end{example}

\begin{topic}{specht-module}{Specht module}
    Let $k$ be a \tref{ring}{commutative ring}, and $\lambda$ a \tref{GM:integer-partition}{partition} of a non-negative integer $n$.
    %
    Define a \textit{Young tabloid} of shape $\lambda$ to be an equivalence class of Young tableaux on a \tref{GM:young-diagram}{Young diagram} of shape $\lambda$, where two Young tableaux are equivalent if one is obtained from the other by permuting the entries of each row.
    %
    The \tref{GT:symmetric-group}{symmetric group} $S_n$ acts naturally on the set $\mathcal{T}$ of Young tabloids of shape $\lambda$.
    %
    The \textbf{Specht module} of $\lambda$ is the $k$-submodule of $\bigoplus_{[T] \in \mathcal{T}} k$ generated by the elements
    \[ E_T = \sum_{\sigma \in Q_T} \operatorname{sign}(\sigma) [\sigma(T)] , \]
    as $T$ runs over all Young tableaux of shape $\lambda$, where $Q_T$ denotes the subgroup of $S_n$ of permutations that preserve the columns of $T$, and $[T]$ denotes the Young tabloid corresponding to $T$.
\end{topic}

\begin{topic}{koszul-complex}{Koszul complex}
    Let $F$ be a \tref{free-module}{free module} of finite rank $r$ over a commutative \tref{ring}{ring} $R$. Then, given an $R$-linear map $s \colon F \to R$ the \textbf{Koszul complex} associated to $s$ is the chain complex of $R$-modules
    \[ K_\bdot(s) \colon \qquad 0 \to \wedge^r F \xrightarrow{d_r} \wedge^{r - 1} F \xrightarrow{d_{r - 1}} \cdots \xrightarrow{d_2} F \xrightarrow{d_1} R \to 0 \]
    with the differentials given by
    \[ d_k(e_1 \wedge e_2 \ldots \wedge e_k) = \sum_{i = 1}^{k} (-1)^{i + 1} s(e_i) e_1 \wedge \cdots \wedge \hat{e_i} \wedge \cdots \wedge e_k , \]
    where the hat indicates the term is missing. Note that $d_1 = s$.
\end{topic}

\begin{example}{koszul-complex}
    When $R = k[x_1, \ldots, x_n]$ and $E$ is of rank $n$ with $s = (x_1, \ldots, x_n)$, we obtain the complex 
    \[ 0 \to \wedge^n E = R \to \cdots \to \wedge^i E = R^{\binom{n}{i}} \to \cdots \to \wedge^1 E = R^n \to R \to 0 \]
    which is a free resolution of $k$ as an $R$-module.
\end{example}

\begin{topic}{shapiro-lemma}{Shapiro's lemma}
    Let $R$ and $S$ be \tref{ring}{rings}, $R \to S$ a morphism, $M$ a \tref{module}{left $R$-module}, and $N$ a left $S$-module. \textbf{Shapiro's lemma} states that
    \begin{itemize}
        \item if $S$ is \tref{projective-module}{projective} as a right $R$-module, then
        \[ \operatorname{Ext}_R^i(M, N) \isom \operatorname{Ext}_S^i(S \otimes_R M, N) , \]
        \item if $S$ is projective as a left $R$-module, then
        \[ \operatorname{Ext}_R^i(N, M) \isom \operatorname{Ext}_S^i(N, \Hom_R(S, M)) . \]
    \end{itemize}
\end{topic}

\begin{example}{shapiro-lemma}
    Let $G$ be a \tref{GT:group}{group} and $H \subset G$ a subgroup of finite \tref{GT:index-subgroup}{index}, so that $\ZZ[G]$ is finitely generated and projective as left and right $\ZZ[H]$-module. 
    For any representations $V$ of $G$ and $W$ of $H$,
    \[ \operatorname{Ext}_{\ZZ[H]}^i(\operatorname{Res}_H^G(V), W) \isom \operatorname{Ext}_{\ZZ[G]}^i(V, \Hom_{\ZZ[H]}(\ZZ[G], W)) \isom \operatorname{Ext}_{\ZZ[G]}^i(V, \operatorname{Ind}_H^G(W)) , \]
    where we used $\Hom_{\ZZ[H]}(\ZZ[G], W) \isom \operatorname{Ind}_H^G(W)$. In particular, for $i = 0$, we obtain \tref{RT:frobenius-reciprocity}{Frobenius reciprocity}.
    
    Specializing to $V = \ZZ$, we obtain an isomorphism in \tref{HA:group-cohomology}{group cohomology}
    \[ \begin{aligned}
        H^i(G, \operatorname{Ind}_H^G(W))
            &= \operatorname{Ext}_{\ZZ[G]}^i(\ZZ, \operatorname{Ind}_H^G(W)) \\
            &= \operatorname{Ext}_{\ZZ[G]}^i(\ZZ, \Hom_{\ZZ[H]}(\ZZ[G], W)) \\
            &= \operatorname{Ext}_{\ZZ[H]}^i(\ZZ, W) \\
            &= H^i(H, W) .
    \end{aligned} \]
\end{example}

\begin{topic}{pseudo-coherent-module}{pseudo-coherent module}
    Let $R$ be a \tref{ring}{ring}. An \tref{module}{$R$-module} $M$ is \textbf{$n$-pseudo-coherent}, for some integer $n \ge 0$, if there exists an \tref{HA:exact-sequence}{exact sequence} of $R$-modules
    \[ \bigoplus_{i = 1}^{r_n} R \to \cdots \to \bigoplus_{i = 1}^{r_1} R \to \bigoplus_{i = 1}^{r_0} R \to M \to 0 . \]
    The module $M$ \textbf{pseudo-coherent} if it is $n$-pseudo-coherent for all $n \ge 0$.
\end{topic}

\begin{example}{pseudo-coherent-module}
    \begin{itemize}
        \item A module is $0$-pseudo-coherent if and only if it is \tref{finitely-generated-module}{finitely generated}.
        \item A module is $1$-pseudo-coherent if and only if it is \tref{finitely-presented-module}{finitely presented}.
    \end{itemize}
\end{example}

\begin{topic}{projective-dimension}{projective dimension}
    The \textbf{projective dimension} of a \tref{module}{module} $M$ over a \tref{ring}{commutative ring} $R$ is the minimal length of a projective resolution of $M$.
    \[ \cdots \to P_n \to \cdots \to P_2 \to P_1 \to P_0 \to M \to 0 \]
    It may be infinite.
\end{topic}

\begin{example}{projective-dimension}
    Consider $R = k[x, y] / (xy)$ and $M = k$ as an $R$-module, where $x$ and $y$ act by multiplication by zero. Then $M$ cannot be resolved by a finite complex of projective $R$-modules, but there does exist an infinite resolution
    \[ \cdots \to R^2 \xrightarrow{\left(\begin{smallmatrix} y & 0 \\ 0 & x \end{smallmatrix}\right)} R^2 \xrightarrow{\left(\begin{smallmatrix} x & 0 \\ 0 & y \end{smallmatrix}\right)} R^2 \xrightarrow{\left(\begin{smallmatrix} y & 0 \\ 0 & x \end{smallmatrix}\right)} R^2 \xrightarrow{\left(\begin{smallmatrix} x & y \end{smallmatrix}\right)} R \to M \to 0 . \]
    Therefore, the projective dimension of $M$ is infinite.
\end{example}

\begin{topic}{injective-dimension}{injective dimension}
    The \textbf{injective dimension} of a \tref{module}{module} $M$ over a \tref{ring}{commutative ring} $R$ is the minimal length of a injective resolution of $M$.
    \[ 0 \to M \to I^0 \to I^1 \to I^2 \to \cdots \]
    It may be infinite.
\end{topic}

\begin{topic}{coextension-of-scalars}{coextension of scalars}
    Let $f \colon R \to S$ be \tref{ring-morphism}{morphism} of \tref{ring}{commutative rings}. \textbf{Coextension of scalars} along $f$ is the \tref{CT:functor}{functor}
    \[ \Hom_R(S, -) \colon \textbf{Mod}_R \to \textbf{Mod}_S \]
    where $S$ is seen as an $R$-module via $f$. The $S$-module structure on $\Hom_R(S, M)$, for an $R$-module $M$, is given by $(s \cdot \varphi)(x) = \varphi(s \cdot x)$ for all $s, x \in S$ and $\varphi \in \Hom_R(S, M)$.
    
    Coextension of scalars is \tref{CT:adjunction}{right adjoint} to \tref{restriction-of-scalars}{restriction of scalars}.
\end{topic}

\begin{topic}{restriction-of-scalars}{restriction of scalars}
    Let $f \colon R \to S$ be \tref{ring-morphism}{morphism} of \tref{ring}{commutative rings}. \textbf{Restriction of scalars} along $f$ is the \tref{CT:functor}{functor}
    \[ U \colon \textbf{Mod}_S \to \textbf{Mod}_R \]
    assigning to an \tref{module}{$S$-module} $N$ the $R$-module with underlying abelian group $N$ and whose $R$-module structure is given by $r \cdot n = f(r) \cdot n$ for all $r \in R$ and $n \in N$.
\end{topic}

\begin{topic}{extension-of-scalars}{extension of scalars}
    Let $f \colon R \to S$ be \tref{ring-morphism}{morphism} of \tref{ring}{commutative rings}. \textbf{Extension of scalars} along $f$ is the \tref{CT:functor}{functor}
    \[ (-) \otimes_R S \colon \textbf{Mod}_R \to \textbf{Mod}_S \]
    assigning to an \tref{module}{$R$-module} $M$ the \tref{tensor-product}{tensor product} $M \otimes_R S$.
    
    Extension of scalars is \tref{CT:adjunction}{left adjoint} to \tref{restriction-of-scalars}{restriction of scalars}.
\end{topic}

\begin{topic}{superfluous-submodule}{superfluous submodule}
    Let $R$ be a \tref{ring}{ring} and $M$ an \tref{module}{$R$-module}. A submodule $N \subset M$ is \textbf{superfluous} if for every submodule $H \subset M$, one has that $N + H = M$ implies that $H = M$.
\end{topic}

\begin{example}{superfluous-submodule}
    \begin{itemize}
        \item As $\ZZ$-modules, $\ZZ \subset \QQ$ is a superfluous submodule.
        \item More generally, a \tref{domain}{domain} $R$ is a superfluous submodule in its \tref{field-of-fractions}{field of fractions}.
        \item For any commutative ring $R$ and finitely generated $R$-module $M$, the submodule $\mathfrak{J}_R M \subset M$ is superfluous, where $\mathfrak{J}_R$ denotes the \tref{jacobson-radical}{Jacobson radical} of $R$.
    \end{itemize}
\end{example}
