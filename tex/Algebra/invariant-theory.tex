\begin{topic}{chevalley-shephard-todd-theorem}{Chevalley--Shephard--Todd theorem}
    Let $V$ be a finite-dimensional \tref{LA:vector-space}{vector space} over a \tref{field}{field} $k$, and $G \subset \textup{GL}(V)$ a finite \tref{GT:group}{group} whose \tref{GT:order}{order} $|G|$ is invertible in $k$. An element $s \in \textup{GL}(V)$ is a \textit{pseudoreflection} if $s \ne 1$ and $s$ fixes a codimension $1$ subspace of $V$. The \textbf{Chevalley--Shephard--Todd theorem} states that the following are equivalent:
    \begin{enumerate}[label=(\roman*)]
        \item $G$ is generated by pseudoreflections,
        \item the ring of invariants $k[V]^G$ is a free polynomial algebra over $k$,
        \item $k[V]$ is a \tref{free-module}{free module} over $k[V]^G$.
    \end{enumerate}
\end{topic}

\begin{example}{chevalley-shephard-todd-theorem}
    Let $G = \ZZ/2\ZZ$ act on $V = k^2$ by $(x, y) \mapsto (x, -y)$. Then $G$ is generated by a pseudoreflection, and indeed
    \[ k[x, y]^G = k[x, y^2] \]
    a free polynomial algebra over $k$. However, if we let $G$ act on $V$ by $(x, y) \mapsto (-x, -y)$, then $G$ is not generated by pseudoreflections, and indeed
    \[ k[x, y]^G = k[x^2, xy, y^2] = k[u, v, w] / (w^2 - uv) \]
    is not a free polynomial algebra over $k$.
\end{example}

\begin{example}{chevalley-shephard-todd-theorem}
    Consider the \tref{GT:symmetric-group}{symmetric group} $S_n$ acting on $V = k^{n - 1}$ via the \tref{RT:standard-representation}{standard representation}. Note that any $2$-cycle acts by a pseudo-reflection, and since $S_n$ is generated by $2$-cycles, we can conclude that $k[V]^G$ is a free polynomial algebra over $k$.
\end{example}

\begin{topic}{molien-formula}{Molien's formula}
    Let $G$ be a finite \tref{GT:group}{group} and $\rho : G \to \textup{GL}(V)$ a finite-dimensional complex \tref{RT:representation}{representation}. Then $G$ acts naturally on the \tref{symmetric-algebra}{symmetric algebra} $\textup{Sym}(V)$. \textbf{Molien's formula} states that the \tref{hilbert-series}{Hilbert series} of the invariant subalgebra $\textup{Sym}(V)^G \subset \textup{Sym}(V)$ is given by
    \[ \textup{HS}_{\textup{Sym}(V)^G}(t) = \frac{1}{|G|} \sum_{g \in G} \frac{1}{\det(1 - gt)} . \]
\end{topic}

\begin{example}{molien-formula}
    \begin{proof}
        Write $S = \textup{Sym}(V)$. The \textit{Reynolds operator} $R^G : S \to S$ given by $f \mapsto \frac{1}{|G|} \sum_{g \in G} g \cdot f$ projects $S_d$ onto $S^G_d$ for each $d \ge 0$, so
        \[ \dim_\CC(S_d) = \tr_{S_d}(R^G) = \frac{1}{|G|} \sum_{g \in G} \tr_{S_d}(g) . \]
        % Hence, it suffices to show that 
        % \[ \sum_{d \ge 0} \tr_{S_d}(g) \cdot t^d = \frac{1}{\det(1 - gt)} \]
        % for all $g \in G$.
        Now fix $g \in G$, and let $\lambda_1, \ldots, \lambda_n$ be the eigenvalues of $\rho(g)$. Then the eigenvalues of $g$ acting on $S_d$ are given by $\mu_\alpha = \prod_{i = 1}^{n} \lambda_i^{\alpha_i}$ for each $\alpha \in \NN^n$ with $\alpha_1 + \ldots + \alpha_n = d$. Now it follows that
        \[ \sum_{d \ge 0} \tr_{S_d}(g) \cdot t^d = \sum_{\alpha} \prod_{i = 1}^{n} (\lambda_i t)^{\alpha_i} = \prod_{i = 1}^{n} \sum_{j \ge 0} (\lambda_i t)^j = \prod_{i = 1}^{n} \frac{1}{1 - \lambda_i t} = \frac{1}{\det(1 - gt)} . \]
        Combined with the above equation, the result follows.
    \end{proof}
\end{example}
