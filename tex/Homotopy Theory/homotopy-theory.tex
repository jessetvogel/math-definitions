\begin{topic}{kan-complex}{Kan complex}
    A \textbf{Kan complex} is a \tref{CT:simplicial-object}{simplicial set} $X$ satisfying the \textit{Kan condition}: any map of simplicial sets $f : \Lambda^n_i \to X$ extends to a map of simplicial sets $\Delta^n \to X$. That is,
    \[ \svg \begin{tikzcd} \Lambda^n_i \arrow{r}{f} \arrow{d} & X \\ \Delta^n \arrow[dashed]{ur} & \end{tikzcd} \]
\end{topic}

\begin{example}{kan-complex}
    For any \tref{TO:topological-space}{topological space} $X$, the singular simplicial set $\textup{Sing}_\bdot(X)$ is a Kan-complex.
\end{example}

\begin{topic}{kan-fibration}{Kan fibration}
    A \textbf{Kan fibration} is a map of \tref{CT:simplicial-object}{simplicial sets} $f : X \to Y$ such that for any $1 \le i \le n$ and for any maps $s : \Lambda^n_i \to X$ and $y : \Delta^n \to Y$ with $f \circ s = y \circ i$ (where $i : \Lambda^n_i \to \Delta^n$ is the inclusion) there exists a map $x : \Delta^n \to X$ such that $s = x \circ i$.
    \[ \svg \begin{tikzcd} \Lambda^n_i \arrow{r}{s} \arrow[swap]{d}{i} & X \arrow{d}{f} \\ \Delta^n \arrow{r}{y} \arrow[dashed]{ur}{x} & Y \end{tikzcd} \]
\end{topic}

\begin{topic}{moore-complex}{(normalized) Moore complex}
    Let $A_\bdot$ be a \tref{CT:simplicial-object}{simplicial} \tref{GT:abelian-group}{abelian group}. The \textbf{Moore complex} of $A_\bdot$ is the \tref{HA:chain-complex}{chain complex} $C(A_\bdot)$ given by
    \[ C(A_\bdot)_n = A_n \quad \textup{ with differential } \quad \partial_n = \sum_{i = 0}^{n} (-1)^i d_{n, i} , \]
    where $d_{n, i} : A_n \to A_{n - 1}$ are the face maps of $A_\bdot$. The \textbf{normalized Moore complex} of $A_\bdot$ is the chain complex $N(A_\bdot)$ given by
    \[ N(A_\bdot)_n = \bigcap_{i = 0}^{n - 1} \ker d_{n, i} \quad \text{ with differential } \quad \partial_n = d_{n, n} . \]
\end{topic}

\begin{example}{moore-complex}
    Let $A_\bdot$ be a constant simplicial abelian group, that is, $A_n = A$ and all face and degeneracy maps are $\id_A$. Then the Moore complex of $A_\bdot$ is
    \[ C(A_\bdot) = \left[ \cdots \xrightarrow{\id_A} \underset{(3)}{A} \xrightarrow{0} \underset{(2)}{A} \xrightarrow{\id_A} \underset{(1)}{A} \xrightarrow{0} \underset{(0)}{A} \to 0 \right] \]
    and the normalized Moore complex is
    \[ N(A_\bdot) = \left[ \cdots \underset{(3)}{0} \to \underset{(2)}{0} \to \underset{(1)}{0} \to \underset{(0)}{A} \to 0 \right] . \]
\end{example}

\begin{topic}{dold-kan-correspondence}{Dold--Kan correspondence}
    The \textbf{Dold--Kan correspondence} is an \tref{CT:equivalence-of-categories}{equivalence} between the category of \tref{CT:simplicial-object}{simplicial} \tref{GT:abelian-group}{abelian groups} and non-negatively graded \tref{HA:chain-complex}{chain complexes} of abelian groups,
    \[ \textbf{Ab}_\Delta \simeq \operatorname{Ch}_{\ge 0}(\textbf{Ab}) . \]
    To any simplicial abelian group $A_\bdot$, one assigns the \tref{moore-complex}{normalized Moore complex} $N(A_\bdot)$.
    Inversely, to any chain complex $C_\bdot$ one assigns the simplicial abelian group
    \[ \sigma(C_\bdot)_n = \bigoplus_{[n] \twoheadrightarrow [k]} C_k . \]
    For any $\nu : [m] \to [n]$ and $\tau : [n] \twoheadrightarrow [k]$, the composition $\tau\nu : [m] \to [k]$ uniquely factors as $[m] \overset{\sigma}{\twoheadrightarrow} [j] \overset{\iota}{\hookrightarrow} [k]$ for some $j \le m, k$, which defines a map
    \[ (C_k \to C_j) = \left\{ \begin{array}{cl} \id_{C_k} & \textup{ if } j = k , \\ (-1)^k \partial_k & \textup{ if } j = k - 1 \textup{ and } \iota = d_{k,k} , \\ 0 & \textup{ otherwise}. \end{array} \right. \]
    The universal property of the direct sum yields morphisms $\sigma(C_\bdot)_\nu : \sigma(C_\bdot)_n \to \sigma(C_\bdot)_m$.
    
    The statement can be generalized to any \tref{HA:abelian-category}{abelian category} $\mathcal{A}$: there is an equivalence
    \[ \mathcal{A}_\Delta \simeq \operatorname{Ch}_{\ge 0}(\mathcal{A}) . \]
\end{topic}

\begin{topic}{geometric-realization}{geometric realization}
    The \textbf{geometric realization} of a \tref{CT:simplicial-object}{simplicial set} $X : \Delta^\textup{op} \to \textbf{Set}$ is the \tref{TO:topological-space}{topological space} $|X|$ defined as follows. The geometric realization of the standard $n$-simplex $\Delta^n$ is given by
    \[ |\Delta^n| = \{ (x_0, \ldots, x_n) \in \RR^{n + 1} : 0 \le x_i \le 1, \sum_i x_i = 1 \} , \]
    and is naturally extended to
    \[ |X| = \underset{\Delta^n \to X}{\textup{colim}} |\Delta^n| , \]
    where the \tref{CT:limit}{colimit} is taken over all maps $\Delta^n \to X$.
    
    This construction gives a functor
    \[ |\cdot| : \textbf{Set}_\Delta \to \textbf{Top} . \]
\end{topic}

\begin{example}{geometric-realization}
    Geometric realization is \tref{CT:adjunction}{left adjoint} to the \tref{AT:singular-homology}{singular functor} $\textup{Sing} : \textbf{Top} \to \textbf{Set}_\Delta$. Indeed, for any simplicial set $X$ and topological space $S$, we have a natural bijection
    \[ \begin{aligned}
        \Hom(\colim_{\Delta^n \to X} |\Delta^n|, S) &= \lim_{\Delta^n \to X} \Hom(|\Delta^n|, S) = \lim_{\Delta^n \to X} \Hom(\Delta^n, \textup{Sing}(S)) \\ & = \Hom(\colim_{\Delta^n \to X} \Delta^n, \textup{Sing}(S)) = \Hom(X, \textup{Sing}(S)) .
    \end{aligned} \]
\end{example}

\begin{topic}{en-algebra}{En-algebra}
    Let $(\mathcal{C}, \otimes, \textbf{1})$ be a \tref{CT:symmetric-monoidal-category}{symmetric monoidal} \tref{CT:infinity-category}{$\infty$-category}, and $n \ge 0$ an integer. An \textbf{$E_n$-algebra} $\mathcal{E}$ of $\mathcal{C}$ consists of
    \begin{itemize}
        \item an object $\mathcal{E}(U) \in \mathcal{C}$ for each open set $U \subset \RR^n$ homeomorphic to a disk,
        \item for every embedding $U_1 \sqcup \cdots \sqcup U_m \subset V$ of disjoint disks into another disk, a multiplication map
        \[ \mu : \bigotimes_{i = 1}^{m} \mathcal{E}(U_i) \to \mathcal{E}(V) . \]
    \end{itemize}
    It is required that the multiplication maps $\mu$ are compatible with composition, and that $\mu$ is an equivalence for $m = 1$.
\end{topic}

\begin{example}{en-algebra}
    Let $X$ be a \tref{TO:topological-space}{topological space} and $\Omega_x X = \{ p : [0, 1] \to X : p(0) = p(1) = x \}$ its \tref{AT:loop-space}{loop space} with respect to some basepoint $x \in X$. One can concatenate paths $p, q : [0, 1] \to X$ as
    \[ (p \star q)(t) = \left\{ \begin{array}{cl}
         p(2t) & 0 \le t \le \tfrac{1}{2} , \\
         q(2t - 1) & \tfrac{1}{2} \le t \le 1 ,
    \end{array} \right. \]
    but this operation is not strictly associative, since loops $(p \star q) \star r$ and $p \star (q \star r)$ differ by their parametrization. However, concatenation is associative `up to coherent homotopy', and this can be made precise by viewing the loop space $\Omega_x X$ as an $E_1$-algebra.
    
    Let $\mathcal{C}$ the $\infty$-category of spaces. To every disk $U \subset \RR^1$ assign the space
    \[ \mathcal{E}(U) = \textup{Map}((U^+, \textup{pt}), (X, x)) , \]
    where $U^+$ denotes the one-point compactification of $U$. In particular, $U$ is homeomorphic to $S^1$, so $\mathcal{E}(U)$ is (non-canonically) homeomorphic to $\Omega_x X$. Furthermore, an embedding of disjoint disks $U_1 \sqcup \cdots \sqcup U_m \subset V$ induces a `collapse map' $V^+ \to U_1^+ \vee \cdots \vee U_m^+$ to the \tref{TO:wedge-sum}{wedge sum}, which yields a multiplication map
    \[ \mu : \mathcal{E}(U_1) \times \cdots \mathcal{E}(U_m) \to \mathcal{E}(V) . \]
    This makes $\mathcal{E}$ into an $E_1$-algebra.
    
    Note that the original concatenation $\star$ corresponds to the multiplication map $\mu$ for the embedding $(0, \tfrac{1}{2}) \sqcup (\tfrac{1}{2}, 1) \subset (0, 1)$.
\end{example}

\begin{example}{en-algebra}
    Suppose that $\mathcal{C}$ is an ordinary category, i.e. all $k$-morphisms for $k \ge 2$ are trivial, and let $\mathcal{E}$ be an $E_n$-algebra in $\mathcal{C}$. Note that for $n \ge 2$, all embeddings $(D^n)^{\sqcup k} \to D^n$ are isotopic, so all corresponding maps $\mathcal{E}^{\otimes k} \to \mathcal{E}$ are equal. This shows that $\mathcal{E}$ is precisely given by a commutative algebra in $\mathcal{C}$. For $n = 1$, all embeddings $(D^1)^{\sqcup k} \to D^1$ are isotopic up to a permutation of the factors. This shows that an $E_1$-algebra $\mathcal{E}$ in $\mathcal{C}$ is precisely an associative algebra in $\mathcal{C}$.
\end{example}

\begin{topic}{simplicial-homotopy-group}{simplicial homotopy group}
    Let $X$ be a \tref{kan-complex}{Kan complex}.
    % The \textbf{$0$th simplicial homotopy group} of $X$ is the set of equivalence classes of vertices $x \in X_0$, where $x \sim{} y$ if there exists some $e \in X_1$ such that $x = d_0(e)$ and $y = d_1(e)$.
    % \[ \pi_0(X) = X_0 / X_1 \]
    For $x \in X_0$ and $n \ge 0$, the \textbf{$n$th simplicial homotopy group} of $X$ at $x$, denoted $\pi_n(X, x)$, is the set of equivalence classes of maps $\alpha : \Delta^n \to X$ sending the boundary $\partial \Delta^n$ to $x$,
    \[ \svg \begin{tikzcd} \partial \Delta^n \arrow{d} \arrow{r} & \Delta^0 \arrow{d}{x} \\ \Delta^n \arrow{r}{\alpha} & X \end{tikzcd} \]
    where two maps $\alpha, \beta : \Delta^n$ are equivalent if there exists a simplicial homotopy $\eta : \Delta^n \times \Delta^1 \to X$ which fixes the boundary:
    \[ \svg \begin{tikzcd} \Delta^n \arrow[swap]{d}{\iota_0} \arrow[bend left=20]{dr}{\alpha} & \\ \Delta^n \times \Delta^1 \arrow{r}{\eta} & X \\ \Delta^n \arrow{u}{\iota_1} \arrow[swap, bend right=20]{ru}{\beta} \end{tikzcd} \quad \textup{ and } \quad \begin{tikzcd} \partial \Delta^n \times \Delta^1 \arrow{d} \arrow{r} & \Delta^0 \arrow{d}{x} \\ \Delta^n \times \Delta^1 \arrow{r}{\eta} & X \end{tikzcd} \]
    For $n \ge 1$, the sets $\pi_n(X, x)$ come equipped with the following \tref{GT:group}{group} structure. For any $f, g : \Delta^n \to X$, consider the map $\Lambda^{n + 1}_n \to X$, where the $i$th face of $\Lambda^{n + 1}_n$ is mapped to
    \[ v_i = \left\{ \begin{array}{cl}
        s_0 \circ \cdots \circ s_0 (x) & \textup{ for } 0 \le i \le n - 2, \\
        f & \textup{ for } i = n - 1, \\
        g & \textup{ for } i = n + 1.
    \end{array} \right. \]
    Since $X$ is a Kan complex, this map extends to some $\theta : \Delta^{n + 1} \to X$, and the group structure on $\pi_n(X, x)$ is given by
    \[ [f] \cdot [g] = [d_n(\theta)] . \]
    The \textbf{simplicial homotopy groups} of any \tref{CT:simplicial-object}{simplicial set} are given by the simplicial homotopy groups of its Kan fibrant replacement.
\end{topic}

\begin{topic}{l-infinity-algebra}{L-infinity algebra}
    An \textbf{$L_\infty$-algebra} is \tref{AA:graded-module}{graded} \tref{LA:vector-space}{vector space} $\mathfrak{g} = \bigoplus_{i \in \ZZ} \mathfrak{g}_i$ together with, for each $n \ge 1$, a linear map
    \[ l_n = [-, -, \cdots, -]_n : \mathfrak{g}^{\otimes n} \to \mathfrak{g} \]
    of degree $n - 2$, called the \textit{$n$-ary bracket}, satisfying
    \begin{itemize}
        \item (\textit{graded skew-symmetry}) for all homogeneous $v_1, \ldots, v_n \in \mathfrak{g}$ and \tref{GT:symmetric-group}{permutations} $\sigma \in S_n$,
        \[ l_n(v_{\sigma(1)}, \ldots, v_{\sigma(n)}) = \chi(\sigma, v_1, \ldots, v_n) l_n(v_1, \ldots, v_n) , \]
        where $\chi(\sigma, v_1, \ldots, v_n) \in \{ \pm 1 \}$ is the product of the \tref{GT:permutation-sign}{sign} of $\sigma$ with a factor of $(-1)^{\deg(v_i) \deg(v_j)}$ for each interchange of neighbors $(\ldots, v_i, v_j, \ldots)$ to $(\ldots, v_j, v_i, \ldots)$ in the decomposition of $\sigma$ as a sequence of swapping neighbor pairs.
        \item (\textit{generalized Jacobi identity}) for all homogeneous $v_1, \ldots, v_n \in \mathfrak{g}$,
        \[ \sum_{i + j = n + 1} \sum_{\sigma} \chi(\sigma, v_1, \ldots, v_n) (-1)^{i(j - 1)} l_j(l_i(v_{\sigma(1)}, \ldots, v_{\sigma(i)}), v_{\sigma(i + 1)}, \ldots, v_{\sigma(n)}) = 0 , \]
        where $\sigma$ runs over the set of \tref{GT:shuffle}{$(i, j - 1)$-unshuffles}.
    \end{itemize}
\end{topic}

\begin{example}{l-infinity-algebra}
    For $n = 1$, the generalized Jacobi identity states that the map $d = l_1 : \mathfrak{g} \to \mathfrak{g}$, of degree $-1$, satisfies
    \[ d^2 = 0 , \]
    making $\mathfrak{g}$ into a \tref{HA:chain-complex}{chain complex}. For $n = 2$, the condition states that
    \[ d[v_1, v_2] = [dv_1, v_2] + (-1)^{\deg(v_1)} [v_1, dv_2] , \]
    for homogeneous $v_1, v_2 \in \mathfrak{g}$, making $d$ a derivation for the bracket $[-, -]$.
    
    If $l_n = 0$ for all $n \ge 2$, then the generalized Jacobi identity for $n = 3$ translates to the graded Jacobi identity
    \[ [v_1, [v_2, v_3] = [[v_1, v_2], v_3] + (-1)^{\deg(v_1) \deg(v_2)} [v_2, [v_1, v_3]] , \]
    for all homogeneous $v_1, v_2, v_3 \in \mathfrak{g}$. In particular, such an $L_\infty$-algebra is equivalent to a \tref{AA:dg-lie-algebra}{differential graded Lie algebra}.
\end{example}

\begin{topic}{stable-infinity-category}{stable infinity category}
    An \tref{CT:infinity-category}{$\infty$-category} $\mathcal{C}$ is \textbf{stable} if
    \begin{itemize}
        \item there exists a \tref{CT:zero-object}{zero object} $0$ in $\mathcal{C}$,
        \item every morphism $f : X \to Y$ in $\mathcal{C}$ has a \textit{fiber} and \textit{cofiber}, that is, a pullback and pushout diagram
        \[ \svg \begin{tikzcd} W \arrow{r} \arrow{d} & X \arrow{d}{f} \\ 0 \arrow{r} & Y \end{tikzcd} \quad \textup{ and } \quad \begin{tikzcd} X \arrow{r}{f} \arrow{d} & Y \arrow{d} \\ 0 \arrow{r} & Z \end{tikzcd} \]
        \item Every diagram in $\mathcal{C}$ of the form
        \[ \svg \begin{tikzcd} X \arrow{r} \arrow{d} & Y \arrow{d} \\ 0 \arrow{r} & Z \end{tikzcd} \]
        is a pullback diagram if and only if it is a pushout diagram.
    \end{itemize}
\end{topic}

\begin{topic}{brown-representability-theorem}{Brown's representability theorem}
    Let $\mathcal{C}$ be the \tref{homotopy-category}{homotopy category} of \tref{TO:connected-space}{connected} pointed \tref{AT:cw-complex}{CW-complexes}, and $\textbf{Set}_*$ the category of pointed sets. Let $F : \mathcal{C}^\textup{op} \to \textbf{Set}_*$ be a \tref{CT:functor}{functor}. \textbf{Brown's representability theorem} states that $F$ is \tref{CT:representable-functor}{representable} if
    \begin{enumerate}[label=(\roman*)]
        \item for any collection of objects $(X_i)_{i \in I}$ of $\mathcal{C}$, the natural map $F(\bigvee_{i \in I} X_i) \to \prod_{i \in I} F(X_i)$ is an isomorphism,
        \item the natural map $F(Y \cup_X Z) \to F(Y) \times_{F(X)} F(Z)$ is a surjection for any two \tref{AT:cofibration}{cofibrations} $X \to Y$ and $X \to Z$ in $\mathcal{C}$.
    \end{enumerate}
\end{topic}

\begin{topic}{alexander-whitney-map}{Alexander--Whitney map}
    Let $C : \textbf{Ab}_\Delta \to \textup{Ch}_{\ge 0}(\textbf{Ab})$ be the \tref{moore-complex}{Moore complex} functor. For $A, B \in \textbf{Ab}_\Delta$, the \textbf{Alexander--Whitney map} is the \tref{CT:natural-transformation}{natural transformation} on chain complexes
    \[ \begin{aligned}
        \Delta_{A, B} : C(A \otimes B) &\to C(A) \otimes C(B) \\
        a \otimes b &\mapsto \bigoplus_{p + q = n} d_\textup{front}^p a \otimes d_\textup{back}^q b ,
    \end{aligned} \]
    for all $a \in A_n$ and $b \in B_n$, where $d_\textup{front}^p : A_n \to A_p$ and $d_\textup{back}^q : B_n \to B_q$ are induced by
    \[ [p] \to [p + q], \quad i \mapsto i \quad \textup{ and } \quad [q] \to [p + q], \quad i \mapsto p + i , \]
    respectively. The Alexander--Whitney map descends to the normalized Moore complex
    \[ \Delta_{A, B} : N(A \otimes B) \to N(A) \otimes N(B) . \]
\end{topic}

\begin{topic}{eilenberg-zilber-map}{Eilenberg--Zilber map}
    Let $C : \textbf{Ab}_\Delta \to \textup{Ch}_{\ge 0}(\textbf{Ab})$ be the \tref{moore-complex}{Moore complex} functor. For $A, B \in \textbf{Ab}_\Delta$, the \textbf{Eilenberg--Zilber map} is the \tref{CT:natural-transformation}{natural transformation} on chain complexes
    \[ \begin{aligned}
        \nabla_{A, B} : C(A) \otimes C(B) &\to C(A \otimes B) \\
        a \otimes b &\mapsto \sum_{(\mu, \nu) \in \textup{Sh}(p, q)} \textup{sign}(\mu, \nu) (s_\nu(a) \otimes s_\mu(b)) ,
    \end{aligned} \]
    for all $a \in A_p$ and $b \in B_q$, where the sum is taken over all \tref{GT:shuffle}{$(p, q)$-shuffles} $(\mu, \nu) = (\mu_1, \ldots, \mu_p, \nu_1, \ldots, \nu_q)$, and $s_\mu$ and $s_\nu$ are given by compositions of the degeneracy maps
    \[ s_\mu = s_{\mu_p - 1} \circ \cdots \circ s_{\mu_1 - 1} \quad \textup{ and } \quad s_\nu = s_{\nu_q - 1} \circ \cdots \circ s_{\nu_1 - 1} . \]
    The Eilenberg--Zilber map descends to the normalized Moore complex
    \[ \nabla_{A, B} : N(A) \otimes N(B) \to N(A \otimes B) . \]
    
    The \textbf{Eilenberg--Zilber theorem} states that for $A, B \in \textbf{Ab}_\Delta$, the composition $\Delta_{A, B} \circ \nabla_{A, B}$ equals the identity on $N(A) \otimes N(B)$, and the composition $\nabla_{A, B} \circ \Delta_{A, B}$ is \tref{HA:chain-homotopy}{homotopic} to the identity on $N(A \otimes B)$, where $\Delta_{A, B}$ denotes the \tref{alexander-whitney-map}{Alexander--Whitney map}. In particular, the chain complexes $N(A \otimes B)$ and $N(A) \otimes N(B)$ are \tref{HA:chain-homotopy-equivalence}{homotopy equivalent}.
\end{topic}
