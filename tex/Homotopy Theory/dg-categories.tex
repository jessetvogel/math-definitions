\begin{topic}{dg-category}{dg category}
    A \textbf{differential graded (dg) category} $\mathcal{A}$ over a \tref{AA:ring}{commutative ring} $k$, is a \tref{CT:category}{category} \tref{CT:enriched-category}{enriched} over \tref{HA:chain-complex}{chain complexes} of $k$-modules. Concretely, morphisms between objects $A$ and $B$ of $\mathcal{A}$ form a chain complex,
    \[ \Hom_\mathcal{A}(A, B) = \bigoplus_{n \in \ZZ} \Hom_\mathcal{A}(A, B)^n , \]
    with differential $d \colon \Hom^n_\mathcal{A}(A, B) \to \Hom^{n + 1}_\mathcal{A}(A, B)$ satisfying $d^2 = 0$. Furthermore, composition of morphisms
    \[ \Hom_\mathcal{A}(A, B) \otimes_k \Hom_\mathcal{A}(B, C) \to \Hom_\mathcal{A}(A, C) \]
    must be a chain map.
\end{topic}

\begin{example}{dg-category}
    Let $\mathcal{A}$ be a \tref{HA:linear-category}{$k$-linear category}. Let $\mathscr{A}$ be the dg category whose objects are \tref{HA:chain-complex}{chain complexes} in $\mathcal{A}$, and whose morphisms are given by the complexes
    \[ \Hom_\mathscr{A}(A^\bdot, B^\bdot) = \bigoplus_{n \in \ZZ} \Hom_\mathscr{A}(A^\bdot, B^\bdot)^n \]
    where
    \[ \Hom_\mathscr{A}(A^\bdot, B^\bdot)^n = \prod_{i \in \ZZ} \Hom_\mathcal{A}(A^i, B^{i + n}) , \]
    with differential
    \[ d_\mathscr{A}^n \left( (f_i)_{i \in \ZZ} \right) = \left( d_B^{i + n} \circ f^i - (-1)^n f^{i + 1} \circ d_A^i \right)_{i \in \ZZ} . \]
    
    One can define the categories $Z^0(\mathscr{A})$ and $H^0(\mathscr{A})$ with the same objects as $\mathscr{A}$ and morphisms given by
    \[ \begin{aligned}
        \Hom_{Z^0(\mathscr{A})}(A, B) &= Z^0(\Hom_\mathscr{A}(A, B)) = \ker \left(d^0 \colon \Hom_\mathscr{A}(A, B)^0 \to \Hom_\mathscr{A}(A, B)^1 \right) , \\
        \Hom_{H^0(\mathscr{A})}(A, B) &= H^0(\Hom_\mathscr{A}(A, B)) = \frac{\ker \left(d^0 \colon \Hom_\mathscr{A}(A, B)^0 \to \Hom_\mathscr{A}(A, B)^1 \right)}{\im \left(d^{-1} \colon \Hom_\mathscr{A}(A, B)^{-1} \to \Hom_\mathscr{A}(A, B)^{0} \right)} .
    \end{aligned} \]
    While this construction works for general dg categories $\mathscr{A}$, in this example $Z^0(\mathscr{A})$ is the category of chain complexes in $\mathcal{A}$ with chain maps as morphisms, and $H^0(\mathscr{A})$ is precisely the \tref{HA:homotopy-category}{homotopy category} of $\mathcal{A}$.
\end{example}

\begin{example}{dg-category}
    Every \tref{AA:dg-algebra}{dg algebra} $A$ can be seen as a dg category $\mathcal{A}$ with a single object $\star$, and dg algebra of endomorphisms $\Hom_\mathcal{A}(\star, \star) = A$.
\end{example}

\begin{topic}{dg-functor}{dg functor}
    A \textbf{differential graded (dg) functor} is a \tref{CT:functor}{functor} $F \colon \mathcal{A} \to \mathcal{B}$ between \tref{dg-category}{dg categories}, such that for all objects $A$ and $B$ of $\mathcal{A}$, the map
    \[ F \colon \Hom_\mathcal{A}(A, B) \to \Hom_\mathcal{B}(F(A), F(B)) \]
    is a morphism of chain complexes.
\end{topic}

\begin{topic}{quasi-equivalence}{quasi-equivalence}
    A \tref{dg-functor}{dg functor} $F \colon \mathcal{A} \to \mathcal{B}$ is a \textbf{quasi-equivalence} if
    \begin{itemize}
        \item $F_{X, Y} \colon \Hom_\mathcal{A}(X, Y) \to \Hom_\mathcal{B}(F(X), F(Y))$ is a \tref{HA:quasi-isomorphism}{quasi-isomorphism} for all objects $X, Y$ of $\mathcal{A}$,
        \item $H^0(F) \colon H^0(\mathcal{A}) \to H^0(\mathcal{B})$ is an \tref{CT:equivalence-of-categories}{equivalence}.
    \end{itemize}
\end{topic}

\begin{topic}{dg-module}{dg module}
    Let $\mathcal{A}$ be a \tref{CT:small-category}{small} \tref{dg-category}{dg category} over $k$. A \textbf{right dg $\mathcal{A}$-module} is a \tref{dg-functor}{dg functor}
    \[ M \colon \mathcal{A}^\textup{op} \to \mathcal{M}\textbf{od}_k , \]
    where $\mathcal{M}\textbf{od}_k$ denotes the \textit{dg category of $k$-modules}, whose objects are \tref{HA:chain-complex}{chain complexes} of $k$-modules, and whose morphisms are given by
    \[ \Hom_{\mathcal{M}\textbf{od}_k}(A, B) = \bigoplus_{n \in \ZZ} \Hom_{\mathcal{M}\textbf{od}_k}(A, B)^n, \quad \textup{ with } \quad \Hom_{\mathcal{M}\textbf{od}_k}(A, B)^n = \prod_{i \in \ZZ} \Hom_k(A^i, B^{i + n}) . \]
    Similarly, a \textbf{left dg $\mathcal{A}$-module} is a dg functor
    \[ M \colon \mathcal{A} \to \textbf{Mod}_k . \]
\end{topic}

\begin{example}{dg-module}
    Any object $x$ of a dg category $\mathcal{A}$ produces the right dg module
    \[ \widehat{x} = \Hom_\mathcal{A}(-, x) . \]
    A dg module of this form is called \textit{representable}. This construction gives the \textit{Yoneda dg functor}
    \[ \widehat{(-)} \colon \mathcal{A} \to \mathcal{M}\textbf{od}_\mathcal{A}, \quad x \mapsto \widehat{x} , \]
    which is \tref{CT:full-functor}{full} and \tref{CT:faithful-functor}{faithful}.
\end{example}

% \begin{topic}{quasi-functor}{quasi-functor}
%     Let $\mathcal{A}$ and $\mathcal{B}$ be \tref{dg-category}{dg categories} over $k$. A \textbf{quasi-functor} from $\mathcal{A}$ to $\mathcal{B}$ is an $\mathcal{A}$-$\mathcal{B}$-bimodule $T$, i.e. a \tref{dg-module}{dg $\mathcal{A}^\textup{op} \otimes \mathcal{B}$-module}, such that the tensor functor
%     \[ (-) \overset{\textup{L}}{\otimes_\mathcal{A}} \colon \textup{D}(\mathcal{A}) \to \textbf{D}(\mathcal{B}) \]
%     maps representable $\mathcal{A}$-modules to representable $\mathcal{B}$-modules (up to isomorphism).
% \end{topic}
