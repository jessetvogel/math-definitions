\begin{topic}{sheaf}{sheaf}
    Let $X$ be a \tref{TO:topological-space}{topological space}. A \textbf{presheaf} $\mathcal{F}$ of abelian groups on $X$ consists of
    \begin{itemize}
        \item an \tref{GT:abelian-group}{abelian group} $\mathcal{F}(U)$ for every open subset $U \subset X$,
        \item a \tref{GT:group-homomorphism}{group morphism} $r_{UV} : \mathcal{F}(U) \to \mathcal{F}(V)$ for every inclusion $V \subset U$ of open subsets of $X$,
    \end{itemize}
    such that
    \begin{itemize}
        \item $r_{UU}$ is the identity map for any open $U \subset X$,
        \item if $W \subset V \subset U$ are open subsets of $X$, then $r_{UW} = r_{VW} \circ r_{UV}$.
    \end{itemize}
    Elements of $\mathcal{F}(U)$ are called \textit{sections}. The maps $r_{UV}$ are thought of as restriction maps, and for this reason $r_{UV}(s)$ is often simply written as $s|_V$, for $s \in \mathcal{F}(V)$. One can similarly define a presheaf of rings, sets, etc.
    
    A presheaf $\mathcal{F}$ is a \textbf{sheaf} if it moreover satisfies:
    \begin{itemize}
        \item for any open $U \subset X$ and open covering $\{ U_i \}$ of $U$, if $s \in \mathcal{F}(U)$ is such that $s|_{U_i} = 0$ for all $i$, then $s = 0$.
        \item for any open $U \subset X$ and open covering $\{ U_i \}$ of $U$, suppose we have elements $s_i \in \mathcal{F}(U_i)$ such that $s_i|{U_i \cap U_j} = s_j|_{U_i \cap U_j}$ for all $i, j$. Then there is an element $s \in \mathcal{F}(U)$ such that $s_i = s|_{U_i}$ for all $i$. (Note that uniqueness follows from the above condition.)
    \end{itemize}
    
    A morphism of presheaves $f : \mathcal{F} \to \mathcal{G}$ consists of a morphism of abelian groups $f(U) : \mathcal{F}(U) \to \mathcal{G}(U)$ for each open set $U$, such that for every inclusion $V \subset U$ the diagram
    \[ \begin{tikzcd} \mathcal{F}(U) \arrow{r}{f(U)} \arrow[swap]{d}{r_{UV}} & \mathcal{G}(U) \arrow{d}{r'_{UV}} \\ \mathcal{F}(V) \arrow{r}{f(V)} & \mathcal{G}(V) \end{tikzcd} \]
    commutes. A morphism of sheaves is a morphism of presheaves.
\end{topic}

\begin{topic}{constant-sheaf}{constant sheaf}
    Let $X$ be a \tref{TO:topological-space}{topological space}, and $A$ an \tref{GT:abelian-group}{abelian group}. The \textbf{constant sheaf} $\underline{A}$ on $X$ determined by $A$ is the \tref{sheaf}{sheaf} given by
    \[ \underline{A}(U) = \{ \textup{locally constant functions } U \to A \} , \]
    and the usual restriction maps. Note that for every \tref{TO:connected-space}{connected} open set $U$ we have $\underline{A}(U) \simeq A$, hence the name `constant sheaf'.
    
    A sheaf $\mathcal{F}$ on $X$ is \textbf{locally constant} if every point has an open neighborhood $U$ such that $\mathcal{F}|_U$ is isomorphic to a \textbf{constant sheaf}.
\end{topic}

\begin{topic}{stalk}{stalk}
    Let $\mathcal{F}$ be a \tref{sheaf}{presheaf} on a topological space $X$, and take a point $x \in X$. The \textbf{stalk} $\mathcal{F}_x$ of $\mathcal{F}$ at $x$ is defined as the direct limit of the groups $\mathcal{F}(U)$ for all open sets $U$ containing $x$, via the restriction maps.
\end{topic}

\begin{topic}{associated-sheaf}{associated sheaf}
    Given a \tref{sheaf}{presheaf} $\mathcal{F}$, there is a \tref{sheaf}{sheaf} $\mathcal{F}^+$ and a morphism $\theta : \mathcal{F} \to \mathcal{F}^+$, with the property that for any sheaf $\mathcal{G}$ and morphism $f : \mathcal{F} \to \mathcal{G}$ there is a unique morphism $g : \mathcal{F}^+ \to \mathcal{G}$ such that $f = g \circ \theta$. The sheaf $\mathcal{F}^+$ is called the \textbf{sheaf associated} to the presheaf $\mathcal{F}$.
    \[ \begin{tikzcd} \mathcal{F} \arrow{rr}{f} \arrow[swap]{dr}{\theta} && \mathcal{G} \\ & \mathcal{F}^+ \arrow[swap,dashed]{ur}{g} & \end{tikzcd} \]
\end{topic}

\begin{topic}{direct-image-sheaf}{direct image sheaf}
    Let $f : X \to Y$ be a map of \tref{TO:topological-space}{topological spaces}, and let $\mathcal{F}$ be a \tref{sheaf}{sheaf} on $X$. The \textbf{direct image sheaf} $f_* \mathcal{F}$ on $Y$ is defined by
    \[ (f_* \mathcal{F})(V) = \mathcal{F}(f^{-1}(V)) \]
    for any open set $V \subset Y$ (indeed this presheaf is a sheaf).
    
    This construction yields the \textbf{direct image functor}
    \[ f_* : \textup{Sh}(X) \to \textup{Sh}(Y) , \]
    which is \tref{CT:adjunction}{right adjoint} to the \tref{inverse-image-sheaf}{inverse image functor} $f^{-1}$.
    
    When $f : X \to Y$ is a morphism of \tref{scheme}{schemes} and $\mathcal{F}$ an \tref{O-module}{$\mathcal{O}_X$-module}, the direct image $f^* \mathcal{F}$ is a $\mathcal{O}_Y$-module, getting its structure via $f^\# : \mathcal{O}_Y \to f_* \mathcal{O}_X$. Again we have a adjunction $f^* \dashv f_*$.
\end{topic}

\begin{topic}{inverse-image-sheaf}{inverse image sheaf}
    Let $f : X \to Y$ be a map of topological spaces, and let $\mathcal{G}$ be a \tref{sheaf}{sheaf} on $Y$. The \textbf{inverse image sheaf} $f^
    {-1}\mathcal{G}$ on $X$ is defined as the \tref{associated-sheaf}{sheaf associated} to the presheaf given by $U \mapsto \lim_{V \supset f(U)} \mathcal{G}(V)$ for any open set $U \subset X$.
    
    This construction yields the \textbf{inverse image functor}
    \[ f^{-1} : \textup{Sh}(Y) \to \textup{Sh}(X) . \]
    It is the \tref{CT:adjunction}{left adjoint} of the \tref{direct-image-sheaf}{direct image functor} $f_* : \textup{Sh}(X) \to \textup{Sh}(Y)$.
    
    When $f : X \to Y$ is a morphism of \tref{scheme}{schemes}, and $\mathcal{G}$ an \tref{O-module}{$\mathcal{O}_Y$-module}, the \textbf{inverse image sheaf} $f^* \mathcal{G}$ is the $\mathcal{O}_X$-module defined by
    \[ f^* \mathcal{G} = f^{-1} \mathcal{G} \otimes_{f^{-1} \mathcal{O}_Y} \mathcal{O}_X . \]
    Again we have an adjunction $f^* \dashv f_*$.
\end{topic}

\begin{topic}{flasque-sheaf}{flasque sheaf}
    A \tref{sheaf}{sheaf} $\mathcal{F}$ on a topological space $X$ is called \textbf{flasque} if for every inclusion $V \subset U$ of open sets, the restriction map $\mathcal{F}(U) \to \mathcal{F}(V)$ is surjective.
\end{topic}

\begin{topic}{skyscraper-sheaf}{skyscraper sheaf}
    Let $X$ be a topological space, $A$ an abelian group, and take a point $x \in X$. The \textbf{skyscraper sheaf} $i_x(A)$ at $x$ with value $A$ is defined as
    \[ i_x(A) (U) = \left\{ \begin{array}{cl} A & \text{if } x \in U, \\ 0 & \text{otherwise.} \end{array} \right. \]
    The stalks of this sheaf are $A$ at any point in the closure of $x$, and zero elsewhere.
    
    Equivalently, it is the \tref{direct-image-sheaf}{direct image sheaf} $i_*(\underline{A})$ for $\underline{A}$ the \tref{constant-sheaf}{constant sheaf} determined by $A$ on the closure $\overline{\{ x \}}$ and $i : \overline{\{ x \}} \to X$ the inclusion.
\end{topic}

\begin{topic}{sheaf-hom}{sheaf hom}
    Let $\mathcal{F}$ and $\mathcal{G}$ be \tref{sheaf}{sheaves} of abelian groups on a topological space $X$. The \textbf{sheaf hom} of $\mathcal{F}$ and $\mathcal{G}$ is the sheaf $\underline{\Hom}(\mathcal{F}, \mathcal{G})$ given by
    \[ \underline{\Hom}(\mathcal{F}, \mathcal{G})(U) = \Hom(\mathcal{F}|_U, \mathcal{G}|_U) . \]
\end{topic}

\begin{topic}{grothendieck-finiteness-theorem}{Grothendieck's finiteness theorem}
    Let $X$ be a \tref{scheme}{scheme} \tref{proper-morphism}{proper} over a field $k$, and $\mathcal{F}$ a \tref{coherent-sheaf}{coherent sheaf} on $X$. \textbf{Grothendieck's finiteness theorem} states that the cohomology groups $H^i(X, \mathcal{F})$ are finite-dimensional over $k$.
\end{topic}

\begin{topic}{grothendieck-vanishing-theorem}{Grothendieck's vanishing theorem}
    Let $X$ be a \tref{TO:noetherian-topological-space}{noetherian} \tref{TO:topological-space}{topological space} and $\mathcal{F}$ a \tref{sheaf}{sheaf} on $X$. \textbf{Grothendieck's vanishing theorem} states that $H^i(X, \mathcal{F}) = 0$ for all $i > \dim(X)$.
\end{topic}

\begin{topic}{euler-characteristic-sheaf}{Euler characteristic sheaf}
    Let $\mathcal{F}$ be a \tref{coherent-sheaf}{coherent sheaf} on a \tref{scheme}{scheme} $X$ \tref{proper-morphism}{proper} over a field $k$. The \textbf{Euler characteristic} of $\mathcal{F}$ is defined as the alternating sum
    \[ \chi(\mathcal{F}) = \sum_{i} (-1)^i \dim_k H^i(X, \mathcal{F}) . \]
    The dimensions are assured to be finite by \tref{grothendieck-finiteness-theorem}{Grothendieck's finiteness theorem}.
\end{topic}

% -- O_X modules --
\begin{topic}{O-module}{sheaf of O-modules}
    Let $(X, \mathcal{O}_X)$ be a \tref{ringed-space}{ringed space}. A \textbf{sheaf of $\mathcal{O}_X$-modules}, or simply $\mathcal{O}_X$-module, is a \tref{sheaf}{sheaf} $\mathcal{F}$ on $X$, such that for each open set $U \subset X$, the group $\mathcal{F}(U)$ is an $\mathcal{O}_X(U)$-module, and for each inclusion of open sets $V \subset U$, the restriction morphism $\mathcal{F}(U) \to \mathcal{F}(V)$ is an $\mathcal{O}_X(U)$-module morphism (here $\mathcal{F}(V)$ is seen as an $\mathcal{O}_X(U)$-module via $\mathcal{O}_X(U) \to \mathcal{O}_X(V)$).
    
    A morphism of $\mathcal{O}_X$-modules $\mathcal{F} \to \mathcal{G}$ is a morphism of sheaves, such that for each open set $U \subset X$, the map $\mathcal{F}(U) \to \mathcal{G}(U)$ is an $\mathcal{O}_X(U)$-module morphism.
\end{topic}

\begin{topic}{free-sheaf}{(locally) free sheaf}
    Let $(X, \mathcal{O}_X)$ be a \tref{ringed-space}{ringed space}. An \tref{O-module}{$\mathcal{O}_X$-module} $\mathcal{F}$ is \textbf{free} if it is isomorphic to a direct sum of copies of $\mathcal{O}_X$.
    
    It is \textbf{locally free} if $X$ can be covered by open sets $U$ for which $\mathcal{F}|_U$ is free. In that case, the \textit{rank} of $\mathcal{F}$ on such an open set is the number of copies of $\mathcal{O}_X$. If $X$ is connected, this rank is the same everywhere.
\end{topic}

\begin{topic}{invertible-sheaf}{invertible sheaf}
    Let $(X, \mathcal{O}_X)$ be a \tref{ringed-space}{ringed space}. An \textbf{invertible sheaf} $\mathcal{L}$ on $X$ is a \tref{free-sheaf}{locally free} \tref{O-module}{$\mathcal{O}_X$-modules} of rank 1.
\end{topic}

\begin{example}{invertible-sheaf}
    Let $D$ be a \tref{cartier-divisor}{Cartier divisor} on a scheme $X$, represented by a collection $\{ (U_i, f_i) \}$ of open subsets $U_i$ and $f_i \in K_X / \mathcal{O}_X^* (U_i)$, where $K_X$ denotes the \tref{sheaf-rational-functions}{sheaf of rational functions}. One defines the invertible sheaf $\mathcal{O}_X(D)$ as the subsheaf of $K_X$ generated as an $\mathcal{O}_X$-submodule by $1/f_i$ on $U_i$. Indeed this gives an invertible sheaf, and moreover this construction gives a canonical homomorphism of groups
    \[ \mathcal{O}_X(-) : \{ \textup{Cartier divisors} \} / \textup{linear equivalence} \to \textup{Pic}(X), \quad D \mapsto \mathcal{O}_X(D) . \]
    This map is injective on any scheme $X$, and when $X$ is \tref{integral-scheme}{integral} it is also surjective.
\end{example}

\begin{topic}{sheaf-of-ideals}{sheaf of ideals}
    Let $(X, \mathcal{O}_X)$ be a \tref{ringed-space}{ringed space}. A \textbf{sheaf of ideals} is an $\mathcal{O}_X$-module $\mathcal{I}$ which is a subsheaf of $\mathcal{O}_X$.
\end{topic}

\begin{topic}{sheaf-associated-to-module}{sheaf associated to module}
    Let $R$ be a ring and let $M$ be an $R$-module. The \textbf{sheaf associated} to $M$ on $X = \Spec R$, denoted $\tilde{M}$, is defined by the gluing data: to each distinguished open $X_f = \{ f \ne 0 \}$ is assigned the localized $R_f$-module $M_f$. For each $X_f \subset X_g$ there is the natural map $M_g \to M_f$. In particular, $\tilde{R} = \mathcal{O}_X$.
    
    For the projective case, let $S$ be a graded ring and $M$ a graded $S$-module. The \textbf{sheaf associated} to $M$ on $X = \Proj S$, denoted $\tilde{M}$, is defined by the gluing data: for each homogeneous $f \in S$, we have $\tilde{M}|_{\{ f \ne 0\}} \simeq (M_f)_0$.
\end{topic}

\begin{topic}{coherent-sheaf}{(quasi-)coherent sheaf}
    Let $X$ be a \tref{scheme}{scheme}. A \tref{O-module}{sheaf of $\mathcal{O}_X$-modules} $\mathcal{F}$ is \textbf{quasi-coherent} if $X$ can be covered by open affine subsets $U_i = \Spec R_i$, such that for each $i$, the restriction $\mathcal{F}|_{U_i}$ is isomorphic to $\tilde{M}_i$ for some $R_i$-module $M_i$.
    
    Furthermore, $\mathcal{F}$ is \textbf{coherent} is each $M_i$ can be taken to be a \tref{AA:finitely-generated-module}{finitely generated} $R_i$-module.
\end{topic}

\begin{topic}{ideal-sheaf}{ideal sheaf}
    Let $i : Y \to X$ be a \tref{closed-immersion}{closed immersion} of \tref{scheme}{schemes}. The \textbf{ideal sheaf} $\mathcal{I}$ of $Y$ is the kernel of $i^\# : \mathcal{O}_X \to i_* \mathcal{O}_Y$. In particular, this is a \tref{sheaf-of-ideals}{sheaf of ideals} on $X$.
    \[ 0 \to \mathcal{I} \to \mathcal{O}_X \to i_* \mathcal{O}_Y \to 0 . \]
\end{topic}

\begin{topic}{twisting-sheaf}{twisting sheaf}
    Let $S$ be a graded ring and let $X = \Proj S$. For any $n \in \ZZ$, the \textbf{twisting sheaf} $\mathcal{O}_X(n)$ is the \tref{sheaf-associated-to-module}{sheaf associated} to $S(n)$ (recall: $S(n)_d = S_{d + n}$).
    
    For any sheaf of \tref{O-module}{$\mathcal{O}_X$-modules} $\mathcal{F}$, the \textbf{twisted sheaf} $\mathcal{F}(n)$ is given by $\mathcal{F} \otimes_{\mathcal{O}_X} \mathcal{O}_X(n)$.
    
    Note that sheaves $\mathcal{O}_X(n)$ are all \tref{invertible-sheaf}{invertible} sheaves, and that $\mathcal{O}_X(n) \otimes \mathcal{O}_X(m) \simeq \mathcal{O}_X(n + m)$.
\end{topic}

\begin{topic}{external-tensor-product}{external tensor product}
    Let $X, Y \to S$ be \tref{scheme}{schemes}, $\mathcal{F}$ an \tref{O-module}{$\mathcal{O}_X$-module} and $\mathcal{G}$ an $\mathcal{O}_Y$-module. The \textbf{external tensor product} of $\mathcal{F}$ and $\mathcal{G}$ is the $\mathcal{O}_{X \times_S Y}$-module
    \[ \mathcal{F} \boxtimes \mathcal{G} = \pi_X^* \mathcal{F} \otimes_{\mathcal{O}_S} \pi_Y^* \mathcal{G} . \]
\end{topic}

\begin{topic}{sheaf-rational-functions}{sheaf of rational functions}
    Let $X$ be a \tref{scheme}{scheme}. The \textbf{sheaf of rational functions} $K_X$ on $X$ is the \tref{associated-sheaf}{sheafification} of the presheaf which assigns to an open $U \subset X$ the \tref{AA:localization}{localization} $S^{-1} \mathcal{O}_X(U)$, where $S$ is the multiplicative set of all $f \in \mathcal{O}_X(U)$ which are stalk-wise not zero-divisors.
\end{topic}

\begin{topic}{sheaf-cohomology}{sheaf cohomology}
    Let $X$ be a \tref{TO:topological-space}{topological space}. The \textit{global section functor} from \tref{sheaf}{sheaves} on $X$ to \tref{GT:abelian-group}{abelian groups},
    \[ \Gamma(X, -) : \textup{Sh}(X) \to \textbf{Ab} , \quad \mathcal{F} \mapsto \Gamma(X, \mathcal{F}) \]
    is \tref{HA:exact-functor}{left exact}, and since $\textup{Sh}(X)$ has enough injectives, one can form the \tref{HA:right-derived-functors}{right derived functors} $H^i(X, -) = \textup{R}^i \Gamma(X, -)$. The group
    \[ H^i(X, \mathcal{F}) \]
    is called the $i$-th \textbf{cohomology group} of $X$ with coefficients in $\mathcal{F}$.
\end{topic}

\begin{example}{sheaf-cohomology}
    Let $X = \Spec A$ for a \tref{AA:noetherian-ring}{noetherian ring} $A$, and $\mathcal{F}$ a \tref{coherent-sheaf}{quasi-coherent sheaf}. Then $H^i(X, \mathcal{F}) = 0$ for $i > 0$. Namely, write $M = \Gamma(X, \mathcal{F})$ and take an injective resolution
    \[ 0 \to M \to I^\bdot \]
    of $A$-modules. Using that localization is exact, from $\mathcal{F} = \widetilde{M}$ and the fact that each $\widetilde{I}^i$ can be shown to be flasque, we obtain a resolution
    \[ 0 \to \mathcal{F} \to \widetilde{I}^\bdot \]
    that can be used to compute the cohomology groups. Applying $\Gamma$ we return to the exact sequence $0 \to M \to I^\bdot$, and taking cohomology gives the desired result.
\end{example}

\begin{topic}{local-cohomology}{local cohomology}
    Let $X$ be a \tref{TO:topological-space}{topological space} and let $i : A \to X$ be the inclusion of a closed subset. Consider the functor
    \[ \Gamma_A(X, -) : \textup{Sh}(X) \to \textbf{Ab}, \quad \mathcal{F} \mapsto \Gamma_A(X, \mathcal{F}) = \{ s \in \Gamma(X, \mathcal{F}) \;|\;\textup{Supp}(s) \subset A \} \]
    of \textit{global sections with support in $A$}. This is a \tref{HA:exact-functor}{left exact functor}, and one can form the \tref{HA:right-derived-functors}{right derived functors} $H_A^i(X, -) = \textup{R}^i \Gamma_A(X, -)$. The group
    \[ H_A^i(X, \mathcal{F}) \]
    is called the $i$-th \textbf{local cohomology group} of $X$ with support in $A$ and coefficients in $\mathcal{F}$.
\end{topic}

\begin{example}{local-cohomology}
    Let $U$ be the complement of $A$ in $X$, and let $\mathcal{F}$ be a sheaf on $X$. Consider the exact sequence
    \[ 0 \to \Gamma_A(X, \mathcal{F}) \to \Gamma(X, \mathcal{F}) \to \Gamma(U, \mathcal{F}) , \]
    and note restriction from $U$ to $X$ is surjective when $\mathcal{F}$ is \tref{flasque-sheaf}{flasque}. Thus an \tref{HA:injective-resolution}{injective resolution} $\mathcal{F} \to \mathcal{I}^\bdot$ gives rise to a short exact sequence
    \[ 0 \to \Gamma_A(X, \mathcal{I}^\bdot) \to \Gamma(X, \mathcal{I}^\bdot) \to \Gamma(U, \mathcal{I}^\bdot) \to 0 , \]
    which gives rise to a long exact sequence
    \[ \cdots \to H_A^i(X, \mathcal{F}) \to H^i(X, \mathcal{F}) \to H^i(U, \mathcal{F}) \to H_A^{i + 1}(X, \mathcal{F}) \to \cdots \]
\end{example}

\begin{topic}{local-system}{local system}
    Let $X$ be a \tref{TO:topological-space}{topological space}. A \textbf{local system} (of abelian groups, modules, ...) on $X$ is a \tref{constant-sheaf}{locally constant sheaf} (of abelian groups, modules, ...) $\mathcal{L}$ on $X$.
\end{topic}
