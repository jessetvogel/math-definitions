\begin{topic}{algebraic-cycle}{algebraic cycle}
    Let $X$ be a \tref{scheme}{scheme} over a field $k$. The group of \textbf{algebraic $r$-cycles} on $X$ is the \tref{GT:free-group}{free abelian group} on $r$-dimensional \tref{closed-immersion}{closed} \tref{integral-scheme}{integral} subschemes $Z$ of $X$,
    \[ Z_r(X) = \bigoplus_{Z \subset X} \ZZ \cdot [Z] . \]
    The group of \textbf{algebraic cycles} is the \tref{AA:graded-module}{graded} group
    \[ Z(X) = \bigoplus_{r \ge 0} Z_r(X) . \]
    An algebraic cycle $\sum_i n_i [Z_i] \in Z(X)$ is \textbf{effective} if $n_i \ge 0$ for all $i$.
\end{topic}

\begin{topic}{rational-equivalence}{rational equivalence}
    Two \tref{algebraic-cycle}{algebraic cycles} $Z, Z'$ on a \tref{variety}{variety} $X$ are \textbf{rationally equivalent} if there is an algebraic cycle $W$ on $X \times \PP^1$ \tref{flat-morphism}{flat} over $\PP^1$, such that
    \[ [Z] - [Z'] = [W \cap X \times \{ 0 \}] - [W \cap X \times \{ \infty \}] . \]
\end{topic}

\begin{topic}{chow-group}{Chow group}
    Let $X$ be a \tref{noetherian-scheme}{noetherian scheme}. The $r$-th \textbf{Chow group} of $X$ is the quotient of $Z_r(X)$, the group of \tref{algebraic-cycle}{algebraic $r$-cycles} on $X$, by the subgroup of $r$-cycles \tref{rational-equivalence}{rationally equivalent} to zero.
    \[ \textup{CH}_r(X) = Z_r(X) / \sim{}_\textup{rat} \]
\end{topic}

\begin{topic}{algebraic-equivalence}{algebraic equivalence}
    Two \tref{algebraic-cycle}{algebraic cycles} $Z, Z'$ on a \tref{variety}{variety} $X$ are \textbf{algebraically equivalent} if there is an \tref{algebraic-curve}{curve} $C$ and an algebraic cycle $W$ on $X \times C$ \tref{flat-morphism}{flat} over $C$, such that
    \[ [Z] - [Z'] = [W \cap X \times \{ c_1 \}] - [W \cap X \times \{ c_2 \}] \]
    for two points $c_1, c_2$ on $C$.
    % The \tref{neron-severi-group}{Néron--Severi group}.
\end{topic}

\begin{topic}{chow-correspondence}{Chow correspondence}
    Let $X$ and $Y$ be \tref{scheme}{schemes} over a field $k$. A \textbf{Chow correspondence} from $X$ to $Y$ is an element from the \tref{chow-group}{Chow group} $\textup{CH}_{\dim X}(X \times_k Y)$.
    
    The \textbf{category of Chow correspondences} over $k$ is the \tref{CT:category}{category} $\textup{Corr}(k)$ whose objects are \tref{smooth-morphism}{smooth} \tref{projective-morphism}{projective} schemes over $k$, and whose morphisms are given by Chow correspondences. Composition is given by
    \[ \beta \circ \alpha = \pi_{XZ*}(\pi_{XY}^*(\alpha) \cdot \pi_{YZ}^*(\beta)) \]
    for all $\alpha \in \textup{CH}_{\dim X}(X \times_k Y)$ and $\beta \in \textup{CH}_{\dim Y}(Y \times_k Z)$, and for any $X$ the identity $\id_X$ is given by the diagonal $\Delta_X(X) \in \textup{CH}_{\dim X}(X, X)$.
\end{topic}

\begin{topic}{chow-motive}{Chow motive}
    Let $k$ be a field. The category of \textbf{effective pure Chow motives} over $k$, denoted $\textup{Chow}^\textup{eff}(k)$, is the \tref{CT:karoubi-envelope}{idempotent completion} of the \tref{chow-correspondence}{category of Chow correspondences} $\textup{Corr}(k)$, that is,
    \begin{itemize}
        \item the objects are pairs $(X, p)$ with $X$ a smooth projective scheme over $k$, and $p \in \textup{CH}_{\dim X}(X, X)$ with $p \circ p = p$,
        \item the morphisms $f : (X, p) \to (Y, q)$ are correspondences $f \in \textup{CH}_{\dim X}(X \times_k Y)$ such that $f \circ p = f = q \circ f$.
    \end{itemize}
    This category has the structure of a \tref{CT:monoidal-category}{monoidal category} given by
    \[ (X, p) \otimes (Y, q) = (X \times_k Y, p \times q) , \]
    with unit $\textbf{1} = (\Spec k, \Gamma_{\id})$.
    
    Consider the morphism $f : \PP^1_k \to \Spec k$ and a point $x : \Spec k \to \PP^1_k$. Since $x \circ f$ is idempotent, and every idempotent splits, $(\PP^1_k, \Gamma_\id) = \textbf{1} \oplus \mathbb{L}$, for some object $\mathbb{L}$ called the \textit{Lefschetz motive}.
    
    The category of \textbf{pure Chow motives} $\textup{Chow}(k)$ is obtained by adjoining a formal inverse $\mathbb{L}^{-1}$ to $\textup{Chow}^\textup{eff}(k)$.
\end{topic}

\begin{example}{chow-motive}
    For any morphism $f : X \to Y$ between smooth projective schemes over $k$, the graph $\Gamma_f = \{ (x, y) \in X \times_k Y \mid y = f(x) \}$ defines a correspondence $\Gamma_f \in \textup{CH}_{\dim X}(X \times_k Y)$. For morphisms $f : X \to Y$ and $g : Y \to Z$, one can check that $\Gamma_g \circ \Gamma_f =\Gamma_{g \circ f}$. In particular, there is a functor
    \[ \textbf{SmProj}_k \to \textup{Chow}^\textup{eff}(k) , \]
    which sends a smooth projective scheme $X$ over $k$ to $(X, \Gamma_{\id_X})$, and a morphism $f : X \to Y$ to $\Gamma_f$.
\end{example}

\begin{topic}{finite-correspondence}{finite correspondence}
    Let $k$ be a field. For any \tref{scheme}{schemes} $X$ and $Y$ over $k$, the group of \textbf{finite correspondences} from $X$ to $Y$ is the subgroup
    \[ C(X, Y) \subset Z(X \times_k Y) \]
    of \tref{algebraic-cycle}{algebraic cycles} generated by integral closed subschemes $W \subset X \times_k Y$ such that $\pi_X : W \to X$ is \tref{finite-morphism}{finite} and $\pi_X(W) \subset X$ is an \tref{TO:irreducible-space}{irreducible component} of $X$.
    
    The category of \textbf{finite correspondences} over $k$ is the \tref{CT:category}{category} $\textup{FinCorr}(k)$ whose objects are \tref{smooth-morphism}{smooth} \tref{scheme}{schemes} over $k$, and whose morphisms from $X$ to $Y$ are given by $C(X, Y)$.
\end{topic}

\begin{topic}{suslin-homology}{Suslin homology}
    Let $X$ be a \tref{smooth-morphism}{smooth} scheme over a field $k$. The \textbf{Suslin complex} of $X$ is the \tref{HA:chain-complex}{complex} given by \tref{finite-correspondence}{finite correspondences}
    \[ C_n(X) = \textup{FinCorr}(\Delta^n, X) , \]
    from $\Delta^n = \Spec k[t_0, \ldots, t_n] / (t_0 + \cdots + t_n - 1)$ to $X$, with differentials
    \[ d_n = \sum_{i = 0}^{n} (-1)^i \delta_{n - 1, i}^* : C_n(X) \to C_{n - 1}(X) , \]
    where $\delta_{n, i} : \Delta^n \to \Delta^{n + 1}$ is given by $(t_0, \ldots, t_n) \mapsto (t_0, \ldots, t_i, 0, t_{i + 1}, \ldots, t_n)$.
    The \textbf{Suslin homology} of $X$ is the \tref{HA:homology-object}{homology}
    \[ H^\textup{Sus}_i(X) = H_i(C_\bdot(X)) . \]
\end{topic}

\begin{topic}{voevodsky-motive}{Voevodsky motive}
    Let $k$ be a \tref{AA:field}{field}. The \textit{category of geometric motives} $\textup{DM}_\textup{gm}(k)$ is constructed as follows.
    \begin{enumerate}[(i)]
        \item Let $\textup{FinCorr}(k)$ be the category of \tref{finite-correspondence}{finite correspondences} over $k$.
        \item Let $\widehat{\textup{DM}}_\textup{gm}^\textup{eff}(k)$ be the category obtained from the bounded \tref{HA:homotopy-category}{homotopy category} $\textbf{K}^\textup{b}(\textup{FinCorr}(k))$ by \tref{CT:localization}{localizing} with respect to
        \begin{itemize}
            \item (\textit{$\AA^1$-invariance}) $\pi_X : X \times_k \AA^1_k \to X$ for all smooth schemes $X$ over $k$,
            \item (\textit{Mayer--Vietoris}) for all smooth schemes $X$ over $k$ and Zariski opens $j_U : U \hookrightarrow X$ and $j_V : V \hookrightarrow X$ with $X = U \cup V$, the map
            \[ \operatorname{Cone}(U \cap V \xrightarrow{j} U \oplus V) \xrightarrow{j_{U*} - j_{V*}} X , \]
            where $j$ is induced by the inclusions of $U \cap V$ into $U$ and $V$.
        \end{itemize}
        \item Define the category of \textbf{effective geometric motives} $\textup{DM}_\textup{gm}^\textup{eff}(k)$ as the \tref{CT:karoubi-envelope}{idempotent completion} of $\widehat{\textup{DM}}_\textup{gm}^\textup{eff}(k)$.
        \item Consider the projection $\pi : \PP^1_k \to \Spec k$ and any section $\sigma : \Spec k \to \PP^1_k$. Writing $\ZZ = \Spec k$, note that the exact triangle
        \[ \begin{tikzcd} \operatorname{Cone}(\pi)[-1] \arrow{r} & \PP^1_k \arrow[swap]{r}{\pi} & \ZZ \arrow{r} \arrow[shift right=0.4em, swap]{l}{\sigma} & \operatorname{Cone}(\pi) \end{tikzcd}  \]
        splits, so that $\PP^1_k \cong \ZZ \oplus \operatorname{Cone}(\pi)[-1]$. Define $\ZZ(1) = \operatorname{Cone}(\pi)[-3]$ so that $\PP^1_k \cong \ZZ \oplus \ZZ(1)[2]$. Furthermore, define $\ZZ(n) = \ZZ(1)^{\otimes n}$ for $n \ge 0$.
        \item The category of \textbf{geometric motives} $\textup{DM}_\textup{gm}(k)$ is defined by inverting the functor $(-) \otimes \ZZ(1)$ on $\textup{DM}_\textup{gm}^\textup{eff}(k)$. That is, its objects are of the form $X(n)$, with $X \in \textup{DM}_\textup{gm}^\textup{eff}(k)$ and $n \in \ZZ$, and the morphisms are given by
        \[ \Hom_{\textup{DM}_\textup{gm}(k)}(X(n), Y(m)) = \varinjlim_{N} \Hom_{\textup{DM}_\textup{gm}^\textup{eff}(k)}(X \otimes \ZZ(n + N), Y \otimes \ZZ(m + N)) . \]
    \end{enumerate}
    The natural functor $i : \textup{DM}_\textup{gm}^\textup{eff}(k) \to \textup{DM}_\textup{gm}(k)$ that sends $X$ to $X(0)$ is \tref{CT:full-functor}{fully} \tref{CT:faithful-functor}{faithful}, and the natural map $i(X \otimes \ZZ(n)) \to X(n)$ is an isomorphism.
\end{topic}

\begin{example}{voevodsky-motive}
    Let us prove inductively that $M_\textup{gm}(\AA^n_k \setminus \{ 0 \}) \cong \ZZ \oplus \ZZ(n)[2n - 1]$ for all $n \ge 1$.
    
    For $n = 1$, apply the Mayer--Vietoris sequence to the usual covering $\PP^1_k = \AA^1_k \cup \AA^1_k$ to find
    \[ M_\textup{gm}(\AA^1_k \setminus \{ 0 \}) \to M_\textup{gm}(\AA^1_k) \oplus M_\textup{gm}(\AA^1_k) \xrightarrow{\left(\begin{smallmatrix} j_1 & -j_2 \end{smallmatrix}\right)} M_\textup{gm}(\PP^1_k) \to M_\textup{gm}(\AA^1_k \setminus \{ 0 \})[1] . \]
    Recall that $M_\textup{gm}(\AA^1_k) = M_\textup{gm}(\Spec k) = \ZZ$ and $M_\textup{gm}(\PP^1) = \ZZ \oplus \ZZ(1)[2]$. Note that $j_1, j_2 : M_\textup{gm}(\AA^1_k) \to M_\textup{gm}(\PP^1_k)$ both factor through $M_\textup{gm}(\Spec k) = \ZZ$ as $M_\textup{gm}(\AA^1_k \to \Spec k)$ is an isomorphism. Therefore, it follows that $M_\textup{gm}(\AA^1_k \setminus \{ 0 \}) = \operatorname{Cone}\left(\begin{smallmatrix} j_1 & -j_2 \end{smallmatrix}\right)[-1] = \ZZ \oplus \ZZ(1)[1]$.
    
    For $n \ge 2$, we use the covering $\AA^n_k \setminus \{ 0 \} = (\AA^n_k \setminus \AA^{n - 1}_k) \cup (\AA^n_k \setminus \AA^1_k)$, the fact that $(\AA^n_k \setminus \AA^{n - 1}_k) \cap (\AA^n_k \setminus \AA^1_k) = (\AA^{n - 1}_k \setminus \{ 0 \}) \times (\AA^1_k \setminus \{ 0 \})$, and the isomorphisms $M_\textup{gm}(\AA^n_k \setminus \AA^{n - 1}_k) \cong M_\textup{gm}(\AA^1_k \setminus \{ 0 \})$ and $M_\textup{gm}(\AA^n_k \setminus \AA^1_k) \cong M_\textup{gm}(\AA^{n - 1}_k \setminus \{ 0 \})$, in order to obtain the Mayer--Vietoris sequence
    \[ (\ZZ \oplus \ZZ(1)[1]) \otimes (\ZZ \oplus \ZZ(n - 1)[2n - 3]) \to (\ZZ \oplus \ZZ(1)[1]) \oplus (\ZZ \oplus \ZZ(n - 1)[2n - 3]) \]
    \[ \to M_\textup{gm}(\AA^n_k \setminus \{ 0 \}) \to (\ZZ \oplus \ZZ(1)[1]) \otimes (\ZZ \oplus \ZZ(n - 1)[2n - 3])[1] . \]
    The first map can be written down explicitly, so that after expanding brackets it follows that $M_\textup{gm}(\AA^n_k \setminus \{ 0 \}) \cong \ZZ \oplus \ZZ(n)[2n - 1]$.
\end{example}

\begin{topic}{motivic-cohomology}{motivic cohomology}
    Let $X$ be a \tref{smooth-morphism}{smooth} \tref{scheme}{scheme} of \tref{finite-type}{finite type} over a field $k$. The \textbf{motivic cohomology} of $X$ is given by
    \[ H^p(X, \ZZ(q)) = \Hom_{\textup{DM}_\textup{gm}(k)}(X, \ZZ(q)[p]) \]
    for any $p, q \ge 0$, where $\textup{DM}_\textup{gm}(k)$ denotes the category of \tref{voevodsky-motive}{geometric motives}.
\end{topic}
