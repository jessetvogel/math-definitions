\begin{topic}{sigma-algebra}{sigma algebra}
    A \textbf{$\sigma$-algebra} on a set $\Omega$ is a collection $\mathcal{A}$ of subsets of $\Omega$, satisfying
    \begin{itemize}
        \item $\Omega \in \mathcal{A}$,
        \item $\Omega \setminus A \in \mathcal{A}$ for all $A \in \mathcal{A}$,
        \item $\bigcup_{n = 1}^\infty A_n \in \mathcal{A}$ for all sequences $(A_n)_{n \in \NN} \in \mathcal{A}$.
    \end{itemize}
\end{topic}

\begin{topic}{measurable-space}{measurable space}
    A \textbf{measurable space} is a pair $(\Omega, \mathcal{A})$, where $\Omega$ is a set and $\mathcal{A}$ a \tref{sigma-algebra}{$\sigma$-algebra} on $\Omega$. The elements of $\mathcal{A}$ are called \textit{measurable sets}.
\end{topic}

\begin{topic}{measure}{measure}
    Let $\mathcal{A}$ be a \tref{sigma-algebra}{$\sigma$-algebra}. A \textbf{measure} on $\mathcal{A}$ is a function $\mu : \mathcal{A} \to [0, \infty]$ satisfying
    \begin{itemize}
        \item (\textit{measure empty set}) $\mu(\varnothing) = 0$,
        \item (\textit{countable additivity}) $\mu(\bigcup_{n = 1}^\infty A_n) = \sum_{n = 1}^\infty \mu(A_n)$ for all $A_n \in \mathcal{A}$ with $n \in \NN$ pairwise disjoint.
    \end{itemize}
\end{topic}

\begin{example}{measure}
    Let $\mathcal{A}$ be any $\sigma$-algebra on a set $\Omega$, and take $\omega \in \Omega$. The \textit{Dirac measure} is given by
    \[ \delta_\omega(A) = \left\{ \begin{array}{cl} 1 & \textup{ if } \omega \in A, \\ 0 & \textup{ if } \omega \not\in A . \end{array} \right. \]
\end{example}

\begin{example}{measure}
    Let $\mathcal{A} = P(\Omega)$ for a set $\Omega$. The \textit{counting measure} on $\mathcal{A}$ is given by $\mu(A) = |A|$.
\end{example}

\begin{topic}{measure-space}{measure space}
    A \textbf{measure space} is a triple $(\Omega, \mathcal{A}, \mu)$ consisting of a set $\Omega$, a \tref{sigma-algebra}{$\sigma$-algebra} $\mathcal{A}$ on $\Omega$, and a \tref{measure}{measure} $\mu : \mathcal{A} \to [0, \infty]$. A measure space is \textit{finite} if $\mu(\Omega) < \infty$.
\end{topic}

\begin{topic}{borel-sigma-algebra}{Borel sigma algebra}
    Let $X$ be a \tref{TO:topological-space}{topological space}. The \textbf{Borel $\sigma$-algebra} is the \tref{sigma-algebra}{$\sigma$-algebra} on $X$ generated by all open sets of $X$. Its elements are called \textbf{Borel subsets}.
\end{topic}

\begin{topic}{dynkin-system}{Dynkin system}
    A \textbf{Dynkin system} on a set $\Omega$ is a collection $\mathcal{D}$ of subsets of $\Omega$, satisfying
    \begin{itemize}
        \item $\Omega \in \mathcal{D}$,
        \item $\Omega \setminus A \in \mathcal{D}$ for all $A \in \mathcal{D}$,
        \item $\bigcup_{n = 1}^\infty A_n \in \mathcal{D}$ for pairwise disjoint $A_n \in \mathcal{A}$ with $n \in \NN$.
    \end{itemize}
\end{topic}

\begin{topic}{complete-measure-space}{complete measure space}
    A \tref{measure-space}{measure space} $(\Omega, \mathcal{A}, \mu)$ is \textbf{complete} if for every $A \in \mathcal{A}$ with $\mu(A) = 0$, all subsets of $A$ are contained in $\mathcal{A}$.
\end{topic}

\begin{topic}{measurable-function}{measurable function}
    Let $(\Omega, \mathcal{A})$ and $(\Omega', \mathcal{A}')$ be \tref{measurable-space}{measurable spaces}. A function $f : \Omega \to \Omega'$ is \textbf{measurable}, or $(\mathcal{A}, \mathcal{A}')$-measurable, if $f^{-1}(A') \in \mathcal{A}$ for all $A' \in \mathcal{A}'$.
\end{topic}

\begin{topic}{haar-measure}{Haar measure}
    Let $G$ be a \tref{TO:locally-compact-space}{locally compact} \tref{TO:hausdorff-space}{Hausdorff} \tref{TO:topological-group}{topological group}, considered with its \tref{borel-sigma-algebra}{Borel $\sigma$-algebra}. A \textbf{left Haar measure} is a \tref{measure}{measure} $\mu$ satisfying
    \begin{itemize}
        \item (\textit{left invariant}) $\mu(gS) = \mu(S)$ for all $g \in G$ and Borel sets $S \subset G$,
        \item (\textit{compact sets}) $\mu(K) < \infty$ for all \tref{TO:compact-space}{compact} $K \subset G$,
        \item (\textit{outer regular}) $\mu$ is \textit{outer regular} on Borel sets $S \subset G$, that is,
        \[ \mu(S) = \inf \{ \mu(U) \;:\; S \subset U, \; U \subset G \textup{ open} \} . \]
        \item (\textit{inner regular}) $\mu$ is inner regular on open sets $U \subset G$, that is,
        \[ \mu(S) = \sup \{ \mu(K) \;:\; K \subset U, \; K \textup{ compact} \} . \]
    \end{itemize}
    \textbf{Haar's theorem} states that there exists, up to a positive multiplicative constant, a unique non-trivial left Haar measure $\mu$ on the Borel subsets of $G$.
\end{topic}

\begin{example}{haar-measure}
    \begin{itemize}
        \item A Haar measure on a \tref{TO:discrete-topology}{discrete} group $G$ is the counting measure $\mu(S) = |S|$.
        \item A Haar measure on the circle group $S^1$ is the measure $\mu(S) = \frac{1}{2 \pi} m(\pi^{-1}(S))$, where $f : [0, 2 \pi] \to S^1$ is the map $f(t) = (\cos t, \sin t)$, and $m$ is the \tref{lebesgue-measure}{Lebesgue measure} on $[0, 2 \pi]$.
        \item A Haar measure on the multiplicative group $(\RR_{> 0}, \cdot)$ is given by $\mu(S) = \int_S \frac{dt}{t}$. For example, $\mu([a, b]) = \log(b/a)$, and $\mu([ca, cb]) = \log(cb/(ca)) = \log(b/a)$.
        \item A Haar measure on the general linear group $\textup{GL}_n(\RR)$ is given by $\mu(S) = \int_S \frac{1}{|\det(X)|^n} dX$, where $dX$ denotes the Lebesgue measure on $\RR^{n^2}$.
    \end{itemize}
\end{example}

\begin{topic}{lebesgue-measure}{Lebesgue measure}
    The \textbf{Lebesgue outer measure} is the \tref{outer-measure}{outer measure} $\mu^*$ on $\RR^n$ given by
    \[ \mu^*(A) = \inf \left\{ \sum_{n = 1}^{\infty} \ell(R_i) \;:\; R_i \subset \RR^n \textup{ closed rectangles}, A \subset \bigcup_{i = 1}^{\infty} R_i \right\} \in [0, \infty] , \]
    for all $A \subset \RR^n$, where a \textit{closed rectangle} is a product of closed intervals $R_i = [a_1, b_1] \times \cdots \times [a_n, b_n]$ with volume $\ell(R_i) =  (b_1 - a_1) \cdots (b_n - a_n)$.
\end{topic}

\begin{topic}{outer-measure}{outer measure}
    Let $\Omega$ be a set and $\mathcal{P}(\Omega)$ its power set. An \textbf{outer measure} on $\Omega$ is a mapping $\mu^* : \mathcal{P}(\Omega) \to [0, \infty]$ such that
    \begin{itemize}
        \item $\mu^*(\varnothing) = 0$,
        \item $\mu^*(A) \le \mu^*(B)$ for all $A \subset B \subset \Omega$,
        \item $\mu^*\left(\cup_{n = 1}^{\infty} A_n \right) \le \sum_{n = 1}^{\infty} \mu^*(A_n)$ for all sequences $(A_n)_{n \in \NN}$ of subsets of $\Omega$.
    \end{itemize}
\end{topic}

\begin{example}{outer-measure}
    An outer measure $\mu^*$ on $\Omega$ can be used to define a \tref{sigma-algebra}{$\sigma$-algebra} on $\Omega$. Let $\mathcal{A}^*$ be the set of subsets $A \subset \Omega$ such that
    \[ \mu^*(Z) = \mu^*(Z \cap A) + \mu^*(Z \cap (\Omega \setminus A)) , \]
    for all $Z \subset \Omega$. Then $\mathcal{A}^*$ is a $\sigma$-algebra on $\Omega$, and the restriction of $\mu^*$ to $\mathcal{A}^*$ is a \tref{complete-measure-space}{complete measure}.
\end{example}
