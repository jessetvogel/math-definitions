% Defining schemes
\begin{topic}{ringed-space}{ringed space}
    A \textbf{ringed space} is a pair $(X, \mathcal{O}_X)$ consisting of a \tref{TO:topological-space}{topological space} $X$ and a \tref{sheaf}{sheaf} of rings $\mathcal{O}_X$ on $X$.
    
    A morphism of ringed spaces from $(X, \mathcal{O}_X)$ to $(Y, \mathcal{O}_Y)$ is a pair $(f, f^\#)$ of a continuous map $f : X \to Y$ and a map $f^\# : \mathcal{O}_Y \to f_* \mathcal{O}_X$ of sheaves of rings on $Y$.
    
    A ringed space $(X, \mathcal{O}_X)$ is a \textbf{locally ringed space} if for each point $x \in X$, the \tref{stalk}{stalk} $\mathcal{O}_{X,x}$ is a \tref{CA:local-ring}{local ring}.
    
    A morphism of locally ringed spaces is a morphism $(f, f^\#)$ of ringed spaces such that for each point $x \in X$ the induced map of local rings $f^\#_x : \mathcal{O}_{Y, f(x)} \to \mathcal{O}_{X, x}$ is a \textit{local morphism}, i.e. the pre-image of the maximal ideal of $\mathcal{O}_{X, x}$ is the maximal ideal of $\mathcal{O}_{Y, f(x)}$.
\end{topic}

\begin{topic}{spectrum}{spectrum}
    Let $R$ be a commutative ring. The \textbf{spectrum} of $R$ is the \tref{ringed-space}{locally ringed space} $(X, \mathcal{O}_X)$ defined as follows.
    \begin{itemize}
        \item The topological space $X$ is the set of \tref{CA:prime-ideal}{prime ideals} of $R$, whose closed sets are given precisely by the sets $V(I) = \{ \textup{prime ideals } \mathfrak{p} \text{ with } \mathfrak{p} \supset I \}$ for all ideals $I$ of $R$.
        
        \item The \tref{sheaf}{sheaf} of rings $\mathcal{O}_X$ is given as follows. For each open set $U \subset X$, $\mathcal{O}_X(U)$ is the set of functions $s : U \to \sqcup_{\mathfrak{p} \in U} R_\mathfrak{p}$ with $s(\mathfrak{p}) \in R_\mathfrak{p}$ for each $\mathfrak{p} \in U$, such that $s$ is locally a quotient of elements of $R$. This means that for each $\mathfrak{p} \in U$, there is a neighborhood $V \subset U$ of $\mathfrak{p}$, and elements $a, f \in R$ such that for each $\mathfrak{q} \in V$, $f \not\in \mathfrak{q}$ and $s(\mathfrak{q}) = a/f$ in $A_\mathfrak{q}$. The sets $\mathcal{O}_X(U)$ are indeed rings.
    \end{itemize}
    The spectrum of $R$ is denoted $\Spec R$.
\end{topic}

\begin{topic}{affine-scheme}{affine scheme}
    An \textbf{affine scheme} is a \tref{ringed-space}{locally ringed space} $(X, \mathcal{O}_X)$ which is isomorphic to the \tref{spectrum}{spectrum} of some ring.
\end{topic}

\begin{topic}{scheme}{scheme}
    A \textbf{scheme} is a \tref{ringed-space}{locally ringed space} $(X, \mathcal{O}_X)$ in which every point has an open neighborhood $U$ such that $(U, \mathcal{O}_X|_U)$ is an \tref{affine-scheme}{affine scheme}. A morphism of schemes is a morphism of locally ringed spaces.
    
    One calls $X$ the \textit{underlying topological space}, and $\mathcal{O}_X$ its \textit{structure sheaf}.
\end{topic}

% Scheme properties
\begin{topic}{reduced-scheme}{reduced scheme}
    A \tref{scheme}{scheme} $X$ is \textbf{reduced} if for every open subset $U \subset X$ the ring $\mathcal{O}_X(U)$ has no \tref{CA:nilpotent}{nilpotent} elements.
    
    Equivalently, this is the case if all stalks $\mathcal{O}_{X, x}$ have no nilpotent elements.
\end{topic}

\begin{topic}{integral-scheme}{integral scheme}
    A \tref{scheme}{scheme} $X$ is \textbf{integral} if for every open subset $U \subset X$ the ring $\mathcal{O}_X(U)$ is a \tref{CA:domain}{domain}.
    
    Equivalently, this is the case if $X$ is \tref{reduced-scheme}{reduced} and \tref{TO:irreducible}{irreducible}.
\end{topic}

\begin{topic}{noetherian-scheme}{noetherian scheme}
    A \tref{scheme}{scheme} $X$ is \textbf{locally noetherian} if it can be covered by open affine subsets $\Spec A_i$, where each $A_i$ is a \tref{CA:noetherian-ring}{noetherian ring}. It is \textbf{noetherian} if it is locally noetherian and \tref{TO:quasi-compact}{quasi-compact}.
\end{topic}

\begin{topic}{finite-type}{finite type}
    A morphism $f : X \to Y$ of \tref{scheme}{schemes} is \textbf{of finite type at $x \in X$} if there exist affine opens $U = \Spec A \subset X$ containing $x$ and $V = \Spec B \subset Y$ with $f(U) \subset V$ such that $A$ is a \tref{CA:finite-type}{finitely generated} $B$-algebra (via the induced map $B \to A$).
    
    The morphism $f$ is \textbf{locally of finite type} if it is of finite type at each $x \in X$, and it is \textbf{of finite type} if it is locally of finite type and \tref{TO:quasi-compact}{quasi-compact}.
\end{topic}

\begin{topic}{finite}{finite}
    A morphism $f : X \to Y$ of \tref{scheme}{schemes} is \textbf{finite} if there exists a covering of $Y$ by open affine subsets $V_i = \Spec B_i$, such that for each $i$, $f^{-1}(V_i)$ is affine, equal to $\Spec A_i$, where $A_i$ is a $B_i$-algebra which is a finitely generated $B_i$-module.
\end{topic}

\begin{topic}{open-immersion}{open immersion}
    An \textbf{open immersion} is a morphism $i : U \to X$ of \tref{scheme}{schemes} which induces an isomorphism of $U$ with an open subscheme of $X$.
\end{topic}

\begin{topic}{closed-immersion}{closed immersion}
    A \textbf{closed immersion} is a morphism $i : Z \to X$ of \tref{scheme}{schemes} such that $i$ induces a \tref{TO:homeomorphism}{homeomorphism} of the underlying space of $Z$ onto a closed subset of that of $X$, and furthermore the induced map $i^\# : \mathcal{O}_X \to i_*\mathcal{O}_Z$ of sheaves on $X$ is surjective.
\end{topic}

\begin{topic}{immersion}{immersion}
    An \textbf{immersion} is a morphism $Y \to X$ of \tref{scheme}{schemes} which factors as $j \circ i$, where $i$ is a \tref{closed-immersion}{closed immersion} and $j$ is an \tref{open-immersion}{open immersion}.
    \[ Y \xrightarrow{i} U \xrightarrow{j} X \]
\end{topic}

\begin{topic}{affine-morphism}{affine morphism}
    A morphism $f : X \to Y$ of \tref{scheme}{schemes} is \textbf{affine} if the inverse image of every affine open in $Y$ is an affine open of $X$. 
\end{topic}

\begin{topic}{normal-scheme}{normal scheme}
    A \tref{scheme}{scheme} $X$ is \textbf{normal} if all of its local rings are integrally closed domains.
    
    Every scheme can be `normalized'.
\end{topic}

\begin{topic}{separated}{(quasi) separated}
    A morphism $f : X \to Y$ of \tref{scheme}{schemes} is \textbf{quasi-separated} if the diagonal $\Delta_{X/Y} : X \to X \times_Y X$ is quasi-compact, and it is \textbf{separated} if the diagonal $\Delta_{X/Y}$ is a closed immersion.
    
    A scheme $X$ is \textbf{(quasi-)separated} if the morphism $X \to \Spec \ZZ$ is (quasi-)separated.
\end{topic}

\begin{example}{separated}
    Any morphism of affine schemes is separated. Namely, for any $f : \Spec A \to \Spec B$, the diagonal $\Delta$ corresponds to the ring map $A \otimes_B A \to A : a \otimes a' \mapsto aa'$, which is clearly surjective. In this way $A$ is seen as a quotient of $A \otimes_B A$, so $\Delta$ is a closed immersion.
\end{example}

\begin{example}{separated}
    Let $X$ be the affine line with two origins, that is, glue two copies of $\AA_k^1$ along $\AA_k^1 - \{ 0 \}$. We claim that $X$ is not separated over $k$. Namely, consider the map $\AA^1_k \to X \times_k X$ obtained from the two different maps $\AA_k^1 \to X$. Then the inverse image of the diagonal is $\AA_k^1 - \{ 0 \}$, which is not closed. Hence, the diagonal is not closed and $\Delta$ is not a closed immersion.
\end{example}

\begin{topic}{proper-morphism}{proper-morphism}
    A morphism $f : X \to Y$ of \tref{scheme}{schemes} is \textbf{proper} if it is \tref{separated}{separated}, \tref{finite-type}{of finite type} and \tref{TO:universally-closed}{universally closed}.
\end{topic}

\begin{topic}{projective-morphism}{(quasi) projective morphism}
    A morphism $f : X \to Y$ of \tref{scheme}{schemes} is \textbf{projective} if it factors as
    \[ X \xrightarrow{i} \PP^n_Y \xrightarrow{p} Y , \]
    with $i$ a \tref{closed-immersion}{closed immersion} and $p$ the projection.
    
    It is \textbf{quasi-projective} if it factors as
    \[ X \xrightarrow{j} X' \xrightarrow{g} Y , \]
    with $j$ an \tref{open-immersion}{open immersion} and $g$ a projective morphism.
\end{topic}

\begin{topic}{variety}{variety}
    An (abstract) \textbf{variety} is an \tref{integral-scheme}{integral} \tref{separated}{separated} \tref{scheme}{scheme} of \tref{finite-type}{finite type} over an algebraically closed field $k$.
\end{topic}

\begin{topic}{complete-variety}{complete variety}
    A \tref{variety}{variety} over $k$ is \textbf{complete} if it is \tref{proper-morphism}{proper} over $k$.
\end{topic}

% -- O_X modules --
\begin{topic}{O-module}{sheaf of O-modules}
    Let $(X, \mathcal{O}_X)$ be a \tref{ringed-space}{ringed space}. A \textbf{sheaf of $\mathcal{O}_X$-modules}, or simply $\mathcal{O}_X$-module, is a \tref{sheaf}{sheaf} $\mathcal{F}$ on $X$, such that for each open set $U \subset X$, the group $\mathcal{F}(U)$ is an $\mathcal{O}_X(U)$-module, and for each inclusion of open sets $V \subset U$, the restriction morphism $\mathcal{F}(U) \to \mathcal{F}(V)$ is an $\mathcal{O}_X(U)$-module morphism (here $\mathcal{F}(V)$ is seen as an $\mathcal{O}_X(U)$-module via $\mathcal{O}_X(U) \to \mathcal{O}_X(V)$).
    
    A morphism of $\mathcal{O}_X$-modules $\mathcal{F} \to \mathcal{G}$ is a morphism of sheaves, such that for each open set $U \subset X$, the map $\mathcal{F}(U) \to \mathcal{G}(U)$ is an $\mathcal{O}_X(U)$-module morphism.
\end{topic}

\begin{topic}{free-sheaf}{(locally) free sheaf}
    Let $(X, \mathcal{O}_X)$ be a \tref{ringed-space}{ringed space}. An \tref{O-module}{$\mathcal{O}_X$-module} $\mathcal{F}$ is \textbf{free} if it is isomorphic to a direct sum of copies of $\mathcal{O}_X$.
    
    It is \textbf{locally free} if $X$ can be covered by open sets $U$ for which $\mathcal{F}|_U$ is free. In that case, the \textit{rank} of $\mathcal{F}$ on such an open set is the number of copies of $\mathcal{O}_X$. If $X$ is connected, this rank is the same everywhere.
\end{topic}

\begin{topic}{invertible-sheaf}{invertible sheaf}
    Let $(X, \mathcal{O}_X)$ be a \tref{ringed-space}{ringed space}. An \textbf{invertible sheaf} $\mathcal{L}$ on $X$ is a \tref{free-sheaf}{locally free} \tref{O-module}{$\mathcal{O}_X$-modules} of rank 1.
\end{topic}

\begin{topic}{sheaf-of-ideals}{sheaf of ideals}
    Let $(X, \mathcal{O}_X)$ be a \tref{ringed-space}{ringed space}. A \textbf{sheaf of ideals} is an $\mathcal{O}_X$-module $\mathcal{I}$ which is a subsheaf of $\mathcal{O}_X$.
\end{topic}

\begin{topic}{sheaf-associated-to-module}{sheaf associated to module}
    Let $R$ be a ring and let $M$ be an $R$-module. The \textbf{sheaf associated} to $M$ on $X = \Spec R$, denoted $\tilde{M}$, is defined by the gluing data: to each distinguished open $X_f = \{ f \ne 0 \}$ is assigned the localized $R_f$-module $M_f$. For each $X_f \subset X_g$ there is the natural map $M_g \to M_f$. In particular, $\tilde{R} = \mathcal{O}_X$.
    
    For the projective case, let $S$ be a graded ring and $M$ a graded $S$-module. The \textbf{sheaf associated} to $M$ on $X = \Proj S$, denoted $\tilde{M}$, is defined by the gluing data: for each homogeneous $f \in S$, we have $\tilde{M}|_{\{ f \ne 0\}} \simeq (M_f)_0$.
\end{topic}

\begin{topic}{coherent-sheaf}{(quasi) coherent sheaf}
    Let $X$ be a \tref{scheme}{scheme}. A \tref{O-module}{sheaf of $\mathcal{O}_X$-modules} $\mathcal{F}$ is \textbf{quasi-coherent} if $X$ can be covered by open affine subsets $U_i = \Spec R_i$, such that for each $i$, the restriction $\mathcal{F}|_{U_i}$ is isomorphic to $\tilde{M}_i$ for some $R_i$-module $M_i$.
    
    Furthermore, $\mathcal{F}$ is \textbf{coherent} is each $M_i$ can be taken to be a finitely generated $R_i$-module.
\end{topic}

\begin{topic}{ideal-sheaf}{ideal sheaf}
    Let $i : Y \to X$ be a \tref{closed-immersion}{closed immersion} of \tref{scheme}{schemes}. The \textbf{ideal sheaf} $\mathcal{I}$ of $Y$ is the kernel of $i^\# : \mathcal{O}_X \to i_* \mathcal{O}_Y$. In particular, this is a \tref{sheaf-of-ideals}{sheaf of ideals} on $X$.
    \[ 0 \to \mathcal{I} \to \mathcal{O}_X \to i_* \mathcal{O}_Y \to 0 . \]
\end{topic}

\begin{topic}{twisting-sheaf}{twisting sheaf}
    Let $S$ be a graded ring and let $X = \Proj S$. For any $n \in \ZZ$, the \textbf{twisting sheaf} $\mathcal{O}_X(n)$ is the \tref{sheaf-associated-to-module}{sheaf associated} to $S(n)$ (recall: $S(n)_d = S_{d + n}$).
    
    For any sheaf of \tref{O-module}{$\mathcal{O}_X$-modules} $\mathcal{F}$, the \textbf{twisted sheaf} $\mathcal{F}(n)$ is given by $\mathcal{F} \otimes_{\mathcal{O}_X} \mathcal{O}_X(n)$.
    
    Note that sheaves $\mathcal{O}_X(n)$ are all \tref{invertible-sheaf}{invertible} sheaves, and that $\mathcal{O}_X(n) \otimes \mathcal{O}_X(m) \simeq \mathcal{O}_X(n + m)$.
\end{topic}

% \begin{topic}{ample-invertible-sheaf}{(very) ample invertible sheaf}
%     Let $X$ be a \tref{scheme}{scheme} over $S$. An \tref{invertible-sheaf}{invertible} sheaf $\mathcal{L}$ on $X$ is called \textbf{very ample} relative to $S$, if there is an \tref{immersion}{immersion} $i : X \to \PP_S^r$ for some $r$, such that $\mathcal{L} \simeq i^*(\mathcal{O}(1))$.
    
%     If $X$ is scheme of finite type over a noetherian ring $A$, then any $\mathcal{L}$ is ample if and only if $\mathcal{L}^m$ is very ample over $\Spec A$ for some $m > 0$.
% \end{topic}

% \begin{topic}{smooth}{smooth}
%     A morphism $f : X \to S$ of \tref{scheme}{schemes} is \textbf{smooth} if one of the following equivalent definitions hold:
%     \begin{itemize}
%         \item it is locally of finite presentation, flat, and for every geometric point $s \to S$, the fiber $X \times_S s$ is regular,
%         \item it is flat and the sheaf of relative differentials $\Omega_{X/S}$ is locally free of rank equal to the relative dimension of $X/S$,
%         \item for any $x \in X$, there exist neighborhoods $\Spec B \subset X$ and $\Spec A \subset S$ of $x$ and $f(x)$ such that $B$ is \tref{standard-smooth} over $A$.
%     \end{itemize}
% \end{topic}

% \begin{topic}{unramified}{unramified}
% \end{topic}

% \begin{topic}{etale}{étale}
% \end{topic}

\begin{topic}{formally-smooth}{formally smooth}
    A morphism $f : X \to S$ of \tref{scheme}{schemes} is \textbf{formally smooth} if for every ring $A$, ideal $I \subset A$ with $I^2 = 0$, and commutative diagram
    \[ \begin{tikzcd} \Spec A/I \arrow{r} \arrow{d} & X \arrow{d}{f} \\ \Spec A \arrow{r} \arrow[dashed]{ur} & Y \end{tikzcd} \]
    there exists a morphism $\Spec A \to X$ making the diagram commute.
\end{topic}

\begin{example}{formally-smooth}
    The variety $\Spec(k[x, y]/(xy))$ is not formally smooth over $\Spec(k)$. Namely, take $A = k[x, y] / (xy)^2$ and $I = (xy)$. Then there exists no morphism $k[x, y]/(xy) \to k[x, y]/(xy)^2$ completing the diagram.
\end{example}

\begin{topic}{formally-unramified}{formally unramified}
    A morphism $f : X \to S$ of \tref{scheme}{schemes} is \textbf{formally unramified} if for every ring $A$, ideal $I \subset A$ with $I^2 = 0$, and commutative diagram
    \[ \begin{tikzcd} \Spec A/I \arrow{r} \arrow{d} & X \arrow{d}{f} \\ \Spec A \arrow{r} \arrow[dashed]{ur} & Y \end{tikzcd} \]
    there is at most one morphism $\Spec A \to X$ making the diagram commute.
\end{topic}

\begin{example}{formally-unramified}
    The affine line $\AA^1_k = \Spec(k[x])$ is not formally unramified over $k$. Namely, take $A = k[\varepsilon] / (\varepsilon^2)$ and $I = (\varepsilon)$. Then for any $k[x] \to k[\varepsilon]/(\varepsilon) = k : x \mapsto x_0$ there exists a whole family of morphisms $k[x] \to k[t]/(t^2)$ completing the diagram: one can send $x \mapsto x_0 + \alpha t + (t^2)$ for any $\alpha \in k$.
\end{example}

\begin{topic}{formally-etale}{formally étale}
    A morphism $f : X \to S$ of \tref{scheme}{schemes} is \textbf{formally étale} if for every ring $A$, ideal $I \subset A$ with $I^2 = 0$, and commutative diagram
    \[ \begin{tikzcd} \Spec A/I \arrow{r} \arrow{d} & X \arrow{d}{f} \\ \Spec A \arrow{r} \arrow[dashed]{ur} & Y \end{tikzcd} \]
    there exists a unique morphism $\Spec A \to X$ making the diagram commute.
    
    That is, $f$ is formally smooth if it is \tref{formally-smooth}{formally smooth} and \tref{formally-unramified}{formally unramified}.
\end{topic}

\begin{topic}{flat-morphism}{flat morphism}
    A morphism $f : X \to Y$ of \tref{scheme}{schemes} is \textbf{flat} if for every $x \in X$, the local ring $\mathcal{O}_{X, x}$ is \tref{CA:flat-module}{flat} as an $\mathcal{O}_{Y, f(x)}$-module.
\end{topic}

\begin{example}{flat-morphism}
    The morphism $\Spec k[x, y, t] / (xy - t) \to \Spec k[t]$ is flat.

    The morphism
    \[ \Spec k[x, y, t] / (txy - t) \to \Spec k[t] \]
    is not flat. Namely, at the maximal ideal $(x, y, t)$, we have that $(k[x, y, t] / (txy - t))_{(x, y, t)}$ is not flat over $k[t]_{(t)}$, as tensoring the injective map $k[t]_{(t)} \xrightarrow{\cdot t} k[t]_{(t)}$ does not give an injective map: $t$ is a zero-divisor in $k[x, y, t] / (txy - t)$.
\end{example}

\begin{topic}{dominant-morphism}{dominant morphism}
    A morphism $f : X \to Y$ of \tref{scheme}{schemes} is \textbf{dominant} if the image of $f$ is a \tref{TO:dense}{dense} subset of $Y$.
\end{topic}

\begin{topic}{serre-duality}{Serre duality}
    Let $X$ be a smooth projective scheme of dimension $n$, and let $\omega_X$ be its canonical sheaf. Then \textbf{Serre duality} states that for every coherent sheaf $\mathcal{F}$ on $X$ there is a natural isomorphism
    \[ H^i(X, \mathcal{F}^\vee \otimes \omega_X) \simeq H^{n - i}(X, \mathcal{F})^\vee . \]
\end{topic}

\begin{topic}{picard-group}{Picard group}
    The \textbf{Picard group} $\text{Pic}(X)$ of a scheme $X$ is the abelian group of isomorphism classes of \tref{invertible-sheaf}{invertible sheaves} on $X$, where addition is given by the tensor product
    \[ [\mathcal{L}_1] + [\mathcal{L}_2] = [\mathcal{L}_1 \otimes \mathcal{L}_2] . \]
\end{topic}

% \begin{example}{picard-group}
%     \begin{itemize}
%         \item $\text{Pic}(\PP^n) \simeq \ZZ$
%         \item $\text{Pic}(\AA^n) \simeq 0$
%     \end{itemize}
% \end{example}

% \begin{example}{picard-group}
%     \[ \text{Pic}(X) \simeq H^1(X, \mathcal{O}_X^*) \]
% \end{example}

\begin{topic}{euler-sequence}{Euler sequence}
    Let $A$ be any ground ring. The \textbf{Euler sequence} is the following exact sequence of sheaves on $\PP_A^n$:
    \[ 0 \rightarrow \Omega^1_{\PP^n_A/A} \rightarrow \mathcal{O}_{\PP_A^n}(-1)^{\oplus n + 1} \rightarrow \mathcal{O}_{\PP_A^n} \rightarrow 0 , \]
    where the latter map is given by the map of graded $A$-modules
    \[ S(-1)^{\oplus n + 1} \to S : e_i \mapsto x_i \qquad \text{ with } S = A[x_0, x_1, \ldots, x_n] . \]
    One can check on the affine patches that the kernel is isomorphic to the relative differential module.
\end{topic}

