\begin{topic}{enriched-category}{enriched category}
    Let $\mathcal{A}$ be a \tref{monoidal-category}{monoidal category}. An \textbf{$\mathcal{A}$-enriched category} $\mathcal{C}$ consists of
    \begin{itemize}
        \item (\textit{objects}) a collection $\textup{Ob}(\mathcal{C})$, whose elements are called \textit{objects} of $\mathcal{C}$,
        \item (\textit{hom-objects}) for every pair of objects $X, Y$ an object $\underline{\Hom}_\mathcal{C}(X, Y)$ of $\mathcal{A}$,
        \item (\textit{composition law}) for every triple of objects $X, Y, Z$ a morphism
        \[ c_{Z, Y, X} : \underline{\Hom}_\mathcal{C}(Y, Z) \otimes \underline{\Hom}_\mathcal{C}(X, Y) \to \underline{\Hom}_\mathcal{C}(X, Z) \]
        in $\mathcal{A}$,
        \item (\textit{identities}) for every object $X$ a morphism $e_X : \textbf{1} \to \underline{\Hom}_\mathcal{C}(X, X)$ in $\mathcal{A}$,
    \end{itemize}
    such that for every quadruple of objects $W, X, Y, Z$ the diagram
    \[ \begin{tikzcd}
        & \underline{\Hom}_\mathcal{C}(Y, Z) \otimes \underline{\Hom}_\mathcal{C}(W, Y) \arrow{dd}{c_{Z, Y, W}} \\ \underline{\Hom}_\mathcal{C}(Y, Z) \otimes \left(\underline{\Hom}_\mathcal{C}(X, Y) \otimes \underline{\Hom}_\mathcal{C}(W, X)\right) \arrow{ur}{\id \otimes c_{Y, X, W}} \arrow{dd}{\alpha} & \\ & \underline{\Hom}_\mathcal{C}(W, Z) \arrow{dd}{c_{Z, X, W}} \\ \left(\underline{\Hom}_\mathcal{C}(Y, Z) \otimes \underline{\Hom}_\mathcal{C}(X, Y)\right) \otimes \underline{\Hom}_\mathcal{C}(W, X) \arrow{dr}{c_{Z, Y, X} \otimes \id} & \\ & \underline{\Hom}_\mathcal{C}(X, Z) \otimes \underline{\Hom}_\mathcal{C}(W, X)
    \end{tikzcd} \]
    commutes, and for every pair of objects $X, Y$, the diagrams
    \[ \begin{tikzcd}
        \textbf{1} \otimes \underline{\Hom}_\mathcal{C}(X, Y) \arrow{rr}{e_Y \otimes \id} \arrow{dr}{\lambda} && \underline{\Hom}_\mathcal{C}(Y, Y) \otimes \underline{\Hom}_\mathcal{C}(X, Y) \arrow{dl}{c_{Y, Y, X}} \\ & \underline{\Hom}_\mathcal{C}(X, Y) & \\
        \underline{\Hom}_\mathcal{C}(X, Y) \otimes \textbf{1} \arrow{rr}{\id \otimes e_X} \arrow{dr}{\rho} && \underline{\Hom}_\mathcal{C}(X, Y) \otimes \underline{\Hom}_\mathcal{C}(X, X) \arrow{dl}{c_{Y, X, X}} \\ & \underline{\Hom}_\mathcal{C}(X, Y) &       
    \end{tikzcd} \]
    commute.
\end{topic}

\begin{example}{enriched-category}
    \begin{itemize}
        \item For $\mathcal{A} = \textbf{Set}$ the category of sets, with monoidal structure given by the cartesian product, an $\mathcal{A}$-enriched category is the same as an ordinary \tref{category}{category}.
        \item For $\mathcal{A} = \textbf{Cat}$ the category of (small) categories, with monoidal structure given by the product of categories, an $\mathcal{A}$-enriched category is the same as a 2-category.
        \item For $\mathcal{A} = \textbf{Ab}$ the category of abelian groups, with monoidal structure given by the product of groups, an $\mathcal{A}$-enriched category is the same as a \textit{preadditive category}.
        \item For $\mathcal{A} = \textbf{Vect}_k$ the category of vector spaces over a field $k$, with monoidal structure given by the tensor product over $k$, an $\mathcal{A}$-enriched category is also known as a \textit{$k$-linear category}.
    \end{itemize}
\end{example}

\begin{topic}{monoid-object}{monoid object}
    Let $(\mathcal{C}, \otimes, \textbf{1})$ be a \tref{monoidal-category}{monoidal category}. A \textbf{monoid object} in $\mathcal{C}$ is an object $M$ together with morphisms $\mu : M \otimes M \to M$ (the \textit{multiplication map}) and $\eta : \textbf{1} \to M$ (the \textit{unit}), such that
    \begin{itemize}
        \item (\textit{associativity}) the diagram
        \[ \begin{tikzcd} (M \otimes M) \otimes M \arrow[swap]{d}{\mu \otimes \id} \arrow{r}{\alpha} & M \otimes (M \otimes M) \arrow{r}{\id \otimes \mu} & M \otimes M \arrow{d}{\mu} \\ M \otimes M \arrow{rr}{\mu} && M \end{tikzcd} \]
        commutes, where $\alpha$ is the \textit{associator}.
        \item (\textit{unit}) the diagram
        \[ \begin{tikzcd} \textbf{1} \otimes M \arrow{r}{\eta \otimes \id} \arrow[swap]{rd}{\lambda} & M \otimes M \arrow{d}{\mu} & M \otimes \textbf{1} \arrow[swap]{l}{\id \otimes \eta} \arrow{dl}{\rho} \\ & M & \end{tikzcd} \]
        commutes, where $\lambda$ and $\rho$ are the left and right unitor, respectively.
    \end{itemize}
\end{topic}

\begin{example}{monoid-object}
    \begin{itemize}
        \item A monoid object in the category of sets $(\textbf{Set}, \times, \{ \star \})$ is a \tref{AA:monoid}{monoid}.
        \item A monoid object in the category of \tref{GT:abelian-group}{abelian groups} $(\textbf{Ab}, \otimes_\ZZ, \ZZ)$ is a \tref{AA:ring}{ring}.
        \item For a \tref{AA:ring}{commutative ring} $R$, a monoid object in the category of \tref{AA:module}{$R$-modules} $(R\textup{-}\textbf{Mod}, \otimes_R, R)$ is an \tref{AA:algebra}{$R$-algebra}.
        \item For any category $\mathcal{C}$, a monoid in the category of endofunctors of $\mathcal{C}$, $(\Hom_\textbf{Cat}(\mathcal{C}, \mathcal{C}), \circ, \id_\mathcal{C})$, is a \tref{monad}{monad} on $\mathcal{C}$.
    \end{itemize}
\end{example}

\begin{topic}{weighted-limit}{weighted (co)limit}
    Let $\mathcal{V}$ be a \tref{monoidal-category}{monoidal category}, and let $F : \mathcal{I} \to \mathcal{C}$ be a \tref{functor}{functor} with $\mathcal{C}$ a category \tref{enriched-category}{enriced} in $\mathcal{V}$. A \textbf{weighted limit} for $F$ with respect to a \textit{weight functor} $W : \mathcal{I} \to \mathcal{V}$ is an object ${\lim}^W F$ in $\mathcal{C}$ represented by
    \[ \Hom_\mathcal{C}(C, {\lim}^W F) \isom \Hom_{\mathcal{V}^\mathcal{I}}(W, \Hom_\mathcal{C}(C, F(-))) . \]
    A \textbf{weighted colimit} for $F$ with respect to $W : \mathcal{I}^\textup{op} \to \mathcal{V}$ is an object ${\colim}_W F$ in $\mathcal{C}$ represented by
    \[ \Hom_\mathcal{C}({\colim}_W F, C) \isom \Hom_{\mathcal{V}^{\mathcal{I}^\textup{op}}}(W, \Hom_\mathcal{C}(F(-), C)) . \]
\end{topic}

\begin{example}{weighted-limit}
    Note that an ordinary \tref{limit}{limit} of a functor $F : \mathcal{I} \to \mathcal{C}$, if it exists, can be computed via the \tref{yoneda-embedding}{Yoneda embedding} on the level of presheaves:
    \[ \begin{aligned} \Hom_\mathcal{C}(C, \lim F) &\isom \lim (\Hom_\mathcal{C}(C, F(-))) \\ &\isom \Hom_\textbf{Set}(\textup{pt}, \lim (\Hom_\mathcal{C}(C, F(-)))) \\ &\isom \Hom_{\textbf{Set}^\mathcal{I}}(\Delta_{\textup{pt}}, \Hom_\mathcal{C}(C, F(-))) . \end{aligned} \]
    In particular, the weighted limit reduces to the ordinary limit when $W : \mathcal{I} \to \textbf{Set}$ is the constant functor $\Delta_\textup{pt}$.
\end{example}

\begin{topic}{operad}{operad}
    Let $(\mathcal{C}, \otimes, \textbf{1})$ be a \tref{CT:symmetric-monoidal-category}{symmetric} \tref{CT:monoidal-category}{monoidal category}. A \textbf{non-symmetric operad} in $\mathcal{C}$ consists of
    \begin{itemize}
        \item (\textit{$n$-ary operations}) an object $\mathcal{O}(n)$ for each integer $n \ge 0$
        \item (\textit{unit}) a morphism $e : \textbf{1} \to \mathcal{O}(1)$,
        \item (\textit{composition}) for all $n \ge 0$ and $k_1, \ldots, k_n \ge 0$,
        \[ \circ : \mathcal{O}(n) \otimes \mathcal{O}(k_1) \otimes \cdots \otimes \mathcal{O}(k_n) \to \mathcal{O}(k_1 + \cdots + k_n) , \]
    \end{itemize}
    satisfying
    \begin{itemize}
        \item (\textit{unit}) $e \circ \theta = \theta = \theta \circ (e \otimes \cdots \otimes e)$ for all $\theta \in \mathcal{O}(n)$,
        \item (\textit{associativity}) $\theta \circ (\theta_1 \circ (\theta_{1,1} \otimes \cdots \otimes \theta_{1,k_1}) \otimes \cdots \otimes \theta_n \circ (\theta_{n,1} \otimes \cdots \otimes \theta_{n,k_n}))$ $= (\theta \circ (\theta_1 \otimes \cdots \otimes \theta_n)) \circ (\theta_{1, 1} \otimes \cdots \otimes \theta_{1,k_1} \otimes \cdots \otimes \theta_{n,1} \otimes \cdots \otimes \theta_{n,k_n})$.
    \end{itemize}
    A \textbf{symmetric operad}, or just \textbf{operad}, consists of the above, together with an action of \tref{GT:symmetric-group}{$S_n$} on $\mathcal{O}(n)$, for each $n \ge 0$, such that
    \begin{itemize}
        \item (\textit{equivariance 1}) $\sigma(\theta) \circ (\theta_{\sigma(1)} \otimes \cdots \otimes \theta_{\sigma(n)}) = \sigma'(\theta \circ (\theta_1 \otimes \cdots \otimes \theta_n))$, for all $\sigma \in S_n$, where $\sigma' \in S_{k_1 + \cdots + k_n}$ acts on $\{ 1, 2, \ldots, k_1 + \cdots + k_n \}$ by permuting the blocks of size $k_1, k_2, \ldots, k_n$.
        \item (\textit{equivariance 2}) $\theta \circ (\sigma_1(\theta_1) \otimes \cdots \otimes \sigma_n(\theta_n)) = (\sigma_1, \ldots, \sigma_n)(\theta \circ (\theta_1 \otimes \cdots \otimes \theta_n))$ for all $\sigma_i \in S_{k_i}$.
    \end{itemize}
\end{topic}
