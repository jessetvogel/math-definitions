\begin{topic}{enriched-category}{enriched category}
    Let $\mathcal{A}$ be a \tref{monoidal-category}{monoidal category}. An \textbf{$\mathcal{A}$-enriched category} $\mathcal{C}$ consists of
    \begin{itemize}
        \item (\textit{objects}) a collection $\text{Ob}(\mathcal{C})$, whose elements are called \textit{objects} of $\mathcal{C}$,
        \item (\textit{hom-objects}) for every pair of objects $X, Y$ an object $\underline{\Hom}_\mathcal{C}(X, Y)$ of $\mathcal{A}$,
        \item (\textit{composition law}) for every triple of objects $X, Y, Z$ a morphism
        \[ c_{Z, Y, X} : \underline{\Hom}_\mathcal{C}(Y, Z) \otimes \underline{\Hom}_\mathcal{C}(X, Y) \to \underline{\Hom}_\mathcal{C}(X, Z) \]
        in $\mathcal{A}$,
        \item (\textit{identities}) for every object $X$ a morphism $e_X : \textbf{1} \to \underline{\Hom}_\mathcal{C}(X, X)$ in $\mathcal{A}$,
    \end{itemize}
    such that for every quadruple of objects $W, X, Y, Z$ the diagram
    \[ \begin{tikzcd}
        & \underline{\Hom}_\mathcal{C}(Y, Z) \otimes \underline{\Hom}_\mathcal{C}(W, Y) \arrow{dd}{c_{Z, Y, W}} \\ \underline{\Hom}_\mathcal{C}(Y, Z) \otimes \left(\underline{\Hom}_\mathcal{C}(X, Y) \otimes \underline{\Hom}_\mathcal{C}(W, X)\right) \arrow{ur}{\id \otimes c_{Y, X, W}} \arrow{dd}{\alpha} & \\ & \underline{\Hom}_\mathcal{C}(W, Z) \arrow{dd}{c_{Z, X, W}} \\ \left(\underline{\Hom}_\mathcal{C}(Y, Z) \otimes \underline{\Hom}_\mathcal{C}(X, Y)\right) \otimes \underline{\Hom}_\mathcal{C}(W, X) \arrow{dr}{c_{Z, Y, X} \otimes \id} & \\ & \underline{\Hom}_\mathcal{C}(X, Z) \otimes \underline{\Hom}_\mathcal{C}(W, X)
    \end{tikzcd} \]
    commutes, and for every pair of objects $X, Y$, the diagrams
    \[ \begin{tikzcd}
        \textbf{1} \otimes \underline{\Hom}_\mathcal{C}(X, Y) \arrow{rr}{e_Y \otimes \id} \arrow{dr}{\lambda} && \underline{\Hom}_\mathcal{C}(Y, Y) \otimes \underline{\Hom}_\mathcal{C}(X, Y) \arrow{dl}{c_{Y, Y, X}} \\ & \underline{\Hom}_\mathcal{C}(X, Y) & \\
        \underline{\Hom}_\mathcal{C}(X, Y) \otimes \textbf{1} \arrow{rr}{\id \otimes e_X} \arrow{dr}{\rho} && \underline{\Hom}_\mathcal{C}(X, Y) \otimes \underline{\Hom}_\mathcal{C}(X, X) \arrow{dl}{c_{Y, X, X}} \\ & \underline{\Hom}_\mathcal{C}(X, Y) &       
    \end{tikzcd} \]
    commute.
\end{topic}

\begin{example}{enriched-category}
    \begin{itemize}
        \item For $\mathcal{A} = \textbf{Set}$ the category of sets, with monoidal structure given by the cartesian product, an $\mathcal{A}$-enriched category is the same as an ordinary \tref{category}{category}.
        \item For $\mathcal{A} = \textbf{Cat}$ the category of (small) categories, with monoidal structure given by the product of categories, an $\mathcal{A}$-enriched category is the same as a 2-category.
        \item For $\mathcal{A} = \textbf{Ab}$ the category of abelian groups, with monoidal structure given by the product of groups, an $\mathcal{A}$-enriched category is the same as a \textit{preadditive category}.
        \item For $\mathcal{A} = \textbf{Vect}_k$ the category of vector spaces over a field $k$, with monoidal structure given by the tensor product over $k$, an $\mathcal{A}$-enriched category is also known as a \textit{$k$-linear category}.
    \end{itemize}
\end{example}

% Simplicial objects and infinity categories
\begin{topic}{simplex-category}{simplex category}
    The \textbf{simplex category}, often denoted by $\Delta$, is the \tref{CT:category}{category} whose objects are sets of the form
    \[ [n] = \{ 0, 1, 2, \ldots, n \}, \qquad \text{ with } n \ge 0 , \]
    and whose morphisms are order-preserving functions between these sets.
\end{topic}

\begin{topic}{simplicial-object}{simplicial object}
    A \textbf{simplicial object} in a \tref{CT:category}{category} $\mathcal{C}$ is a \tref{CT:functor}{functor}
    \[ X : \Delta^\text{op} \to \mathcal{C} , \]
    where $\Delta$ denotes the \tref{simplex-category}{simplex category}. Concretely, a simplicial object consists of a family of objects $X_n$ in $\mathcal{C}$ and a morphism $X_m \to X_n$ for each function $f : [n] \to [m]$, which compose nicely.
    
    The \textit{face maps} are the maps $d_{n, i} : X_n \to X_{n - 1}$ corresponding to the function $\delta^{n, i} : [n - 1] \to [n]$ that misses $i$.
    
    The \textit{degeneracy maps} are the maps $s_{n, i} : X_n \to X_{n + 1}$ corresponding to the function $\sigma^{n, i} : [n + 1] \to [n]$ that repeat $i$.
\end{topic}

\begin{example}{simplicial-object}
    Let $X$ be a topological space. The \textbf{singular simplicial set} of a topological space is the simplicial set $\mathcal{S}(X)_\bdot$ given by
    \[ \mathcal{S}(X)_n = \{ \text{cont. maps } \Delta^n \to X \} . \]
    Every $[n] \to [m]$ induces a map $\Delta^n \to \Delta^m$ by labelling the vertices, which in turn induces a map $\mathcal{S}(X)_m \to \mathcal{S}(X)_m$. Also see \tref{AT:singular-homology}{singular homology}.
\end{example}

\begin{example}{simplicial-object}
    The \textit{standard simplicial set} $\Delta^n$ is given by $\Hom_\Delta(-, [n])$. By the \tref{yoneda-lemma}{Yoneda lemma}, we can identify
    \[ X_n = X([n]) \simeq \Hom_{\textbf{Set}_\Delta}(\Delta^n, X) . \]
\end{example}

\begin{topic}{nerve}{nerve}
    The \textbf{nerve} of a \tref{CT:category}{category} $\mathcal{C}$ is the \tref{simplicial-object}{simplicial set} $\text{N}_\bdot(\mathcal{C})$, where $\text{N}_n(\mathcal{C})$ is the set of \tref{CT:functor}{functors} from $[n] = \{ 0, 1, \ldots, n \}$ (viewed as category: a unique morphism from $i$ to $j$ iff $i \le j$) to $\mathcal{C}$.
    Indeed, for any non-decreasing map $\alpha : [m] \to [n]$, precomposition with $\alpha$ gives a map $\text{N}_n(\mathcal{C}) \to \text{N}_m(\mathcal{C})$.
    
    It can be shown that the \textit{nerve functor}
    \[ \text{N}_\bdot : \textbf{Cat} \to \textbf{Set}_\Delta \]
    is \tref{CT:full-functor}{fully} \tref{CT:faithful-functor}{faithful}.
\end{topic}

\begin{example}{nerve}
    Concretely, $N_n(\mathcal{C})$ is the set of all composable sequences of morphisms
    \[ C_0 \xrightarrow{f_1} C_1 \xrightarrow{f_2} \cdots \xrightarrow{f_n} C_n \]
    of length $n$ in $\mathcal{C}$. Applying the face map $d_{n, i}$ to this sequence gives
    \[ C_0 \xrightarrow{f_1} \cdots \xrightarrow{f_{i - 1}} C_{i - 1} \xrightarrow{f_{i + 1} \circ f_i} \cdots C_{i + 1} \xrightarrow{f_{i + 2}} \cdots \xrightarrow{f_n} C_n , \]
    while the degeneracy map $s_{n, i}$ maps it to
    \[ C_0 \xrightarrow{f_1} \cdots \xrightarrow{f_i} C_i \xrightarrow{\id} C_i \xrightarrow{f_{i + 1}} \cdots \xrightarrow{f_n} C_n . \]
\end{example}

\begin{example}{nerve}
    The standard simplicial set $\Delta^n = \Hom_\Delta(-, [n])$ can be seen as the nerve of $[n]$.
\end{example}

\begin{example}{nerve}
    Consider a \tref{GT:group}{group} $G$ as a category with one object. Note that any $f \in N_n(G)$ can be described by the images $f(k - 1 \le k) \in G$ for $1 \le k \le n$, so that $N_n(G) \simeq G^n$. Hence, the nerve $N_\bdot(G)$ is the simplicial set
    \[ \cdots G \times G \times G \; \substack{\rightarrow \\[-0.9em] \rightarrow \\[-0.9em] \rightarrow \\[-0.9em] \rightarrow} \; G \times G \; \substack{\rightarrow \\[-0.9em] \rightarrow \\[-0.9em] \rightarrow} \; G \rightrightarrows * \]
    with face and degeneracy maps given by
    \[ d_{n, i} : G^n \to G^{n - 1}, \quad (g_1, \ldots, g_n) \mapsto \left\{ \begin{array}{cl}
        (g_2, \ldots, g_n) & \textup{ for } i = 0, \\
        (g_1, \ldots, g_{i + 1} g_i, \ldots, g_n) & \textup{ for } 0 < i < n, \\
        (g_1, \ldots, g_{n - 1}) & \textup{ for } i = n,
    \end{array} \right. \]
    \[ s_{n, i} : G^n \to G^{n + 1}, \quad (g_1, \ldots, g_n) \mapsto (g_1, \ldots, g_i, 1, g_{i + 1}, \ldots, g_n) \quad \textup{ for } 0 \le i \le n . \]
    More generally, a group $G$ acting on a set $X$ can be viewed as a category $[X/G]$ whose objects are points of $X$, and a morphism $x \to gx$ for every $x \in X$ and $g \in G$. Similar to the above, we have $N_n([X/G]) \simeq G^n \times X$, so the nerve $N_\bdot([X/G])$ is the simplicial set
    \[ \cdots G \times G \times G \times X \; \substack{\rightarrow \\[-0.9em] \rightarrow \\[-0.9em] \rightarrow \\[-0.9em] \rightarrow} \; G \times G \times X \; \substack{\rightarrow \\[-0.9em] \rightarrow \\[-0.9em] \rightarrow} \; G \times X \rightrightarrows X \]
    with face and degeneracy maps given by
    \[ d_{n, i} : G^n \times X \to G^{n - 1} \times X, \quad (g_1, \ldots, g_n, x) \mapsto \left\{ \begin{array}{cl}
        (g_2, \ldots, g_n, x) & \textup{ for } i = 0, \\
        (g_1, \ldots, g_{i + 1} g_i, \ldots, g_n, x) & \textup{ for } 0 < i < n, \\
        (g_1, \ldots, g_{n - 1}, g_n x) & \textup{ for } i = n,
    \end{array} \right. \]
    \[ s_{n, i} : G^n \times X \to G^{n + 1} \times X, \quad (g_1, \ldots, g_n, x) \mapsto (g_1, \ldots, g_i, 1, g_{i + 1}, \ldots, g_n, x) \quad \textup{ for } 0 \le i \le n . \]
\end{example}

\begin{topic}{infinity-category}{infinity category}
    An \textbf{$\infty$-category} is a \tref{simplicial-object}{simplicial set} $\mathcal{C}$ such that every map of simplicial sets $\Lambda^n_i \to \mathcal{C}$ with $0 < i < n$ can be extended to a map $\Delta^n \to \mathcal{C}$.
    \[ \begin{tikzcd} \Lambda^n_i \arrow{r} \arrow{d} & \mathcal{C} \\ \Delta^n \arrow[dashed]{ur} & \end{tikzcd} \]
    One thinks of the vertices $\mathcal{C}_0$ as the objects, the edges $\mathcal{C}_1$ as the $1$-morphisms (with $\id_x = s_0(x)$ and $d_0(f) = \text{cod}(f)$ and $d_1(f) = \text{dom}(f)$), the triangles $\mathcal{C}_2$ as a $2$-morphism from $g \circ f$ to $h$, etc.
    
    A \textit{functor} between $\infty$-categories is a map of simplicial sets.
    
    An \textbf{$(\infty, r)$-category} is an $\infty$-category for which all $k$-morphisms with $k > r$ are invertible.
\end{topic}

\begin{example}{infinity-category}
    \begin{itemize}
        \item Any \tref{HT:kan-complex}{Kan complex} is an $\infty$-category.
        \item The \tref{nerve}{nerve} $N_\bdot(\mathcal{C})$ of a \tref{CT:category}{category} $\mathcal{C}$ is an $\infty$-category.
    \end{itemize}
\end{example}

\begin{topic}{monoid-object}{monoid object}
    Let $(\mathcal{C}, \otimes, \textbf{1})$ be a \tref{monoidal-category}{monoidal category}. A \textbf{monoid object} in $\mathcal{C}$ is an object $M$ together with morphisms $\mu : M \otimes M \to M$ (the \textit{multiplication map}) and $\eta : \textbf{1} \to M$ (the \textit{unit}), such that
    \begin{itemize}
        \item (\textit{associativity}) the diagram
        \[ \begin{tikzcd} (M \otimes M) \otimes M \arrow[swap]{d}{\mu \otimes \id} \arrow{r}{\alpha} & M \otimes (M \otimes M) \arrow{r}{\id \otimes \mu} & M \otimes M \arrow{d}{\mu} \\ M \otimes M \arrow{rr}{\mu} && M \end{tikzcd} \]
        commutes, where $\alpha$ is the \textit{associator}.
        \item (\textit{unit}) the diagram
        \[ \begin{tikzcd} \textbf{1} \otimes M \arrow{r}{\eta \otimes \id} \arrow[swap]{rd}{\lambda} & M \otimes M \arrow{d}{\mu} & M \otimes \textbf{1} \arrow[swap]{l}{\id \otimes \eta} \arrow{dl}{\rho} \\ & M & \end{tikzcd} \]
        commutes, where $\lambda$ and $\rho$ are the left and right unitor, respectively.
    \end{itemize}
\end{topic}

\begin{example}{monoid-object}
    \begin{itemize}
        \item A monoid object in the category of sets $(\textbf{Set}, \times, \{ \star \})$ is a \tref{AA:monoid}{monoid}.
        \item A monoid object in the category of \tref{GT:abelian-group}{abelian groups} $(\textbf{Ab}, \otimes_\ZZ, \ZZ)$ is a \tref{AA:ring}{ring}.
        \item For a \tref{AA:ring}{commutative ring} $R$, a monoid object in the category of \tref{AA:module}{$R$-modules} $(R\textup{-}\textbf{Mod}, \otimes_R, R)$ is an \tref{AA:algebra}{$R$-algebra}.
        \item For any category $\mathcal{C}$, a monoid in the category of endofunctors of $\mathcal{C}$, $(\Hom_\textbf{Cat}(\mathcal{C}, \mathcal{C}), \circ, \id_\mathcal{C})$, is a \tref{monad}{monad} on $\mathcal{C}$.
    \end{itemize}
\end{example}

\begin{topic}{weighted-limit}{weighted (co)limit}
    Let $\mathcal{V}$ be a \tref{monoidal-category}{monoidal category}, and let $F : \mathcal{I} \to \mathcal{C}$ be a \tref{functor}{functor} with $\mathcal{C}$ a category \tref{enriched-category}{enriced} in $\mathcal{V}$. A \textbf{weighted limit} for $F$ with respect to a \textit{weight functor} $W : \mathcal{I} \to \mathcal{V}$ is an object ${\lim}^W F$ in $\mathcal{C}$ represented by
    \[ \Hom_\mathcal{C}(C, {\lim}^W F) \simeq \Hom_{\mathcal{V}^\mathcal{I}}(W, \Hom_\mathcal{C}(C, F(-))) . \]
    A \textbf{weighted colimit} for $F$ with respect to $W : \mathcal{I}^\textup{op} \to \mathcal{V}$ is an object ${\colim}_W F$ in $\mathcal{C}$ represented by
    \[ \Hom_\mathcal{C}({\colim}_W F, C) \simeq \Hom_{\mathcal{V}^{\mathcal{I}^\textup{op}}}(W, \Hom_\mathcal{C}(F(-), C)) . \]
\end{topic}

\begin{example}{weighted-limit}
    Note that an ordinary \tref{limit}{limit} of a functor $F : \mathcal{I} \to \mathcal{C}$, if it exists, can be computed via the \tref{yoneda-embedding}{Yoneda embedding} on the level of presheaves:
    \[ \begin{aligned} \Hom_\mathcal{C}(C, \lim F) &\simeq \lim (\Hom_\mathcal{C}(C, F(-))) \\ &\simeq \Hom_\textbf{Set}(\textup{pt}, \lim (\Hom_\mathcal{C}(C, F(-)))) \\ &\simeq \Hom_{\textbf{Set}^\mathcal{I}}(\Delta_{\textup{pt}}, \Hom_\mathcal{C}(C, F(-))) . \end{aligned} \]
    In particular, the weighted limit reduces to the ordinary limit when $W : \mathcal{I} \to \textbf{Set}$ is the constant functor $\Delta_\textup{pt}$.
\end{example}
