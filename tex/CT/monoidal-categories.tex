\begin{topic}{monoidal-category}{monoidal category}
    A \textbf{monoidal category} is a \tref{category}{category} $\mathcal{C}$ with a functor $\otimes : \mathcal{C} \times \mathcal{C} \to \mathcal{C}$ (the \textit{tensor product}), an object $\textbf{1}$ in $\mathcal{C}$ (the \textit{unital object}) and \tref{natural-transformation}{natural} isomorphisms
    \[ \begin{aligned}
        \textup{(associator)} &\quad \alpha : - \otimes (- \otimes -) \Rightarrow (- \otimes -) \otimes - \qquad  \\
        \textup{(left unitor)} &\quad \lambda : \textbf{1} \otimes - \Rightarrow \id_{\mathcal{C}} \\
        \textup{(right unitor)} &\quad \rho : - \otimes \textbf{1} \Rightarrow \id_{\mathcal{C}}
    \end{aligned} \]
    such that the triangle
    \[ \begin{tikzcd}[row sep=0em] (X \otimes \textbf{1}) \otimes Y \arrow{rr}{\alpha_{X,\textbf{1},Y}} \arrow[swap]{rd}{\rho_X \otimes \id_Y} & & X \otimes (\textbf{1} \otimes Y) \arrow{ld}{\id_X \otimes \lambda_Y} \\ & X \otimes Y & \end{tikzcd} \]
    and the pentagon
    \[ \begin{tikzcd} & (X \otimes Y) \otimes (Z \otimes W) \arrow{rd}{\alpha_{X, Y, Z \otimes W}} & \\ ((X \otimes Y) \otimes Z) \otimes W \arrow{ur}{\alpha_{X \otimes Y, Z, W}} \arrow{d}{\alpha_{X,Y,Z} \otimes \id_W} & & X \otimes (Y \otimes (Z \otimes W)) \\ (X \otimes (Y \otimes Z)) \otimes W \arrow{rr}{\alpha_{X, Y \otimes Z, W}} & & X \otimes ((Y \otimes Z) \otimes W) \arrow{u}{\id_X \otimes \alpha_{Y, Z, W}} \end{tikzcd} \]
    commute for all objects $X, Y, Z, W$ in $\mathcal{C}$. In case the natural isomorphisms $\alpha, \lambda$ and $\rho$ are all equalities, we say that such a category is \textit{strict}. The above triangle and pentagon then commute automatically.
\end{topic}

\begin{example}{monoidal-category}
    The category $\textbf{Set}$ with the disjoint union $\sqcup$ as tensor product, and the empty set $\varnothing$ as unital object is a monoidal category. Also $\textbf{Set}$ with the Cartesian product $\times$ as tensor product and a choice of singleton set as unital object is a monoidal category.
\end{example}

\begin{example}{monoidal-category}
    Let $R$ be a \tref{AA:ring}{commutative ring}, then the category $R\text{-Mod}$ (resp. $R\text{-Alg}$) with the tensor product $\otimes_R$ and $R$ as unital object is a monoidal category. Also $R\text{-Mod}$ (resp. $R\text{-Alg}$) with $\oplus$ as tensor product and $0$ as unital object is a monoidal category.
\end{example}

\begin{example}{monoidal-category}
    The category $\textbf{Sch}/S$ of \tref{AG:scheme}{schemes} over $S$, with the fiber product $\times_S$ as tensor product, and $S$ as unital object is a monoidal category.
\end{example}

\begin{topic}{symmetric-monoidal-category}{symmetric monoidal category}
    A \textbf{symmetric monoidal category} is a \tref{monoidal-category}{monoidal category} $\mathcal{C}$ together with isomorphisms
    \[ \tau_{X, Y} : X \otimes Y \to Y \otimes X , \]
    \tref{natural-transformation}{natural} in $X$ and $Y$, such that
    \[ \tau_{Y, X} \circ \tau_{X, Y} = \id_{X \otimes Y} \]
    and the diagrams
    \[ \begin{tikzcd}
        (X \otimes Y) \otimes Z \arrow{r}{\alpha_{X, Y, Z}} \arrow[swap]{d}{\tau_{X, Y} \otimes \id_Z} & X \otimes (Y \otimes Z) \arrow{r}{\tau_{X, Y \otimes Z}} & (Y \otimes Z) \otimes X \arrow{d}{\alpha_{Y, Z, X}} \\ (Y \otimes X) \otimes Z \arrow{r}{\alpha_{Y, X, Z}} & Y \otimes (X \otimes Z) \arrow{r}{\id_Y \otimes \tau_{X, Z}} Y & Y \otimes (Z \otimes X)
    \end{tikzcd} \]
    and
    \[ \begin{tikzcd}
        X \otimes (Y \otimes Z) \arrow{r}{\alpha^{-1}_{X, Y, Z}} \arrow[swap]{d}{\id_X \otimes \tau_{Y, Z}} & (X \otimes Y) \otimes Z \arrow{r}{\tau_{X \otimes Y, Z}} & Z \otimes (X \otimes Y) \arrow{d}{\alpha^{-1}_{Z, X, Y}} \\ X \otimes (Y \otimes Z) \arrow{r}{\alpha^{-1}_{X, Z, Y}} & (X \otimes Z) \otimes Y \arrow{r}{\tau_{X, Z} \otimes \id_Y} & (Z \otimes X) \otimes Y
    \end{tikzcd} \]
    commute for all $X, Y, Z$ in $\mathcal{C}$.
\end{topic}

\begin{example}{symmetric-monoidal-category}
    \begin{itemize}
        \item The category $\textbf{Set}$ of sets, with either $\sqcup$ or $\times$ as tensor product, is naturally symmetric.
        \item Let $R$ be a \tref{AA:ring}{commutative ring}. The category $\textbf{Mod}_R$ of $R$-modules, with either $\oplus$ or $\otimes_R$ as tensor product, is naturally symmetric.
    \end{itemize}
\end{example}

\begin{example}{symmetric-monoidal-category}
    The category $R\textup{-}\textbf{Bimod}$ of $R$-bimodules is monoidal, with $\otimes_R$ as tensor product and $R$ as unital object, but not necessarily symmetric. For example, take $R = k$ a field with two non-commuting automorphisms $\sigma, \tau$. Let $M = {}_1 k_\sigma$ be the abelian group $k$ with $k$-bimodule structure given by $a \cdot x \cdot b = ax \sigma(b)$, and let $N = {}_1 k_\tau$ similarly. Then we have a $k$-bimodule isomorphism $\varphi : M \otimes_k N \xrightarrow{\sim} {}_1 k_{\sigma \tau}$ given by $x \otimes y \mapsto x \sigma(y)$, and similarly $N \otimes_k M = {}_1 k_{\tau \sigma}$. If there were to exist some $k$-bimodule isomorphism $\psi : {}_1 k_{\sigma \tau} \xrightarrow{\sim} {}_1 k_{\tau \sigma}$ it would be given by $\psi(x) = x \psi(1)$ (since $\psi$ is a left $k$-module isomorphism). However, as $\psi$ is a right $k$-module isomorphism as well, we must have $\psi(x) = \psi(1) \tau \sigma \tau^{-1} \sigma^{-1} (x)$, which implies that $x = \tau \sigma \tau^{-1} \sigma^{-1} (x)$ for all $x \in k$, but we assumed $\tau$ and $\sigma$ did not commute. Hence such $\psi$ does not exist, and $k\textup{-}\textbf{Bimod}$ cannot be symmetric.
\end{example}

\begin{example}{symmetric-monoidal-category}
    A monoidal category can be symmetric in multiple ways. Let $\textbf{GrVect}_k$ be the category whose objects are graded vector spaces $V = \oplus_{n \in \ZZ} V_n$ over a field $k$, and whose morphisms are linear maps that respect the grading. The tensor product of two graded vector spaces $V$ and $W$ is again graded, with grading $(V \otimes W)_n = \oplus_{p + q = n} (V_p \otimes W_q)$, making $\textbf{GrVect}_k$ into a monoidal category, where the unital object is the ground field $k$ concentrated in degree zero.
        
    There is more than one way to make $\textbf{GrVect}_k$ symmetric. As the map $V \otimes W \to W \otimes V$ one could take the more obvious map $\tau : v \otimes w \to w \otimes v$. However, one could also take the map $\kappa : v \otimes w \mapsto (-1)^{pq} w \otimes v$ with $p = \deg(v)$ and $q = \deg(w)$, known as \textit{Koszul's sign change}. One can check that $\kappa$ indeed satisfies the axioms.
\end{example}

\begin{topic}{braided-monoidal-category}{braided monoidal category}
    A \textbf{braided monoidal category} is a \tref{symmetric-monoidal-category}{symmetric monoidal category} $(\mathcal{C}, \otimes, \textbf{1}, \tau)$ without the condition that $\tau_{Y, X} \circ \tau_{X, Y} = \id_{X \otimes Y}$.
    % \[ \gamma_{X, Y} : X \otimes Y \to Y \otimes X \]
    % such that the diagrams
    % \[ \begin{tikzcd}
    %     (X \otimes Y) \otimes Z \arrow{r}{\alpha_{X, Y, Z}} \arrow[swap]{d}{\tau_{X, Y} \otimes \id_Z} & X \otimes (Y \otimes Z) \arrow{r}{\tau_{X, Y \otimes Z}} & (Y \otimes Z) \otimes X \arrow{d}{\alpha_{Y, Z, X}} \\ (Y \otimes X) \otimes Z \arrow{r}{\alpha_{Y, X, Z}} & Y \otimes (X \otimes Z) \arrow{r}{\id_Y \otimes \tau_{X, Z}} Y & Y \otimes (Z \otimes X)
    % \end{tikzcd} \]
    % and
    % \[ \begin{tikzcd}
    %     X \otimes (Y \otimes Z) \arrow{r}{\alpha^{-1}_{X, Y, Z}} \arrow[swap]{d}{\id_X \otimes \tau_{Y, Z}} & (X \otimes Y) \otimes Z \arrow{r}{\tau_{X \otimes Y, Z}} & Z \otimes (X \otimes Y) \arrow{d}{\alpha^{-1}_{Z, X, Y}} \\ X \otimes (Y \otimes Z) \arrow{r}{\alpha^{-1}_{X, Z, Y}} & (X \otimes Z) \otimes Y \arrow{r}{\tau_{X, Z} \otimes \id_Y} & (Z \otimes X) \otimes Y
    % \end{tikzcd} \]
    % commute for all $X, Y, Z$ in $\mathcal{C}$.
\end{topic}

\begin{topic}{monoidal-functor}{(lax) monoidal functor}
    A \textbf{lax monoidal functor} is a \tref{functor}{functor} $F : \mathcal{C} \to \mathcal{D}$ between \tref{monoidal-category}{monoidal categories} together with a \tref{natural-transformation}{natural transformation}
    \[ \mu : F(-) \otimes_\mathcal{D} F(-) \Rightarrow F(- \otimes_\mathcal{C} -) \]
    and a morphism $\varepsilon : \textbf{1}_\mathcal{D} \to F(\textbf{1}_\mathcal{C})$, such that the diagrams
    \[ \begin{tikzcd}[column sep=7em]
        (F(X) \otimes_\mathcal{D} F(Y)) \otimes_\mathcal{D} F(Z) \arrow{r}{\alpha^\mathcal{D}_{F(X), F(Y), F(Z)}} \arrow[swap]{d}{\mu_{X, Y} \otimes \id_{F(Z)}} & F(X) \otimes_\mathcal{D} (F(Y) \otimes_\mathcal{D} F(Z)) \arrow{d}{\id_{F(X)} \otimes \mu_{Y, Z}} \\ F(X \otimes_\mathcal{C} Y) \otimes_\mathcal{D} F(Z) \arrow[swap]{d}{\mu_{X \otimes_\mathcal{C} Y, Z}} &  F(X) \otimes_\mathcal{D} F(Y \otimes_\mathcal{C} Z) \arrow{d}{\mu_{X, Y \otimes_\mathcal{C} Z}} \\ F((X \otimes_\mathcal{C} Y) \otimes_\mathcal{C} Z) \arrow{r}{F(\alpha^\mathcal{C}_{X, Y, Z})} & F(X \otimes_\mathcal{C} (Y \otimes_\mathcal{C} Z))
    \end{tikzcd} \]
    and
    \[ \begin{tikzcd}[column sep=3em]
        \textbf{1}_\mathcal{D} \otimes_\mathcal{D} F(X) \arrow{r}{\varepsilon \otimes \id_{F(X)}} \arrow[swap]{d}{\lambda^\mathcal{D}_F(X)} & F(\textbf{1}_\mathcal{C}) \otimes_\mathcal{D} F(X) \arrow{d}{\mu_{\textbf{1}_\mathcal{C}, X}} \\ F(X) & F(\textbf{1}_\mathcal{C} \otimes_\mathcal{C} X) \arrow{l}{F \left(\lambda^\mathcal{C}_X \right)}
    \end{tikzcd} \text{ and } \begin{tikzcd}[column sep=3em]
        F(X) \otimes_\mathcal{D} \textbf{1}_\mathcal{D} \arrow{r}{\id_{F(X)} \otimes \varepsilon} \arrow[swap]{d}{\rho^\mathcal{D}_F(X)} & F(X) \otimes_\mathcal{D} F(\textbf{1}_\mathcal{C}) \arrow{d}{\mu_{X, \textbf{1}_\mathcal{C}}} \\ F(X) & F(X \otimes_\mathcal{C} \textbf{1}_\mathcal{C}) \arrow{l}{F \left(\rho^\mathcal{C}_X \right)}
    \end{tikzcd} \]
    commute for all objects $X, Y, Z$ in $\mathcal{C}$. Such a functor is said to be a \textbf{strong monoidal functor}, or simply a \textbf{monoidal functor}, if all $\mu_{X, Y}$ and $\varepsilon$ are isomorphisms.
\end{topic}

\begin{topic}{dualizable-object}{dualizable object}
    Let $(\mathcal{C}, \otimes, \textbf{1})$ be a \tref{monoidal-category}{monoidal category}. An object $X$ of $\mathcal{C}$ is \textbf{left dualizable} if there exist an object $Y$, called a \textbf{left dual} of $X$, and morphisms $\varepsilon : Y \otimes X \to \textbf{1}$ (\textit{evaluation}) and $\eta : \textbf{1} \to X \otimes Y$ (\textit{coevaluation}) such that
    \[ X \xrightarrow{\eta \otimes \id_X} (X \otimes Y) \otimes X \xrightarrow{\alpha_{X, Y, X}^{-1}} X \otimes (Y \otimes X) \xrightarrow{\id_X \otimes \varepsilon} X \]
    and
    \[ Y \xrightarrow{\eta \otimes \id_Y} Y \otimes (X \otimes Y) \xrightarrow{\alpha_{Y, X, Y}} (Y \otimes X) \otimes Y \xrightarrow{\varepsilon \otimes \id_Y} Y \]
    are equal to $\id_X$ and $\id_Y$, respectively.
    
    Similarly, one defines \textbf{right dualizable} objects by switching the roles of $X$ and $Y$.
    An object of $\mathcal{C}$ is \textbf{dualizable} if it is both left and right dualizable.
\end{topic}

\begin{topic}{rigid-monoidal-category}{rigid monoidal category}
    A \tref{monoidal-category}{monoidal category} $(\mathcal{C}, \otimes, \textbf{1})$ is \textbf{left rigid} (resp. \textbf{right rigid}) (resp. \textbf{rigid}) if all objects are \tref{dualizable-object}{left dualizable} (resp. right dualizable) (resp. dualizable).
\end{topic}

\begin{example}{rigid-monoidal-category}
    Let $k$ be a field. The category $\textbf{FinVect}_k$ of finite-dimensional \tref{LA:vector-space}{vector spaces} over $k$, with tensor product $\otimes_k$, is a rigid monoidal category. Namely, the left and right dual of a vector space $V$ is the \tref{LA:dual-vector-space}{dual vector space} $V^*$. Indeed, for every $V$ we have maps
    \[ \eta_V : k \to V \otimes_k V^*, \quad 1 \mapsto \sum_{i = 1}^{n} e_i \otimes e^*_i , \quad \textup{ and } \quad \varepsilon_V : V^* \otimes_k V \to k, \quad f \otimes v \mapsto f(v) , \]
    where $e_1, \ldots, e_n$ is some basis for $V$ with dual basis $e^*_1, \ldots, e^*_n$ for $V^*$.
\end{example}

\begin{topic}{internal-hom}{internal hom}
    Let $(\mathcal{C}, \otimes, \textbf{1})$ be a \tref{symmetric-monoidal-category}{symmetric} \tref{monoidal-category}{monoidal category}. An \textbf{internal hom} in $\mathcal{C}$ is a \tref{functor}{functor}
    \[ \iHom_\mathcal{C}(-, -) : \mathcal{C}^\textup{op} \times \mathcal{C} \to \mathcal{C} \]
    such that for every object $X$ of $\mathcal{C}$ there is an \tref{adjunction}{adjunction}
    \[ (-) \otimes X \dashv \iHom_\mathcal{C}(X, -) . \]
    If such an internal hom exists, $\mathcal{C}$ is called a \tref{closed-monoidal-category}{closed monoidal category}.
\end{topic}

\begin{example}{internal-hom}
    \begin{itemize}
        \item In the monoidal category $(\textbf{Set}, \times, \{ \star \})$, an internal hom is given by
        \[ \iHom_\textbf{Set}(A, B) = \Hom_\textbf{Set}(A, B) . \]
        Indeed, for any set $X$ there is an adjunction $(-) \otimes X \dashv \iHom_\textbf{Set}(X, -)$ since we have a natural bijection
        \[ \begin{array}{ccc}
            \Hom_\textbf{Set}(A \times X, B) & \isom & \Hom_\textbf{Set}(A, \iHom_\textbf{Set}(X, B)) \\
            f & \mapsto & (a \mapsto f(a, -)) \\
            ((a, x) \mapsto g(a)(x)) & \mapsfrom & g
        \end{array} \]
        \item Let $G$ be a \tref{GT:group}{group} and let $\mathcal{C} = \textbf{Rep}_k(G)$ be the monoidal category of representations of $G$ (over a field $k$), with tensor product $\otimes_k$. An internal hom in $\mathcal{C}$ is given by
        \[ \iHom_\mathcal{C}(V, W) = \Hom_{k}(V, W) , \]
        which follows from the natural bijection
        \[ \Hom_\mathcal{C}(U \otimes_k V, W) \isom \Hom_\mathcal{C}(U, V^* \otimes_k W) \isom \Hom_\mathcal{C}(U, \Hom_k(V, W)) . \]
    \end{itemize}
\end{example}

\begin{topic}{closed-monoidal-category}{closed monoidal category}
    A \tref{symmetric-monoidal-category}{symmetric} \tref{monoidal-category}{monoidal category} $(\mathcal{C}, \otimes, \textbf{1})$ is \textbf{closed} if it has an \tref{internal-hom}{internal hom}.
\end{topic}

\begin{topic}{ribbon-category}{ribbon category}
    A \textbf{ribbon category} is a \tref{rigid-monoidal-category}{rigid} \tref{braided-monoidal-category}{braided monoidal category} $(\mathcal{C}, \otimes, \textbf{1}, \tau)$ together with a collection of isomorphisms $\theta_X : X \to X$, called a \textit{twist}, satisfying
    \begin{itemize}
        \item (\textit{unit twist}) $\theta_\textbf{1} = \id_\textbf{1}$,
        \item (\textit{braided compatible}) $\theta_{X \otimes Y} = \tau_{Y, X} \tau_{X, Y} (\theta_X \otimes \theta_Y)$ for all objects $X, Y$ of $\mathcal{C}$,
        \item (\textit{dual compatible}) $\theta_{X^*} = \theta_X^*$ for all objects $X$ of $\mathcal{C}$ with left dual $X^*$, where
        \[ \theta_X^* = (\varepsilon_X \otimes \id_{X^*}) (\id_{X^*} \otimes \theta_X \otimes \id_{X^*}) (\id_{X^*} \otimes \eta_X) . \]
    \end{itemize}
\end{topic}

\begin{example}{ribbon-category}
    Let $A$ be a \tref{AA:hopf-algebra}{Hopf algebra} over a field $k$ with a \tref{AA:universal-r-matrix}{universal $R$-matrix} $R$. The category of (left) $A$-modules, finite-dimensional over $k$, has a monoidal structure:
    \begin{itemize}
        \item The tensor product of $A$-modules $V$ and $W$ is given by $V \otimes_k W$, where $a \cdot (v \otimes w) = \Delta(a) \cdot (v, w)$, and the unit is the one-dimensional vector space $\textbf{1} = k$ with $a \cdot \lambda = \varepsilon(a) \lambda$. This being a monoidal category follows from $A$ being a \tref{AA:bialgebra}{bialgebra}.
        \item A braiding is given by $c_{V, W}^R(v \otimes w) = \tau_{V, W}(R (v \otimes w))$.
        \item The dual of an $A$-module $V$ is the \tref{LA:dual-vector-space}{dual vector space} $V^*$ with $(a \cdot \varphi)(v) = \varphi(S^{-1}(a) \cdot v)$. There are morphisms $\eta_V : k \to V \otimes_k V^*$ and $\varepsilon_V : V^* \otimes_k V \to k$ given by $\eta_V(1) = \sum_i v_i \otimes v^i$ and $\varepsilon(\varphi \otimes v) = \varphi(v)$. 
        \item For any central invertible element $\theta \in A$ satisfying $\varepsilon(\theta) = 1$, $S(\theta) = \theta$ and $\Delta(\theta) = (R_{21} R)^{-1} (\theta \otimes \theta)$, the collection of maps
        \[ \theta_V : V \to V, \quad v \mapsto \theta^{-1} \cdot v \]
        defines a twist, which is compatible with the brading and duals. Conversely, any twist defines such an element $\theta = \theta_A(1)^{-1}$.
    \end{itemize}
\end{example}

\begin{topic}{module-category}{module category}
    Let $(\mathcal{C}, \otimes, \textbf{1})$ be a \tref{monoidal-category}{monoidal category}. A \textbf{(left) module category} over $\mathcal{C}$ is a category $\mathcal{M}$ together with a \tref{functor}{functor} $\boxtimes : \mathcal{C} \times \mathcal{M} \to \mathcal{M}$ and \tref{natural-transformation}{natural} isomorphisms
    \[ \begin{aligned}
        \textup{(associator)}& \quad \alpha : - \boxtimes (- \boxtimes -) \Rightarrow (- \otimes -) \boxtimes - \\
        \textup{(unitor)}& \quad \lambda : \textbf{1} \boxtimes - \Rightarrow \id_\mathcal{M}
    \end{aligned} \]
    which are compatible with the associator and left and right unitor of $\mathcal{C}$.
\end{topic}

\begin{topic}{cosmos}{cosmos}
    A \textbf{cosmos} is a \tref{symmetric-monoidal-category}{symmetric} \tref{closed-monoidal-category}{closed} \tref{monoidal-category}{monoidal category} which is both \tref{complete-category}{complete} and cocomplete.
\end{topic}
