\begin{topic}{category}{category}
    A \textbf{category} $\mathcal{C}$ is given by a collection of \textit{objects} and \textit{morphisms}.
    \begin{itemize}
        \item Each morphism has a \textit{domain} and \textit{codomain}, which are objects. We write $f : X \to Y$ or $X \overset{f}{\to} Y$ if $X$ is the domain of $f$ and $Y$ the codomain. We also write $X = \text{dom}(X)$ and $Y = \text{cod}(Y)$.
        
        \item Given two morphisms $f$ and $g$ such that $\text{cod}(f) = \text{dom}(g)$, the \textbf{composition} of $f$ and $g$, written $gf$, is defined and has domain $\text{dom}(f)$ and codomain $\text{cod}(g)$.
        
        \item Composition is associative, i.e. $(fg)h = f(gh)$.
        
        \item For every object $X$ there is an \textit{identity} morphism $\id_X : X \to X$ satisfying $f \id_X = f$ and $\id_X g = g$ for all $f, g$.
    \end{itemize}
\end{topic}

\begin{topic}{full-subcategory}{full subcategory}
    A \textbf{full subcategory} $\mathcal{C}'$ of a \tref{category}{category} $\mathcal{C}$ is a subcategory which has all morphisms between its objects, that is
    \[ \Hom_{\mathcal{C}'}(X, Y) = \Hom_{\mathcal{C}}(X, Y) \]
    for all $X, Y$ in $\mathcal{C}'$.
\end{topic}

\begin{topic}{functor}{functor}
    Given two categories $\mathcal{C}$ and $\mathcal{D}$, a \textbf{functor} $F : \mathcal{C} \to \mathcal{D}$ assigns an object (resp. a morphism) in $\mathcal{D}$ to each object (resp. morphism) in $\mathcal{C}$, such that
    \begin{itemize}
        \item $F(f) : F(X) \to F(Y)$ for each $f : X \to Y$,
        \item $F(gf) = F(g) F(f)$,
        \item $F(\id_X) = \id_{F(X)}$.
    \end{itemize}
\end{topic}

\begin{topic}{monomorphism}{monomorphism}
    A morphism $f : X \to Y$ in a \tref{category}{category} $\mathcal{C}$ is a \textbf{monomorphism} if $f \circ g = f \circ h$ implies $g = h$.
    \[ \begin{tikzcd} W \arrow[shift left=0.25em]{r}{g} \arrow[shift right=0.25em, swap]{r}{h} & X \arrow{r}{f} & Y \end{tikzcd} \]
\end{topic}

\begin{topic}{epimorphism}{epimorphism}
    A morphism $f : X \to Y$ in a \tref{category}{category} $\mathcal{C}$ is a \textbf{epimorphism} if $g \circ f = h \circ f$ implies $g = h$.
    \[ \begin{tikzcd} X \arrow{r}{f} & Y \arrow[shift left=0.25em]{r}{g} \arrow[shift right=0.25em, swap]{r}{h} & Z \end{tikzcd} \]
\end{topic}

\begin{topic}{split-mono}{split monomorphism}
    A morphism $f : X \to Y$ in a \tref{category}{category} $\mathcal{C}$ is a \textbf{split monomorphism} if there exists a $g : Y \to X$ such that $gf = \id_X$. That is, it has a left-inverse. In particular, it is an \tref{monomorphism}{monomorphism}.
    \[ \begin{tikzcd} X \arrow[shift left=0.25em]{r}{f} & Y \arrow[shift left=0.25em]{l}{g} \end{tikzcd} \]
\end{topic}

\begin{topic}{split-epi}{split epimorphism}
    A morphism $f : X \to Y$ in a \tref{category}{category} $\mathcal{C}$ is a \textbf{split epimorphism} if there exists a $g : Y \to X$ such that $fg = \id_Y$. That is, it has a right-inverse. In particular, it is an \tref{epimorphism}{epimorphism}.
    \[ \begin{tikzcd} X \arrow[shift left=0.25em]{r}{f} & Y \arrow[shift left=0.25em]{l}{g} \end{tikzcd} \]
\end{topic}

\begin{topic}{isomorphism}{isomorphism}
    A morphism $f : X \to Y$ in a \tref{category}{category} $\mathcal{C}$ is an \textbf{isomorphism} if there exists a $g : Y \to X$ such that $fg = \id_Y$ and $gf = \id_X$. We call $g$ the \textbf{inverse} of $f$ and write $g = f^{-1}$. If it exists, it is unique.
\end{topic}

\begin{topic}{terminal-object}{terminal object}
    An object $X$ in a \tref{category}{category} $\mathcal{C}$ is called \textbf{terminal} if for any object $Y$ there is exactly one morphism $Y \to X$. Any two terminal objects are isomorphic.
\end{topic}

\begin{topic}{initial-object}{initial object}
    An object $X$ in a \tref{category}{category} $\mathcal{C}$ is called \textbf{initial} if for any object $Y$ there is exactly one morphism $X \to Y$. Any two initial objects are isomorphic.
\end{topic}

\begin{topic}{full-functor}{full functor}
    A \tref{functor}{functor} $F : \mathcal{C} \to \mathcal{D}$ is called \textbf{full} if for every two objects $X$ and $Y$ of $\mathcal{C}$, the map
    \[ F : \Hom_{\mathcal{C}}(X, Y) \to \Hom_{\mathcal{D}}(F(X), F(Y)) \]
    is surjective.
\end{topic}

\begin{topic}{faithful-functor}{faithful functor}
    A \tref{functor}{functor} $F : \mathcal{C} \to \mathcal{D}$ is called \textbf{full} if for every two objects $X$ and $Y$ of $\mathcal{C}$, the map
    \[ F : \Hom_{\mathcal{C}}(X, Y) \to \Hom_{\mathcal{D}}(F(X), F(Y)) \]
    is injective.
\end{topic}

\begin{topic}{natural-transformation}{natural transformation}
    A \textbf{natural transformation} between two functors $F, G : \mathcal{C} \to \mathcal{D}$ consists of a collection of morphisms $\mu_X : F(X) \to G(X)$ for each object $C$ in $\mathcal{C}$, such that for each $f : X \to Y$ the diagram
    \[ \begin{tikzcd} F(X) \arrow{r}{\mu_X} \arrow[swap]{d}{F(f)} & G(X) \arrow{d}{G(f)} \\ F(Y) \arrow[swap]{r}{\mu_Y} & G(Y) \end{tikzcd} \]
    commutes. We denote this by $\mu : F \Rightarrow G$.
\end{topic}

\begin{topic}{yoneda-embedding}{Yoneda embedding}
    Let $\mathcal{C}$ be a \tref{category}{category}. The \textbf{Yoneda embedding} is the functor
    \[ y_{(-)} : \mathcal{C} \to \textbf{Set}^{\mathcal{C}^\text{op}} \]
    given by $y_X(Y) = \Hom_{\mathcal{C}}(Y, X)$.
\end{topic}

\begin{topic}{yoneda-lemma}{Yoneda lemma}
    Let $\mathcal{C}$ be a \tref{category}{category}, and write $y_X = \Hom_{\mathcal{C}}(-, X)$. The \textbf{Yoneda lemma} states that for every $F : \mathcal{C}^\text{op} \to \textbf{Set}$ and $X$ in $\mathcal{C}$ there is a bijection
    \[ \Hom(y_X, F) \xrightarrow{\sim} F(X)  \]
    natural in $X$ and $F$. As a consequence, the \tref{yoneda-embedding}{Yoneda embedding}
    \[ y_{(-)} : \mathcal{C} \to \textbf{Set}^{\mathcal{C}^\text{op}} \]
    is \tref{full-functor}{full} and \tref{faithful-functor}{faithful}.
\end{topic}

\begin{topic}{groupoid}{groupoid}
    A \textbf{groupoid} is a category where every morphism is an isomorphism.
\end{topic}

\begin{topic}{equivalence-of-categories}{equivalence of categories}
    A \tref{functor}{functor} $F : \mathcal{C} \Rightarrow \mathcal{D}$ is said to be an \textbf{equivalence of categories} if there exists a functor $G : \mathcal{D} \Rightarrow \mathcal{C}$ and natural isomorphisms $\mu : F \Rightarrow G$ and $\nu : G \Rightarrow F$. In this case $F$ and $G$ are called \textit{pseudo-inverses} of each other.
\end{topic}

\begin{topic}{essentially-surjective-functor}{essentially surjective functor}
    A \tref{functor}{functor} $F : \mathcal{C} \to \mathcal{D}$ is \textbf{essentially surjective} if each object in $\mathcal{D}$ is isomorphic to $F(X)$ for some $X$ in $\mathcal{C}$.
\end{topic}

\begin{topic}{equalizer}{equalizer}
    Let $f, g : X \to Y$ be morphisms in a \tref{category}{category} $\mathcal{C}$. An object $E$ of $\mathcal{C}$ together with morphism $e : E \to X$ is said to be an \textbf{equalizer} of the pair $f, g$ if $fe = ge$ and for any other such $e' : E' \to X$ there exists a unique morphism $h : E' \to E$ such that $e' = eh$.
    \[ \begin{tikzcd} E \arrow{r}{e} & X \arrow[shift left=0.25em]{r}{f} \arrow[swap, shift right=0.25em]{r}{g} & Y \\ E' \arrow[dashed]{u}{\exists!} \arrow[swap]{ur}{e'} && \end{tikzcd} \]
\end{topic}

\begin{topic}{coequalizer}{coequalizer}
    Let $f, g : X \to Y$ be morphisms in a \tref{category}{category} $\mathcal{C}$. An object $Q$ of $\mathcal{C}$ together with morphism $q : Y \to Q$ is said to be a \textbf{coequalizer} of the pair $f, g$ if $qf = qg$ and for any other such $q' : Y \to Q'$ there exists a unique morphism $h : Q \to Q'$ such that $q' = hq$.
    \[ \begin{tikzcd} X \arrow[shift left=0.25em]{r}{f} \arrow[swap, shift right=0.25em]{r}{g} & Y \arrow{r}{q} \arrow[swap]{dr}{q'} & Q \arrow[dashed]{d}{\exists!} \\ && Q' \end{tikzcd} \]
\end{topic}

\begin{topic}{fiber-product}{fiber product}
    Let $f : X \to Z$ and $g : Y \to Z$ be morphisms in a \tref{category}{category} $\mathcal{C}$. An object $W$ of $\mathcal{C}$ together with morphisms $\pi_X: W \to X$ and $\pi_Y : W \to Y$ is said to be a \textbf{fiber product} of the pair $f, g$ if $f \circ \pi_X = g \circ \pi_Y$ and for any other such $W'$ with maps $\pi_X'$ and $\pi_Y'$ there exists a unique morphism $h : W' \to W$ such that $\pi_X' = \pi_X \circ h$ and $\pi_Y' = \pi_Y \circ h$.
    \[ \begin{tikzcd} W' \arrow[bend left=30]{rrd}{\pi_X'} \arrow[swap, bend right=30]{ddr}{\pi_Y'} \arrow[dashed]{rd}{\exists!} & & \\ & W \arrow{r}{\pi_X} \arrow[swap]{d}{\pi_Y} & X \arrow{d}{f} \\ & Y \arrow[swap]{r}{g} & Z \end{tikzcd} \]
\end{topic}

\begin{topic}{pushout}{pushout}
    Let $f : X \to Y$ and $g : X \to Z$ be morphisms in a \tref{category}{category} $\mathcal{C}$. An object $W$ of $\mathcal{C}$ together with morphisms $i_Y: Y \to W$ and $i_Z : Z \to W$ is said to be a \textbf{pushout} of the pair $f, g$ if $i_Y \circ f = i_Z \circ g$ and for any other such $W'$ with maps $i'_Y$ and $i'_Z$ there exists a unique morphism $h : W \to W'$ such that $i'_Y = h \circ i_Y$ and $i'_Z = h \circ i_Z$.
    \[ \begin{tikzcd} X \arrow{r}{f} \arrow[swap]{d}{g} & Y \arrow{d}{i_Y} \arrow[bend left=30]{rdd}{i'_Y} & \\ Z \arrow[swap]{r}{i_Z} \arrow[swap, bend right=30]{drr}{i'_Z} & W \arrow[dashed]{rd}{\exists!} & \\ & & W' \end{tikzcd} \]
\end{topic}

\begin{topic}{regular-monomorphism}{regular monomorphism}
    A \tref{monomorphism}{monomorphism} $f : X \to Y$ is called a \textbf{regular monomorphism} if fits into an equalizer diagram:
    \[ \begin{tikzcd} X \arrow{r}{f} & Y \arrow[shift left=0.25em]{r} \arrow[shift right=0.25em]{r} & Z \end{tikzcd} \]
\end{topic}

\begin{topic}{adjoint-functors}{adjoint functors}
    Let $F : \mathcal{C} \to \mathcal{D}$ and $G : \mathcal{D} \to \mathcal{C}$ be a pair of functors. We say that $F$ is \textbf{left adjoint} to $G$, or $G$ is \textbf{right adjoint} to $F$, denoted $F \dashv G$, if there is a natural bijection
    \[ \Hom_{\mathcal{D}}(F(C), D) \xrightarrow{\sim} \Hom_{\mathcal{D}}(C, G(D)) \]
    for each object $C$ in $\mathcal{C}$ and $D$ in $\mathcal{D}$. Naturality means that for all $f : C \to C'$ in $\mathcal{C}$ and $g : D' \to D$ in $\mathcal{D}$, the following square commutes.
    \[ \begin{tikzcd}[row sep=3em] \Hom_{\mathcal{D}}(F(C), D) \arrow{r} & \Hom_{\mathcal{C}}(C, G(D)) \\ \Hom_{\mathcal{D}}(F(C'), D') \arrow{r} \arrow{u}{g \circ (-) \circ F(f)} & \Hom_{\mathcal{C}}(C', G(D')) \arrow[swap]{u}{G(g) \circ (-) \circ f} \end{tikzcd} \]
    Two maps $f : F(C) \to D$ and $g : C \to G(D)$ which correspond to each other are called \textit{transposes}.
\end{topic}

\begin{topic}{retraction}{retraction}
    A \textbf{retraction} of a morphism $i : X \to Y$ is a morphism $r : Y \to X$ such that $r \circ i = \id_X$.
\end{topic}

\begin{topic}{section}{section}
    A \textbf{section} of a morphism $p : X \to Y$ is a morphism $s : Y \to X$ such that $p \circ s = \id_Y$.
\end{topic}

\begin{topic}{retract}{retract}
    An object $X$ is called a \textbf{retract} of an object $Y$ if there are morphisms $i : X \to Y$ and $r : Y \to X$ such that $r \circ i = \id_X$, i.e. $r$ is a \tref{retraction}{retraction} of $i$.
\end{topic}

\begin{topic}{sieve}{sieve}
    Let $\mathcal{C}$ be a category, and $C$ an object of $\mathcal{C}$. A \textbf{sieve} on $C$ is a set of morphisms $S = \{ f : C' \to C \}$ such that if $f : C' \to C$ is in $S$ and $g : C'' \to C'$ is arbitrary, then $f \circ g$ is in $S$.
    
    Equivalently, it is a subpresheaf of $y_C$, the \tref{yoneda-embedding}{yoneda functor}.
\end{topic}

\begin{topic}{grothendieck-topology}{Grothendieck topology}
    Let $\mathcal{C}$ be a \tref{category}{category}. A \textbf{Grothendieck topology} on $\mathcal{C}$ consists of a family $\text{Cov}(C)$ of \tref{sieve}{sieves} on $C$ for every object $C$ of $\mathcal{C}$, called \textit{covering sieves}, such that
    \begin{itemize}
        \item the maximal sieve $\text{mac}(C) = \{ \text{all } f : C' \to C \}$ is in $\text{Cov}(C)$,
        \item if $R \in \text{Cov}(C)$ then for every $f : C' \to C$, $f^*(R) \in \text{Cov}(C')$,
        \item if $R$ is any sieve on $C$ and $S$ is a covering sieve on $C$, such that for every $f : C' \to C$ from $S$ we have $f^*(R) \in \text{Cov}(C')$, then $R \in \text{Cov}(C)$.
    \end{itemize}
\end{topic}

\begin{topic}{site}{site}
    A small \tref{category}{category} $\mathcal{C}$ together with a \tref{grothendieck-topology}{Grothendieck topology} on it is called a \textbf{site}.
\end{topic}

\begin{topic}{sheaf}{sheaf}
    A presheaf $\mathcal{F} : \mathcal{C}^\text{op} \to \textbf{Set}$ on a \tref{site}{site} $\mathcal{C}$ is a \textbf{sheaf} if for every covering sieve $\{ U_i \to U \}_{i \in I}$, the diagram
    \[ \mathcal{F}(U) \to \prod_{i \in I} \mathcal{F}(U_i) \rightrightarrows \prod_{i, j \in I} \mathcal{F}(U_i \times_U U_j) \]
    is an \tref{equalizer}{equalizer}.
\end{topic}

\begin{topic}{localization}{localization}
    Let $\mathcal{C}$ be a \tref{category}{category}, and $W$ a collection of morphisms of $\mathcal{C}$. A \textbf{localization} of $\mathcal{C}$ by $W$ is a category $\mathcal{C}\left[\frac{1}{W}\right]$ with a functor $Q : \mathcal{C} \to \mathcal{C}\left[\frac{1}{W}\right]$ such that
    \begin{itemize}
        \item $Q(f)$ is an isomorphism for all $f \in W$,
        \item for any category $\mathcal{D}$ and functor $F : \mathcal{C} \to \mathcal{D}$ such that $F(f)$ is an isomorphism for all $f \in W$, there exists a unique functor $G : \mathcal{C}\left[\frac{1}{W}\right] \to \mathcal{D}$ such that $F \simeq G \circ Q$.
        \[ \begin{tikzcd} \mathcal{C} \arrow{r}{Q} \arrow[swap]{rd}{F} & \mathcal{C}\left[\frac{1}{W}\right] \arrow[dashed]{d}{G} \\ & \mathcal{D} \end{tikzcd} \]
    \end{itemize}
\end{topic}

\begin{example}{localization}
    A \tref{CA:ring}{ring} $R$ can be considered as an \tref{HA:additive-category}{additive category} $\mathcal{R}$ with one object $\bdot$ via $R = \text{End}_\mathcal{R}(\bdot)$: multiplication is given by composition. Let $S$ is a multiplicatively closed subset of $R$ containing $1$. If $R$ is commutative, or more generally if $S$ is in the center of $R$, then the \tref{CA:localization}{localization} $S^{-1} R$ can be defined, which corresponds precisely to the localization $\mathcal{R}\left[\frac{1}{S}\right]$.
\end{example}

\begin{topic}{limit}{(co)limit}
    Given a \tref{functor}{functor} $F : \mathcal{C} \to \mathcal{D}$, a \textit{cone} for $F$ is an object $D$ of $\mathcal{D}$ together with a family of morphisms $\mu_C : D \to F(C)$ for all objects $C$ in $\mathcal{C}$, such that for all $f : C \to C'$ in $\mathcal{C}$,
    \[ \begin{tikzcd} & D \arrow[swap]{ld}{\mu_C} \arrow{rd}{\mu_{C'}} & \\ F(C) \arrow{rr}{F(f)} && F(C') \end{tikzcd} \]
    commutes in $\mathcal{D}$. A map of cones $(D, \mu) \to (D', \mu')$ is a map $g : D \to D'$ such that $\mu'_C g = \mu_C$ for all $C$ in $\mathcal{C}$. A \textit{limiting cone} or a \textbf{limit} for $F$ is a \tref{terminal-object}{terminal} object in the category of cones for $F$.
    
    Dually, one can define a \textit{cocone} for $F$ and \textit{colimiting cocones} to define a \textbf{colimit} for $F$.
\end{topic}

\begin{example}{limit}
    Let $\textbf{0}$ be the empty category. A limit for $! : \textbf{0} \to \mathcal{C}$ is a terminal object of $\mathcal{C}$, and a colimit is an initial object of $\mathcal{C}$.
\end{example}

\begin{example}{limit}
    Let $\textbf{2}$ be the discrete category with two objects. A functor $\textbf{2} \to \mathcal{C}$ is a pair $(A, B)$ of objects of $\mathcal{C}$, and a cone for this functor is an object $C$ with maps $C \to A$ and $C \to B$. Such a cone is a limiting cone iff for any $D$ with morphisms $D \to A$ and $D \to B$ there is a unique morphism $D \to C$ such that
    \[ \begin{tikzcd} & D \arrow[dashed]{d} \arrow{ld} \arrow{rd} & \\ A & \arrow{l} C \arrow{r} & B \end{tikzcd} \]
    commutes. In this case, $C$ is also known as the \textit{product} of $A$ and $B$, and denoted $A \times B$.
\end{example}

\begin{topic}{category-of-elements}{category of elements}
    The \textbf{category of elements} of a \tref{functor}{functor} $F : \mathcal{C} \to \textbf{Set}$ is the \tref{category}{category} $\int^{\mathcal{C}} F$ whose
    \begin{itemize}
        \item objects are pairs $(C, x)$ with $C$ an object of $\mathcal{C}$ and $x \in F(C)$,
        \item morphisms $(C, x) \to (C', x')$ are morphisms $f : C \to C'$ with $F(f)(x) = x'$.
    \end{itemize}
\end{topic}

\begin{topic}{small-category}{small category}
    A \tref{category}{category} $\mathcal{C}$ is \textbf{small} if its objects and morphisms form a set.
    
    A category $\mathcal{C}$ is \textbf{locally small} if for all objects $X, Y$, the hom-class $\Hom(X, Y)$ is a set.
\end{topic}

\begin{topic}{presheaf}{presheaf}
    A \textbf{presheaf} on a \tref{category}{category} $\mathcal{C}$ is a \tref{functor}{functor} $F : \mathcal{C}^\text{op} \to \textbf{Set}$.
\end{topic}

\begin{topic}{representable-functor}{representable functor}
    A \tref{functor}{functor} $F : \mathcal{C} \to \textbf{Set}$ is \textbf{representable} if it is \tref{natural-transformation}{naturally} isomorphic to $\Hom_\mathcal{C}(X, -)$ for some object $X$ of $\mathcal{C}$.
    
    Similarly, a functor $F : \mathcal{C}^\text{op} \to \textbf{Set}$ (i.e. a \tref{presheaf}{presheaf}) is \textbf{representable} if it is naturally isomorphic to $\Hom_\mathcal{C}(-, X)$ for some object $X$ of $\mathcal{C}$.
\end{topic}

\begin{topic}{opposite-category}{opposite category}
    The \textbf{opposite category} $\mathcal{C}^\text{op}$ of a \tref{category}{category} $\mathcal{C}$ is formed by reversing the morphisms. That is, the objects of $\mathcal{C}^\text{op}$ are the objects of $\mathcal{C}$, and there is a morphism $f^\text{op} : Y \to X$ in $\mathcal{C}^\text{op}$ for each morphism $f : X \to Y$ in $\mathcal{C}$.
\end{topic}

\begin{topic}{zero-object}{zero object}
    In a \tref{category}{category} $\mathcal{C}$, a \textbf{zero object} is an object which is both \tref{initial-object}{initial} and \tref{terminal-object}{terminal}.
\end{topic}

\begin{topic}{pointed-category}{pointed category}
    A \textbf{pointed category} is a \tref{category}{category} with a \tref{zero-object}{zero object}.
\end{topic}

\begin{topic}{fiber}{(co)fiber}
    Let $\mathcal{C}$ be a \tref{category}{category} with a \tref{terminal-object}{terminal object} $1$. A \textbf{fiber} of a morphism $f : X \to Y$ in $\mathcal{C}$ is a \tref{fiber-product}{pullback diagram}
    \[ \begin{tikzcd}
        X \times_Y 1 \arrow{r} \arrow{d} & X \arrow{d}{f} \\ 1 \arrow{r} & Y .
    \end{tikzcd} \]
    Dually, a \textbf{cofiber} of $f$ is a \tref{pushout}{pushout diagram}
    \[ \begin{tikzcd}
        X \arrow{r}{f} \arrow{d} & Y \arrow{d} \\ 1 \arrow{r} & Y \amalg_X 1 .
    \end{tikzcd} \]
\end{topic}

\begin{example}{fiber}
    If $\mathcal{C}$ is an \tref{HA:additive-category}{additive category}, fibers are the same as kernels, and cofibers are the same as cokernels.
\end{example}

\begin{topic}{slice-category}{slice category}
    Given a \tref{category}{category} $\mathcal{C}$ over an object $X$, the \textbf{slice category} $\mathcal{C}/X$ is the category whose
    \begin{itemize}
        \item objects are morphisms $f : Y \to X$ of $\mathcal{C}$,
        \item morphisms from $f : Y \to X$ to $f' : Y' \to X$ are morphisms $g : Y \to Y'$ such that $f' \circ g = f$.
        \[ \begin{tikzcd}[row sep=0em] Y \arrow{rd}{f} \arrow[swap]{dd}{g} & \\ & X \\ Y' \arrow[swap]{ur}{f'} & \end{tikzcd} \]
    \end{itemize}
\end{topic}

\begin{topic}{arrow-category}{arrow category}
    Given a \tref{category}{category} $\mathcal{C}$, the \textbf{arrow category} $\mathcal{C}^\rightarrow$ is the category whose
    \begin{itemize}
        \item objects are morphisms $f : X \to Y$ of $\mathcal{C}$,
        \item morphisms from $f : X \to Y$ to $f' : X' \to Y'$ are pairs $(g, h)$, where $g : X \to X'$ and $h : Y \to Y'$ such that $f' \circ g = h \circ f$.
        \[ \begin{tikzcd} X \arrow{r}{f} \arrow[swap]{d}{g} & Y \arrow{d}{h} \\ X' \arrow{r}{f'} & Y' \end{tikzcd} \]
    \end{itemize}
\end{topic}

\begin{topic}{comma-category}{comma category}
    The \textbf{comma category} of two \tref{functor}{functors} $F : \mathcal{C} \to \mathcal{E}$ and $G : \mathcal{D} \to \mathcal{E}$ is the category $F \downarrow G$ whose
    \begin{itemize}
        \item objects are triples $(C, D, f)$ where $C, D$ are objects of $\mathcal{C}$ and $\mathcal{D}$, respectively, and $f : F(C) \to G(D)$ a morphism in $\mathcal{E}$,
        \item morphisms from $(C, D, f)$ to $(C', D', f')$ are pairs $(g, h)$, where $g : C \to C'$ and $h : D \to D'$ are morphisms in $\mathcal{C}$ and $\mathcal{D}$, respectively, such that $f' \circ F(g) = G(h) \circ f$.
        \[ \begin{tikzcd}
            F(C) \arrow{r}{F(g)} \arrow[swap]{d}{f} & F(C') \arrow{d}{f'} \\ G(D) \arrow{r}{G(h)} & G(D')
        \end{tikzcd} \]
    \end{itemize}
\end{topic}

\begin{example}{comma-category}
    When $F = \id_\mathcal{C}$ and $G : \textbf{1} \to \mathcal{C}$, given by $G(\star) = X$, the comma category $F \downarrow G$ is the same the \tref{slice-category}{slice category} $\mathcal{C}/X$.
\end{example}

\begin{example}{comma-category}
    When $F = G = \id_\mathcal{C}$, the comma category $F \downarrow G$ is the same the \tref{arrow-category}{arrow category} $\mathcal{C}^\rightarrow$.
\end{example}

\begin{topic}{inverse-limit}{inverse limit}
    Let $\mathcal{C}$ be a \tref{category}{category}. An \textbf{inverse system} in $\mathcal{C}$ consists of a directed set $I$, an object $X_i$ for each $i \in I$, and a morphism $f_{ij} : X_j \to X_i$ for all $i \le j$, such that $f_{ii}$ is the identity on $X_i$ and
    \[ f_{ij} \circ f_{jk} = f_{ik} \quad \text{for all } i \le j \le k . \]
    An \textbf{inverse limit} of an inverse system is an object $X = \varprojlim_{i \in I} X_i$ in $\mathcal{C}$ with maps $\pi_i : X \to X_i$ for each $i \in I$ such that $\pi_i = f_{ij} \circ \pi_j$ for all $i \le j$, satisfying the universal property: for any other such object $Y$ with maps $p_i : Y \to Y_i$ there is a unique map $h : Y \to X$ such that $p_i = \pi_i \circ h$ for all $i \in I$.
    \[ \begin{tikzcd} Y \arrow[dashed, swap]{d}{h} \arrow{dr}{p_i} & \\ X \arrow{r}{\pi_i} & X_i \end{tikzcd} \]
    Equivalently, an inverse system is a \tref{functor}{functor} $F : I^\text{op} \to \mathcal{C}$ for a directed set $I$, and an inverse limit is a \tref{limit}{limit} for $F$.
\end{topic}

\begin{example}{inverse-limit}
    When $\mathcal{C}$ is the category of sets, groups, rings or modules, an inverse limit can explicitly be described by
    \[ \varprojlim_{i \in I} X_i = \left\{ (x_i)_{i \in I} \in \prod_{i \in I} X_i : f_{ij}(x_j) = x_i \text{ for all } i \le j \right\} . \]
\end{example}

\begin{topic}{direct-limit}{direct limit}
    Let $\mathcal{C}$ be a \tref{category}{category}. A \textbf{direct system} in $\mathcal{C}$ consists of a directed set $I$, an object $X_i$ for each $i \in I$, and a morphism $f_{ij} : X_i \to X_j$ for all $i \le j$, such that $f_{ii}$ is the identity on $X_i$ and
    \[ f_{jk} \circ f_{ij} = f_{ik} \quad \text{for all } i \le j \le k . \]
    A \textbf{direct limit} of a direct system is an object $X = \varinjlim_{i \in I} X_i$ in $\mathcal{C}$ with maps $\iota_i : X_i \to X$ for each $i \in I$ such that $\iota_i = \iota_j \circ f_{ij}$ for all $i \le j$, satisfying the universal property: for any other such object $Y$ with maps $\iota'_i : Y_i \to Y$ there is a unique map $h : X \to Y$ such that $\iota'_i = h \circ \iota_i$ for all $i \in I$.
    \[ \begin{tikzcd} X_i \arrow{r}{\iota_i} \arrow[swap]{dr}{\iota'_i} & X \arrow[dashed]{d}{h} \\ & Y \end{tikzcd} \]
    Equivalently, a direct system is a \tref{functor}{functor} $F : I \to \mathcal{C}$ for a directed set $I$, and a direct limit is a \tref{limit}{colimit} for $F$.
\end{topic}

\begin{example}{direct-limit}
    When $\mathcal{C}$ is the category of sets, groups, rings or modules, a direct limit can     explicitly be described by
    \[ \varinjlim_{i \in I} X_i = \bigsqcup_{i \in I} X_i / \sim{} , \]
    where $x_i \in X_i$ is equivalent to $x_j \in X_j$ if and only if there exists some $k \in I$ with $i, j \le k$ such that $f_{ik}(x_i) = f_{jk}(x_j)$. That is, $x_i$ is equivalent to $x_j$ if they `eventually become equal'.
\end{example}

\begin{topic}{conservative-functor}{conservative functor}
    A \tref{functor}{functor} $F : \mathcal{C} \to \mathcal{D}$ is \textbf{conservative} if it reflects \tref{isomorphism}{isomorphisms}. That is, if $F(f)$ is an isomorphism for some morphism $f$ in $\mathcal{C}$, then $f$ is an isomorphism as well.
\end{topic}

\begin{topic}{galois-category}{Galois category}
    A \textbf{Galois category} is a \tref{category}{category} $\mathcal{C}$ with a \tref{functor}{functor} $F : \mathcal{C} \to \textbf{FSet}$ to the category of finite sets, such that
    \begin{itemize}
        \item $\mathcal{C}$ has finite \tref{limit}{(co)limits},
        \item any morphism $f : X \to Y$ in $\mathcal{C}$ can be written as $f = m \circ e$ with $e$ an \tref{epimorphism}{epimorphism} and $m$ a \tref{monomorphism}{monomorphism} onto a direct summand of $Y$,
        \item $F$ preserves finite (co)limits and \tref{conservative-functor}{reflects isomorphisms}.
    \end{itemize}
\end{topic}

\begin{example}{galois-category}
    Let $X$ be a \tref{TO:connected-space}{connected} \tref{TO:topological-space}{topological space} and $x \in X$ a basepoint. The category $\textbf{FCov}_X$ of finite \tref{TO:covering-space}{coverings} $Y \to X$ together with the functor $F : \textbf{FCov}_X \to \textbf{FSet}$ which maps $Y \to X$ to the fiber $Y_x$ is a Galois category.
\end{example}

\begin{example}{galois-category}
    Let $k$ be a field, and $\mathcal{C}$ the \tref{opposite-category}{opposite} of the category $\textbf{SAlg}_k$ of free separable $k$-algebras, with $F(A) = \text{Alg}_k(A, k_s)$, where $k_s$ is a separable closure of $k$. Then $\mathcal{C}$ with $F$ is a Galois category.
\end{example}

\begin{example}{galois-category}
    Let $\pi$ be a \tref{GT:profinite-group}{profinite group}. The category $\pi\text{-}\textbf{set}$ of finite sets with a continuous $\pi$-action, and $F : \pi\text{-}\textbf{set} \to \textbf{Set}$ the forgetful functor, is a Galois category.
    
    It is a theorem that any essentially small Galois category is equivalent to $\pi\text{-}\textbf{set}$ for the uniquely determined profinite group $\pi = \text{Aut}(F)$.
\end{example}

\begin{topic}{fiber-category}{Fiber category}
    Let $\mathfrak{S}$ be a \tref{category}{category} and $\pi : \mathfrak{X} \to \mathfrak{S}$ a category over $\mathfrak{X}$. The \textbf{fiber category} $\mathfrak{X}_S$ of an object $S$ of $\mathfrak{S}$ is the subcategory of $\mathfrak{X}$ of all objects $x$ with $\pi(x) = S$ and morphisms $\alpha : x \to y$ with $\pi(\alpha) = \id_S$.
\end{topic}

\begin{topic}{strongly-cartesian-morphism}{strongly cartesian morphism}
    Let $\mathfrak{S}$ be a \tref{category}{category} and $\pi : \mathfrak{X} \to \mathfrak{S}$ a category over $\mathfrak{S}$. A morphism $\alpha : y \to x$ in $\mathfrak{X}$ is \textbf{strongly cartesian} if the map
    \[ \begin{aligned}
        \Hom_\mathfrak{X}(z, y) &\to \Hom_\mathfrak{X}(z, x) \times_{\Hom_\mathfrak{S}(\pi(z), \pi(x))} \Hom_\mathfrak{S}(\pi(z), \pi(y)) \\
        \beta &\mapsto (\alpha \circ \beta, \pi(\beta))
    \end{aligned} \]
    is a bijection. Intuitively, $y$ acts like a `fiber product' of $x$ and $\pi(y)$ over $\pi(x)$.
\end{topic}

\begin{topic}{fibered-category}{fibered category}
    Let $\mathfrak{S}$ be a \tref{category}{category}. A \textbf{fibered category} over $\mathfrak{S}$ is a category $\pi : \mathfrak{X} \to \mathfrak{S}$ over $\mathfrak{S}$ such that for every object $S$ in $\mathfrak{S}$ and $x$ in $\mathfrak{X}$ lying over $S$, and morphism $f : T \to S$, there exists a \tref{strongly-cartesian-morphism}{strongly cartesian morphism} $\alpha : y \to x$ with $\pi(y) = T$.
    \[ \begin{tikzcd} y \arrow{r}{\alpha} \arrow[rightsquigarrow]{d} & x \arrow[rightsquigarrow]{d} \\ T \arrow{r}{f} & S \end{tikzcd} \]
\end{topic}

\begin{topic}{descent-datum}{descent datum}
    Let $\mathfrak{S}$ be a \tref{category}{category} and $\pi : \mathfrak{X} \to \mathfrak{S}$ a \tref{fibered-category}{fibered category} over $\mathfrak{S}$. Let $\mathcal{U} = \{ f_i : S_i \to S \}_{i \in I}$ be a family of morphisms in $\mathfrak{S}$, and assume that all fiber products $S_{ij} = S_i \times_S S_j$ and $S_{ijk} = S_i \times_S S_j \times_S S_k$ exist. A \textbf{descent datum} $(x_i, \varphi_{ij})$ in $\mathfrak{X}$ relative to $\mathcal{U}$ consists of an object $x_i$ in $\mathfrak{X}$ over $S_i$ for each $i \in I$ and an isomorphism $\varphi_{ij} : \pi_0^* x_i \to \pi_1^* x_j$ (where $\pi_0 : S_{ij} \to S_i$ and $\pi_1 : S_{ij} \to S_j$) for each $i, j \in I$, satisying the \textit{cocycle condition}: for each $i, j, k \in I$ the diagram
    \[ \begin{tikzcd} \pi_0^* x_i \arrow{rr}{\pi_{02}^* \varphi_{ik}} \arrow[swap]{rd}{\pi_{01}^* \varphi_{ij}} && \pi_2^* x_k \\ & \pi_1^* x_j \arrow[swap]{ur}{\pi_{12}^* \varphi_{jk}} & \end{tikzcd} \]
    in $S_{ijk}$ commutes.
    
    A \textbf{morphism} of descent data $\psi : (x_i, \varphi_{ij}) \to (y_i, \phi_{ij})$ is a collection of morphisms $(\psi_i : x_i \to y_i)_{i \in I}$ in $\mathfrak{X}_{S_i}$ (i.e. $\pi(\psi_i) = \id_{S_i}$) such that for all $i, j \in I$ the diagram
    \[ \begin{tikzcd} \pi_0^* x_i \arrow{r}{\varphi_{ij}} \arrow[swap]{d}{\pi_0^* \psi_i} & \pi_1^* x_j \arrow{d}{\pi_1^* \psi_j} \\ \pi_0^* y_i \arrow{r}{\phi_{ij}} & \pi_1^* y_j \end{tikzcd} \]
    in $S_{ij}$ commutes.
    
    A descent datum $(x_i, \varphi_{ij})$ is \textbf{effective} if there exists an object $x$ of $\mathfrak{X}$ over $S$ such that $(X_i, \varphi_{ij})$ is isomorphic to the \textit{canonical descent datum} $(f_i^* x, \text{can})$.
\end{topic}

\begin{topic}{category-fibered-in-groupoids}{category fibered in groupoids}
    A \tref{category}{category} $\mathfrak{X}$ over $\mathfrak{S}$ is called a \textbf{category fibered in groupoids} over $\mathfrak{S}$ if for any $f : T \to S$ in $\mathfrak{S}$ and object $x$ over $S$, there exists a lift $\overline{f} : y \to x$ of $f$, which is unique up to unique isomorphism. That is, for any other lift $\overline{f}' : y' \to x$ of $f$, there exists a unique isomorphism $\alpha : y' \to y$ such that $\overline{f}' = \overline{f} \circ \alpha$.
\end{topic}

\begin{example}{category-fibered-in-groupoids}
    Motivating the terminology, if $\mathfrak{X}$ is a category fibered in groupoids over $\mathfrak{S}$, then every morphism $\varphi : y \to x$ of $\mathfrak{X}$ that lies over an isomorphism $f : T \to S$ of $\mathfrak{S}$, is an isomorphism as well. In particular, all the \tref{fiber-category}{fibers} of $\mathfrak{X}$ are \tref{groupoid}{groupoids}.

    Namely, let $g$ be the inverse of $f$, and choose a lifting $\overline{g} : z \to y$ of $g$. Now $\varphi \circ \overline{g} : z \to x$ lies over $f \circ g = \id_S$, so is a lifting of $\id_S$ with target $x$. Since $\id_x$ is so as well, there exists an isomorphism $\alpha : z \to x$ such that $\varphi \circ \overline{g} = \alpha$. Now it is clear that $\overline{g} \circ \alpha^{-1}$ is the inverse of $\varphi$.
\end{example}

\begin{topic}{monad}{monad}
    A \textbf{monad} on a \tref{category}{category} $\mathcal{C}$ is a \tref{functor}{functor} $T : \mathcal{C} \to \mathcal{C}$ together with two \tref{natural-transformation}{natural transformations} $\mu : T^2 \Rightarrow T$ and $\eta : \id_{\mathcal{C}} \Rightarrow T$, such that
    \[ \begin{tikzcd} T^3 \arrow{r}{T \mu} \arrow[swap]{d}{\mu T} & T^2 \arrow{d}{\mu} \\ T^2 \arrow{r}{\mu} & T \end{tikzcd} \qquad \text{and} \qquad \begin{tikzcd} T \arrow{r}{\eta T} \arrow[swap]{dr}{\id_T} & T^2 \arrow{d}{\mu} & \arrow[swap]{l}{T \eta} T \arrow{ld}{\id_T} \\ & T & \end{tikzcd} \]
    commute.
\end{topic}

\begin{example}{monad}
    Any \tref{CT:adjoint-functors}{adjunction} $F : \mathcal{C} \to \mathcal{D}$ and $G : \mathcal{D} \to \mathcal{C}$ gives rise to a monad with $T = GF$. The map $\eta : \id_{\mathcal{C}} \to T$ is the unit of the adjunction, and the map $\mu : T^2 \to T$ is $G \varepsilon F$, where $\varepsilon : FG \to \id_{\mathcal{D}}$ is the counit of the adjunction.
    % Conversely, any monad can be found as an explicit adjunction of functors, using the \textit{Eilenberg--Moore category}.
\end{example}

\begin{topic}{eilenberg-moore-category}{Eilenberg--Moore category}
    Let $(T, \mu, \eta)$ be a \tref{monad}{monad} on a \tref{category}{category} $\mathcal{C}$.  A \textbf{$T$-algebra} in $\mathcal{C}$ is a pair $(X, h)$ of an object $X$ in $\mathcal{C}$ and a morphism $h : TX \to X$, such that
    \[ \begin{tikzcd} T^2 X \arrow[swap]{d}{\mu_X} \arrow{r}{Th} & TX \arrow{d}{h} \\ TX \arrow{r}{h} & X \end{tikzcd} \qquad \text{and} \qquad \begin{tikzcd} X \arrow[swap]{dr}{\id_X} \arrow{r}{\eta_X} & TX \arrow{d}{h} \\ & X \end{tikzcd} \]
    commute. Morphisms between $T$-algebras $(X, h) \to (Y, k)$ are morphisms $f : X \to Y$ in $\mathcal{C}$ for which
    \[ \begin{tikzcd} TX \arrow{r}{Tf} \arrow[swap]{d}{h} & TY \arrow{d}{k} \\ X \arrow{r}{f} & Y \end{tikzcd} \]
    commutes. The resulting category $T\text{-Alg}$ of $T$-algebras and morphisms is called the \textbf{Eilenberg--Moore category} of the monad.
\end{topic}

\begin{example}{eilenberg-moore-category}
    There is an \tref{adjoint-functors}{adjunction} between $T\text{-Alg}$ and $\mathcal{C}$ yielding the initial monad. The forgetful functor $U : T\text{-Alg} \to \mathcal{C}$, mapping $(X, h)$ to $h$, has a left adjoint $F : \mathcal{C} \to T\text{-Alg}$, mapping an object $X$ to the $T$-algebra $(TX, \mu_X)$, and a morphism $f : X \to Y$ to $Tf$. One can show $F \dashv U$ and clearly $T = UF$.
\end{example}

\begin{topic}{cartesian-closed-category}{cartesian closed category}
    A \tref{category}{category} $\mathcal{C}$ is \textbf{cartesian closed} if it has finite products, and for every object $X$ of $\mathcal{C}$ the product functor $(-) \times X$ has a \tref{adjoint-functors}{right adjoint} $(-)^X$.
\end{topic}

\begin{example}{cartesian-closed-category}
    The category \textbf{Set} is cartesian closed, where $Y^X = \Hom_\textbf{Set}(X, Y)$.
\end{example}

\begin{topic}{thin-category}{thin category}
    A \tref{category}{category} is \textbf{thin} if there exists at most one morphism between any two given objects.
\end{topic}

% \begin{topic}{exponential-object}{exponential object}
%     Let $X$ and $Y$ be objects of a \tref{category}{category} $\mathcal{C}$
% \end{topic}
