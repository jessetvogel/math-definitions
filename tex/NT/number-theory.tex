\begin{topic}{number-field}{number field/ring}
    A \textbf{number field} is a finite \tref{AA:field-extension}{field extension} of the field of rational numbers $\QQ$. A \textbf{number ring} is a subring of a number field.
\end{topic}

\begin{topic}{diophantine-equation}{Diophantine equation}
    A \textbf{Diophantine equation} is a polynomial equation, usually in two or more unknowns, such that the only solutions of interest are the integer ones.
\end{topic}

\begin{topic}{pell-equation}{Pell equation}
    The \textbf{Pell equation} is any \tref{diophantine-equation}{Diophantine equation} of the form $x^2 - dy^2 = 1$, where $d \in \ZZ$ is not a square.
\end{topic}

\begin{topic}{wilson-theorem}{Wilson's theorem}
    Let $n > 1$ be a natural number. \textbf{Wilson's theorem} states that $n$ is a prime number if and only if
    \[ (n - 1) ! \equiv -1 \mod n . \]
\end{topic}

\begin{topic}{gaussian-integers}{Gaussian integers}
    The ring of \textbf{Gaussian integers} is the ring
    \[ \ZZ[i] = \{ a + bi : a, b \in \ZZ \}, \]
    where $i^2 = -1$.
\end{topic}

\begin{topic}{legendre-symbol}{Legendre symbol}
    For $p$ an odd prime number and $d$ an integer, the \textbf{Legendre symbol} $\left(\tfrac{d}{p}\right)$ is defined as
    \[ \left(\frac{d}{p}\right) = \left\{ \begin{array}{cl}
        0 & \text{ if $a \equiv 0 \text{ mod } p$} , \\
        1 & \text{ if $a \text{ mod } p$ is a square in $\ZZ/p\ZZ$} ,  \\
        -1 & \text{ otherwise} .
    \end{array} \right. \]
\end{topic}

\begin{topic}{bezout-identity}{Bézout's identity}
    \textbf{Bézout's identity} states that for any integers $a, b$ with greatest common divisor $d$, there exists integers $x, y$ such that $ax + by = d$.
    
    This statement also works for \tref{AA:principal-ideal-domain}{PID's}: $(a) + (b) = (\gcd(a, b))$.
\end{topic}

\begin{topic}{p-adic-numbers}{p-adic numbers}
    Let $p$ be a prime number. The \textbf{ring of $p$-adic integers} is defined as the inverse limit
    \[ \ZZ_p = \varprojlim_{n \ge 1} \ZZ / p^n \ZZ . \]
    That is, a $p$-adic integer is a sequence $(a_n)_{n \ge 1}$ such that $a_n \in \ZZ / p^n \ZZ$ and $a_m \equiv a_n \mod p^m$ for $m \le n$.
    
    The ring of $p$-adic rationals $\QQ_p$ is defined as the \tref{AA:field-of-fractions}{field of fractions} of $\ZZ_p$.
\end{topic}

\begin{example}{p-adic-numbers}
    Any $p$-adic integer can be uniquely expressed as $\sum_{i \ge 0} a_i p^i$ for some $a_i \in \{ 0, 1, 2, \ldots, p - 1 \}$. For example,
    \[ 7 = 2 \cdot 3^1 + 1 \in \ZZ_3 . \]
    We can compute some $p$-adic digits of $\tfrac{1}{7}$ using long division.
    \[ \begin{array}{ccccccccl}
        \ldots & 0 & 0 & 0 & 0 & 0 & 0 & 1 & / \; 2 \; 1 \\
               &   &   &   &   &   & 2 & 1 & (1) \\ \hline
        % --------------------------------- -
        \ldots & 2 & 2 & 2 & 2 & 2 & 1 \\
               &   &   &   &   & 2 & 1 &   & (1) \\ \hline
        % --------------------------------- -
        \ldots & 2 & 2 & 2 & 2 & 0 \\
               &   &   &   &   & 0 &   &   & (0) \\ \hline
        % --------------------------------- -
        \ldots & 2 & 2 & 2 & 2 \\
               &   &   & 4 & 2 &   &   &   & (2) \\ \hline
        % --------------------------------- -
        \ldots & 2 & 1 & 1 \\
               &   & 2 & 1 &   &   &   &   & (1) \\ \hline
        % --------------------------------- -
        \ldots & 1 & 2
    \end{array} \]
    Continuing this process, one finds
    \[ \tfrac{1}{7} = 1 \; 1 0 2 1 2 0 \; 1 0 2 1 2 0 \; 1 0 2 1 2 0 \; \ldots \in \ZZ_3 . . \]
\end{example}

\begin{topic}{discriminant}{discriminant}
    Let $f \in R[X]$ be a polynomial over some \tref{AA:domain}{domain} $R$. Then $f$ can be written as
    \[ f = a \prod_{i = 1}^n (X - \alpha_i) \]
    in some sufficiently large extension $R' \supset R$ of domains, with $\alpha_i \in R'$ and $a \in R$. The \textbf{discriminant} of $f$ is defined as
    \[ \Delta(f) = a^{2n - 2} \prod_{1 \le i < j \le n} (\alpha_i - \alpha_j)^2 \in R . \]
\end{topic}

\begin{example}{discriminant}
    For quadratic polynomials,
    \[ \Delta(aX^2 + bX + c) = b^2 - 4ac . \]
    For cubic polynomials,
    \[ \Delta(X^3 + pX + q) = -4p^3 - 27q^2 . \]
\end{example}

\begin{topic}{resultant}{resultant}
    The \textbf{resultant} of two polynomials
    \[ f = a \prod_{i = 1}^{n} (X - \alpha_i) \quad \text{ and } \quad g = b \prod_{j = 1}^{m} (X - \beta_j) \]
    in $k[X]$ for some field $k$, is defined as
    \[ R(f, g) = a^m b^n \prod_{i = 1}^{n} \prod_{j = 1}^{m} (\alpha_i - \beta_j) \in k . \]
\end{topic}

\begin{example}{resultant}
    The resultant can be used to compute \tref{discriminant}{discriminants}. In particular, for a polynomial $f = a \prod_{i = 1}^{n} (X - \alpha_i)$ we have
    \[ R(f, f') = a^{n - 1} \prod_{i = 1}^{n} f'(\alpha_i) = \frac{(-1)^{n(n - 1)/2}}{a} \Delta(f) . \]
\end{example}

\begin{topic}{abel-ruffini-theorem}{Abel--Ruffini theorem}
    The \textbf{Abel--Ruffini theorem} states that polynomials of degree $\ge 5$ cannot be solved in general with solutions in radicals.
\end{topic}

\begin{example}{abel-ruffini-theorem}
    \begin{proof}
        Let $f \in k[x]$ be a polynomial over a field $k$, and let $\ell/k$ be a \tref{AA:splitting-field}{splitting field} of $f$. Then in particular $\ell/k$ is a \tref{AA:galois-extension}{Galois extension}. Now if $f$ is solvable by radicals, then we can write
        \[ k = \ell_0 \subset \ell_1 \subset \ell_2 \subset \cdots \subset \ell_m = \ell , \]
        where each $\ell_{i + 1}/\ell_i$ is a simple radical extension, that is, $\ell_{i + 1} = \ell_i(\sqrt[n_i]{\alpha_i})$ for some $n_i \in \NN$ and $\alpha_i \in \ell_i$. This implies the Galois groups $\textup{Gal}(\ell_{i + 1}/\ell_i) \isom \ZZ/n_i\ZZ$ are all \tref{GT:cyclic-group}{cyclic}, and via the \tref{AA:galois-correspondence}{Galois correspondence} we obtain
        \[ \textup{Gal}(\ell/k) = \textup{Gal}(\ell/\ell_0) \supset \textup{Gal}(\ell/\ell_1) \supset \cdots \supset \textup{Gal}(\ell/\ell) = \{ 1 \} . \]
        Since each quotient $\textup{Gal}(\ell/\ell_i) / \textup{Gal}(\ell/\ell_{i + 1}) \isom \textup{Gal}(\ell_{i + 1}/\ell)$ is cyclic, it follows that $\textup{Gal}(\ell/k)$ is a \tref{GT:solvable-group}{solvable group}.
        
        However, when $\deg(f) \ge 5$, the Galois group is generally not solvable. For example, the Galois group of the quintic polynomial $f = x^5 - x - 1 \in \QQ[x]$ is the symmetric group $S_5$, which is not solvable.
    \end{proof}
\end{example}
