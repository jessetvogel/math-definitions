% Abelian categories
\begin{topic}{additive-category}{additive category}
    An \textbf{additive category} is a category $\mathcal{A}$ such that
    \begin{itemize}
        \item every $\Hom_\mathcal{A}(A, B)$ is an abelian group, and composition is bilinear,
        \item there is a zero object $0$ which is both \tref{CT:terminal-object}{terminal} and \tref{CT:initial-object}{initial},
        \item direct sums and direct products exist, and coincide.
    \end{itemize}
\end{topic}

\begin{topic}{abelian-category}{abelian category}
    An \textbf{abelian category} is an \tref{additive-category}{additive category} category $\mathcal{A}$ in which
    \begin{itemize}
        \item all kernels and cokernels exist,
        \item every monomoprhism is the kernel of some morphism, and every epimoprhism is the cokernel of some morphism.
    \end{itemize}
\end{topic}

\begin{topic}{additive-functor}{additive functor}
    Let $\mathcal{A}$ and $\mathcal{B}$ be \tref{additive-category}{additive categories}. A functor $F : \mathcal{A} \to \mathcal{B}$ is called \textbf{additive} if for all $A, B$ in $\mathcal{A}$ the map
    \[ \Hom_\mathcal{A}(A, B) \to \Hom_\mathcal{B}(FA, FB) \]
    is a group morphism.
    
    It follows that an additive functor preserves finite direct sums and sends $0$ to $0$.
\end{topic}

\begin{topic}{exact-functor}{exact functor}
    Let $F : \mathcal{A} \to \mathcal{B}$ be an \tref{additive-functor}{additive functor} between \tref{abelian-category}{abelian categories}. Then $F$ is called \textbf{left exact} if for all short exact sequences
    \[ 0 \to M \to N \to L \to 0 \]
     in $\mathcal{A}$, the sequence
    \[ 0 \to FM \to FN \to FL \]
    is exact. Similarly, $F$ is called \textbf{right exact} if for all such short exact sequences
    \[ FM \to FN \to FL \to 0 \]
    is exact. Finally, $F$ is called \textbf{exact} if it is both left and right exact.
\end{topic}

\begin{topic}{injective-object}{injective object}
    Let $\mathcal{A}$ be an \tref{abelian-category}{abelian category}. An object $I$ of $\mathcal{A}$ is called \textbf{injective} if the functor
    \[ \Hom_\mathcal{A}(-, I) : \mathcal{A}^\text{op} \to \textbf{Ab} \]
    is \tref{exact-functor}{exact}. Since it is already left exact for any object $I$, this is equivalent to the condition that for all monomorphisms $f : M \to N$ and morphisms $g : M \to I$ there exists a morphism $h : N \to I$ such that $hf = g$.
    \[ \begin{tikzcd} M \arrow[hookrightarrow]{r}{f} \arrow[swap]{d}{g} & N \arrow[dashed]{ld}{\exists h} \\ I & \end{tikzcd} \]
\end{topic}

\begin{topic}{projective-object}{projective object}
    Let $\mathcal{A}$ be an \tref{abelian-category}{abelian category}. An object $P$ of $\mathcal{A}$ is called \textbf{projective} if the functor
    \[ \Hom_\mathcal{A}(P, -) : \mathcal{A} \to \textbf{Ab} \]
    is \tref{exact-functor}{exact}. Since it is already right exact for any object $P$, this is equivalent to the condition that for all epimorphisms $f : N \to M$ and morphisms $g : P \to M$ there exists a morphism $h : P \to N$ such that $fh = g$.
    \[ \begin{tikzcd} & N \arrow[twoheadrightarrow]{d}{f} \\ P \arrow[swap]{r}{g} \arrow[dashed]{ur}{\exists h} & M \end{tikzcd} \]
\end{topic}

% Complexes
\begin{topic}{chain-complex}{chain complex}
    Let $\mathcal{A}$ be an \tref{abelian-category}{abelian category}. A \textbf{chain complex} $A_\bdot$ in $\mathcal{A}$ is a sequence of objects $A_i$ in $\mathcal{A}$, for $i \in \ZZ$, together with morphisms $d_i : A_i \to A_{i - 1}$ such that $d_{i - 1} \circ d_i = 0$ for all $i$.
    \[ \cdots \xleftarrow{d_{i - 1}} A_{i - 1} \xleftarrow{d_i} A_i \xleftarrow{d_{i + 1}} A_{i + 1} \xleftarrow{d_{i + 2}} \cdots \]
    Dually, a \textbf{cochain complex} $A^\bdot$ in $\mathcal{A}$ is a sequence of objects $A^i$ in $\mathcal{A}$, for $i \in \ZZ$, together with morphisms $d^i : A^i \to A^{i + 1}$ such that $d^{i + 1} \circ d^i = 0$ for all $i$.
    \[ \cdots \xrightarrow{d^{i - 2}} A^{i - 1} \xrightarrow{d^{i - 1}} A^i \xrightarrow{d^i} A^{i + 1} \xrightarrow{d^{i + 1}} \cdots \]
    
    A \textbf{morphism of chain complexes} or \textbf{chain map} $f : A_\bdot \to B_\bdot$ is given by a collection of morphisms $f_i : A_i \to B_i$ for each $i \in \ZZ$ such that $d^B_i \circ f_i = f_{i - 1} \circ d^A_i$. Dually, there are \textbf{cochain maps}.
    
    If the objects of $\mathcal{A}$ are sets, then elements of $A_n$ are called \textit{$n$-chains}, elements of $\ker d_n$ are called \textit{$n$-cycles}, and elements of $\im d_{n - 1}$ are called \textit{$n$-boundaries}. Similarly, elements of $A^n$ are called \textit{$n$-cochains}, elements of $\ker d^n$ are called \textit{$n$-cocycles}, and elements of $\im d^{n - 1}$ are called \textit{$n$-coboundaries}. Note that any $n$-(co)boundary is an $n$-(co)cycle.
\end{topic}

\begin{topic}{homology-object}{homology object}
    Let $\mathcal{A}$ be an \tref{abelian-category}{abelian category}, and $A^\bdot$ a \tref{chain-complex}{complex} in $\mathcal{A}$. For each $i \in \ZZ$, the quotient
    \[ H^i(A^\bdot) = \ker d_n / \im d_{n + 1} \]
    is called the \textbf{$i$-th homology object} of $A^\bdot$.
\end{topic}

\begin{topic}{long-exact-sequence-homology}{long exact sequence in homology}
    Let $\mathcal{A}$ be an \tref{abelian-category}{abelian category}, and let
    \[ 0 \rightarrow A^\bdot \rightarrow B^\bdot \rightarrow C^\bdot \rightarrow 0 \]
    be a short exact sequence of \tref{chain-complex}{complexes} in $\mathcal{A}$. Then there is an induced long exact sequence
    \[ \cdots \rightarrow H^i(A^\bdot) \rightarrow H^i(B^\bdot) \rightarrow H^i(C^\bdot) \rightarrow H^{i + 1}(A^\bdot) \rightarrow \cdots . \]
\end{topic}

\begin{topic}{homotopy}{homotopy}
    Let $\mathcal{A}$ be an \tref{abelian-category}{abelian category}, and let $f, g : M^\bdot \to N^\bdot$ be morphisms of \tref{chain-complex}{complexes} in $\mathcal{A}$. A \textbf{homotopy} from $f$ to $g$ is a collection of morphisms $k^i : M^i \to N^{i - 1}$ such that $f - g = dk + kd$.
    
    If a homotopy exists from $f$ to $g$, we say that $f$ and $g$ are \textbf{homotopic}. This gives an equivalence relation on the set of morphisms from $M^\bdot$ to $N^\bdot$.
\end{topic}

\begin{topic}{homotopy-equivalence}{homotopy equivalence}
    Let $\mathcal{A}$ be an \tref{abelian-category}{abelian category}. A morphism of complexes $f : M^\bdot \to N^\bdot$ is called a \textbf{homotopy equivalence} if there exists a morphism $g : N^\bdot \to M^\bdot$ such that $fg$ and $gf$ are \tref{homotopy}{homotopic} to the identity morphism.
\end{topic}

\begin{topic}{quasi-isomorphism}{quasi-isomorphism}
    Let $\mathcal{A}$ be an \tref{abelian-category}{abelian-category}. A morphism $f : M^\bdot \to N^\bdot$ of \tref{chain-complex}{complexes} in $\mathcal{A}$ is called a \textbf{quasi-isomorphism} if the induced maps on homology
    \[ H^i(f) : H^i(M^\bdot) \to H^i(N^\bdot) \]
    are all isomorphisms.
\end{topic}

\begin{topic}{injective-resolution}{injective resolution}
    Let $\mathcal{A}$ be an \tref{abelian-category}{abelian category} and $M$ an object in $\mathcal{A}$. An \textbf{injective resolution} of $M$ is an exact sequence
    \[ 0 \to M \to I^0 \to I^1 \to \cdots \]
    where all $I^i$ are \tref{injective-object}{injective objects}. Equivalently, this can be seen as a \tref{quasi-isomorphism}{quasi-isomorphism} $M[0] \to I^\bdot$ in $\text{Comp}(\mathcal{A})$.
\end{topic}

\begin{topic}{projective-resolution}{projective resolution}
    Let $\mathcal{A}$ be an \tref{abelian-category}{abelian category} and $M$ an object in $\mathcal{A}$. A \textbf{projective resolution} of $M$ is an exact sequence
    \[ \cdots \to P^1 \to P^0 \to M \to 0 \]
    where all $P^i$ are \tref{projective-object}{projective objects}. Equivalently, this can be seen as a \tref{quasi-isomorphism}{quasi-isomorphism} $M[0] \to P^\bdot$ in $\text{Comp}(\mathcal{A})$.
\end{topic}

% Derived functors
\begin{topic}{right-derived-functors}{right derived functors}
    Let $\mathcal{A}$ and $\mathcal{B}$ be \tref{abelian-category}{abelian categories}, assume that $\mathcal{A}$ has \textit{enough injectives}, and let $F : \mathcal{A} \to \mathcal{B}$ be a \tref{exact-functor}{left exact} functor. Take $M$ to be an object of $\mathcal{A}$ and pick an \tref{injective-resolution}{injective resolution} $M[0] \to I^\bdot$. Then we define
    \[ R^i F M = H^i(F(I^\bdot)) . \]
    This is well-defined (independent of the chosen resolution) and yields additive functors $R^i F : \mathcal{A} \to \mathcal{B}$, called the \textbf{right derived functors} of $F$. Note that $R^0 F \simeq F$.
    
    For each short exact sequence $0 \to A \to B \to C \to 0$ in $\mathcal{A}$ there is an associated long exact sequence
    \[ 0 \to FA \to FB \to FC \to R^1FA \to R^1FB \to R^1FC \to R^2FA \to \cdots \]
\end{topic}

% Homological lemmas
\begin{topic}{five-lemma}{five lemma}
    Let $\mathcal{A}$ be an abelian category, and consider the following commutative diagram.
    \[ \begin{tikzcd} A \arrow{r}{f} \arrow{d}{a} & B \arrow{r}{g} \arrow{d}{b} & C \arrow{r}{h} \arrow{d}{c} & D \arrow{r}{i} \arrow{d}{d} & E \arrow{d}{e} \\ A' \arrow{r}{f'} & B' \arrow{r}{g'} & C' \arrow{r}{h'} & D \arrow{r}{i'} & E' \end{tikzcd} \]
    Suppose that both rows are exact. The \textbf{five lemma} states that
    \begin{itemize}
        \item if $b$ and $d$ are injective and $a$ is surjective, then $c$ is injective.
        
        \item if $b$ and $d$ are surjective and $e$ is injective, then $c$ is surjective.
        
        \item if $b$ and $d$ are isomorphisms, $a$ is surjective and $e$ is injective, then $c$ is an isomorphism.
    \end{itemize}
\end{topic}

\begin{topic}{snake-lemma}{snake lemma}
    Let $\mathcal{A}$ be an abelian category, and consider the following commutative diagram.
    \[ \begin{tikzcd} & A \arrow{d}{a} \arrow{r}{f} & B \arrow{d}{b} \arrow{r}{g} & C \arrow{d}{c} \arrow{r} & 0 \\ 0 \arrow{r} & A' \arrow{r}{f'} & B' \arrow{r}{g'} & C' \end{tikzcd} \]
    Suppose that the rows are exact. The \textbf{snake lemma} states that there is an exact sequence of kernels and cokernels
    \[ \ker a \rightarrow \ker b \rightarrow \ker c \xrightarrow{d} \coker a \rightarrow \coker b \rightarrow \coker c \]
    where $d$ is known as the \textit{connecting homomorphism}. Furthermore, if $f$ is \tref{CT:monomorphism}{mono} then so is $\ker a \to \ker b$, and if $g'$ is \tref{CT:epimorphism}{epi} then so is $\coker b \to \coker c$.
\end{topic}

\begin{topic}{spectral-sequence}{spectral sequence}
    Let $\mathcal{A}$ be an \tref{abelian-category}{abelian category}. A \textbf{spectral sequence} in $\mathcal{A}$ is a collection of objects
    \[ (E_r^{p, q}, E^n) \qquad n, p, q, r \in \ZZ, r \ge 1 \]
    and morphisms
    \[ d_r^{p, q} : E_r^{p, q} \to E_r^{p + r, q - r + 1} \]
    satisfying
    \begin{itemize}
        \item $d_r^2 = 0$ (i.e. the $E_r^{p + \bdot r, q - \bdot r + \bdot}$ are complexes),
        \item there are isomorphisms (which are part of the data)
        \[ E_{r + 1}^{p, q} \simeq H^0(E_r^{p + \bdot r, q - \bdot r + \bdot}), \]
        \item for any $(p, q)$ there exists an $r_0$ such that $d_r^{p, q} = d_r^{p - r, q + r - 1} = 0$ for all $r \ge r_0$. In particular, $E_r^{p, q} \simeq E_{r_0}^{p, q}$ for all $r \ge r_0$, and this object is denoted by $E_\infty^{p, q}$.
        \item There is a decreasing filtration
        \[ 0 \subset \cdots \subset F^{p + 1} E^n \subset F^p E^n \subset \cdots E^n , \]
        with $\cap_p F^p E^n = 0$ and $\cup_p F^p E^n = E^n$, and isomorphisms
        \[ E_\infty^{p, q} \simeq F^p E^{p + q} / F^{p + 1} E^{p + q} . \]
    \end{itemize}
    
    In some sense, the objects $E_r^{p, q}$ converge towards subquotients of a certain filtration of $E^n$. Usually, the objects of one layer of some fixed $r$ are given. Then one writes
    \[ E_r^{p, q} \Rightarrow E^{p + q} . \]
\end{topic}

\begin{topic}{serre-subcategory}{Serre subcategory}
    A non-empty \tref{CT:full-subcategory}{full subcategory} $\mathcal{B}$ of an \tref{abelian-category}{abelian category} $\mathcal{A}$ is a \textbf{Serre subcategory} if for any exact sequence in $\mathcal{A}$,
    \[ 0 \to A \to B \to C \to 0 , \]
    $B$ is in $\mathcal{B}$ if and only if $A$ and $C$ are in $\mathcal{B}$. In words, $\mathcal{B}$ is closed under taking subobjects, quotient objects and extensions.
\end{topic}
