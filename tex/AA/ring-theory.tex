\begin{topic}{ring}{ring}
    A \textbf{ring} is an \tref{GT:abelian-group}{abelian group} $R$ with an operation called multiplication and an element $1 \in R$ satisfying
    \begin{itemize}
        \item (\textit{associativity}) $a(bc) = (ab)c$,
        \item (\textit{distributivity}) $a(b + c) = ab + ac$ and $(a + b)c = ac + bc$,
        \item (\textit{unit}) $1 \cdot a = a \cdot 1 = a$,
    \end{itemize}
    for all $a, b, c \in R$.
    
    A ring $R$ is called \textbf{commutative} if moreover $ab = ba$ for all $a, b \in R$.
\end{topic}

\begin{topic}{ring-morphism}{ring morphism}
    A \textbf{ring morphism} is a map $f : R \to S$ between \tref{ring}{rings} satisfying
    \begin{itemize}
        \item $f(1) = 1$,
        \item $f(a + b) = f(a) + f(b)$,
        \item $f(ab) = f(a) f(b)$,
    \end{itemize}
    for all $a, b \in R$.
\end{topic}

\begin{topic}{unit}{unit}
    Let $R$ be a \tref{ring}{ring}. An element $a \in R$ is called a \textbf{unit} if there exists some $b \in R$ such that $ab = 1$. This is denoted $b = a^{-1}$.
\end{topic}

\begin{example}{unit}
    If $a \in R$ is a unit, its inverse is unique. Namely, if $ab = ca = 1$ for some $b, c \in R$, then $b = cab = c$.
\end{example}

\begin{topic}{zero-divisor}{zero-divisor}
    Let $R$ be a \tref{ring}{ring}. An element $a \in R$ is called a \textbf{left zero-divisor} if $a \ne 0$ and $ab = 0$ for some $b \ne 0$. Similarly, an element $a \in R$ is called a \textbf{right zero-divisor} if $a \ne 0$ and $ba = 0$ for some $b \ne 0$. An element $a \in R$ is called a \textbf{zero-divisor} if it is both a left zero-divisor and a right zero-divisor.
\end{topic}

\begin{example}{zero-divisor}
    The zero-divisors of the ring $\ZZ / 6 \ZZ$ are $\{ \overline{2}, \overline{3}, \overline{4} \}$, since $\overline{2} \cdot \overline{3} = \overline{3} \cdot \overline{4} = 0$.
\end{example}

\begin{example}{zero-divisor}
    Let $R$ be the ring of endomorphisms of $V = k^\infty$, for some \tref{field}{field} $k$. Let $f : V \to V$ be given by $f(e_i) = e_{i + 1}$ with $\{ e_i : i \in \NN \}$ denotes the standard basis of $V$. Then $f$ is a right zero-divisor of $R$ since
    \[ g \circ f = 0 \quad \textup{ for }g(e_i) = \left\{ \begin{array}{cl} e_0 & \textup{ if } i = 0 , \\ 0 & \textup{ otherwise.} \end{array} \right. \]
    However, $f$ is not a left zero-divisor of $R$ as it is easy to verify that $f \circ h = 0$ implies $h = 0$.
\end{example}

\begin{topic}{nilpotent-element}{nilpotent element}
    Let $R$ be a \tref{ring}{ring}. An element $a \in R$ is \textbf{nilpotent} if $a^n = 0$ for some positive integer $n$.
\end{topic}

\begin{example}{nilpotent-element}
    If $a \in R$ is nilpotent, then $1 - a$ is \tref{unit}{invertible} in $R$. Namely, $a^n = 0$ for some $n \ge 0$, so
    \[ (1 - a) (1 + a + a^2 + \cdots + a^{n - 1}) = 1 - a^n = 1 . \]
\end{example}

\begin{topic}{unipotent-element}{unipotent element}
    Let $R$ be a \tref{ring}{ring}. An element $a \in R$ is \textbf{unipotent} if $(a - 1)^n = 0$ for some positive integer $n$.
\end{topic}

\begin{example}{unipotent-element}
    Let $R = \textup{Mat}_{n \times n}(k)$ be the ring of $n \times n$ matrices over some field $k$. Then any matrix of the form
    \[ \begin{pmatrix} 1 & * & \cdots & * \\ 0 & 1 & \cdots & * \\ \vdots & \vdots & \ddots & \vdots \\ 0 & 0 & \cdots & 1 \end{pmatrix} \]
    is unipotent.
\end{example}

\begin{topic}{reduced-ring}{reduced ring}
    A \tref{ring}{ring} $R$ is \textbf{reduced} if it has no non-zero \tref{nilpotent-element}{nilpotent} elements.
\end{topic}

\begin{example}{reduced-ring}
    The ring $\ZZ[x] / (x^2)$ is not reduced, since $x \text{ mod } (x^2)$ squares to $0$.
\end{example}

\begin{topic}{ideal}{ideal}
    Let $R$ be a \tref{ring}{ring}. A subgroup under addition $I \subset R$ is called a
    \begin{itemize}
        \item \textbf{left ideal} if $ra \in I$  for all $r \in R$ and $a \in I$,
        \item \textbf{right ideal} if $ar \in I$ for all $r \in R$ and $a \in I$,
        \item \textbf{two-sided ideal} if $I$ is both a left and right ideal.
    \end{itemize}
    When $R$ is commutative, these notions coincide, and $I$ is simply called an \textbf{ideal}.
\end{topic}

\begin{topic}{principal-ideal}{principal ideal}
    Let $R$ be a \tref{ring}{ring} and $I \subset R$ an \tref{ideal}{ideal} of $R$. Then $I$ is called a
    \begin{itemize}
        \item \textbf{left principal ideal} if $I = Ra = \{ ra : r \in R \}$ for some $a \in R$,
        \item \textbf{right principal ideal} if $I = aR = \{ ar : r \in R \}$ for some $a \in R$,
        \item \textbf{two-sided principal ideal} if $I$ is both a left and right principal ideal.
    \end{itemize}
    When $R$ is commutative, these notions coincide, and $I$ is simply called a \textbf{principal ideal}, and denoted as $I = (a)$.
\end{topic}

\begin{topic}{domain}{domain}
    A non-zero \tref{ring}{commutative ring} $R$ is called a \textbf{domain} if it has no zero-divisors, that is, $ab = 0$ implies $a = 0$ or $b = 0$ for all $a, b \in R$.
\end{topic}

\begin{topic}{irreducible-element}{irreducible element}
    Let $R$ be a \tref{domain}{domain}. An element $a \in R$ is called a \textbf{irreducible} if $a$ is not a \tref{unit}{unit}, and for all $b, c \in R$ such that $bc = a$ either $b$ or $c$ is a unit.
\end{topic}

\begin{topic}{field}{field}
    A non-zero \tref{ring}{commutative ring} $R$ is called a \textbf{field} if all non-zero elements are units, that is, for all non-zero $a \in R$ there exists a $b \in R$ such that $ab = 1$. 
\end{topic}

\begin{topic}{coprime-ideals}{coprime ideals}
    Two \tref{ideal}{ideals} $I, J$ of a \tref{ring}{commutative ring} $R$ are \textbf{coprime} if $I + J = R$.
\end{topic}

\begin{topic}{radical-ideal}{radical ideal}
    Let $R$ be a \tref{ring}{commutative ring}. The \textbf{radical} of an \tref{ideal}{ideal} $I \subset R$ is given by
    \[ \sqrt{I} = \{ x \in R : x^n \in I \} . \]
    An ideal $I$ is called \textbf{radical} if $I = \sqrt{I}$.
\end{topic}

\begin{topic}{prime-ideal}{prime ideal}
    Let $R$ be a \tref{ring}{commutative ring}. An \tref{ideal}{ideal} $I \subset R$ is called \textbf{prime} if $I \ne R$ and
    \[ ab \in I \implies a \in I \text{ or } b \in I . \]
    Equivalently, $I \subset R$ is a prime ideal if and only if the quotient ring $R / I$ is a \tref{domain}{domain}.
\end{topic}

\begin{topic}{maximal-ideal}{maximal ideal}
    Let $R$ be a \tref{ring}{commutative ring}. An \tref{ideal}{ideal} $I \subset R$ is called \textbf{maximal} if $I \ne R$ and if $I \subset J \subsetneq R$ implies $I = J$ for any other ideal $J \subset R$.
    
    Equivalently, $I \subset R$ is a maximal ideal if and only if the quotient ring $R / I$ is a \tref{field}{field}.
    
    Every maximal ideal is \tref{prime-ideal}{prime}.
\end{topic}

\begin{example}{maximal-ideal}
    Every ideal $I \subsetneq R$ is contained in some maximal ideal $\mathfrak{m}$. Namely, consider the partially ordered set $P$ of ideals $J \subsetneq R$ containing $I$, ordered by inclusion. Then $P$ is non-empty as it contains $I$, and every chain $J_1 \subset J_2 \subset \ldots$ in $P$ has an upper bound $J = \cup_{i \ge 1} J_i$. Note that indeed $J$ is an ideal, and $J \ne R$ as $1 \not\in J$ because $1 \not\in J_i$ for all $i$.
    Now from \tref{GM:zorns-lemma}{Zorn's lemma} it follows $P$ has a maximal element, which corresponds to a maximal ideal $\mathfrak{m}$ containing $I$.
\end{example}

\begin{topic}{irreducible-ideal}{irreducible ideal}
    Let $R$ be a \tref{ring}{ring}. An \tref{ideal}{ideal} $I \subset R$ is called \textbf{irreducible} if
    \[ I = J_1 \cap J_2 \quad \implies \quad I = J_1 \text{ or } I = J_2 . \]
\end{topic}

\begin{topic}{quotient-ring}{quotient ring}
    Given a \tref{ring}{commutative ring} $R$ and \tref{ideal}{ideal} $I$, the \textbf{quotient ring} $R/I$ is the ring
    \[ R/I = \{ a + I : a \in R \} \]
    where addition and multiplication are given by
    \[ (a + I) + (b + I) = (a + b) + I \quad \text{ and } \quad (a + I) (b + I) = (ab) + I . \]
    It has the universal property that each morphism of rings $f : R \to S$ with $f(I) = 0$ uniquely extends to a morphism $R/I \to S$.
    \[ \begin{tikzcd} R \arrow{r}{f} \arrow[swap]{d}{\pi} & S \\ R/I \arrow[dashed]{ur} & \end{tikzcd} \]
\end{topic}

\begin{topic}{local-ring}{local ring}
    A \textbf{local ring} is a \tref{ring}{ring} $R$ with exactly one \tref{maximal-ideal}{maximal ideal}. The maximal is often denoted by $\mathfrak{m}$.
    
    The quotient $k = R/\mathfrak{m}$ is called the \textbf{residue field}.
\end{topic}

\begin{topic}{local-morphism}{local morphism}
    A \textbf{local morphism} of \tref{local-ring}{local rings} $f : R \to S$ is a ring morphism such that $f(\mathfrak{m}_R) \subset \mathfrak{m}_S$.
\end{topic}

\begin{topic}{finite-type}{finite type}
    A \tref{ring-morphism}{ring morphism} $R \to S$ is said to be of \textbf{finite type} if $S$ is isomorphic to a quotient of $R[x_1, \ldots, x_n]$ for some integer $n$.
    
    That is, $S$ is a \tref{finitely-generated-algebra}{finitely generated $R$-algebra}.
\end{topic}

\begin{topic}{finitely-presented-algebra}{finitely presented algebra}
    A \tref{ring-morphism}{ring morphism} $R \to S$ is said to be of \textbf{finite presentation} if $S$ is isomorphic to $R[x_1, \ldots, x_n] / (f_1, \ldots, f_m)$ for some integer $n$ and some $f_i \in R[x_1, \ldots, x_n]$.
    
    In this case, $S$ is also said to be a \textbf{finitely presented $R$-algebra}.
\end{topic}

\begin{example}{finitely-presented-algebra}
    Let $R = k[x_1, x_2, \ldots]$ and $S = R / (x_1^2, x_2^2, \ldots)$. Then the quotient map $R \to S$ is of \tref{finite-type}{finite type}, but not of finite presentation.
\end{example}

\begin{topic}{krull-dimension}{Krull dimension}
    The \textbf{Krull dimension} of a commutative ring $R$ is the supremum of the lengths of all chains of \tref{prime-ideal}{prime ideals}, where a chain of the form
    \[ \mathfrak{p}_0 \subsetneq \mathfrak{p}_1 \subsetneq \cdots \subsetneq \mathfrak{p}_n \]
    has length $n$.
\end{topic}

\begin{topic}{group-ring}{group ring}
    Let $R$ be a \tref{ring}{ring} and $G$ a group. The \textbf{group ring} $R[G]$ of $G$ over $R$ is defined as
    \[ R[G] = \bigoplus_{g \in G} R , \]
    where multiplication is induced by
    \[ (a \cdot g) \cdot (b \cdot h) = (ab) \cdot gh . \]
    % where multiplication is given by
    % \[ (a_g)_{g \in G} \cdot (b_g)_{g \in G} = \left(\sum_{h \in G} a_h b_{h^{-1}g}\right)_{g \in G} \]
\end{topic}

\begin{example}{group-ring}
    \begin{itemize}
        \item For $G = \ZZ$, we have $R[G] \simeq R[x, x^{-1}]$.
        \item For $G = \ZZ/n\ZZ$ with $n \ge 1$, we have $R[G] \simeq R[x] / (x^n - 1)$.
    \end{itemize}
\end{example}

\begin{topic}{augmentation-ideal}{augmentation ideal}
    The \textbf{augmentation ideal} of a \tref{group-ring}{group ring} $R[G]$ is the \tref{ideal}{(two-sided) ideal}
    \[ A = \ker\left( \varepsilon : R[G] \to G, \quad \sum_i r_i g_i \mapsto \sum_i r_i  \right) = (g - g' : g, g' \in G). \]
\end{topic}

\begin{topic}{chinese-remainder-theorem}{Chinese remainder theorem}
    Let $R$ be a \tref{ring}{commutative ring} and suppose $I, J$ are coprime ideals of $R$, i.e. $I + J = R$. Then the \textbf{Chinese remainder theorem} states that $I \cap J = I \cdot J$ and that there is an isomorphism of rings
    \[ R / (I \cdot J) \xrightarrow{\sim} (R / I) \times (R/J), \qquad a \text{ mod } (I \cdot J) \mapsto (a \text{ mod } I, b \text{ mod } J) . \]
    In particular, when $R = \ZZ$ with $I = (n)$ and $J = (m)$ for $n, m$ relatively prime, there is
    \[ \ZZ / nm \ZZ \simeq (\ZZ/n\ZZ) \times (\ZZ/m\ZZ), \qquad a \text{ mod } nm \mapsto (a \text{ mod } n, a \text{ mod } m) . \]
\end{topic}

\begin{topic}{idempotent-element}{idempotent element}
    An element $e \in R$ of a \tref{ring}{ring} $R$ is called \textbf{idempotent} if $e^2 = e$.
\end{topic}

\begin{example}{idempotent-element}
    The idempotent elements of a commutative ring $R$ are in bijection with the decompositions of $R$ as a product $R = R_1 \times R_2$. Indeed, every such decomposition yields the idempotent element $(1, 0)$. Conversely, for every idempotent $e \in R$, we have that $Re$ and $R(1 - e)$ are subrings of $R$, and we have an isomorphism
    \[ \begin{aligned}
        R &\simeq Re \times R(1 - e) \\
        r &\mapsto (re, r - re)
    \end{aligned} \]
    whose inverse is $(a, b) \mapsto a + b$. Clearly these two constructions are inverse to each other.
\end{example}

\begin{topic}{dual-numbers}{dual numbers}
    Let $R$ be a \tref{ring}{commutative ring}. The \textbf{ring of dual numbers} over $R$ is the quotient ring
    \[ R[\varepsilon] / (\varepsilon^2) . \]
\end{topic}

\begin{topic}{noetherian-ring}{noetherian ring}
    A \tref{ring}{commutative ring} $R$ is called \textbf{noetherian} if it satisfies the \textit{ascending chain condition}: for any increasing sequence of ideals
    \[ I_1 \subset I_2 \subset I_3 \subset \cdots \]
    there exists some $N \in \NN$ such that $I_n = I_N$ for any $n \ge N$.
    
    Equivalently, $R$ is noetherian if all its ideals are finitely generated.
\end{topic}

\begin{example}{noetherian-ring}
    The following are all noetherian rings.
    \begin{itemize}
        \item Any \tref{field}{field}: their only ideal is $(0)$.
        \item Any \tref{principal-ideal-domain}{PID}: every ideal is generated by a single element.
        \item If $R$ is noetherian, then so is $R[x]$: this is \textit{Hilbert's basis theorem}.
    \end{itemize}
\end{example}

\begin{example}{noetherian-ring}
    The ring $k[x_1, x_2, x_3, \ldots]$ is not noetherian. Namely, the sequence of ideals
    \[ (x_1) \subset (x_1, x_2) \subset (x_1, x_2, x_3) \subset \ldots \]
    is ascending, but does not stabilize.
\end{example}

\begin{topic}{localization}{localization}
    Let $R$ be a \tref{ring}{commutative ring}, and $S \subset R$ a \textit{multiplicative set} (that is, $1 \in S$ and $xy \in S$ for all $x, y \in S$). Then the \textbf{localization} of $R$ w.r.t. $S$ is the ring
    \[ S^{-1} R = R \times S / \sim{} \quad \text{ where } (r_1, s_1) \sim{} (r_2, s_2) \text{ if } t(r_1 s_2 - r_2 s_1) = 0 \text{ for some } t \in S . \]
\end{topic}

\begin{example}{localization}
    If $R$ is a \tref{domain}{domain}, then $S = R - \{ 0 \}$ is a multiplicative set, and $S^{-1} R$ is the \tref{field-of-fractions}{field of fractions} of $R$.
\end{example}

\begin{example}{localization}
    For any commutative ring $R$ and $f \in R$, the set
    \[ S = \{ 1, f, f^2, \ldots \} \]
    is a multiplicative set, and the localization is
    \[ R_f := S^{-1} R = \left\{ \frac{a}{f^n} : a \in R, \; n \ge 0 \right\} . \]
\end{example}

\begin{example}{localization}
    For any commutative ring $R$ and \tref{prime-ideal}{prime ideal} $\mathfrak{p}$, the set
    \[ S = \{ f \in R : f \not\in \mathfrak{p} \} \]
    is a multiplicative set, and the localization
    \[ R_\mathfrak{p} := S^{-1} R = \left\{ \frac{f}{g} : f, g \in R, \; g \not\in \mathfrak{p} \right\} \]
    is a \tref{local-ring}{local ring} with maximal ideal $\mathfrak{p} R_\mathfrak{p}$.
\end{example}

\begin{topic}{total-quotient-ring}{total quotient ring}
    The \textbf{total quotient ring} of a \tref{ring}{commutative ring} $R$ is the \tref{localization}{localization} $S^{-1} R$, with $S$ the set of elements of $R$ which are not \tref{zero-divisor}{zero-divisors}.
    
    When $R$ is a \tref{domain}{domain}, this construction gives the \tref{field-of-fractions}{field of fractions} of $R$.
\end{topic}

\begin{topic}{regular-ring}{regular (local) ring}
    A \textbf{regular local ring} is a commutative \tref{noetherian-ring}{noetherian} \tref{local-ring}{local} ring such that the minimal number of generators of its maximal ideal is equal to its \tref{krull-dimension}{Krull dimension}. Equivalently, a local ring $R$ with maximal ideal $\mathfrak{m}$ and residue field $k = R / \mathfrak{m}$ is regular if and only if $\dim_k(\mathfrak{m} / \mathfrak{m}^2) = \dim R$.
    
    A \textbf{regular ring} is a commutative noetherian ring, such that the \tref{localization}{localization} at every \tref{prime-ideal}{prime ideal} is a regular local ring.
\end{topic}

\begin{example}{regular-ring}
    Every field is a regular local ring: their Krull dimension is zero, and their maximal ideal is $(0)$.
\end{example}

\begin{example}{regular-ring}
    \begin{itemize}
        \item The local ring $R = k[x]/(x^2)$ is not a regular local ring. Its only prime ideal is its maximal ideal $\mathfrak{m} = (x)/(x^2)$, so the Krull dimension is zero, but the minimal number of generators of $\mathfrak{m}$ is one.
        \item The ring $R = k[x, y]/(y^2 - x^3)$ is not regular. Namely, at the maximal ideal $\mathfrak{m} = (x, y)$ we have $\dim R_\mathfrak{m} = 1$ since the only non-zero prime ideal in $R$ is $\mathfrak{m}$. However, $\dim_k(\mathfrak{m} / \mathfrak{m}^2) = \dim_k(k \cdot x \oplus k \cdot y) = 2$.
    \end{itemize}
\end{example}

\begin{topic}{principal-ideal-domain}{principal ideal domain (PID)}
    A \textbf{principal ideal domain} (PID) is a \tref{domain}{domain} $R$ in which every \tref{ideal}{ideal} is \tref{principal-ideal}{principal}. That is, every ideal $I \subset R$ is of the form $I = (x)$ for some element $x \in R$.
\end{topic}

\begin{topic}{unique-factorization-domain}{unique factorization domain (UFD)}
    A \textbf{unique factorization domain} (UFD) is a \tref{domain}{domain} $R$ for which every non-zero $x \in R$ can be written as the product of a \tref{unit}{unit} and a finite number of \tref{irreducible-element}{irreducible} elements:
    \[ a = u \cdot p_1 \cdot p_2 \cdot \cdots \cdot p_k \qquad u \in R^\times, \; k \ge 0, \; p_i \in R \text{ irreducible}. \]
\end{topic}

\begin{topic}{euclidean-ring}{Euclidean ring}
    A \textbf{Euclidean ring} is a \tref{domain}{domain} $R$ for which there exists a function
    \[ g : R^\times \to \ZZ_{\ge 0} \]
    such that for all $a, b \in R$ with $b \ne 0$, there exists $q, r \in R$ with $a = qb + r$ and either $r = 0$ or $g(r) < g(b)$.
    
    That is, a ring in which one can perform division with remainder. The function $g$ is used to say that the `remainder' $r$ is `smaller' than the element $b$ one divides by.
    
    In particular, one can find the \textit{gcd} of elements by means of the \textit{Euclidean algorithm}.
\end{topic}

\begin{topic}{field-of-fractions}{field of fractions}
    The \textbf{field of fractions} of a \tref{domain}{domain} $R$ is the \tref{field}{field} given by
    \[ K = \left\{ (a, b) : a, b \in R, b \ne 0 \right\} / \sim{} \]
    where $(a, b) \sim{} (c, d)$ if and only if $ad = bc$. Indeed $(a, b)$ represents the fraction $\frac{a}{b}$. Addition and multiplication are given by
    \[ \frac{a}{b} + \frac{c}{d} = \frac{ad + bc}{bd} \quad \text{and} \quad \frac{a}{b} \cdot \frac{c}{d} = \frac{ac}{bd} . \]
\end{topic}

\begin{example}{field-of-fractions}
    The field of fractions of $\ZZ$ is $\QQ$.
\end{example}

\begin{topic}{valuation-ring}{valuation ring}
    A \textbf{valuation ring} is a \tref{domain}{domain} $R$ such that for every element $x$ of its \tref{field-of-fractions}{field of fractions}, at least one of $x$ or $x^{-1}$ belongs to $R$.
\end{topic}

\begin{topic}{discrete-valuation-ring}{discrete valuation ring}
    A domain \tref{domain}{domain} $R$ is a \textbf{discrete valuation ring} if there is a \textit{discrete valuation} $v$ of its \tref{field-of-fractions}{field of fractions} $K$, such that $R$ is the \textit{valuation ring} of $v$. This means there is a homomorphism
    \[ v : K^\times \to \ZZ \]
    such that
    \[ R = \{ x \in K : v(x) \ge 0 \} . \]
    
    In particular, $R$ is \tref{local-ring}{local} with maximal ideal $\mathfrak{m} = \{ x \in K : v(x) > 0 \}$. Indeed, any element $x \in R$ with $v(x) = 0$ is a unit in $R$.
\end{topic}

\begin{example}{discrete-valuation-ring}
    Let $K = \QQ$ and fix a prime number $p$. Any non-zero $x \in \QQ$ can be written uniquely as $x = p^m y$ where $k \in \ZZ$ and both the numerator and denominator of $y$ are coprime to $p$. Now define the discrete valuation $v_p(x) = m$, then the valuation ring is the local ring $\ZZ_{(p)}$.
\end{example}

% \begin{topic}{dedekind-domain}{Dedekind domain}
%     A \textbf{Dedekind domain} is a \tref{domain}{domain} in which every non-zero proper ideal factors into a product of prime ideals.
% \end{topic}

\begin{topic}{monic-polynomial}{monic polynomial}
    A polynomial $f$ is called \textbf{monic} if its leading coefficient is $1$.
\end{topic}

\begin{topic}{integral-element}{integral element}
    Let $A$ be a \tref{ring}{commutative ring}, and $B$ an $A$-algebra. An element $x \in B$ is said to be \textbf{integral} over $A$ if $x$ is a root of a monic polynomial with coefficients in $A$. That is,
    \[ x^n + a_1 x^{n - 1} + \cdots + a_n = 0 \]
    for some $a_i \in A$.
\end{topic}

\begin{topic}{integral-closure}{integral closure}
    Let $A$ be a \tref{ring}{commutative ring}, and $B$ an $A$-algebra. The \textbf{integral closure} of $A$ in $B$ is the subring of $B$ of elements which are \tref{integral-element}{integral} over $A$.
    
    If the integral closure is equal to $A$, then $A$ is said to be \textbf{integrally closed} in $B$. If the integral closure is equal to $B$, the ring $B$ is said to be \textbf{integral} over $A$.
\end{topic}

\begin{topic}{artin-ring}{artin ring}
    A \tref{ring}{commutative ring} $R$ is called \textbf{artin} if it satisfies the \textit{descending chain condition}: for any decreasing sequence of ideals
    \[ I_1 \supset I_2 \supset I_3 \supset \cdots \]
    there exists some $N \in \NN$ such that $I_n = I_N$ for any $n \ge N$.
\end{topic}

\begin{example}{artin-ring}
    The ring $R = k[x] / (x^n)$, for a \tref{field}{field} $k$ and integer $n \ge 0$, is artin. Namely, the only ideals of $R$ are of the form $(x^k)$ with $0 \le k \le n$. 
\end{example}

\begin{topic}{fractional-ideal}{fractional ideal}
    Let $R$ be a \tref{domain}{domain} and $K$ its \tref{field-of-fractions}{field of fractions}. An $R$-submodule $M$ of $K$ is a \textbf{fractional ideal} of $R$ if $xM \subset R$ for some $x \ne 0$ in $R$.
\end{topic}

\begin{topic}{invertible-ideal}{invertible ideal}
    Let $R$ be a \tref{domain}{domain} and $K$ its \tref{field-of-fractions}{field of fractions}. An $R$-submodule $M$ of $K$ is an \textbf{invertible ideal} if there exists a submodule $N$ of $K$ such that $MN = R$. This module $N$ is then unique and equal to
    \[ N = (R : M) := \{ x \in K : xM \subset R \} . \]
    The invertible ideals form a group with respect to multiplication, whose unit element is $R = (1)$.
\end{topic}

\begin{topic}{quotient-ideal}{quotient ideal}
    Let $R$ be a \tref{ring}{commutative ring}, and $I, J \subset R$ two \tref{ideal}{ideals}. Their \textbf{quotient ideal} is the ideal
    \[ (I : J) = \{ x \in R \;|\; xJ \subset I \} . \]
\end{topic}

\begin{example}{quotient-ideal}
    \begin{itemize}
        \item In $\ZZ$, we have $((6) : (2)) = (3)$.
        \item In $k[x, y]$, we have $((xy), (x)) = (y)$.
    \end{itemize}
\end{example}

\begin{topic}{syzygy-module}{syzygy module}
    Let $g_1, g_2, \ldots, g_k$ be generators of \tref{module}{module} $M$ over a \tref{ring}{commutative ring} $R$. The \textbf{syzygy module} is the $R$-module consisting of all relations between the generators, i.e. the kernel of
    \[ \bigoplus_{i = 1}^{k} R \to M, \qquad (r_1, \ldots, r_k) \mapsto r_1 g_1 + \cdots r_k g_k . \]
    Inductively, one can define the $n$-th syzygy module for any $n \ge 1$ after choosing generators.
\end{topic}

% Hilbert's three theorems
\begin{topic}{hilbert-basis-theorem}{Hilbert's basis theorem}
    \textbf{Hilbert's basis theorem} states that if a \tref{ring}{commutative ring} $R$ is \tref{noetherian-ring}{noetherian}, then so is the polynomial ring $R[x]$.
\end{topic}

% \begin{topic}{hilbert-nullstellensatz}{Hilbert's Nullstellensatz}
%     \textbf{Hilbert's Nullstellensatz} states that there is a bijection
%     \[ \left\{ \begin{array}{c} \text{radical ideals of} \\ k[x_1, \ldots, x_n] \end{array} \right\} \leftrightarrow \left\{ \begin{array}{c} \text{closed subsets} \\ \text{of $\AA^n$} \end{array} \right\} \]
%     \[ I \mapsto Z(I) \]
%     \[ I \mapsfrom Z(I) \]
% \end{topic}

\begin{topic}{hilbert-syzygy-theorem}{Hilbert's syzygy theorem}
    \textbf{Hilbert's syzygy theorem} states that if $M$ is a finitely generated module over a polynomial ring $k[x_1, \ldots, x_n]$, then the $n$-th \tref{syzygy-module}{syzygy module} $M$ is always \tref{free-module}{free}.
    
    In particular, this implies that there exists a free resolution
    \[ 0 \to F_k \to F_{k - 1} \to \cdots \to F_0 \to M \to 0 \]
    of length $k \le n$.
\end{topic}

\begin{topic}{primary-ideal}{primary ideal}
    Let $R$ be a \tref{ring}{commutative ring}. An \tref{ideal}{ideal} $\mathfrak{q}$ of $R$ is \textbf{primary} if $\mathfrak{q} \ne R$ and if
    \[ xy \in \mathfrak{q} \implies \text{ either } x \in \mathfrak{q} \text{ or } y^n \in \mathfrak{q} \text{ for some } n > 0 . \]
    In other words,
    \[ \mathfrak{q} \text{ is primary } \iff A / \mathfrak{q} \ne 0 \text{ and every zero-divisor in } A / \mathfrak{q} \text{ is nilpotent} . \]
\end{topic}

\begin{topic}{primary-decomposition}{primary decomposition}
    Let $R$ be a \tref{ring}{commutative ring}. A \textbf{primary decomposition} of an ideal $I \subset R$ is an expression of $I$ as a finite intersection of \tref{primary-ideal}{primary ideals},
    \[ I = \bigcap_{i = 1}^{n} \mathfrak{q}_i . \]
    The primary decomposition is said to be \textbf{minimal} if (i) the $\mathfrak{p}_i = r(\mathfrak{q}_i)$ are all distinct, and (ii) no $\mathfrak{q}_i$ contains the intersection of the other primary ideals.
\end{topic}

\begin{topic}{annihilator}{annihilator}
    Let $R$ be a \tref{ring}{commutative ring} and $M$ an $R$-module. The \textbf{annihilator} of $M$ is the subring
    \[ \text{Ann}(M) = \{ x \in R : xM = 0 \} \]
\end{topic}

\begin{topic}{nilradical}{nilradical}
     The \textbf{nilradical} of a \tref{ring}{commutative ring} $R$ is the ideal of \tref{nilpotent-element}{nilpotent} elements
     \[ \mathfrak{N}_R = \{ x \in R : x \text{ is nilpotent } \} . \]
     It is equal to the intersection of all prime ideals of $R$,
     \[ \mathfrak{N}_R = \bigcap_{\mathfrak{p} \subset R \text{ prime}} \mathfrak{p} . \]
\end{topic}

\begin{topic}{jacobson-radical}{Jacobson radical}
     The \textbf{Jacobson radical} of a \tref{ring}{commutative ring} $R$ is the ideal given by the intersection of all \tref{maximal-ideal}{maximal} ideals,
     \[ \mathfrak{J}_R = \bigcap_{\mathfrak{m} \subset R \text{ maximal}} \mathfrak{m} . \]
     It can also be characterized by
     \[ x \in \mathfrak{J}_R \iff 1 - xy \text{ is a unit for all } y \in R . \]
\end{topic}

\begin{topic}{regular-sequence}{regular sequence}
    Let $R$ be a \tref{ring}{commutative ring}, and $M$ an $R$-module. An $M$-\textbf{regular sequence} is a sequence
    \[ r_1, r_2, \ldots, r_d \in R \]
    such that $r_i$ is not a \textit{zero-divisor} on $M/(r_1, \ldots, r_{i - 1})$. That is, if $r_i m = 0$ for some $m \in M / (r_1, \ldots, r_{i - 1})$, then $m = 0$.
    
    When $M = R$, such a sequence is simply called a \textbf{regular sequence}.
\end{topic}

\begin{example}{regular-sequence}
    Consider $R = k[x, y]$ and the sequence $(xy, x^2)$. This sequence is not regular, since $x^2 \cdot y = 0$ in $\QQ[x, y] / (xy)$, although $y \ne 0$.
\end{example}

\begin{example}{regular-sequence}
    Consider $R = k[x, y, z]$ and the sequence $(x, y(1 - x), z(1 - x))$. This sequence is regular since
    \[ y(1 - x) = y \in k[x, y, z]/(x) = k[y, z] \quad \text{ and } \quad z(1 - x) = z \in k[x, y, z]/(x, y - xy) = k[z] \]
    are both non-zero-divisors.
    
    However, note that the order matters as $(y(1 - x), z(1 - x), x)$ is not a regular sequence. Namely, $z(1 - x) \cdot y = 0 \in k[x, y, z] / (y(1 - x))$ even though $y \ne 0$.
\end{example}

\begin{topic}{hilbert-series}{Hilbert series}
    Let $S = \bigoplus_{i \ge 0} S_i$ be a \tref{finitely-generated-algebra}{finitely generated} \tref{graded-ring}{graded} commutative algebra over a field $k$, with $S_0 = k$. The \textbf{Hilbert series} of $S$ is defined as
    \[ \text{HS}_S(t) = \sum_{i = 0}^{\infty} \dim_k S_i \cdot t^i . \]
\end{topic}

\begin{example}{hilbert-series}
    Let $R = k[x_1, \ldots, x_n]$ and $I = (f)$ for some homogenous polynomial of degree $d$. Then we have an exact sequence (a free resolution)
    \[ 0 \to R(-d) \xrightarrow{\cdot f} R \to R / I \to 0 , \]
    which implies that
    \[ \text{HS}_{R/I}(t) = \text{H}_R(t) - \text{HS}_{R(-d)}(t) = \text{HS}_R(t) (1 - t^d) = \frac{1 - t^d}{(1 - t)^n}. \]
\end{example}

\begin{topic}{hilbert-polynomial}{Hilbert polynomial}
    Let $S = \bigoplus_{i \ge 0} S_i$ be a \tref{finitely-generated-algebra}{finitely generated} graded commutative \tref{algebra}{algebra} over a \tref{field}{field} $k$, with $S_0 = k$. The \textit{Hilbert function}
    \[ i \mapsto \dim_k S_i \]
    becomes polynomial for large enough values of $i$, and the resulting polynomial is called the \textbf{Hilbert polynomial} $\text{HP}_S(i)$ of $S$.
\end{topic}

\begin{example}{hilbert-polynomial}
    Let $R = k[x_1, \ldots, x_n]$, then
    \[ \dim R_i = \binom{n - 1 + i}{i} = \frac{(n - 1 + i) \cdots (i + 1)}{(n - 1)!} , \]
    which is a polynomial in $i$ for every value of $n$. In particular,
    \[ \begin{aligned}
        \text{HP}_{k[x]}(i) &= 1, \\
        \text{HP}_{k[x, y]}(i) &= i + 1, \\
        \text{HP}_{k[x, y, z]}(i) &= \tfrac{1}{2} (i + 1)(i + 2), \\
        \text{HP}_{k[x, y, z, w]}(i) &= \tfrac{1}{6} (i + 1)(i + 2)(i + 3) .
    \end{aligned} \]
\end{example}

\begin{topic}{absolutely-flat-ring}{absolutely flat ring}
    A \tref{ring}{commutative ring} $R$ is called \textbf{absolutely flat} if every $R$-module is \tref{flat-module}{flat}.
\end{topic}

\begin{topic}{standard-smooth-algebra}{standard smooth algebra}
    Let $A$ be \tref{ring}{commutative ring}, and $B = A[x_1, \ldots, x_n] / (f_1, \ldots, f_c)$ a \tref{finitely-presented-algebra}{finitely presented $A$-algebra}. Then $B$ is a \textbf{standard smooth} over $A$ if the determinant
    \[ \det \begin{pmatrix}
        \frac{\partial f_1}{\partial x_1} & \frac{\partial f_1}{\partial x_2} & \cdots & \frac{\partial f_1}{\partial x_c} \\
        \frac{\partial f_2}{\partial x_1} & \frac{\partial f_2}{\partial x_2} & \cdots & \frac{\partial f_2}{\partial x_c} \\ 
        \vdots & \vdots & \ddots & \vdots \\ 
        \frac{\partial f_c}{\partial x_1} & \frac{\partial f_c}{\partial x_2} & \cdots & \frac{\partial f_c}{\partial x_c}
    \end{pmatrix} \]
    maps to an invertible element in $B$. Note that this definition is dependent on the presentation of $B$.
\end{topic}

\begin{example}{standard-smooth-algebra}
    \begin{itemize}
        \item Take $A = k$ a field and $B = k[x, y] / (xy)$. Then $\partial (xy) / \partial x = y$ which is not invertible in $B$, so $k[x, y] / (xy)$ is not standard smooth over $k$.
        \item Taking $B = k[x, y] / (xy - 1)$, we see that $\partial (xy - 1) / \partial x = y$, which is invertible in $B$. Hence $k[x, y] / (xy - 1)$ is standard smooth over $k$.
    \end{itemize}
\end{example}

\begin{topic}{standard-etale-algebra}{standard étale algebra}
    Let $R$ be \tref{ring}{commutative ring}. A \textbf{standard étale algebra} over $R$ is an \tref{algebra}{$R$-algebra} of the form
    \[ R[x]_g / (f) , \]
    where $f, g \in R[x]$ with $f$ monic and the derivative $f'$ is invertible in the \tref{localization}{localization} $R[x]_g$.
\end{topic}

\begin{example}{standard-etale-algebra}
    Let $k$ be a field, and $n \ge 1$ an integer with $n \nmid \operatorname{char}(k)$. Then $k[t^{\pm 1}, x] / (x^n - t)$ is a standard étale algebra over $k[t^{\pm 1}]$, with $f = x^n - t$ and $g = 1$ as in the definition. Indeed, $f' = n x^{n - 1}$ has inverse $\tfrac{1}{n} x t^{-1}$ in $k[t^{\pm 1}, x]$.
    
    However, the algebra $k[t, x] / (x^n - t)$ over $k[t]$ is not standard étale, with $f = x^n - t$ and $g = 1$ as in the definition. Indeed, the derivative $f' = n x^{n - 1}$ is not invertible in $k[t, x] / (x^n - t)$.
\end{example}

\begin{topic}{etale-algebra}{étale algebra}
    Let $A$ be a \tref{ring}{commutative ring} and $B$ an $A$-algebra. Then $B$ is an \textbf{étale algebra} over $A$ if $B$ is a \tref{flat-module}{flat} $A$-module and $\Omega_{B/A} = 0$.
\end{topic}

\begin{topic}{fitting-ideal}{Fitting ideal}
    Let $R$ be a \tref{ring}{commutative ring} and $M$ an $R$-\tref{module}{module} generated by elements $m_1, \ldots, m_n \in M$, with relations
    \[ a_{i1} m_1 + \cdots a_{in} m_n = 0 \text{ for } i = 1, 2, \ldots, \ell . \]
    Then the \textbf{$i$-th Fitting ideal} $\text{Fitt}_i(M)$ of $M$ is generated by the minors (determinants of submatrices) of order $n - i$ of the matrix $a_{ij}$. It can be shown that this does not depend on the choice of generators or relations. One has the inclusions
    \[ \text{Fitt}_0(M) \subset \text{Fitt}_1(M) \subset \text{Fitt}_2(M) \subset \cdots \]
    Intuitively, the $i$-th fitting ideal (or actually the quotient $R / \text{Fitt}_i(M)$) measures the obstruction for $M$ to be generated by $i$ elements.
    
    Sometimes the \textbf{Fitting ideal} of $M$ is defined as the first non-zero fitting ideal.
\end{topic}

\begin{example}{fitting-ideal}
    If $M$ is \tref{free-module}{free} of rank $n$, the matrix $a_{ij}$ will be of size $0 \times n$. Hence $\text{Fitt}_i(M) = 0$ for $i < n$ as there are no submatrices of size $n - i > 0$, and $\text{Fitt}_i(M) = R$ for $i \ge n$ as the determinant of a $0 \times 0$ matrix is one.
\end{example}

\begin{example}{fitting-ideal}
    Consider $M = \ZZ / p \ZZ \times \ZZ / p^2 \ZZ$ as $\ZZ$-module. It can be generated by $m_1 = (1, 0)$ and $m_2 = (0, 1)$ with relations $p m_1 = 0$ and $p^2 m_2 = 0$, so the corresponding matrix is
    \[ a_{ij} = \begin{pmatrix} p & 0 \\ 0 & p^2 \end{pmatrix} \]
    and thus
    \[ \text{Fitt}_0(M) = (p^3), \quad \text{Fitt}_1(M) = (p, p^2) = (p) \quad \text{ and } \quad \text{Fitt}_{\ge 2}(M) = \ZZ . \]
\end{example}

\begin{example}{fitting-ideal}
    Consider the finitely generated abelian group $M = \ZZ^r \times \ZZ / d_1 \ZZ \times \cdots \times \ZZ / d_k \ZZ$ with $d_1 | d_2 | \cdots | d_k$. Using the natural generators, the relation matrix is
    \[ a_{ij} = \begin{pmatrix} \textbf{0}_{r \times r} & & &  \\  & d_1 & \\ & & \ddots & \\ & & & d_k \end{pmatrix} \]
    and thus
    \[ \begin{array}{rcl}
         \text{Fitt}_{0 \le i \le r}(M) &=& (0) , \\
        \text{Fitt}_{r + 1}(M) &=& (d_1 d_2\cdots d_k) , \\
        \text{Fitt}_{r + 2}(M) &=& (d_1 d_2\cdots d_{k - 1}) , \\
        & \vdots & \\
        \text{Fitt}_{r + k - 1}(M) &=& (d_1) , \\
        \text{Fitt}_{\ge r + k}(M) &=& \ZZ .
    \end{array} \]
\end{example}

\begin{example}{fitting-ideal}
    Consider the scheme $X = \Spec k[x_1, \ldots, x_n] / I$ with $I = (f_1, \ldots, f_k)$ for some polynomials $f_i$. The module of differentials
    \[ \Omega_{X/k} = \left(\bigoplus_{i = 1}^{n} k[x_1, \ldots, x_n] / I \cdot dx_i \right) / (df_1, df_2, \ldots, df_k) \]
    is generated by $dx_1, \ldots, dx_n$ and relations
    \[ df_i = \sum_{j = 1}^{n} \frac{\partial f_i}{\partial x_j} dx_j = 0 \quad \text{ for } i = 1, 2, \ldots, k , \]
    so the corresponding matrix is the Jacobian matrix
    \[ J = \begin{pmatrix}
        \frac{\partial f_1}{\partial x_1} & \frac{\partial f_1}{\partial x_2} & \cdots & \frac{\partial f_1}{\partial x_n} \\
        \frac{\partial f_2}{\partial x_1} & \frac{\partial f_2}{\partial x_2} & \cdots & \frac{\partial f_2}{\partial x_n} \\
        \vdots & \vdots & \ddots & \vdots \\
        \frac{\partial f_k}{\partial x_1} & \frac{\partial f_k}{\partial x_2} & \cdots & \frac{\partial f_k}{\partial x_n}
    \end{pmatrix} . \]
    Now $X$ is smooth over $k$ if and only if $\Omega_{X/k}$ is locally free of rank $n = \dim X$, which is equivalent to the Fitting ideal $\text{Fitt}_{\dim X}(\Omega_{X/k})$ generating the unit ideal in the localization of each prime ideal of $k[x_1, \ldots, x_n] / I$, which is equivalent to the Fitting ideal $\text{Fitt}_{\dim X}(\Omega_{X/k})$ generating the unit ideal in $k[x_1, \ldots, x_n] / I$ itself.
\end{example}

\begin{topic}{gcd}{greatest common divisor (GCD)}
    Let $R$ be a \tref{ring}{commutative ring}. A \textbf{greatest common divisor} of two elements $a, b \in R$ is an element $d \in R$ such that $d | a, b$ and for any other $d' \in R$ with $d' | a, b$ one has $d' | d$. It is denoted $d = \text{gcd}(a, b)$.
\end{topic}

\begin{topic}{lcm}{least common multiple (LCM)}
    Let $R$ be a \tref{ring}{commutative ring}. A \textbf{least common multiple} of two elements $a, b \in R$ is an element $m \in R$ such that $a, b | m$ and for any other $m' \in R$ with $a, b | m$ one has $m | m'$. It is denoted $m = \text{lcm}(a, b)$.
\end{topic}

\begin{topic}{gcd-domain}{GCD domain}
    A \textbf{GCD domain} is a \tref{domain}{domain} $R$ in which any two elements $a, b$ have a \tref{gcd}{greatest common divisor}.
\end{topic}

\begin{topic}{projective-dimension}{projective dimension}
    The \textbf{projective dimension} of a \tref{module}{module} $M$ over a \tref{ring}{commutative ring} $R$ is the minimal length of a projective resolution of $M$.
    \[ \cdots \to P_n \to \cdots \to P_2 \to P_1 \to P_0 \to M \to 0 \]
    It may be infinite.
\end{topic}

\begin{topic}{injective-dimension}{injective dimension}
    The \textbf{injective dimension} of a \tref{module}{module} $M$ over a \tref{ring}{commutative ring} $R$ is the minimal length of a injective resolution of $M$.
    \[ 0 \to M \to I^0 \to I^1 \to I^2 \to \cdots \]
    It may be infinite.
\end{topic}

% \begin{example}{projective-dimension}
%     Consider $R = k[x, y] / (xy)$ and $M = k$ as an $R$-module, where $x$ and $y$ act by multiplication by zero.
% \end{example}

\begin{topic}{graded-ring}{graded ring}
    A \textbf{graded ring} is a \tref{ring}{ring} $R$ that is decomposed as a direct sum
    \[ R = \bigoplus_{i \ge 0} R_i \]
    of additive groups, such that $R_i R_j \subset R_{i + j}$ for all $i, j \ge 0$.
\end{topic}

\begin{example}{graded-ring}
    The polynomial ring $R = k[x_1, \ldots, x_n]$ is a graded ring, with
    \[ R_i = \{ f \in R : f \text{ is homogeneous of degree $i$} \} . \]
\end{example}

\begin{topic}{morita-equivalence}{Morita equivalence}
    Two \tref{ring}{rings} $R$ and $S$ are said to be \textbf{Morita equivalent} if there is an \tref{CT:equivalence-of-categories}{equivalence of categories} between the category of (left) $R$-modules, and (left) $S$-modules.
\end{topic}

\begin{example}{morita-equivalence}
    Any ring $R$ is Morita equivalent to the ring $\text{M}_n(R)$ of $n \times n$ matrices with elements in $R$, for any $n > 0$. Indeed, take
    \[ R\text{-Mod} \to \text{M}_n(R)\text{-Mod}, \quad M \to R^{n \times 1} \otimes_R M \simeq M^n , \]
    where $\text{M}_n(R)$ acts on $M^n$ by matrix multiplication on the left. Inversely, take
    \[ \text{M}_n(R)\text{-Mod} \to R\text{-Mod}, \quad N \to R^{1 \times n} \otimes_{\text{M}_n(R)} N . \]
    Indeed these are inverse to each other since
    \[ R^{1 \times n} \otimes_{\text{M}_n(R)} R^{n \times 1} \simeq R \quad \text{and} \quad R^{n \times 1} \otimes_R R^{1 \times n} \simeq \text{M}_n(R) . \]
\end{example}

\begin{topic}{gorenstein-ring}{Gorenstein ring}
    A \textbf{Gorenstein local ring} is a commutative \tref{noetherian-ring}{noetherian} \tref{local-ring}{local} \tref{ring}{ring} $R$ with finite \tref{injective-dimension}{injective dimension} as an $R$-module.
    
    A \textbf{Gorenstein ring} is a commutative noetherian ring $R$ such that the localization $R_\mathfrak{p}$ is Gorenstein for each \tref{prime-ideal}{prime ideal} $\mathfrak{p}$.
\end{topic}

\begin{topic}{cohen-macaulay}{Cohen--Macaulay}
    A \tref{finitely-generated-module}{finitely generated} \tref{module}{module} $M$ over a commutative \tref{noetherian-ring}{noetherian} \tref{local-ring}{local} \tref{ring}{ring} $R$ is \textbf{Cohen--Macaulay} if its \tref{depth-module}{depth} $\text{depth}_{\mathfrak{m}}(M)$ equals its \tref{krull-dimension}{dimension} $\dim_R(M) := \dim(R/\text{Ann}_R(M))$.
    
    More generally, for a commutative noetherian ring $R$, a finitely generated $R$-module $M$ is \textbf{Cohen--Macaulay} if the localization $M_\mathfrak{m}$ is Cohen--Macaulay over $R_\mathfrak{m}$ for each maximal ideal $\mathfrak{m}$ of $R$.
    
    A commutative noetherian ring $R$ is \textbf{Cohen--Macaulay} if it is so as a module over itself.
\end{topic}

\begin{topic}{homogeneous-ideal}{homogeneous ideal}
    A \textbf{homogeneous ideal} in a \tref{graded-ring}{graded ring} is an \tref{ideal}{ideal} generated by homogeneous elements.
\end{topic}

\begin{example}{homogeneous-ideal}
    In $S = k[x, y, z]$, the ideals $(x, y, z)$ and $(x^2, y + z)$ are homogeneous, while $(x + y^2)$ is not.
\end{example}

\begin{topic}{irrelevant-ideal}{irrelevant ideal}
    The \textbf{irrelevant ideal} of a \tref{graded-ring}{graded ring} is the \tref{ideal}{ideal} generated by the homogeneous elements of degree greater than zero. More generally, a \tref{homogeneous-ideal}{homogeneous ideal} of a graded ring is called \textbf{irrelevant} if its \tref{radical-ideal}{radical} contains the irrelevant ideal.
\end{topic}

\begin{example}{irrelevant-ideal}
    The irrelevant ideal of the graded ring $S = k[x_1, \ldots, x_n]$ is the ideal $S_+ = (x_1, \ldots, x_n)$.
\end{example}

\begin{topic}{algebra}{algebra}
    Let $A$ be a \tref{ring}{commutative ring}. An \textbf{$A$-algebra} is ring $B$ with a \tref{ring-morphism}{ring morphism} $f : A \to B$.
\end{topic}

\begin{example}{algebra}
    Let $k$ be a field, and $B = \text{Mat}_{n}(k)$ the ring of $n \times n$ matrices over $k$. Then $B$ is a $k$-algebra under the morphism $i : k \to B$ which sends $a \mapsto a \cdot I$.
\end{example}

\begin{topic}{derivation}{derivation}
    Let $A$ be a \tref{ring}{commutative ring}, $B$ an \tref{algebra}{$A$-algebra}, and $M$ a \tref{module}{$B$-module}. An \textbf{$A$-derivation} is a map $d : B \to M$ such that
    \begin{itemize}
        \item $d(b + b') = db + db'$ for all $b, b' \in B$,
        \item $d(bb') = bdb' + b'db$ for all $b, b' \in B$,
        \item $da = 0$ for all $a \in A$.
    \end{itemize}
    The set of $A$-derivations $d : B \to M$ is often denoted $\text{Der}_A(B, M)$.
    
    An $A$-derivation $d : B \to M$ is \textbf{universal} if for any other $A$-derivation $d' : B \to N$ there exists a unique $B$-module morphism $\phi : M \to N$ such that $d' = \phi \circ d$. In this case one often writes $M = \Omega_{B/A}$.
\end{topic}

\begin{example}{derivation}
    Let $\mu : B \otimes_A B \to B$ be the multiplication map $b_1 \otimes b_2 \mapsto b_1 b_2$, and let $I = \ker \mu$. Consider $B \otimes_A B$ as a $B$-module by multiplication on the left, that is $b \cdot (b_1 \otimes b_2) = (bb_1 \otimes b_2)$, then $I/I^2$ inherits the structure of a $B$-module, and the map
    \[ d : B \to I/I^2, \quad b \mapsto 1 \otimes b - b \otimes 1 (\text{mod } I^2) \]
    is a universal $A$-derivation. This can be seen as follows.
    
    First note that $I$ is generated as an $B$-module by elements of the form $1 \otimes b - b \otimes 1$ with $b \in B$. Namely, for any $x \in I$ we have $\mu(x) = 0$ so $x \equiv \mu(x) \otimes 1 \text{ mod } (1 \otimes b - b \otimes 1 : b \in B) = 0$, and the other inclusion is obvious. In other words, $d$ is surjective.
    
    Now for any other $A$-derivation $d' : B \to M$ we can (and must) define $\phi : I/I^2 \to M$ by putting $\phi(d(b)) = d' b$ and extending $B$-linearly. Indeed this is well-defined as any element of $I^2$ is a sum of elements of the form $xy$ with $x, y \in I$, and writing $x = \sum_i (1 \otimes b_i - b_i \otimes 1)$ and $y = \sum_j (1 \otimes b'_j - b'_j \otimes 1)$ we find that
    \[ xy = \sum_{i,j} (1 \otimes b_i b'_j - b_i b'_j \otimes 1) - b_i (1 \otimes b'_j - b'_j \otimes 1) - b'_j (1 \otimes b_i - b_i \otimes 1) , \]
    so $\phi(xy) = \sum_{i, j} d'(b_i b'_j) - b_i d' b'_j - b'_j d' b_i = 0$.
\end{example}

\begin{example}{derivation}
    If $B = A[x_1, \ldots, x_n]$ is a polynomial ring over $A$, then
    \[ \Omega_{B/A} = \bigoplus_{i = 1}^{n} B \cdot dx_i . \]
\end{example}

\begin{topic}{height-ideal}{height ideal}
    The \textbf{height} of a \tref{prime-ideal}{prime ideal} $\mathfrak{p}$ in a \tref{ring}{commutative-ring} $R$ is the supremum of lengths of chains of prime ideals, where a chain
    \[ \mathfrak{p}_0 \subsetneq \mathfrak{p}_1 \subsetneq \cdots \subsetneq \mathfrak{p}_n = \mathfrak{p} \]
    has length $n$.
\end{topic}

\begin{topic}{associated-graded-ring}{associated graded ring}
    The \textbf{associated graded ring} of a \tref{ring}{commutative ring} $R$ and an \tref{ideal}{ideal} $\mathfrak{a}$ is the \tref{graded-ring}{graded ring}
    \[ G_\mathfrak{a}(R) = \bigoplus_{n = 0}^{\infty} \mathfrak{a}^n / \mathfrak{a}^{n + 1} . \]
\end{topic}

\begin{topic}{completion}{completion}
    Let $R$ be a \tref{ring}{commutative ring} and $M$ an \tref{module}{$R$-module}. Each \tref{ideal}{ideal} $\mathfrak{a} \subset R$ determines a \tref{TO:topological-space}{topology} on $M$ called the \textbf{$\mathfrak{a}$-adic topology}: a subset $U \subset M$ is \textit{open} if and only if for each $x \in U$ there exists a positive integer $n$ such that $x + \mathfrak{a}^n M \subset U$.
    
    The \textbf{completion} of $M$ with respect to $\mathfrak{a}$ is the \tref{CT:inverse-limit}{inverse limit}
    \[ \widehat{M} = \varprojlim_{n \ge 1} M / \mathfrak{a}^n M = \left\{ (x_n)_{n \ge 1} \in \prod_{n \ge 1} M / \mathfrak{a}^n M : x_m = x_n \text{ mod } \mathfrak{a}^n M \text{ for } m \le n \right\} \]
    with the \tref{TO:subspace-topology}{subspace} \tref{TO:product-topology}{product topology}.
    
    When $R = M$, the completion $\widehat{R}$ has a ring structure and is called the \textbf{completion} of $R$, with repsect to $\mathfrak{a}$.
\end{topic}

\begin{example}{completion}
    When $R = k[x_1, \ldots, x_n]$ and $\mathfrak{a} = (x_1, \ldots, x_n)$, the completion is the power series ring
    \[ \widehat{R} = k\llbracket x_1, \ldots, x_n \rrbracket . \]
\end{example}

\begin{topic}{azumaya-algebra}{Azumaya algebra}
    An \textbf{Azumaya algebra} over a \tref{ring}{commutative ring} $R$ is an $R$-algebra $A$ that is \tref{finitely-generated-module}{finitely generated} \tref{faithful-module}{faithful} and \tref{projective-module}{projective} as an $R$-module, such that $A \otimes_R A^\text{op}$ is isomorphic to $\text{End}_R(A)$ via the map $a \otimes b \mapsto (x \mapsto axb)$.
\end{topic}

\begin{example}{azumaya-algebra}
    When $R = k$ is a field, Azumaya algebras over $k$ are precisely \tref{central-simple-algebra}{central simple algebras} over $k$.
\end{example}

\begin{topic}{simple-ring}{simple ring}
    A \tref{ring}{ring} $R$ is \textbf{simple} if it is non-trivial and it has no two-sided ideal besides the zero ideal and itself.
\end{topic}

\begin{example}{simple-ring}
    A commutative ring is simple if and only if it is a \tref{field}{field}.
\end{example}

\begin{topic}{ring-center}{ring center}
    The \textbf{center} of a \tref{ring}{ring} $R$ is the subring $Z(R)$ of all elements $x \in R$ such that $xy = yx$ for all $y \in R$.
\end{topic}

\begin{topic}{central-algebra}{central algebra}
    Let $S$ be a \tref{ring}{commutative ring}. A \textbf{central algebra} over $S$ is an $S$-algebra whose \tref{ring-center}{center} is exactly $S$.
\end{topic}

\begin{topic}{central-simple-algebra}{central simple algebra}
    A \textbf{central simple algebra (CSA)} over a field $k$ is a finite-dimensional $k$-algebra that is \tref{simple-ring}{simple} and whose \tref{ring-center}{center} is $k$.
\end{topic}

\begin{example}{central-simple-algebra}
    The quaternions $\mathbb{H}$ are a central simple algebra over the reals $\RR$.
\end{example}

\begin{topic}{involutive-ring}{involutive ring}
    An \textbf{involutive ring} is a \tref{ring}{ring} $R$ together with a map $^* : R \to R$ satsifying
    \begin{itemize}
        \item (\textit{involution}) $(x^*)^* = x$ for all $x \in R$,
        \item (\textit{antiautomorphism}) $(x + y)^* = x^* + y^*$ and $(xy)^* = y^* x^*$ for all $x, y \in R$.
    \end{itemize}
\end{topic}

\begin{example}{involutive-ring}
    The complex numbers $\CC$ are an involutive ring, with complex conjugation as involution.
\end{example}

\begin{topic}{involutive-algebra}{involutive algebra}
    An \textbf{involutive algebra} is an \tref{involutive-ring}{involutive ring} $A$ with involution $^*$, which is an algebra over a \tref{ring}{commutative ring} $R$ with involution $'$, satisfying
    \[ (ra)^* = r' a^* \]
    for all $r \in R$ and $a \in A$.
\end{topic}

\begin{example}{involutive-algebra}
    The ring of $n \times n$ matrices over the complex numbers $\CC$ are an involutive algebra, with the conjugate transpose as involution.
\end{example}

\begin{topic}{standard-complex}{standard complex (bar resolution)}
    Let $k$ be a \tref{field}{field} and $A$ an \tref{algebra}{$k$-algebra}. The \textbf{standard complex} is the \tref{HA:chain-complex}{chain complex} with $A^{\otimes n + 2}$ at degree $n$,
    \[ \cdots \xrightarrow{d_{n + 1}} A^{\otimes n + 2} \xrightarrow{d_n} \cdots \xrightarrow{d_1} A^{\otimes 2} \xrightarrow{d_0} A \to 0 \]
    with differential given by
    \[ d_n(a_0 \otimes \cdots \otimes a_{n + 1}) = \sum_{i = 0}^{n} (-1)^i a_0 \otimes \cdots \otimes a_i a_{i + 1} \otimes \cdots \otimes a_n . \]
\end{topic}

\begin{topic}{hereditary-ring}{hereditary ring}
    A \tref{ring}{ring} $R$ is \textbf{hereditary} if every submodule of a \tref{projective-module}{projective modules} over $R$ is also projective.
    
    If this is only true for \tref{finitely-generated-module}{finitely generated submodules}, the ring $R$ is called \textbf{semi-hereditary}.
\end{topic}

\begin{topic}{cotangent-complex}{cotangent complex}
    Let $A$ be a \tref{ring}{commutative ring} and let $B$ be an $A$-algebra. The \textbf{cotangent complex} $\mathbb{L}_{B/A}$ is the complex of $B$-modules associated to the simplicial $B$-module (via the \tref{HT:dold-kan-correspondence}{Dold--Kan correspondence})
    \[ \Omega_{P_\bdot/A} \otimes_{P_\bdot, \varepsilon} B \]
    where $\varepsilon : P_\bdot \to B$ is a free resolution of simplicial $A$-algebras.
\end{topic}

% \begin{example}{cotangent-complex}
%     ???
% \end{example}

\begin{topic}{division-ring}{division ring}
    A \textbf{division ring} is a non-zero \tref{ring}{ring} $R$ in which every non-zero element $x \in R$ is a \tref{unit}{unit}, i.e. there exists $y \in R$ such that $xy = yx = 1$
\end{topic}

\begin{example}{division-ring}
    Every \tref{field}{field} is a division ring, but the converse is not true. The \tref{quaternion-algebra}{quaternion algebra} $\mathbb{H}$ is a division ring, but not a field since it is not commutative.
\end{example}

\begin{topic}{prime-avoidance-lemma}{prime avoidance lemma}
    Let $R$ be a \tref{ring}{commutative ring}. The \textbf{prime avoidance lemma} states that if $\mathfrak{p}_1, \ldots, \mathfrak{p}_n$ are \tref{prime-ideal}{prime ideals} and $I$ an \tref{ideal}{ideal} with $I \subset \bigcup_{i = 1}^{n} \mathfrak{p}_i$, then $I \subset \mathfrak{p}_i$ for some $i$.
\end{topic}

\begin{example}{prime-avoidance-lemma}
    \begin{proof}
        Proof by induction on $n$. The base case $n = 1$ is trivial. Let $n > 1$ and assume the result holds for $n - 1$. Suppose for a contradiction that $I \not\subset \mathfrak{p}_i$ for all $i$, then by applying the induction hypothesis to each collection $\mathfrak{p}_1, \ldots, \mathfrak{p}_{i - 1}, \mathfrak{p}_{i + 1}, \ldots, \mathfrak{p}_n$, there exist elements $x_i \in I$ such that $x_i \not\in \mathfrak{p}_j$ whenever $j \ne i$. If for some $i$ we have $x_i \not\in \mathfrak{p}_i$ we are done. Otherwise, $x_i \in \mathfrak{p}_i$ for all $i$ and then
        \[ y = \sum_{i = 1}^{n} x_1 x_2 \cdots x_{i - 1} x_{i + 1} \cdots x_n \]
        is contained in $I$ but not in $\mathfrak{p}_i$ for any $i$. This shows $I \not\in \bigcup_{i = 1}^{n} \mathfrak{p}_i$.
    \end{proof}
\end{example}

\begin{topic}{integrally-closed-domain}{integrally closed domain}
    A \tref{domain}{domain} $R$ is \textbf{integrally closed} if it is equal to its \tref{integral-closure}{integral closure} in its \tref{field-of-fractions}{field of fractions}.
\end{topic}

\begin{example}{integrally-closed-domain}
    The ring $R = \ZZ[\sqrt{5}]$ is not integrally closed as it does not contain the integral element $\tfrac{1}{2} \sqrt{5} + \tfrac{1}{2} \in \QQ(\sqrt{5})$, which satisfies $x^2 - x - 1 = 0$.
\end{example}

\begin{topic}{finitely-generated-algebra}{finitely generated algebra}
    Let $R$ be a \tref{ring}{commutative ring}. A \textbf{finitely generated algebra} over $R$ is an \tref{algebra}{$R$-algebra} isomorphic to a quotient of $R[x_1, \ldots, x_n]$ for some $n$.
\end{topic}

\begin{example}{finitely-generated-algebra}
    The ring of rational numbers $\QQ$ is not finitely generated as a $\ZZ$-algebra. Namely, for any $q_1, \ldots, q_n \in \QQ$, the subring of $\QQ$ generated as a $\ZZ$-algebra by the $q_i$, can only contain denominators with prime factors that are also prime factors of the denominators of the $q_i$. But since there are infinitely many prime numbers, this subring cannot be the whole of $\QQ$.
\end{example}

\begin{topic}{catenary-ring}{(universally) catenary ring}
    A \tref{ring}{commutative ring} $R$ is \textbf{catenary} if for any \tref{prime-ideal}{prime ideals} $\mathfrak{p}$ and $\mathfrak{q}$, any two strictly increasing chains of prime ideals
    \[ \mathfrak{p} = \mathfrak{p}_0 \subsetneq \mathfrak{p}_1 \subsetneq \cdots \subsetneq \mathfrak{p}_n = \mathfrak{q} \]
    are contained in maximally strictly increasing chains from $\mathfrak{p}$ to $\mathfrak{q}$ of the same (finite) length.
    
    The ring $R$ is \textbf{universally catenary} if all \tref{finitely-generated-algebra}{finitely generated algebras} over $R$ are catenary.
\end{topic}

\begin{topic}{G-ring}{G-ring}
    A \tref{ring}{commutative ring} $R$ is a \textbf{G-ring} or \textbf{Grothendieck ring} if it is \tref{noetherian-ring}{noetherian} and for every localization $R_{\mathfrak{p}}$ the map $R_{\mathfrak{p}} \to \widehat{R_{\mathfrak{p}}}$ to its \tref{completion}{completion} (with respect to the maximal ideal) is a \tref{AG:regular-morphism}{regular morphism}.
\end{topic}

\begin{topic}{j0-j1-j2-ring}{J-0/J-1/J-2 ring}
    Let $R$ be a \tref{noetherian-ring}{noetherian} \tref{ring}{commutative ring}. The \textit{regular locus} $\textup{Reg}(X)$ of $X = \Spec R$ is the set of \tref{prime-ideal}{prime ideals} $\mathfrak{p}$ of $R$ such that the \tref{localization}{localization} $R_\mathfrak{p}$ is \tref{regular-ring}{regular}.
    
    The ring $R$ is \textbf{J-0} if $\textup{Reg}(X)$ contains a nonempty open.
    
    The ring $R$ is \textbf{J-1} if $\textup{Reg}(X)$ is open.
    
    The ring $R$ is \textbf{J-2} if $\textup{Reg}(X)$ if any \tref{finitely-generated-algebra}{finitely generated $R$-algebra} is J-1.
\end{topic}

\begin{topic}{excellent-ring}{(quasi-)excellent ring}
    A \tref{ring}{commutative ring} $R$ is \textbf{quasi-excellent} if it is \tref{noetherian-ring}{noetherian}, a \tref{G-ring}{G-ring}, and \tref{j0-j1-j2-ring}{J-2}.
    The ring $R$ is \textbf{excellent} if it is quasi-excellent and \tref{catenary-ring}{universally catenary}.
\end{topic}

\begin{topic}{henselian-ring}{Henselian ring}
    A \tref{local-ring}{local-ring} $A$ with \tref{maximal-ideal}{maximal-ideal} $\mathfrak{m}$ is \textbf{Henselian} if \tref{NT:hensel-lifting-lemma}{Hensel's lifting lemma} holds: for every \tref{monic-polynomial}{monic polynomial} $f \in A[x]$, any factorization $\overline{f} = \overline{g} \cdot \overline{h}$ in $(A/\mathfrak{m})[x]$ with $\overline{g}$ and $\overline{h}$ monic and coprime can be lifted to a factorization $f = gh$ in $A[x]$ such that $\deg(\overline{g}) = \deg(g)$ and $g, h$ have reduction $\overline{g}, \overline{h}$ in $A/\mathfrak{m}$.
    
    The ring $A$ is called \textbf{strictly Henselian} if in addition the residue field $k = A/\mathfrak{m}$ is \tref{separably-closed-field}{separably closed}.
\end{topic}

\begin{topic}{going-up}{going-up}
    Let $A$ be a \tref{ring}{commutative ring}, and $B$ an $A$-algebra. The ring extension $A \subset B$ satisfies the \textbf{going-up property} if whenever
    \[ \mathfrak{p}_1 \subset \mathfrak{p}_2 \subset \cdots \subset \mathfrak{p}_n \quad \textup{and} \quad \mathfrak{q}_1 \subset \mathfrak{q}_2 \subset \cdots \subset \mathfrak{q}_m \]
    are chains of \tref{prime-ideal}{prime ideals} of $A$ and $B$, respectively, with $m < n$ such that $\mathfrak{q}_i$ lies over $\mathfrak{p}_i$ (that is, $\mathfrak{q}_i \cap A = \mathfrak{p}_i$) for each $1 \le i \le m$, the latter chain can be extended to a chain $\mathfrak{q}_1 \subset \cdots \subset \mathfrak{q}_m$ so that $\mathfrak{q}_i$ lies over $\mathfrak{p}_i$ for all $1 \le i \le n$.
    
    The \textbf{going-up theorem} states that any \tref{integral-closure}{integral extension} $A \subset B$ satisfies the going-up property.
\end{topic}

\begin{topic}{going-down}{going-down}
    Let $A$ be a \tref{ring}{commutative ring}, and $B$ an $A$-algebra. The ring extension $A \subset B$ satisfies the \textbf{going-down property} if whenever
    \[ \mathfrak{p}_1 \supset \mathfrak{p}_2 \supset \cdots \supset \mathfrak{p}_n \quad \textup{and} \quad \mathfrak{q}_1 \supset \mathfrak{q}_2 \supset \cdots \supset \mathfrak{q}_m \]
    are chains of \tref{prime-ideal}{prime ideals} of $A$ and $B$, respectively, with $m < n$ such that $\mathfrak{q}_i$ lies over $\mathfrak{p}_i$ (that is, $\mathfrak{q}_i \cap A = \mathfrak{p}_i$) for each $1 \le i \le m$, the latter chain can be extended to a chain $\mathfrak{q}_1 \supset \cdots \supset \mathfrak{q}_m$ so that $\mathfrak{q}_i$ lies over $\mathfrak{p}_i$ for all $1 \le i \le n$.
    
    The \textbf{going-down theorem} states that any \tref{integral-closure}{integral extension} $A \subset B$ with $B$ a \tref{domain}{domain} and $A$ \tref{integral-closure}{integrally closed} in its \tref{field-of-fractions}{field of fractions} satisfies the going-down property.
\end{topic}

\begin{topic}{eisenstein-polynomial}{Eisenstein polynomial}
    Let $R$ be a \tref{unique-factorization-domain}{unique factorization domain} and $p \in R$ an \tref{irreducible-element}{irreducible element}. A polynomial
    \[ f = a_n x^n + \cdots + a_1 x + a_0 \in R[x] \]
    is \textbf{Eisenstein} (for $p$) if
    \begin{itemize}
        \item $p \nmid a_n$,
        \item $p | a_i$ for $i = 0, 1, \ldots, n - 1$,
        \item $p^2 \nmid a_0$.
    \end{itemize}
\end{topic}

\begin{example}{eisenstein-polynomial}
    \begin{itemize}
        \item The polynomial $f = x^5 + 6x^2 - 12 \in \ZZ[x]$ is Eisenstein for $p = 3$, but not for $p = 2$.
        \item The polynomial $f = x^3 + (y^4 - 1) x - (y^2 - 1) \in R[x]$ with $R = \ZZ[y]$ is Eisenstein for $p = y^2 + 1$.
    \end{itemize}
\end{example}

\begin{example}{eisenstein-polynomial}
    \textit{Eisenstein's criterion} states that if $f \in R[x]$ is an Eisenstein polynomial, then it is irreducible in $K[x]$, where $K$ is the \tref{field-of-fractions}{field of fractions} of $R$, and if $f$ is primitive it is also irreducible in $R[x]$.
    
    Namely, suppose that $f = a_n x^n + \cdots + a_0$ is Eisenstein, and that $f = gh$ for some $g, h \in R[x]$ with $\deg(g), \deg(h) > 0$. Reducing modulo $p$, we find that
    \[ \overline{f} = f \mod p = \overline{a}_n x^n \in (R/pR)[x] \]
    is non-zero as $p \nmid a_n$. Moreover $\overline{f} = \overline{g} \overline{h}$, which is only possible if $\overline{g} = \overline{b} x^k$ and $\overline{h} = \overline{c} x^\ell$ for some $b, c \in R$ and $k, l > 0$ with $k + \ell = n$. Hence, the constant coefficients of $g$ and $h$ are both divisible by $p$, which implies the constant coefficient $a_0$ of $f$ is divisible by $p^2$, a contradiction. Therefore $f$ is irreducible in $R[x]$, and if $f$ is primitive also in $K[x]$.
\end{example}

\begin{topic}{primitive-polynomial}{primitive polynomial}
    Let $R$ be a \tref{unique-factorization-domain}{unique factorization domain}. A polynomial $f = a_n x^n + \cdots + a_1 x + a_0 \in R[x]$ is \textbf{primitive} if the \tref{gcd}{greatest common divisor} of its coefficients, $\gcd(a_n, \ldots, a_0)$, also known as its \tref{content-polynomial}{content} is a \tref{unit}{unit} in $R$.
\end{topic}

\begin{example}{primitive-polynomial}
    The polynomial $f = 3x^4 + 5x \in \ZZ[x]$ is primitive since $\gcd(3, 5) = 1$, while $g = 2x^3 + 4 \in \ZZ[x]$ is not as $\gcd(2, 4) = 2$. However, both $f$ and $g$ are primitive in $\QQ[x]$.
\end{example}

\begin{topic}{content-polynomial}{content polynomial}
    Let $R$ be a \tref{unique-factorization-domain}{unique factorization domain}. The \textbf{content} of a polynomial $f \in R[x]$ is the the \tref{gcd}{greatest common divisor} of its coefficients,
    \[ \textup{co}(f) = \gcd(a_n, \ldots, a_0) , \]
    which is well-defined up to a unit in $R$.
\end{topic}

\begin{topic}{laurent-polynomial}{Laurent polynomial}
    A \textbf{Laurent polynomial} over a \tref{ring}{commutative ring} $R$ is an element $f \in R[x, x^{-1}]$, or explicitly
    \[ f = \sum_{i} a_i x^i \]
    where the sum is taken over finitely many $i \in \ZZ$.
\end{topic}

\begin{topic}{rees-algebra}{Rees algebra}
    Let $R$ be a \tref{ring}{commutative ring} and $I \subset R$ and \tref{ideal}{ideal}. Then the \textbf{Rees algebra} of $I$ in $R$ is the ring given by
    \[ R[It] = \bigoplus_{n = 0}^\infty I^n t^n \subset R[t] , \]
    where $I^0 = R$. % The \textbf{extended Rees algebra} of $I$ in $R$ is the ring given by
    % \[ R[It, t^{-1}] = \bigoplus_{n = -\infty}^\infty I^n t^n \subset R[t^{\pm 1}] . \]
\end{topic}

\begin{topic}{formal-group-law}{formal group law}
    Let $R$ be a \tref{ring}{commutative ring}. A \textbf{(1-dimensional) formal group law} on $R$ is a power series $F(x, y) \in R\llbracket x, y \rrbracket$ such that
    \begin{itemize}
        \item (\textit{linear terms}) $F(x, 0) = x$ and $F(0, y) = y$,
        \item (\textit{associativity}) $F(x, F(y, z)) = F(F(x, y), z)$.
    \end{itemize}
    Such a formal group law is called \textbf{commutative} if $F(x, y) = F(y, x)$.
\end{topic}

\begin{example}{formal-group-law}
    \begin{itemize}
        \item The \textit{additive group law} is given by $F(x, y) = x + y$.
        \item The \textit{multiplicative group law} is given by $F(x, y) = x + y + xy$.
        \item The group law $F(x, y) = (x + y) / (1 + xy)$ appears in \textit{special relativity} as the addition law for velocities (here the speed of light is set to $c = 1$).
        \item (Campbell--Hausdorff formula) Let $G$ be a \tref{DG:lie-group}{Lie group} and $\mathfrak{g}$ the corresponding \tref{lie-algebra}{Lie algebra}. For small enough $x, y \in \mathfrak{g}$ one has $\exp(x) \exp(y) = \exp(\mu(x, y))$ with
        \[ \mu(x, y) = x + y + \frac{1}{2} [x, y] + \frac{1}{12}([x, [x, y]] + [y, [y, x]]) + \cdots \]
    \end{itemize}
\end{example}

\begin{topic}{lazard-universal-ring}{Lazard's universal ring}
    \textbf{Lazard's universal ring} is \tref{ring}{commutative ring} $\mathbb{L}$ with a \tref{formal-group-law}{formal group law}
    \[ F(x, y) = x + y + \sum_{i + j \ge 2} a_{i,j} x^i y^j \in \mathbb{L}\llbracket x, y \rrbracket , \]
    with the universal property that for any formal group law $G(x, y) \in R \llbracket x, y \rrbracket$ over a commutative ring $R$, there exists a unique \tref{ring-morphism}{ring morphism} $f : \mathbb{L} \to R$ such that $G(f(x), f(y)) = f(F(x, y))$. The ring $\mathbb{L}$ is constructed as the quotient of the polynomial ring $\ZZ[a_{i, j} : i + j \ge 2]$ by the relations on the $a_{i, j}$ in order to make $F$ associative.
    
    It can be shown that $\mathbb{L}$ is isomorphic to the \tref{graded-ring}{graded ring} $\ZZ[x_1, x_2, \ldots]$, where $x_i$ has degree $i$, such that $a_{i, j}$ has degree $(i + j - 1)$.
\end{topic}

\begin{topic}{dedekind-domain}{Dedekind domain}
    A \textbf{Dedekind domain} is a \tref{noetherian-ring}{noetherian} \tref{integrally-closed-domain}{integrally closed domain} of \tref{krull-dimension}{dimension} $\le 1$ (that is, every non-zero \tref{prime-ideal}{prime ideal} is \tref{maximal-ideal}{maximal}).
\end{topic}

\begin{example}{dedekind-domain}
    Dedekind domains can be characterized in a number of ways. The following are all equivalent.
    \begin{enumerate}[(i)]
        \item $R$ is a Dedekind domain.
        \item $R$ is \tref{noetherian-ring}{noetherian}, and for every prime ideal $\mathfrak{p} \subset R$, the \tref{localization}{localization} $R_\mathfrak{p}$ is a \tref{discrete-valuation-ring}{discrete valuation ring}.
        \item Every \tref{fractional-ideal}{fractional ideal} of $R$ is \tref{invertible-ideal}{invertible}.
        \item Every non-zero \tref{ideal}{ideal} $I \subsetneq R$ factors into a product of prime ideals.
    \end{enumerate}
    % \textbf{Proof}. $(i \Rightarrow ii)$ Let $I$ be any non-zero ideal, and take some non-zero $a \in I$. Then $(a) \subset I$, so we can write
    % \[ I = \mathfrak{p}_1^{r_1} \cdots \mathfrak{p}_m^{r_m} \quad \textup{ and } \quad (a) = \mathfrak{p}_1^{s_1} \cdots \mathfrak{p}_m^{s_m} , \]
    % with $r_i \le s_i$. In particular, the fractional ideal $J = a^{-1} \mathfrak{p}_1^{s_1 - r_1} \cdots \mathfrak{p}_m^{s_m - r_m}$ is the inverse of $I$.
    % Why noetherian?
    % $(ii \Rightarrow iii)$ 
\end{example}

\begin{example}{dedekind-domain}
    \begin{itemize}
        \item Any \tref{field}{field} is a Dedekind domain.
        \item The \tref{NT:ring-of-integers}{ring of integers} $\mathcal{O}_K$ of a \tref{NT:number-field}{number field} $K$ is a Dedekind domain.
        \item The ring $\ZZ[\sqrt{5}]$ is not a Dedekind domain as it is not integrally closed: the element $\alpha = \frac{1 + \sqrt{5}}{2}$ satisfies $\alpha^2 - \alpha - 1 = 0$.
    \end{itemize}
\end{example}

\begin{topic}{corner-ring}{corner ring}
    Let $R$ be a \tref{ring}{ring}. A \textbf{corner} of $R$ is a subset $eRe \subset R$ with $e \in R$ an \tref{idempotent-element}{idempotent}. Such a corner is itself a ring, with $e$ as multiplicative identity.
\end{topic}

\begin{example}{corner-ring}
    Let $R = \textup{Mat}_{3 \times 3}(k)$ and $e = \begin{pmatrix} 1 & 0 & 0 \\ 0 & 1 & 0 \\ 0 & 0 & 0 \end{pmatrix}$. The associated corner is
    \[ eRe = \left\{ \begin{pmatrix} a & b & 0 \\ c & d & 0 \\ 0 & 0 & 0 \end{pmatrix} \right\} \simeq \textup{Mat}_{2 \times 2}(k) , \]
    with $e$ as multiplicative identity.
\end{example}

\begin{topic}{characteristic}{characteristic}
    The \textbf{characteristic} of a \tref{ring}{ring} $R$ is the integer $\textup{char}(R) \in \ZZ$ such that
    \[ \ker(\ZZ \to R) = \textup{char}(R) \ZZ . \]
\end{topic}

\begin{example}{characteristic}
    \begin{itemize}
        \item The characteristic of $\ZZ, \QQ, \RR, \CC$ is zero.
        \item The characteristic of $\FF_{p^n}$ is $p$.
        \item The characteristic of a \tref{field}{field} is always zero or a prime number. Namely, if $\textup{char}(R) = ab$ for non-zero $a, b \in \ZZ$, then $a$ and $b$ are \tref{zero-divisor}{zero-divisors} in $R$.
    \end{itemize}
\end{example}

\begin{topic}{coherent-ring}{coherent ring}
    A \tref{ring}{ring} $R$ is \textbf{coherent} if it is a \tref{coherent-module}{coherent module} over itself. That is, if every \tref{finitely-generated-module}{finitely generated} \tref{ideal}{ideal} $I \subset R$ is \tref{finitely-presented-module}{finitely presented}.
\end{topic}

\begin{topic}{symmetric-polynomial}{symmetric polynomial}
    Let $R$ be a \tref{ring}{commutative ring}. A polynomial $f \in R[x_1, \ldots, x_n]$ is \textbf{symmetric} if it is invariant under any \tref{GT:symmetric-group}{permutation} of the variables, that is,
    \[ f(x_{\sigma(1)}, \ldots, x_{\sigma(n)}) = f(x_1, \ldots, x_n) \quad \textup{ for all } \sigma \in S_n . \]
    \begin{itemize}
        \item For any $0 \le k \le n$, the \textbf{$k$th elementary symmetric polynomial} is given by
        \[ e_k = \sum_{1 \le i_1 < \ldots < i_k \le n} x_{i_1} \cdots x_{i_k} . \]
        \item For any \tref{GM:integer-partition}{partition} $\lambda = (\lambda_1, \ldots, \lambda_n)$, the \textbf{monomial symmetric polynomial} of $\lambda$ is given by
        \[ m_\lambda = \sum_{\mu} x_1^{\mu_1} \cdots x_n^{\mu_n} , \]
        where $\mu$ ranges over all distinct permutations of $\lambda$.
        \item For any partition $\lambda = (\lambda_1, \ldots, \lambda_r)$, the \textbf{homogeneous symmetric polynomial} of $\lambda$ is given by
        \[ h_\lambda = h_{\lambda_1} \cdots h_{\lambda_r} , \]
        where $h_k$ is the sum of all monomials of degree $k$.
        \item For any partition $\lambda = (\lambda_1, \ldots, \lambda_r)$, the \textbf{power symmetric polynomial} of $\lambda$ is given by
        \[ p_\lambda = p_{\lambda_1} \cdots p_{\lambda_r} , \]
        where $p_k = m_{(k)}$ is the sum of all $k$-th powers.
    \end{itemize}
    The \textbf{fundamental theorem of symmetric polynomials} states that any symmetric polynomial $f \in R[x_1, \ldots, x_n]$ can be written uniquely as a polynomial in the elementary symmetric polynomials $e_k$. In other words, there is an isomorphism $R[x_1, \ldots, x_n]^{S_n} \simeq R[e_1, \ldots, e_n]$.
\end{topic}

\begin{example}{symmetric-polynomial}
    The elementary symmetric polynomials appear (up to sign) as the coefficients of the polynomial
    \[ F = \sum_{i = 1}^{n} (T - x_i) = \sum_{k = 0}^{n} (-1)^k e_k (x_1, \ldots, x_n) T^{n - k} \in R[x_1, \ldots, x_n][T]  .\]
    Now let $f = \sum_{k = 0}^{n} a_k T^k \in R[T]$ be a monic polynomial over a field $k$. Over some \tref{splitting-field}{splitting field} $\ell/k$, one may write $f = \prod_{i = 1}^{n} (T - \alpha_i)$, where $\alpha_i \in \ell$ are the zeros of $f$. In particular, from the above we find
    \[ f = \sum_{k = 0}^{n} (-1)^k e_k(\alpha_1, \ldots, \alpha_n) T^{n - k} , \]
    that is, the coefficients of $f$ are given by $a_i = (-1)^{n - i} e_{n - i}(\alpha_1, \ldots, \alpha_n)$. Hence, by the fundamental theorem of symmetric polynomials, any symmetric function of the roots $\alpha_i$ can be written as a function of the coefficients of $f$. For example,
    \[ \sum_{i = 1}^{n} \alpha_i = - a_{n - 1} \quad \textup{ and } \quad \prod_{i = 1}^{n} \alpha_i = (-1)^n a_0 . \]
    As another example, the \tref{NT:discriminant}{discriminant} $\Delta(f) = \prod_{1 \le i < j \le n} (\alpha_i - \alpha_j)$ of $f$ is symmetric, so it can be expressed in terms of the coefficients of $f$.
\end{example}

\begin{topic}{ore-extension}{Ore extension}
    Let $R$ be a \tref{ring}{ring}. Given an endomorphism $\sigma : R \to R$, a \textit{$\sigma$-derivation} is a group morphism $\delta : R \to R$ satisfying
    \[ \delta(rs) = \delta(r) s + \sigma(r) \delta(s) , \quad \textup { for all } r, s \in R . \]
    The \textbf{Ore extension} $R[x; \sigma, \delta]$ is the (noncommutative) ring $R[x]$, whose multiplication is determined by
    \[ xr = \sigma(r) x + \delta(r), \quad \textup{ for all } r \in R . \]
\end{topic}

\begin{example}{ore-extension}
    Let $R$ be the polynomial ring $k[x]$, with $\sigma : R \to R$ given by $x \mapsto qx$ for some $q \in k$, and $\delta = 0$. Then, the Ore extension $R[y; \sigma, \delta]$ is the quotient of the free algebra $k \langle x, y \rangle$ by the relation $yx - qxy = 0$, also known as the \textit{quantum plane}
    \[ k \langle x, y \rangle / (yx - qxy) . \]
\end{example}

\begin{example}{ore-extension}
    Let $R$ be the polynomial ring $k[x]$, with $\sigma : R \to R$ the identity, and $\delta : R \to R$ given by $f \mapsto \frac{\partial f}{\partial x}$. Then, the Ore extension $R[p; \sigma, \delta]$ is the quotient of the free algebra $k \langle x, p \rangle$ by the relation $xp - px + 1 = 0$,
    \[ k \langle x, y \rangle / (xp - px + 1) . \]
    that is, a \tref{weyl-algebra}{Weyl algebra}.
\end{example}

\begin{topic}{divided-power-structure}{divided power structure}
    Let $R$ be a \tref{ring}{commutative ring} and $I \subset R$ an \tref{ideal}{ideal}. A \textbf{divided power structure} on $I$ is a sequence of maps $\delta_n : I \to A$ for $n \ge 0$, satisfying
    \begin{itemize}
        \item $\delta_0(x) = 1$ and $\delta_1(x) = x$ and $\delta_n(x) \in I$ for all $x \in I$ and $n > 0$,
        \item $\delta_n(x + y) = \sum_{i = 0}^{n} \delta_{n - i}(x) \delta_i(y)$ for all $x, y \in I$,
        \item $\delta_n(a x) = a^n \delta_n(x)$ for all $a \in R$ and $x \in I$,
        \item $\delta_m(x) \delta_n(x) = \frac{(m + n)!}{m! n!} \delta_{m + n}(x)$ for all $x \in I$ and $m, n \ge 0$,
        \item $\delta_n(\delta_m(x)) = \frac{(mn)!}{(m!)^n n!} \delta_{mn}(x)$ for all $x \in I$ and $m > 0$.
    \end{itemize}
\end{topic}

\begin{example}{divided-power-structure}
    \begin{itemize}
        \item For an $\QQ$-algebra $A$, every ideal $I$ has a unique divided power structure with $\delta_n(x) = \frac{x^n}{n!}$.
        \item For $R = \ZZ/4\ZZ$ with $I = 2\ZZ/4\ZZ$, there are precisely two divided power structures, given by
        \[ \delta_n(2) = \left\{ \begin{array}{cl} 2 & \textup{ if } n \in 2^{\ZZ_{\ge 0}} , \\ 0 & \textup{ otherwise,} \end{array} \right. \quad \textup{ and } \quad \delta'_n(2) = \left\{ \begin{array}{cl} 2 & \textup{ if } n = 1 , \\ 0 & \textup{ otherwise.} \end{array} \right. \]
    \end{itemize}
\end{example}

\begin{topic}{lambda-structure}{lambda structure}
    Let $R$ be a \tref{ring}{commutative ring}. A \textbf{$\lambda$-structure} on $R$ is a sequence of maps $\lambda^n : R \to R$ for $n \ge 0$, satisfying
    \begin{itemize}
        \item $\lambda^0(r) = 1$ for all $r \in R$,
        \item $\lambda^1(r) = r$ for all $r \in R$,
        \item $\lambda^n(1) = 0$ for all $n > 1$,
        \item $\lambda^{n}(rs) = \sum_{k = 0}^{n} \lambda^k(r) \lambda^{n - k}(s)$ for all $r, s \in R$,
        \item $\lambda^n(rs) = P_n(\lambda^1(r), \ldots, \lambda^n(r), \lambda^1(s), \ldots,  \lambda^n(s))$ for all $r, s \in R$,
        \item $\lambda^m(\lambda^n(r)) = P_{m, n}(\lambda^1(r), \ldots, \lambda^{mn}(r))$ for all $r \in R$,
    \end{itemize}
    where $P_n$ and $P_{m, n}$ are certain universal polynomials with integer coefficients.
\end{topic}

\begin{example}{lambda-structure}
    For any \tref{GT:group}{group} $G$, the \tref{RT:representation-ring}{representation ring} $R(G)$ has a natural $\lambda$-structure. For any \tref{RT:representation}{representation} $\rho : G \to \textup{GL}(V)$ and $n \ge 0$, the $\lambda^n(\rho)$ is given by induced representation on the \tref{exterior-algebra}{exterior power} $\wedge^n V$.
\end{example}

\begin{topic}{conductor}{conductor}
    Let $A$ be and $B$ be \tref{ring}{commutative rings} with $A \subset B$. The \textbf{conductor} of $A$ in $B$ is the \tref{ideal}{ideal}
    \[ \mathfrak{f}(B/A) = \textup{Ann}_A(B/A) = \{ a \in A \;|\; a B \subset A \} . \]
\end{topic}
