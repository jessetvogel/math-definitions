\begin{topic}{monoid}{monoid}
    A \textbf{monoid} is a set $M$ with a binary operation $\cdot : M \times M \to M$ satisfying
    \begin{itemize}
        \item (\textit{associative}) $(a \cdot b) \cdot c = a \cdot (b \cdot c)$ for all $a, b, c \in M$,
        \item (\textit{identity element}) there exists an element $e \in M$ such that $e \cdot a = a = a \cdot e$ for all $a \in M$.
    \end{itemize}
    A monoid $M$ is said to be \textbf{commutative} if $a \cdot b = b \cdot a$ for all $a, b \in M$.
\end{topic}

\begin{topic}{integral-monoid}{integral monoid}
    A \tref{monoid}{commutative monoid} $M$ is \textbf{integral} if $a + b = a + c$ implies $b = c$ for all $a, b, c \in M$.
\end{topic}

\begin{topic}{saturated-monoid}{saturated monoid}
    A \tref{monoid}{commutative monoid} $M$ is \textbf{saturated} if $a^k \in M$ implies $a \in M$ for all $a \in M^\textup{gp}$.
\end{topic}

\begin{example}{saturated-monoid}
    The natural numbers $\NN$ form a saturated monoid under addition. However, when removing the number $1$, the remaining $M = \NN \setminus \{ 1 \}$ is no longer saturated. Namely, $3 - 2 = 1 \in M^\textup{gp} \setminus M$ while $(3 - 2) + (3 - 2) = 2 \in M$.
\end{example}

\begin{topic}{fine-monoid}{fine monoid}
    A \tref{monoid}{commutative monoid} $M$ is \textbf{fine} if it is finitely generated and \tref{integral-monoid}{integral}.
\end{topic}

\begin{topic}{semigroup}{semigroup}
    A \textbf{semigroup} is a set $S$ with a binary operation $\cdot : S \times S \to S$ which is associative, that is, $(a \cdot b) \cdot c = a \cdot (b \cdot c)$ for all $a, b, c \in S$.
\end{topic}

\begin{topic}{semiring}{semiring}
    A \textbf{semiring} is a set $R$ with two binary operations $+ : R \times R \to R$, called addition, satisfying
    \begin{itemize}
        \item (\textit{associative}) $(a + b) + c = a + (b + c)$ for all $a, b, c \in R$,
        \item (\textit{commutative}) $a + b = b + a$ for all $a, b \in R$,
        \item (\textit{zero element}) there exists an element $0 \in R$ such that $a + 0 = 0 + a = a$ for all $a \in R$,
    \end{itemize}
    and $\cdot : R \times R \to R$, called multiplication, satisfying
    \begin{itemize}
        \item (\textit{associative}) $(a \cdot b) \cdot c = a \cdot (b \cdot b)$ for all $a, b, c \in R$,
        \item (\textit{unit element}) there exists an element $1 \in R$ such that $a \cdot 1 = 1 \cdot a = a$ for all $a \in R$,
        \item (\textit{distributive}) $a \cdot (b + c) = (a \cdot b) + (a \cdot c)$ and $(a + b) \cdot c = (a \cdot c) + (b \cdot c)$ for all $a, b, c \in R$,
        \item (\textit{zero element}) $0 \cdot a = a \cdot 0 = 0$ for all $a \in R$.
    \end{itemize}
\end{topic}

\begin{example}{semiring}
    The natural numbers $\NN$ (including zero) form a semiring under usual addition and multiplication. Similarly, $\QQ_{\ge 0}$ and $\RR_{\ge 0}$ are semirings.
\end{example}

\begin{topic}{bialgebra}{bialgebra}
    A \textbf{bialgebra} over a \tref{CA:field}{field} $k$ is an \tref{CA:algebra}{algebra} $A$ over $k$ together with morphisms of $k$-algebras $\Delta : A \to A \otimes A$ (the \textit{comultiplication}) and $\varepsilon : A \to k$ (the \textit{counit}), satisfying
    \begin{itemize}
        \item (\textit{coassociativity}) $(\Delta \otimes \id) \circ \Delta = (\id \otimes \Delta) \circ \Delta$,
        \item (\textit{counit}) $(\varepsilon \otimes \id) \circ \Delta = \id = (\id \otimes \varepsilon) \otimes \Delta$.
    \end{itemize}
\end{topic}

\begin{topic}{hopf-algebra}{Hopf algebra}
    A \textbf{Hopf algebra} over a \tref{CA:field}{field} $k$ is a \tref{bialgebra}{bialgebra} $A$ over $k$ together with a $k$-linear map $S : A \to A$ (the \textit{antipode}), such that the diagram
    \[ \begin{tikzcd}[column sep=1em, row sep=2em] & A \otimes A \arrow{rr}{S \otimes \id} && A \otimes A \arrow{dr}{\mu} & \\ A \arrow{rr}{\varepsilon} \arrow{ur}{\Delta} \arrow[swap]{dr}{\Delta} && k \arrow{rr}{\eta} && A \\ & A \otimes A \arrow[swap]{rr}{\id \otimes S} && A \otimes A \arrow[swap]{ur}{\mu} &  \end{tikzcd} \]
    commutes, where $\mu$ denotes the multiplication, $\Delta$ the comultiplication, $\eta$ the unit, and $\varepsilon$ the counit of the algebra.
\end{topic}

\begin{example}{hopf-algebra}
    The category of commutative \tref{CA:finitely-generated-algebra}{finitely generated} Hopf algebras over $k$ is \tref{CT:equivalence-of-categories}{equivalent} to the category of \tref{AG:affine-scheme}{affine} \tref{AG:algebraic-group}{algebraic groups} over $k$, via the functors
    \[ A \mapsto \Spec A, \quad \text{and} \quad G \mapsto \Gamma(G, \mathcal{O}_G) . \]
    The antipode corresponds to group inversion.
\end{example}

\begin{topic}{frobenius-algebra}{Frobenius algebra}
    A \textbf{Frobenius algebra} over a field $k$ is a $k$-algebra $A$ equipped with a bilinear form $\beta : A \otimes_k A \to k$, called the \textit{Frobenius pairing}, satisfying
    \begin{itemize}
        \item (\textit{associativity}) $\beta(ab \otimes c) = \beta(a \otimes bc)$ for all $a, b, c \in A$,
        \item (\textit{non-degeneracy}) there exists a $k$-linear map $\gamma : k \to A \otimes A$ such that 
        \[ (\beta \otimes \id_A)(a \otimes \gamma(1)) = a = (\id_A \otimes \beta)(\gamma(1) \otimes a) \]
        for all $a \in A$.
    \end{itemize}
\end{topic}

\begin{example}{frobenius-algebra}
    \begin{itemize}
        \item Let $A$ be a finite field extension of $k$, and $\varepsilon : A \to k$ any non-zero $k$-linear map. Then $A$ with the pairing $\beta : A \otimes A \to k, a \otimes b \mapsto \varepsilon(ab)$ is a Frobenius algebra.
        \item The ring $\textup{Mat}_n(k)$ of $n \times n$ matrices over $k$, with pairing $A \otimes B \mapsto \operatorname{tr}(AB)$, is a Frobenius algebra.
    \end{itemize}
\end{example}

\begin{example}{frobenius-algebra}
    Any Frobenius algebra $A$ is finite-dimensional. Namely, write $\gamma(1) = \sum_{i = 1}^{n} a_i \otimes b_i$ for some $a_i, b_i \in A$, and note for any $a \in A$ we have
    \[ a = (\id_A \otimes \beta) \circ (\gamma \otimes \id_A)(1 \otimes a) = (\id_A \otimes \beta) \left( \sum_{i = 1}^{n} a_i \otimes b_i \otimes a \right) = \sum_{i = 1}^{n} a_i \cdot \beta(b_i \otimes a) , \]
    so in particular $a \in \langle a_1, \ldots, a_n \rangle$, implying that $A$ is finite-dimensional.
\end{example}

\begin{example}{frobenius-algebra}
    A Frobenius algebra $A$ naturally carries a $k$-coalgebra structure. Take as counit the map
    \[ \varepsilon : A \to k, \quad a \mapsto \beta(1 \otimes a) = \beta(a \otimes 1) \]
    which is well-defined by associativity of $\beta$. As comultiplication, take
    \[ \delta : A \to A \otimes A, \quad a \mapsto (\mu \otimes \id_A)(a \otimes \gamma(1)) = (\id_A \otimes \mu)(\gamma(1) \otimes a) . \]
    To see why this is well-defined, write $\gamma(1) = \sum_i a_i \otimes b_i$ as before, then the associativity and non-degeneracy of $\beta$ imply that
    \[ \begin{aligned}
        (\mu \otimes \id_A)(a \otimes \gamma(1)) &= \sum_{i = 1}^{n} a a_i \otimes b_i = \sum_{i,j = 1}^{n} \beta(b_j \otimes a a_i) a_j \otimes b_i \\ &= \sum_{i,j = 1}^{n} \beta(b_i a \otimes a_j) a_i \otimes b_j = \sum_{i = 1}^{n} a_i \otimes b_i a = (\id_A \otimes \mu)(\gamma(1) \otimes a) 
    \end{aligned} \]
    for all $a \in A$. Note that $\delta$ is coassociative since
    \[ (\id_A \otimes \delta) \circ \delta(a) = \sum_{i, j = 1}^{n} a a_i \otimes b_i a_j \otimes b_j = \sum_{i, j = 1}^{n} a a_i \otimes b_i a_j \otimes b_j = \sum_{i, j = 1}^{n} a a_j a_i \otimes b_i \otimes b_j = (\delta \otimes \id_A) \circ \delta(a) \]
    for all $a \in A$. Furthermore, $\varepsilon$ is a counit for the comultiplication since
    \[ (\id_A \otimes \varepsilon)(\delta(a)) = \sum_{i = 1}^{n} a a_i \cdot \beta(b_i \otimes 1) = a \cdot 1 = a \]
    for all $a \in A$. Similarly $(\varepsilon \otimes \id_A)(\delta(a)) = a$ for all $a \in A$.
\end{example}

\begin{topic}{augmentation}{augmentation}
    An \textbf{augmentation} of an \tref{CA:algebra}{algebra} over a field $k$, is a morphism of $k$-algebras $\varepsilon : A \to k$.
\end{topic}

\begin{topic}{weyl-algebra}{Weyl algebra}
    Let $(V, \omega)$ be a \tref{LA:symplectic-vector-space}{symplectic vector space}. The \textbf{Weyl algebra} of $V$ is the quotient
    \[ W(V) = T(V) / (v \otimes w - w \otimes v - \omega(v, w) : v, w \in V) , \]
    of the \tref{CA:tensor-algebra}{tensor algebra} $T(V)$ by the given ideal.
\end{topic}

\begin{example}{weyl-algebra}
    Let $V = \RR^{2n}$ with coordinates $q^1, \ldots, q^n, p_1, \ldots, p_n$ and the standard symplectic form $\omega$ given by $\omega(q^i, p_j) = 1$ and $\omega(q^i, q^j) = \omega(p_i, p_j) = 0$ for all $i, j$. The corresponding Weyl algebra is given by
    \[ \RR \langle q^1, \ldots, q^n, p_1, \ldots, p_n \rangle / (q^i q^j - q^j q^i, p_i p_j - p_j p_i, q^i p_j - p_j q^i - 1) . \]
    Note that this algebra is isomorphic to the algebra over $\RR$ generated by polynomials in $q^i$ together with the differential operators $\partial/\partial q^i$, via the isomorphism $p_i \mapsto \partial/\partial q^i$.
\end{example}
