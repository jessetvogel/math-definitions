\begin{topic}{monoid}{monoid}
    A \textbf{monoid} is a set $M$ with a binary operation $\cdot : M \times M \to M$ satisfying
    \begin{itemize}
        \item (\textit{associative}) $(a \cdot b) \cdot c = a \cdot (b \cdot c)$ for all $a, b, c \in M$,
        \item (\textit{identity element}) there exists an element $e \in M$ such that $e \cdot a = a = a \cdot e$ for all $a \in M$.
    \end{itemize}
    A monoid $M$ is said to be \textbf{commutative} if $a \cdot b = b \cdot a$ for all $a, b \in M$.
\end{topic}

\begin{topic}{integral-monoid}{integral monoid}
    A \tref{monoid}{commutative monoid} $M$ is \textbf{integral} if $a + b = a + c$ implies $b = c$ for all $a, b, c \in M$.
\end{topic}

\begin{topic}{saturated-monoid}{saturated monoid}
    A \tref{monoid}{commutative monoid} $M$ is \textbf{saturated} if it is \tref{integral-monoid}{integral} and if $a^k \in M$ implies $a \in M$ for all $a \in M^\textup{gp}$.
\end{topic}

\begin{example}{saturated-monoid}
    The natural numbers $\NN$ form a saturated monoid under addition. However, when removing the number $1$, the remaining $M = \NN \setminus \{ 1 \}$ is no longer saturated. Namely, $3 - 2 = 1 \in M^\textup{gp} \setminus M$ while $(3 - 2) + (3 - 2) = 2 \in M$.
\end{example}

\begin{topic}{fine-monoid}{fine monoid}
    A \tref{monoid}{commutative monoid} $M$ is \textbf{fine} if it is finitely generated and \tref{integral-monoid}{integral}.
\end{topic}

\begin{topic}{sharp-monoid}{sharp monoid}
    A \tref{monoid}{commutative monoid} $M$ is \textbf{sharp} if the identity element $e \in M$ is the only unit in $M$.
\end{topic}

\begin{example}{sharp-monoid}
    The commutative monoid $(\NN, +)$ is sharp, but $(\ZZ, +)$ is not.
\end{example}

\begin{topic}{semigroup}{semigroup}
    A \textbf{semigroup} is a set $S$ with a binary operation $\cdot : S \times S \to S$ which is associative, that is, $(a \cdot b) \cdot c = a \cdot (b \cdot c)$ for all $a, b, c \in S$.
\end{topic}

\begin{topic}{semiring}{semiring}
    A \textbf{semiring} is a set $R$ with two binary operations $+ : R \times R \to R$, called addition, satisfying
    \begin{itemize}
        \item (\textit{associative}) $(a + b) + c = a + (b + c)$ for all $a, b, c \in R$,
        \item (\textit{commutative}) $a + b = b + a$ for all $a, b \in R$,
        \item (\textit{zero element}) there exists an element $0 \in R$ such that $a + 0 = 0 + a = a$ for all $a \in R$,
    \end{itemize}
    and $\cdot : R \times R \to R$, called multiplication, satisfying
    \begin{itemize}
        \item (\textit{associative}) $(a \cdot b) \cdot c = a \cdot (b \cdot b)$ for all $a, b, c \in R$,
        \item (\textit{unit element}) there exists an element $1 \in R$ such that $a \cdot 1 = 1 \cdot a = a$ for all $a \in R$,
        \item (\textit{distributive}) $a \cdot (b + c) = (a \cdot b) + (a \cdot c)$ and $(a + b) \cdot c = (a \cdot c) + (b \cdot c)$ for all $a, b, c \in R$,
        \item (\textit{zero element}) $0 \cdot a = a \cdot 0 = 0$ for all $a \in R$.
    \end{itemize}
\end{topic}

\begin{example}{semiring}
    The natural numbers $\NN$ (including zero) form a semiring under usual addition and multiplication. Similarly, $\QQ_{\ge 0}$ and $\RR_{\ge 0}$ are semirings.
\end{example}

\begin{topic}{bialgebra}{bialgebra}
    A \textbf{bialgebra} over a \tref{field}{field} $k$ is an \tref{algebra}{algebra} $A$ over $k$ together with morphisms of $k$-algebras $\Delta : A \to A \otimes A$ (the \textit{comultiplication}) and $\varepsilon : A \to k$ (the \textit{counit}), satisfying
    \begin{itemize}
        \item (\textit{coassociativity}) $(\Delta \otimes \id) \circ \Delta = (\id \otimes \Delta) \circ \Delta$,
        \item (\textit{counit}) $(\varepsilon \otimes \id) \circ \Delta = \id = (\id \otimes \varepsilon) \otimes \Delta$.
    \end{itemize}
\end{topic}

\begin{topic}{hopf-algebra}{Hopf algebra}
    A \textbf{Hopf algebra} over a \tref{field}{field} $k$ is a \tref{bialgebra}{bialgebra} $A$ over $k$ together with a $k$-linear map $S : A \to A$ (the \textit{antipode}), such that the diagram
    \[ \begin{tikzcd}[column sep=1em, row sep=2em] & A \otimes A \arrow{rr}{S \otimes \id} && A \otimes A \arrow{dr}{\mu} & \\ A \arrow{rr}{\varepsilon} \arrow{ur}{\Delta} \arrow[swap]{dr}{\Delta} && k \arrow{rr}{\eta} && A \\ & A \otimes A \arrow[swap]{rr}{\id \otimes S} && A \otimes A \arrow[swap]{ur}{\mu} &  \end{tikzcd} \]
    commutes, where $\mu$ denotes the multiplication, $\Delta$ the comultiplication, $\eta$ the unit, and $\varepsilon$ the counit of the algebra.
\end{topic}

\begin{example}{hopf-algebra}
    The category of commutative \tref{finitely-generated-algebra}{finitely generated} Hopf algebras over $k$ is \tref{CT:equivalence-of-categories}{equivalent} to the category of \tref{AG:affine-scheme}{affine} \tref{AG:algebraic-group}{algebraic groups} over $k$, via the functors
    \[ A \mapsto \Spec A, \quad \text{and} \quad G \mapsto \Gamma(G, \mathcal{O}_G) . \]
    The antipode corresponds to group inversion.
\end{example}

\begin{topic}{group-like-element}{group-like element}
    Let $A$ be a \tref{hopf-algebra}{Hopf algebra}. An element $a \in A$ is \textbf{group-like} if
    \[ \Delta(a) = a \otimes a , \]
    where $\Delta : A \to A \otimes A$ denotes the comultiplication map.
\end{topic}

\begin{example}{group-like-element}
    If $A = \mathcal{O}_G(G)$ for some \tref{AG:algebraic-group}{algebraic group} $G$, then a group-like element $a \in A$ corresponds to a group morphism
    \[ \chi : G \to \mathbb{G}_m = \Spec(k[t, t^{-1}]), \quad t \mapsto a , \]
    known as a \textit{character}, and vice-versa, any such character corresponds to a group-like element $a = \chi^\#(t) \in A$.
    
    For example, for $G = \textup{GL}_n(k)$ the group of invertible matrices, the determinant $\det \in \mathcal{O}_G(G)$ is a group-like element, since $\det(AB) = \det(A) \det(B)$ for all $A, B \in \textup{GL}_n(k)$.
\end{example}

% \begin{topic}{dual-hopf-algebra}{dual Hopf algebra}
%     Let $(A, \mu, \eta, \Delta, \varepsilon, S)$ be a \tref{hopf-algebra}{Hopf algebra} over a \tref{AA:field}{field} $k$. The \textbf{dual Hopf algebra} of $A$ is the $k$-algebra
%     \[ A^\circ = \{ a \in A \mid \mu^*(a) \in A^* \otimes_k A^* \subset (A \otimes_k A)^* \} \]
%     with multiplication $\Delta^* : A^\circ \otimes_k A^\circ \to A^\circ$, unit $\varepsilon^* : k \to A^\circ$, comultiplication $\mu^* : A^\circ \to A^\circ \otimes_k A^\circ$, counit $\eta^* : A^\circ \to k$ and antipode $S^* : A^\circ \to A^\circ$.
% \end{topic}

\begin{topic}{frobenius-algebra}{Frobenius algebra}
    A \textbf{Frobenius algebra} over a \tref{field}{field} $k$ is a \tref{algebra}{$k$-algebra} $A$ equipped with a bilinear form $\beta : A \otimes_k A \to k$, called the \textit{Frobenius pairing}, satisfying
    \begin{itemize}
        \item (\textit{associativity}) $\beta(ab \otimes c) = \beta(a \otimes bc)$ for all $a, b, c \in A$,
        \item (\textit{non-degeneracy}) there exists a $k$-linear map $\gamma : k \to A \otimes_k A$ such that 
        \[ (\beta \otimes \id_A)(a \otimes \gamma(1)) = a = (\id_A \otimes \beta)(\gamma(1) \otimes a) \]
        for all $a \in A$.
    \end{itemize}
\end{topic}

\begin{example}{frobenius-algebra}
    \begin{itemize}
        \item Let $A$ be a finite field extension of $k$, and $\varepsilon : A \to k$ any non-zero $k$-linear map. Then $A$ with the pairing $\beta : A \otimes_k A \to k, a \otimes b \mapsto \varepsilon(ab)$ is a Frobenius algebra.
        \item The ring $\textup{Mat}_n(k)$ of $n \times n$ matrices over $k$, with pairing $A \otimes B \mapsto \operatorname{tr}(AB)$, is a Frobenius algebra.
        \item The complex numbers $\CC$ are a Frobenius algebra over $\RR$ with pairing $\beta(z_1 \otimes z_2) = \operatorname{Re}(z_1 z_2)$. The map $\gamma : \RR \to \CC \otimes_\RR \CC$ is given by $\gamma(1) = 1 \otimes 1 - i \otimes i$.
    \end{itemize}
\end{example}

\begin{example}{frobenius-algebra}
    Any Frobenius algebra $A$ is finite-dimensional. Namely, write $\gamma(1) = \sum_{i = 1}^{n} a_i \otimes b_i$ for some $a_i, b_i \in A$, and note for any $a \in A$ we have
    \[ a = (\id_A \otimes \beta) \circ (\gamma \otimes \id_A)(1 \otimes a) = (\id_A \otimes \beta) \left( \sum_{i = 1}^{n} a_i \otimes b_i \otimes a \right) = \sum_{i = 1}^{n} a_i \cdot \beta(b_i \otimes a) , \]
    so in particular $a \in \langle a_1, \ldots, a_n \rangle$, implying that $A$ is finite-dimensional.
\end{example}

\begin{example}{frobenius-algebra}
    A Frobenius algebra $A$ naturally carries a $k$-coalgebra structure. Take as counit the map
    \[ \varepsilon : A \to k, \quad a \mapsto \beta(1 \otimes a) = \beta(a \otimes 1) \]
    which is well-defined by associativity of $\beta$. As comultiplication, take
    \[ \delta : A \to A \otimes_k A, \quad a \mapsto (\mu \otimes \id_A)(a \otimes \gamma(1)) = (\id_A \otimes \mu)(\gamma(1) \otimes a) . \]
    To see why this is well-defined, write $\gamma(1) = \sum_i a_i \otimes b_i$ as before, then the associativity and non-degeneracy of $\beta$ imply that
    \[ \begin{aligned}
        (\mu \otimes \id_A)(a \otimes \gamma(1)) &= \sum_{i = 1}^{n} a a_i \otimes b_i = \sum_{i,j = 1}^{n} \beta(b_j \otimes a a_i) a_j \otimes b_i \\ &= \sum_{i,j = 1}^{n} \beta(b_i a \otimes a_j) a_i \otimes b_j = \sum_{i = 1}^{n} a_i \otimes b_i a = (\id_A \otimes \mu)(\gamma(1) \otimes a) 
    \end{aligned} \]
    for all $a \in A$. Note that $\delta$ is coassociative since
    \[ (\id_A \otimes \delta) \circ \delta(a) = \sum_{i, j = 1}^{n} a a_i \otimes b_i a_j \otimes b_j = \sum_{i, j = 1}^{n} a a_i \otimes b_i a_j \otimes b_j = \sum_{i, j = 1}^{n} a a_j a_i \otimes b_i \otimes b_j = (\delta \otimes \id_A) \circ \delta(a) \]
    for all $a \in A$. Furthermore, $\varepsilon$ is a counit for the comultiplication since
    \[ (\id_A \otimes \varepsilon)(\delta(a)) = \sum_{i = 1}^{n} a a_i \cdot \beta(b_i \otimes 1) = a \cdot 1 = a \]
    for all $a \in A$. Similarly $(\varepsilon \otimes \id_A)(\delta(a)) = a$ for all $a \in A$.
\end{example}

\begin{topic}{augmentation}{augmentation}
    An \textbf{augmentation} of an \tref{algebra}{algebra} over a field $k$, is a morphism of $k$-algebras $\varepsilon : A \to k$.
\end{topic}

\begin{topic}{weyl-algebra}{Weyl algebra}
    Let $(V, \omega)$ be a \tref{LA:symplectic-vector-space}{symplectic vector space}. The \textbf{Weyl algebra} of $V$ is the quotient
    \[ W(V) = T(V) / (v \otimes w - w \otimes v - \omega(v, w) : v, w \in V) , \]
    of the \tref{tensor-algebra}{tensor algebra} $T(V)$ by the given ideal.
\end{topic}

\begin{example}{weyl-algebra}
    Let $V = \RR^{2n}$ with coordinates $q^1, \ldots, q^n, p_1, \ldots, p_n$ and the standard symplectic form $\omega$ given by $\omega(q^i, p_j) = -1$ and $\omega(q^i, q^j) = \omega(p_i, p_j) = 0$ for all $i, j$. The corresponding Weyl algebra is given by
    \[ \RR \langle q^1, \ldots, q^n, p_1, \ldots, p_n \rangle / (q^i q^j - q^j q^i, p_i p_j - p_j p_i, q^i p_j - p_j q^i - 1) . \]
    Note that this algebra is isomorphic to the algebra over $\RR$ generated by polynomials in $q^i$ together with the differential operators $\partial/\partial q^i$, via the isomorphism $p_i \mapsto \partial/\partial q^i$.
\end{example}

\begin{topic}{division-algebra}{division algebra}
    Let $k$ be a \tref{field}{field}. An \tref{algebra}{algebra} $A$ over $k$ is a \textbf{division algebra} if for every non-zero $a \in A$ there exists an inverse element $b \in A$ satisfying $ab = ba = 1$.
\end{topic}

\begin{example}{division-algebra}
    \begin{itemize}
        \item Any \tref{field-extension}{field extension} $\ell / k$ is a division algebra over $k$.
        \item The \tref{quaternion-algebra}{quaternions} are a (non-commutative) division algebra over $\RR$.
    \end{itemize}
\end{example}

\begin{example}{division-algebra}
    The only finite-dimensional division algebra over an \tref{algebraically-closed-field}{algebraically closed field} $k$ is $k$ itself. Namely, let $A$ be a finite-dimensional division algebra over $k$, and take any element $x \in A$. Since $A$ is finite-dimensional, the powers of $x$ must be linearly dependent, that is, there exists some relation
    \[ f(x) = x^n + c_1 x^{n - 1} + \cdots + c_{n - 1} x + c_n = 0 \]
    for some $c_i \in k$. Since $k$ is algebraically closed, the polynomial $f$ factors into linear factors $f(x) = (x - \alpha_1) \cdots (x - \alpha_n)$ for some $\alpha_i \in k$. Assuming $f$ has minimal degree, it follows that $f(x) = x - \alpha_1$ and thus $x = \alpha_1 \in k$, so $A = k$.
\end{example}

\begin{topic}{koszul-algebra}{Koszul algebra}
    A \textbf{Koszul algebra} $A$ over a \tref{field}{field} $k$ is a \tref{graded-ring}{graded} \tref{algebra}{$k$-algebra}, such that the ground field $k$ has a linear minimal graded free resolution, that is, there exists an exact sequence
    \[ \cdots \to A(-i)^{b_i} \to \cdots \to A(-2)^{b_2} \to A(-1)^{b_1} \to A \to k \to 0 , \]
    for some $b_i \ge 0$.
\end{topic}

\begin{example}{koszul-algebra}
    \begin{itemize}
        \item The polynomial ring $R = k[x_1, \ldots, x_n]$ is a Koszul algebra over $k$, as a desired resolution is given by the \tref{koszul-complex}{Koszul complex}
        \[ 0 \to R(-n) \to \cdots \to R(-i)^{\binom{n}{i}} \to \cdots \to R(-1)^n \to R \to k \to 0 . \]
        \item The ring $R = k[x, y]/(xy)$ is a Koszul algebra over $k$, even though a resolution of $k$
        \[ \cdots \to R^2 \xrightarrow{\begin{pmatrix} y & 0 \\ 0 & x \end{pmatrix}} R^2 \xrightarrow{\begin{pmatrix} x & 0 \\ 0 & y \end{pmatrix}} R^2 \xrightarrow{\begin{pmatrix} y & 0 \\ 0 & x \end{pmatrix}} R^2 \xrightarrow{\begin{pmatrix} x & y \end{pmatrix}} R \to k \to 0 \]
        has infinite length.
        \item The ring $R = k[x] / (x^3)$ is not a Koszul algebra over $k$. Namely, when trying to construct a resolution, we obtain
        \[ R \xrightarrow{x^2 \cdot} R \xrightarrow{x \cdot} R \to k \to 0 , \]
        but $x^2$ is not a linear term.
    \end{itemize}
\end{example}

\begin{topic}{kac-moody-algebra}{Kac--Moody algebra}
    Let $A = (a_{ij})$ be an $n \times n$ \tref{LA:cartan-matrix}{generalized Cartan matrix}. The \textbf{Kac--Moody algebra} of $A$ is the \tref{lie-algebra}{Lie algebra} $\mathfrak{g}(A)$ over $\CC$, defined by generators $H_i, X_i, Y_i$ for $i = 1, \ldots, n$ and relations
    \[ [H_i, H_j] = 0, \quad [H_i, X_j] = a_{ij} X_j, \quad [H_i, Y_j] = -a_{ij} Y_j, \quad [X_i, Y_j] = \delta_{ij} H_i , \textup{ for all } i, j \]
    \[ \textup{ and } \left(\textup{ad}_{X_i}\right)^{1 - a_{ij}}(X_j) = 0, \quad \left(\textup{ad}_{Y_i}\right)^{1 - a_{ij}}(Y_j) = 0, \textup{ for } i \ne j , \]
    where $\textup{ad}_X(Y) = [X, Y]$ denotes the \tref{DG:adjoint-representation}{adjoint representation}.
\end{topic}

\begin{topic}{hochschild-homology}{Hochschild (co)homology}
    Let $k$ be a \tref{field}{field}, $A$ an \tref{algebra}{$k$-algebra}, and $M$ a \tref{bimodule}{bimodule} over $A$. Let $C_n(A, M) := M \otimes A^{\otimes n}$, and define boundary maps $d_{n, i} : C_n(A, M) \to C_{n - 1}(A, M)$ for $0 \le i \le n$ by
    \[ d_{n, i}(m \otimes a_1 \otimes \cdots \otimes a_n) = \left\{ \begin{array}{cl}
        m a_1 \otimes a_2 \otimes \cdots \otimes a_n & \textup{ if } i = 0, \\
         m \otimes a_1 \otimes \cdots \otimes a_i a_{i + 1} \otimes \cdots \otimes a_n & \textup{ if } 1 < i < n , \\
        a_n m \otimes a_1 \otimes \cdots \otimes a_{n - 1} & \textup{ if } i = n .
    \end{array} \right. \]
    Then $C_n(A, M)$ with differential $d_n = \sum_{i = 0}^{n} = (-1)^i d_{n, i}$ is a \tref{HA:chain-complex}{chain complex} called the \textbf{Hochschild complex}.
    \[ \cdots \xrightarrow{d_{n + 1}} M \otimes A^{\otimes n} \xrightarrow{d_n} M \otimes A^{\otimes n - 1} \xrightarrow{d_{n - 1}} \cdots \xrightarrow{d_2} M \otimes A \xrightarrow{d_1} M \to 0 \]
    The corresponding \tref{HA:homology-object}{homology} $HH_i(A, M)$ is called the \textbf{Hochschild homology} of $A$ with coefficients in $M$.
    
    Similarly, \textbf{Hochschild cohomology} is defined by from the cocomplex
    \[ 0 \to M \xrightarrow{\partial} \Hom_k(A, M) \xrightarrow{\partial} \cdots \xrightarrow{\partial} \Hom_k(A^{\otimes n}, M) \xrightarrow{\partial} \Hom_k(A^{\otimes n + 1}, M) \xrightarrow{\partial} \cdots \]
    where $\partial^n = \sum_{i = 0}^n (-1)^i \partial^{n, i}$ is given by
    \[ (\partial^{n, i} f)(a_1 \otimes \cdots \otimes a_n) = \left\{ \begin{array}{cl}
        a_1 f(a_2 \otimes \cdots \otimes a_n)  & \textup{ if } i = 0, \\
         f(a_1 \otimes \cdots \otimes a_i a_{i + 1} \otimes \cdots \otimes a_n) & \textup{ if } 1 < i < n, \\
        f(a_1 \otimes \cdots \otimes a_{n - 1}) a_n & \textup{ if } i = n .
    \end{array} \right. \]
    The corresponding cohomology $HH^n(A, M)$ is the \textbf{Hochschild cohomology} of $A$ with coefficients in $M$.
\end{topic}

\begin{example}{hochschild-homology}
    Note that $\ker \partial^1$ consists of all $f : A \to M$ for which $\partial f$, given by $(a \otimes b) \mapsto a f(b) - f(ab) + f(a) b$, is zero, that is, $f$ is a \tref{derivation}{$k$-derivation}. Furthermore, $\im \partial^0$ consists of all $f$ such that $f(a) = ma - am$, which are known as \textit{inner derivations}. Therefore, $HH^1(A, M) = \{ \text{derivations} \} / \{ \text{inner derivations} \}$.
\end{example}

\begin{example}{hochschild-homology}
    The Hochschild (co)homology can be expressed in terms of \tref{HA:tor-functors}{Tor} (\tref{HA:ext-functors}{Ext}) as
    \[ HH_i(A, M) \isom \textup{Tor}_i^{A^\textup{e}}(A, M) \quad \text{and} \quad HH^i(A, M) \isom \textup{Ext}^i_{A^\textup{e}}(A, M) , \]
    where $A^\textup{e}$ denotes the \tref{enveloping-algebra}{enveloping algebra} of $A$. Indeed, one can resolving $A$ by free (left) $A^\textup{e}$-modules using the \tref{standard-complex}{bar resolution}. Then, tensoring with $M$ over $A^\textup{e}$ gives a complex isomorphic to the Hochschild complex: in degree $n$ we have an isomorphism
    \[ M \otimes_{A^\textup{e}} A^{\otimes n + 2} \to M \otimes A^{\otimes n}, \quad m \otimes a_0 \otimes \cdots \otimes a_{n + 1} \mapsto a_0 m a_{n + 1} \otimes a_1 \otimes \cdots \otimes a_n . \]
    Taking homology gives the result. Similarly for the cohomology complex: in degree $n$ we have an isomorphism
    \[ \Hom_{A^\textup{e}}(A^{\otimes n + 2}, M) \to \Hom_k(A^{\otimes n}, M), \quad f \mapsto (a_1 \otimes \cdots \otimes a_n \mapsto f(1 \otimes a_1 \otimes \cdots \otimes a_n \otimes 1)) . \]
\end{example}

% \begin{example}{hochschild-homology}
%     Using the \tref{UN:koszul-complex}{Koszul complex}, one can compute the Hochschild homology for the polynomial ring $R = k[x_1, \ldots, x_n]$:
%     \[ HH_i(R) \isom HH^i(R) \isom \left\{ \begin{array}{cl} \bigwedge^i R^n & \text{for } 0 \le i \le n \\ 0 & \text{for } i > n . \end{array} \right. \]
% \end{example}

\begin{example}{hochschild-homology}
    The (first order) deformations of a ring $A$ up to trivial deformations are parametrized by $HH^2(A)$. Namely, any deformed multiplication $a \star b := ab + \varepsilon f(a, b)$ with $f : A \otimes A \to A$ and $\varepsilon^2 = 0$ is associative if and only if $(a \star b) \star c = a \star (b \star c)$ for all $a, b, c \in A$. This translates to
    \[ a f(b, c) - f(ab, c) + f(a, bc) - f(a, b) c = 0 , \]
    that is, $\partial f = 0$. If $f$ is a trivial deformation, then there exists some $k[\varepsilon]$-linear automorphism $\phi$ of $A[\varepsilon]$ such that $\phi(a_1 + \varepsilon a_2) \phi(b_1 + \varepsilon b_2) = \phi((a_1 + \varepsilon a_2) \star (b_1 + \varepsilon b_2))$. Expanding gives $f(a, b) = a \phi_1(b) - \phi(ab) + \phi(a) b = (\partial^1 \phi)(a, b)$ for all $a, b \in A$, that is, $f \in \im \partial$.
\end{example}

\begin{example}{hochschild-homology}
    The Hochschild--Kostant--Rosenberg theorem states that for a smooth \tref{finitely-presented-algebra}{finitely presented} commutative algebra $A$ over a field $k$, there is an isomorphism of graded $k$-algebras
    \[ \Omega^\bdot_{A/k} \isom HH_\bdot(A, A) . \]
    Moreover, for cohomology there is an isomorphism
    \[ \bigwedge^\bdot_A \textup{Der}_k(A, A) \isom HH_\bdot(A, A) . \]
\end{example}

\begin{topic}{enveloping-algebra}{enveloping algebra}
    Let $k$ be a \tref{ring}{commutative ring}. The \textbf{enveloping algebra} of a \tref{algebra}{$k$-algebra} $A$ is the algebra $A \otimes_k A^\textup{op}$, where $A^\textup{op}$ denotes the \tref{opposite-ring}{opposite} algebra.
\end{topic}

\begin{topic}{opposite-ring}{opposite ring}
    The \textbf{opposite ring} of a \tref{ring}{ring} $A$ is the ring $A^\textup{op}$ whose underlying additive group is the same as $A$, but with multiplication given by $a \cdot b = ba$.
\end{topic}

% \begin{topic}{finitely-presented-algebra}{finitely presented algebra}
%     An \tref{algebra}{algebra} $A$ over a \tref{ring}{commutative ring} $R$ is \textbf{finitely presented} if
%     \[ A \isom R[x_1, \ldots, x_n] / (f_1, \ldots, f_k) \]
%     for some $n, k \ge 0$ and $f_1, \ldots, f_k \in R[x_1, \ldots, x_n]$.
% \end{topic}

\begin{topic}{gerstenhaber-algebra}{Gerstenhaber algebra}
    A \textbf{Gerstenhaber algebra} is a \tref{graded-ring}{graded} \tref{algebra}{$k$-algebra} $A = \bigoplus_{n \in \ZZ} A^n$, equipped with a bilinear map $[-, -] : A \times A \to A$ of degree $-1$, such that
    \begin{itemize}
        \item (\textit{graded commutative}) $ab = (-1)^{\deg(a) \deg(b)} ba$ for all homogeneous $a, b \in A$,
        \item (\textit{graded skew-symmetric}) $[a, b] = -(-1)^{(\deg(a) - 1)(\deg(b) - 1)} [b, a]$ for all homogeneous $a, b \in A$,
        % \item (automatic?) $[a, a] = 0$ for all homogeneous $a \in A$ of odd degree,
        % \item (automatic?) $[[a, a], a] = 0$ for all homogeneous $a \in A$ of even degree,
        \item (\textit{graded Jacobi identity}) $[a, [b, c]] = [[a, b], c] + (-1)^{(\deg(a) - 1)(\deg(b) - 1)} [b, [a, c]]$ for all homogeneous $a, b, c \in A$,
        \item (\textit{graded Poisson identity}) $[a, bc] = [a, b]c + (-1)^{(\deg(a) - 1) \deg(b)} b [a, c]$ for all homogeneous $a, b, c \in A$.
    \end{itemize}
\end{topic}

\begin{topic}{noether-bound}{Noether's bound}
    Let $G$ be a finite \tref{GT:group}{group} and $\rho : G \to \textup{GL}(V)$ a finite-dimensional \tref{RT:representation}{representation} over a field $k$. The action of $G$ naturally extends to the polynomial ring $k[x_1, \ldots, x_n]$, where $n = \dim_k V$. \textbf{Noether's bound} states that, if $|G|$ is invertible in $k$, the invariant subring
    \[ k[x_1, \ldots, x_n]^G = \{ f \in k[x_1, \ldots, x_n] \mid g \cdot f = f \textup{ for all } g \in G \} \]
    is \tref{finitely-generated-algebra}{finitely generated} over $k$ by elements of degree at most $|G|$.
\end{topic}

\begin{example}{noether-bound}
    Let $G = \ZZ/2\ZZ$ act on $k^2$ by $(x, y) \mapsto (-x, -y)$. Then,
    \[ k[x, y]^G = k[x^2, xy, y^2] = k[u, v, w] / (w^2 - uv) \]
    is indeed finitely generated over $k$ by elements of degree at most $|G| = 2$.
\end{example}

\begin{topic}{artin-wedderburn-theorem}{Artin--Wedderburn theorem}
    Let $R$ be a semisimple \tref{ring}{ring}, that is, a ring which is \tref{simple-module}{semisimple} as a \tref{module}{(left) module} over itself. The \textbf{Artin--Wedderburn theorem} states that $R$ is isomorphic to a product of finitely many $n_i \times n_i$ matrix rings over \tref{division-ring}{division rings} $D_i$, that is,
    \[ R \isom \prod_{i = 1}^{k} \textup{Mat}_{n_i}(D_i) , \]
    where the $n_i$ and $D_i$ are unique up to permutation.
\end{topic}

\begin{topic}{noether-normalization-lemma}{Noether's normalization lemma}
    Let $A$ be a (non-trivial) \tref{finitely-generated-algebra}{finitely generated algebra} over a \tref{field}{field} $k$. \textbf{Noether's normalization lemma} states that there exists an integer $n \ge 0$ and an injective morphism of $k$-algebras
    \[ k[x_1, \ldots, x_n] \to A , \]
    such that $A$ is a \tref{finitely-generated-module}{finitely generated module} over $k[x_1, \ldots, x_n]$.
\end{topic}

\begin{example}{noether-normalization-lemma}
    \textit{Claim}: For any non-zero polynomial $f \in k[x_1, \ldots, x_n]$ there exist $\lambda \in k$ and $a_1, \ldots, a_{n - 1} \in \NN$ such that $\lambda f(x_1 + x_n^{a_1}, \ldots, x_{n - 1} + x_n^{a_{n - 1}}, x_n)$ is monic in $x_n$.
    
    \textit{Proof of claim}: Choose $0 \le e \le n - 1$ maximally such that $f$ depends non-trivially on $x_1, \ldots, x_e$. If $e = 0$ then $f \in k[x_n]$, so we can choose $a_1 = \ldots = a_{n - 1} = 0$ and appropriate $\lambda$. Otherwise $e > 0$ and by induction we can assume the claim holds for $e - 1$. Write $f = \sum_{i = 1}^{m} g_i x_e^i$ with $g_i \in k[x_1, \ldots, x_{e - 1}, x_n]$. By induction, there exist $\lambda \in k$ and $a_1, \ldots, a_{e - 1} \in \NN$ such that $\lambda g_m(x_1 + x_n^{a_1}, \ldots, x_{e - 1} + x_n^{a_{e - 1}}, x_n)$ is monic in $x_n$. Now choose $a_{e + 1} = \ldots = a_{n - 1} = 0$, and note that
        \[ \lambda f(x_1 + x_n^{a_1}, \ldots, x_{n - 1} + x_n^{a_{n - 1}}, x_n) = \sum_{i = 1}^{m} \lambda g_i(x_1 + x_n^{a_1}, \ldots, x_{e - 1} + x_n^{a_{e - 1}}, x_n) (x_e + x_n^{a_e})^i \]
        becomes monic in $x_n$ for sufficiently large $a_e$.

    \begin{proof}
        Let $y_1, \ldots, y_n$ be generators of $A$ over $k$, and continue by induction on $n$, the case $n = 0$ being trivial. Suppose $n > 0$ and suppose the result holds for $n - 1$. If the $y_i$ are algebraically independent, then $A \isom k[x_1, \ldots, x_n]$, so we are done. Otherwise, there exists some non-zero polynomial $f \in k[x_1, \ldots, x_n]$ such that $f(y_1, \ldots, y_n) = 0$. By the above claim, there exist $\lambda \in k$ and $a_1, \ldots, a_{n - 1} \in \NN$ such that $\lambda f(x_1 + x_n^{a_1}, \ldots, x_{n - 1} + x_n^{a_{n - 1}}, x_n)$ is monic in $x_n$. Hence, $A$ is a finitely generated module over the subalgebra $B \subset A$ generated by $y_1 + y_n^{a_1}, \ldots, y_{n - 1} + y_n^{a_{n - 1}}$. By induction, $B$ is itself a finitely generated module over some subalgebra $k[z_1, \ldots, z_r] \subset B$ for some $r$, and hence so is $A$.
    \end{proof}
\end{example}

\begin{example}{noether-normalization-lemma}
    Let $k$ be a field of characteristic not equal to $2$. The algebra $A = k[x, y] / (xy - 1)$ is finitely generated over $k$, but it is not finitely generated (as module) over $k[x]$ or $k[y]$. However, after a change of coordinates $x = z + w$ and $y = z - w$, we see that $A \isom k[z, w] / (z^2 - w^2 - 1)$ is finitely generated (as module) over $k[z]$ and $k[w]$.
\end{example}

\begin{topic}{huber-ring}{Huber ring}
    An \textit{adic ring} is a topological \tref{ring}{ring} $A$ for which there exists an \tref{ideal}{ideal} $I \subset A$, called the \textit{ideal of definition}, such that $\{ I^n : n \in \NN \}$ is a \tref{TO:neighborhood-basis}{neighborhood basis} of $0$ in $A$.

    A \textbf{Huber ring} is a topological ring $A$ for which there exists an open adic subring $A_0 \subset A$ with finitely generated ideal of definition $I$.
\end{topic}

\begin{topic}{huber-pair}{Huber pair}
    Let $A$ be \tref{huber-ring}{Huber ring}. A subset $S \subset A$ is \textit{bounded} if for every \tref{TO:neighborhood}{neighborhood} $U$ of $0$ in $A$ there exists a neighborhood $V$ of $0$ with $\{ vs : v \in V, s \in S \} \subset U$. Denote by $A^\circ$ the subset of elements $a \in A$ for which $\{ a \in A \mid \{ a^n : n \ge 1 \}$ is bounded.

    A \textbf{Huber pair} is a pair $(A, A^+)$ where $A$ is a Huber ring and $A^+ \subset A^\circ$ is an open subring of $A$ that is \tref{integral-closure}{integrally closed} in $A$.

    A morphism of Huber pairs $(A, A^+) \to (B, B^+)$ is a continuous ring morphism $f : A \to B$ such that $f(A^+) \subset B^+$.
\end{topic}

