\begin{topic}{monoid}{monoid}
    A \textbf{monoid} is a set $M$ with a binary operation $\cdot : M \times M \to M$ satisfying
    \begin{itemize}
        \item (\textit{associative}) $(a \cdot b) \cdot c = a \cdot (b \cdot c)$ for all $a, b, c \in M$,
        \item (\textit{identity element}) there exists an element $e \in M$ such that $e \cdot a = a = a \cdot e$ for all $a \in M$.
    \end{itemize}
\end{topic}

\begin{topic}{semigroup}{semigroup}
    A \textbf{semigroup} is a set $S$ with a binary operation $\cdot : S \times S \to S$ which is associative, that is, $(a \cdot b) \cdot c = a \cdot (b \cdot c)$ for all $a, b, c \in S$.
\end{topic}

\begin{topic}{semiring}{semiring}
    A \textbf{semiring} is a set $R$ with two binary operations $+ : R \times R \to R$, called addition, satisfying
    \begin{itemize}
        \item (\textit{associative}) $(a + b) + c = a + (b + c)$ for all $a, b, c \in R$,
        \item (\textit{commutative}) $a + b = b + a$ for all $a, b \in R$,
        \item (\textit{zero element}) there exists an element $0 \in R$ such that $a + 0 = 0 + a = a$ for all $a \in R$,
    \end{itemize}
    and $\cdot : R \times R \to R$, called multiplication, satisfying
    \begin{itemize}
        \item (\textit{associative}) $(a \cdot b) \cdot c = a \cdot (b \cdot b)$ for all $a, b, c \in R$,
        \item (\textit{unit element}) there exists an element $1 \in R$ such that $a \cdot 1 = 1 \cdot a = a$ for all $a \in R$,
        \item (\textit{distributive}) $a \cdot (b + c) = (a \cdot b) + (a \cdot c)$ and $(a + b) \cdot c = (a \cdot c) + (b \cdot c)$ for all $a, b, c \in R$,
        \item (\textit{zero element}) $0 \cdot a = a \cdot 0 = 0$ for all $a \in R$.
    \end{itemize}
\end{topic}

\begin{example}{semiring}
    The natural numbers $\NN$ (including zero) form a semiring under usual addition and multiplication. Similarly, $\QQ_{\ge 0}$ and $\RR_{\ge 0}$ are semirings.
\end{example}

% \begin{topic}{universal-enveloping-algebra}{universal enveloping algebra}
%     The \textbf{universal enveloping algebra} of a \tref{lie-algebra}{Lie algebra} $\mathfrak{g}$ is defined as follows. The Lie bracket $[\cdot, \cdot]$ lifts to the \tref{CA:tensor-algebra}{tensor algebra} $T(\mathfrak{g})$.
    
%     Now the universal enveloping algebra of $\mathfrak{g}$ is defined as the quotient
%     \[ U(\mathfrak{g}) = T(\mathfrak{g}) / (a \otimes b - b \otimes a = [a, b]) . \]
% \end{topic}

\begin{topic}{bialgebra}{bialgebra}
    A \textbf{bialgebra} over a \tref{CA:field}{field} $k$ is a \tref{algebra}{$k$-algebra} with morphisms of algebras $\Delta : A \to A \otimes A$ (\textit{comultiplication}) and $\varepsilon : A \to k$ (\textit{counit}). (TODO)
\end{topic}

\begin{topic}{hopf-algebra}{Hopf algebra}
    A \textbf{Hopf algebra} over a \tref{CA:field}{field} $k$ is a \tref{bialgebra}{$k$-bialgebra} $A$ with a $k$-linear map $S : A \to A$ (the \textit{antipode}), such that the diagram
    \[ \begin{tikzcd}[column sep=1em, row sep=2em] & A \otimes A \arrow{rr}{S \otimes \id} && A \otimes A \arrow{dr}{\mu} & \\ A \arrow{rr}{\varepsilon} \arrow{ur}{\Delta} \arrow[swap]{dr}{\Delta} && k \arrow{rr}{\eta} && A \\ & A \otimes A \arrow[swap]{rr}{\id \otimes S} && A \otimes A \arrow[swap]{ur}{\mu} &  \end{tikzcd} \]
    commutes, where $\mu$ denotes the multiplication, $\Delta$ the comultiplication, $\eta$ the unit, and $\varepsilon$ the counit of the algebra.
\end{topic}

\begin{example}{hopf-algebra}
    The category of commutative \tref{CA:finitely-generated-algebra}{finitely generated} Hopf algebras over $k$ is \tref{CT:equivalence-of-categories}{equivalent} to the category of \tref{AG:affine-scheme}{affine} \tref{AG:algebraic-group}{algebraic groups} over $k$, via the functors
    \[ A \mapsto \Spec A, \quad \text{and} \quad G \mapsto \Gamma(G, \mathcal{O}_G) . \]
    The antipode corresponds to group inversion.
\end{example}
