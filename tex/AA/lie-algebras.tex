\begin{topic}{lie-algebra}{Lie algebra}
    A \textbf{Lie algebra} is is a \tref{LA:vector-space}{vector space} $\mathfrak{g}$ over a field $k$, together with an operation $\mathfrak{g} \times \mathfrak{g} \to \mathfrak{g}, (x, y) \mapsto [x, y]$ called the \textbf{Lie bracket}, satisfying
    \begin{itemize}
        \item (\textit{bilinearity}) $[ax + by, z] = a[x, z] + b[y, z]$ and $[z, ax + by] = a[z, x] + b[z, y]$ for all $x, y, z \in \mathfrak{g}$ and $a, b \in k$,
        \item (\textit{alternativity}) $[x, x] = 0$ for all $x \in \mathfrak{g}$,
        \item (\textit{Jacobi identity}) $[x, [y, z]] + [z, [x, y]] + [y, [z, x]] = 0$ for all $x, y, z \in \mathfrak{g}$.
    \end{itemize}
\end{topic}

\begin{example}{lie-algebra}
    Consider $\mathfrak{g} = \RR^3$ with bracket operation defined by the \textit{cross product} $[x, y] = x \times y$.
\end{example}

\begin{example}{lie-algebra}
    For any \tref{DG:lie-group}{Lie group} $G$, the tangent space at the identity $\mathfrak{g} = T_e G$ has a Lie algebra structure, called the \textit{Lie algebra} of $G$. Tangent vectors $v \in T_e G$ correspond one-to-one to \textit{left invariant vector fields} on $G$, i.e. vector fields $X$ satisfying $X_g = ((L_g)_* X)_e$ for all $g \in G$, where $L_g : G \to G$ denotes left multiplication by $g$. The Lie bracket on $\mathfrak{g}$ is then given by the \tref{DG:lie-bracket-vector-fields}{Lie bracket of vector fields}. Note that the Lie bracket of left-invariant vector fields is indeed also left-invariant.
\end{example}

\begin{example}{lie-algebra}
    \begin{itemize}
        \item $\mathfrak{sl}_n(k) = \{ X \in \textup{Mat}_{n \times n}(k) : \tr X = 0 \}$ is the Lie algebra associated to the \tref{LA:special-linear-group}{special linear group}.
        \item $\mathfrak{so}_n = \{ X \in \textup{Mat}_{n \times n}(\RR) : X + X^T = 0 \}$ is the Lie algebra associated to the \tref{LA:orthogonal-group}{(special) orthogonal group}.
        \item $\mathfrak{u}_n = \{ A \in \textup{Mat}_{n \times n}(\CC) : X + X^H = 0 \}$ is the Lie algebra associated to the \tref{LA:unitary-group}{special unitary group}.
        \item $\mathfrak{su}_n = \{ A \in \textup{Mat}_{n \times n}(\CC) : X + X^H = 0 \textup{ and } \tr X = 0 \}$ is the Lie algebra associated to the \tref{LA:unitary-group}{special unitary group}.
        \item $\mathfrak{sp}_{2n}(k) = \{ A \in \textup{Mat}_{2n \times 2n}(k) : \Omega X + X^T \Omega = 0 \}$ is the Lie algebra associated to the \tref{LA:symplectic-group}{symplectic group}.
    \end{itemize}
\end{example}

\begin{example}{lie-algebra}
    For any \tref{CA:ring}{ring} $A$ over a field $k$ (in particular matrix rings), the commutator bracket $[x, y] = xy - yx$ makes $A$ into a Lie algebra.
\end{example}

\begin{topic}{lie-algebra-ideal}{Lie algebra ideal}
    An \textbf{ideal} of a \tref{lie-algebra}{Lie algebra} $\mathfrak{g}$ is a subalgebra $\mathfrak{h} \subset \mathfrak{g}$ such that $[X, H] \in \mathfrak{h}$ for all $X \in \mathfrak{g}$ and $H \in \mathfrak{h}$.
\end{topic}

\begin{example}{lie-algebra-ideal}
    For any Lie algebra $\mathfrak{g}$, the \textit{commutator ideal} $[\mathfrak{g}, \mathfrak{g}]$ is an ideal of $\mathfrak{g}$, given as the set of all elements in $\mathfrak{g}$ of the form
    \[ c_1 [X_1, Y_1] + \cdots + c_m [X_m, Y_m] , \]
    for some constants $c_i$ and vectors $X_i, Y_i \in \mathfrak{g}$.
\end{example}

\begin{topic}{ado-theore}{Ado's theorem}
    \textbf{Ado's theorem} states that any \tref{lie-algebra}{Lie algebra} over a field of characteristic zero is isomorphic to a subalgebra of $\mathfrak{gl}_n(k)$, the Lie algebra of square matrices with the commutator bracket.
\end{topic}

\begin{topic}{nilpotent-lie-algebra}{nilpotent Lie algebra}
    A \tref{lie-algebra}{Lie algebra} $\mathfrak{g}$ is called \textbf{nilpotent} if its lower central series
    \[ \mathfrak{g} \supset [\mathfrak{g}, \mathfrak{g}] \supset [\mathfrak{g}, [\mathfrak{g}, \mathfrak{g}]] \supset [\mathfrak{g}, [\mathfrak{g}, [\mathfrak{g}, \mathfrak{g}]]] \supset \cdots \]
    terminates in the zero algebra.
\end{topic}

\begin{example}{nilpotent-lie-algebra}
    \begin{itemize}
        \item The Lie algebra given by
        \[ \mathfrak{g} = \left\{ \begin{pmatrix} 0 & x & y \\ 0 & 0 & z \\ 0 & 0 & 0 \end{pmatrix} \right\} , \]
        and the commutator bracket, is nilpotent. Namely, $[\mathfrak{g}, \mathfrak{g}]$ is the span of $\begin{pmatrix} 0 & 0 & 1 \\ 0 & 0 & 0 \\ 0 & 0 & 0 \end{pmatrix}$, and $[\mathfrak{g}, [\mathfrak{g}, \mathfrak{g}]] = \{ 0 \}$.
        
        \item The Lie algebra $\mathfrak{sl}_2(\CC)$ is not nilpotent since $[\mathfrak{sl}_2(\CC), \mathfrak{sl}_2(\CC)] = \mathfrak{sl}_2(\CC)$.
    \end{itemize}
\end{example}

\begin{topic}{solvable-lie-algebra}{solvable Lie algebra}
    A \tref{lie-algebra}{Lie algebra} $\mathfrak{g}$ is \textbf{solvable} if the \textit{derived series} of subalgebras
    \[ \mathfrak{g} = \mathfrak{g}_0 \supset \mathfrak{g}_1 \supset \mathfrak{g}_2 \supset \cdots \]
    given by $\mathfrak{g}_{i + 1} = [\mathfrak{g}_i, \mathfrak{g}_i]$, results in the zero algebra. That is, $\mathfrak{g}_i = \{ 0 \}$ for some $i > 0$.
\end{topic}

\begin{example}{solvable-lie-algebra}
    Every \tref{nilpotent-lie-algebra}{nilpotent} Lie algebra is solvable, but the converse does not hold. Namely, consider the Lie algebra
    \[ \mathfrak{g} = \left\{ \begin{pmatrix} x & y \\ 0 & z \end{pmatrix} \right\} , \]
    with the commutator bracket. Then
    \[ \left[ \begin{pmatrix} x & y \\ 0 & z \end{pmatrix}, \begin{pmatrix} a & b \\ 0 & c \end{pmatrix} \right] = \begin{pmatrix} 0 & w \\ 0 & 0 \end{pmatrix} \]
    with $w = xb + yc - ya - zb$, which shows that $\mathfrak{g}_1 = [\mathfrak{g}, \mathfrak{g}]$ is one-dimensional, and thus $\mathfrak{g}_2 = [\mathfrak{g}_1, \mathfrak{g}_1] = \{ 0 \}$, so $\mathfrak{g}$ is solvable. However, since
    \[ \left[ \begin{pmatrix} 1 & 0 \\ 0 & -1 \end{pmatrix}, \begin{pmatrix} 0 & 1 \\ 0 & 0 \end{pmatrix} \right] = \begin{pmatrix} 0 & 2 \\ 0 & 0 \end{pmatrix} , \]
    it follows that $[\mathfrak{g}, [\mathfrak{g}, \mathfrak{g}]] = [\mathfrak{g}, \mathfrak{g}]$, so the lower central series does not terminate in the zero algebra, and thus $\mathfrak{g}$ is not nilpotent.
\end{example}

\begin{topic}{cartan-subalgebra}{Cartan subalgebra}
    A \textbf{Cartan subalgebra} of a \tref{lie-algebra}{Lie algebra} $\mathfrak{g}$ is a \tref{nilpotent-lie-algebra}{nilpotent} subalgebra $\mathfrak{h}$ whose \textit{normalizer} $N_\mathfrak{g}(\mathfrak{h}) := \{ X \in \mathfrak{g} : [X, \mathfrak{h}] \subset \mathfrak{h} \}$ equals $\mathfrak{h}$.
\end{topic}

\begin{example}{cartan-subalgebra}
    The Cartan subalgebra of $\mathfrak{gl}_n$ is the subalgebra of diagonal matrices.
    
    The Cartan subalgebra of $\mathfrak{sl}_n$ is the subalgebra of diagonal matrices with trace zero.
\end{example}

\begin{topic}{universal-enveloping-algebra}{universal enveloping algebra}
    Let $\mathfrak{g}$ be a \tref{lie-algebra}{Lie algebra}. The \textbf{universal enveloping algebra} of $\mathfrak{g}$ is the algebra
    \[ U(\mathfrak{g}) = T(\mathfrak{g}) / I , \]
    where $T(\mathfrak{g})$ is the \tref{CA:tensor-algebra}{tensor algebra} of $\mathfrak{g}$, and $I$ is the (two-sided) ideal generated by elements of the form $[a, b] - a \otimes b - b \otimes a$ for all $a, b \in \mathfrak{g}$.
    
    It has the universal property that for any \tref{CA:algebra}{algebra} $A$ and Lie algebra morphism $\varphi : \mathfrak{g} \to A$, where the Lie bracket on $A$ is given by the commutator, there exists a unique algebra morphism $U(\mathfrak{g}) \to A$ such that
    \[ \begin{tikzcd} \mathfrak{g} \arrow{r} \arrow[swap]{dr}{\varphi} & U(\mathfrak{g}) \arrow[dashed]{d} \\ & A \end{tikzcd} \]
    commutes.
\end{topic}

\begin{example}{universal-enveloping-algebra}
    The Lie algebra $\mathfrak{g} = \mathfrak{sl}_2$ has a basis
    \[ H = \begin{pmatrix} 1 & 0 \\ 0 & -1 \end{pmatrix}, \quad X = \begin{pmatrix} 0 & 1 \\ 0 & 0 \end{pmatrix}, \quad Y = \begin{pmatrix} 0 & 0 \\ 1 & 0 \end{pmatrix} , \]
    satisfying the relations
    \[ [H, X] = 2X, \quad [H, Y] = -2Y, \quad [X, Y] = H . \]
    This shows that the universal enveloping algebra of $\mathfrak{sl}_2$ is given by
    \[ U(\mathfrak{sl}_2) = \CC \langle x, y, z \rangle / (zx - xz - 2x, zy - yz + 2y, xy - yz - z) . \]
\end{example}

\begin{example}{universal-enveloping-algebra}
    The universal enveloping algebra $U(\mathfrak{g})$ naturally has the structure of a \tref{hopf-algebra}{Hopf algebra}, where comultiplication, counit and the antipode are given by
    \[ \Delta(x) = x \otimes 1 + 1 \otimes x, \qquad \varepsilon(x) = 0 , \qquad S(x) = -x , \]
    for all $x \in \mathfrak{g}$. Note that $\Delta : U(\mathfrak{g}) \to U(\mathfrak{g}) \otimes U(\mathfrak{g})$ is well-defined since
    \[ \begin{aligned}
        \Delta([x, y])
            &= [x, y] \otimes 1 + 1 \otimes [x, y] \\
            &= xy \otimes 1 - yx \otimes 1 + 1 \otimes xy - 1 \otimes yx \\
            &= xy \otimes 1 - x \otimes y - y \otimes x + 1 \otimes xy - yx \otimes 1 + y \otimes x + x \otimes y - 1 \otimes yx \\
            &= (x \otimes 1 + 1 \otimes x)(y \otimes 1 + 1 \otimes y) - (y \otimes 1 + 1 \otimes y)(x \otimes 1 + 1 \otimes x) \\
            &= \Delta(x) \Delta(y) - \Delta(y) \Delta(x) \\
            &= \Delta(xy - yx)
    \end{aligned} \]
    for all $x, y \in \mathfrak{g}$.
\end{example}

\begin{topic}{poisson-algebra}{Poisson algebra}
    A \textbf{Poisson algebra} is a commutative \tref{CA:algebra}{algebra} $A$ over a \tref{CA:field}{field} $k$, together with an operation $\{ \cdot, \cdot \} : A \otimes_k A \to A$, called the \textbf{Poisson bracket}, making $A$ into a \tref{lie-algebra}{Lie algebra} and satisfying the \textit{Leibniz rule} $\{ f, gh \} = \{ f, g \} h + g \{ f, h \}$. % $\{ a, - \} : A \to A$ is a \tref{CA:derivation}{$k$-derivation} for every $a \in A$.
\end{topic}

\begin{topic}{simple-lie-algebra}{(semi)simple Lie algebra}
    A \tref{lie-algebra}{Lie algebra} is \textbf{simple} if it is non-commutative and contains no non-zero proper \tref{lie-algebra-ideal}{ideals}.
    
    A Lie algebra is \textbf{semisimple} if it is the direct sum of simple Lie algebras.
\end{topic}

\begin{example}{simple-lie-algebra}
    The Lie algebra $\mathfrak{sl}_2(\CC)$ is simple. A basis for $\mathfrak{sl}_2(\CC)$is given by
    \[ H = \begin{pmatrix} 1 & 0 \\ 0 & -1 \end{pmatrix}, \quad X = \begin{pmatrix} 0 & 1 \\ 0 & 0 \end{pmatrix}, \quad Y = \begin{pmatrix} 0 & 0 \\ 1 & 0 \end{pmatrix} , \]
    satisfying the relations
    \[ [H, X] = 2X, \quad [H, Y] = -2Y, \quad [X, Y] = H . \]
    Let $\mathfrak{h}$ be a non-zero ideal of $\mathfrak{sl}_2(\CC)$. Then $\mathfrak{h}$ contains an element $Z = aX + bH + cY$ where $a, b, c \in \CC$ are not all zero. If $c \ne 0$, then $[X, [X, Z]] = -2cX$, so $X \in \mathfrak{h}$, and also $[Y, X] = -H \in \mathfrak{h}$ and hence also $[Y, H] = 2Y \in \mathfrak{h}$, which shows $\mathfrak{h} = \textup{sl}_2(\CC)$. If $c = 0$ and $b \ne 0$, then $[X, Z] = -2bX$, so $X \in \mathfrak{h}$ and similarly $\mathfrak{h} = \mathfrak{sl}_2(\CC)$. Finally, if $b = c = 0$ and $a \ne 0$, then $Z = aX$, so $X \in \mathfrak{h}$, so $h = \mathfrak{sl}_2(\CC)$.
\end{example}

\begin{topic}{root-system-lie-algebra}{root system Lie algebra}
    Let $\mathfrak{g}$ be a complex \tref{simple-lie-algebra}{semisimple} \tref{lie-algebra}{Lie algebra}, and $\mathfrak{h}$ a \tref{cartan-subalgebra}{Cartan subalgebra}. A \textbf{root} of $\mathfrak{g}$ (relative to $\mathfrak{h}$) is an element $\lambda \in \mathfrak{h}^*$ such that
    \[ \mathfrak{g}^\lambda := \{ x \in \mathfrak{g} : [h, x] = \lambda(h) x \textup{ for all } h \in \mathfrak{h} \} \]
    is non-empty. The \textbf{root system} of $\mathfrak{g}$ (relative to $\mathfrak{h}$) is the set of all roots, often denoted $\Phi$.
\end{topic}

\begin{topic}{killing-form}{Killing form}
    Let $\mathfrak{g}$ be a \tref{lie-algebra}{Lie algebra} of finite dimension over a field $k$. The \textbf{Killing form} of $\mathfrak{g}$ is the symmetric bilinear form
    \[ B : \mathfrak{g} \times \mathfrak{g} \to k, \quad B(x, y) = \textup{tr}(\textup{ad}_{\mathfrak{g}}(x) \circ \textup{ad}_{\mathfrak{g}}(y)) \]
    where $\text{ad}_\mathfrak{g} : \mathfrak{g} \to \mathfrak{gl}(\mathfrak{g})$ is the \tref{DG:adjoint-representation}{adjoint representation} of $\mathfrak{g}$.
\end{topic}
