\begin{topic}{lie-algebra}{Lie algebra}
    A \textbf{Lie algebra} is is a \tref{LA:vector-space}{vector space} $\mathfrak{g}$ over a field $k$, together with an operation $\mathfrak{g} \times \mathfrak{g} \to \mathfrak{g}, (x, y) \mapsto [x, y]$ called the \textbf{Lie bracket}, satisfying
    \begin{itemize}
        \item (\textit{bilinearity}) $[ax + by, z] = a[x, z] + b[y, z]$ and $[z, ax + by] = a[z, x] + b[z, y]$ for all $x, y, z \in \mathfrak{g}$ and $a, b \in k$,
        \item (\textit{alternativity}) $[x, x] = 0$ for all $x \in \mathfrak{g}$,
        \item (\textit{Jacobi identity}) $[x, [y, z]] + [z, [x, y]] + [y, [z, x]] = 0$ for all $x, y, z \in \mathfrak{g}$.
    \end{itemize}
\end{topic}

\begin{example}{lie-algebra}
    Consider $\mathfrak{g} = \RR^3$ with bracket operation defined by the \textit{cross product} $[x, y] = x \times y$.
\end{example}

\begin{example}{lie-algebra}
    For any \tref{DG:lie-group}{Lie group} $G$, the tangent space at the identity $\mathfrak{g} = T_e G$ has a Lie algebra structure, called the \textit{Lie algebra} of $G$. Tangent vectors $v \in T_e G$ correspond one-to-one to \textit{left invariant vector fields} on $G$, i.e. vector fields $X$ satisfying $X_g = ((L_g)_* X)_e$ for all $g \in G$, where $L_g : G \to G$ denotes left multiplication by $g$. The Lie bracket on $\mathfrak{g}$ is then given by the \tref{DG:lie-bracket-vector-fields}{Lie bracket of vector fields}. Note that the Lie bracket of left-invariant vector fields is indeed also left-invariant.
\end{example}

\begin{example}{lie-algebra}
    For any \tref{CA:ring}{ring} $R$ over a field $k$ (in particular matrix rings), the commutator bracket $[x, y] = xy - yx$ makes $R$ into a Lie algebra.
\end{example}

\begin{topic}{ado-theore}{Ado's theorem}
    \textbf{Ado's theorem} states that any \tref{lie-algebra}{Lie algebra} over a field of characteristic zero is isomorphic to a subalgebra of $\mathfrak{gl}_n(k)$, the Lie algebra of square matrices with the commutator bracket.
\end{topic}

\begin{topic}{nilpotent-lie-algebra}{nilpotent Lie algebra}
    A \tref{lie-algebra}{Lie algebra} $\mathfrak{g}$ is called \textbf{nilpotent} if its lower central series
    \[ \mathfrak{g} \supset [\mathfrak{g}, \mathfrak{g}] \supset [\mathfrak{g}, [\mathfrak{g}, \mathfrak{g}]] \supset [\mathfrak{g}, [\mathfrak{g}, [\mathfrak{g}, \mathfrak{g}]]] \supset \cdots \]
    terminates in the zero algebra.
\end{topic}

\begin{topic}{cartan-subalgebra}{Cartan subalgebra}
    A \textbf{Cartan subalgebra} of a \tref{lie-algebra}{Lie algebra} $\mathfrak{g}$ is a \tref{nilpotent-lie-algebra}{nilpotent} subalgebra $\mathfrak{h}$ whose \textit{normalizer} $N_\mathfrak{g}(\mathfrak{h}) := \{ X \in \mathfrak{g} : [X, \mathfrak{h}] \subset \mathfrak{h} \}$ equals $\mathfrak{h}$.
\end{topic}

\begin{example}{cartan-subalgebra}
    The Cartan subalgebra of $\mathfrak{gl}_n$ is the subalgebra of diagonal matrices.
    
    The Cartan subalgebra of $\mathfrak{sl}_n$ is the subalgebra of diagonal matrices with trace zero.
\end{example}

\begin{topic}{universal-enveloping-algebra}{universal enveloping algebra}
    Let $\mathfrak{g}$ be a \tref{lie-algebra}{Lie algebra}. The \textbf{universal enveloping algebra} of $\mathfrak{g}$ is the algebra
    \[ U(\mathfrak{g}) = T(\mathfrak{g}) / I , \]
    where $T(\mathfrak{g})$ is the \tref{CA:tensor-algebra}{tensor algebra} of $\mathfrak{g}$, and $I$ is the (two-sided) ideal generated by elements of the form $[a, b] - a \otimes b - b \otimes a$ for all $a, b \in \mathfrak{g}$.
    
    It has the universal property that for any \tref{CA:algebra}{algebra} $A$ and Lie algebra morphism $\varphi : \mathfrak{g} \to A$, where the Lie bracket on $A$ is given by the commutator, there exists a unique algebra morphism $U(\mathfrak{g}) \to A$ such that
    \[ \begin{tikzcd} \mathfrak{g} \arrow{r} \arrow[swap]{dr}{\varphi} & U(\mathfrak{g}) \arrow[dashed]{d} \\ & A \end{tikzcd} \]
    commutes.
\end{topic}

\begin{example}{universal-enveloping-algebra}
    The Lie algebra $\mathfrak{g} = \mathfrak{sl}_2$ has a basis
    \[ H = \begin{pmatrix} -1 & 0 \\ 0 & 1 \end{pmatrix}, \quad X = \begin{pmatrix} 0 & 1 \\ 0 & 0 \end{pmatrix}, \quad Y = \begin{pmatrix} 0 & 0 \\ 1 & 0 \end{pmatrix} , \]
    satisfying the relations
    \[ [H, X] = -2X, \quad [H, Y] = -2Y, \quad [X, Y] = -H . \]
    This shows that the universal enveloping algebra of $\mathfrak{sl}_2$ is given by
    \[ U(\mathfrak{sl}_2) = \CC \langle x, y, z \rangle / (zx - xz + 2x, zy - yz + 2y, xy - yz + z) . \]
\end{example}

\begin{topic}{poisson-algebra}{Poisson algebra}
    A \textbf{Poisson algebra} is a commutative \tref{CA:algebra}{algebra} $A$ over a \tref{CA:field}{field} $k$, together with an operation $\{ \cdot, \cdot \} : A \otimes_k A \to A$, called the \textbf{Poisson bracket}, making $A$ into a \tref{lie-algebra}{Lie algebra} and satisfying the \textit{Leibniz rule} $\{ f, gh \} = \{ f, g \} h + g \{ f, h \}$. % $\{ a, - \} : A \to A$ is a \tref{CA:derivation}{$k$-derivation} for every $a \in A$.
\end{topic}

\begin{topic}{simple-lie-algebra}{(semi)simple Lie algebra}
    A \tref{lie-algebra}{Lie algebra} $\mathfrak{g}$ is \textbf{simple} if it is non-abelian and contains no non-zero proper ideals.
    
    A Lie algebra is \textbf{semisimple} if it is the direct sum of simple Lie algebras.
\end{topic}

\begin{topic}{root-system-lie-algebra}{root system Lie algebra}
    Let $\mathfrak{g}$ be a complex \tref{simple-lie-algebra}{semisimple} \tref{lie-algebra}{Lie algebra}, and $\mathfrak{h}$ a \tref{cartan-subalgebra}{Cartan subalgebra}. A \textbf{root} of $\mathfrak{g}$ (relative to $\mathfrak{h}$) is an element $\lambda \in \mathfrak{h}^*$ such that
    \[ \mathfrak{g}^\lambda := \{ x \in \mathfrak{g} : [h, x] = \lambda(h) x \textup{ for all } h \in \mathfrak{h} \} \]
    is non-empty. The \textbf{root system} of $\mathfrak{g}$ (relative to $\mathfrak{h}$) is the set of all roots, often denoted $\Phi$.
\end{topic}

\begin{topic}{killing-form}{Killing form}
    Let $\mathfrak{g}$ be a \tref{lie-algebra}{Lie algebra} of finite dimension over a field $k$. The \textbf{Killing form} of $\mathfrak{g}$ is the symmetric bilinear form
    \[ B : \mathfrak{g} \times \mathfrak{g} \to k, \quad B(x, y) = \textup{tr}(\textup{ad}_{\mathfrak{g}}(x) \circ \textup{ad}_{\mathfrak{g}}(y)) \]
    where $\text{ad}_\mathfrak{g} : \mathfrak{g} \to \mathfrak{gl}(\mathfrak{g})$ is the \tref{DG:adjoint-representation}{adjoint representation} of $\mathfrak{g}$.
\end{topic}
