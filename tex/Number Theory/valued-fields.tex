\begin{topic}{valuation}{valuation}
    A \textbf{valuation} on a \tref{AA:field}{field} $K$ is a function $\phi : K \to \RR_{\ge 0}$ satisfying
    \begin{itemize}
        \item $\phi(x) = 0$ if and only if $x = 0$,
        \item $\phi(xy) = \phi(x) \phi(y)$ for $x, y \in K$,
        \item there exists a constant $C > 0$ such that $\phi(x + y) \le C \max \{ \phi(x), \phi(y) \}$ for all $x, y \in K$.
    \end{itemize}
    The smallest possible constant $C$, called the \textbf{norm} $\norm{\phi}$ of $\phi$, is given by
    \[ \norm{\phi} = \sup \{ \phi(1 + x) : x \in K \textup{ s.t. } \phi(x) \le 1 \} . \]
\end{topic}

\begin{example}{valuation}
    For $K$ equal to $\CC$ or a subfield of it such as $\RR$ or $\QQ$, the absolute value $\phi = |\cdot|$ defines a valuation. Its norm is $2$.
\end{example}

\begin{example}{valuation}
    Let $K$ be a \tref{number-field}{number field}, and $\mathcal{O}_K$ its \tref{ring-of-integers}{ring of integers}. For any prime $\mathfrak{p}$ of $\mathcal{O}_K$, there is the valuation
    \[ \phi_\mathfrak{p} : K \to \RR_{\ge 0}, \quad x \mapsto c^{\textup{ord}_\mathfrak{p}(x)} , \]
    for some fixed $c \in (0, 1)$. From
    \[ \textup{ord}_\mathfrak{p}(x + y) \ge \min \{ \textup{ord}_\mathfrak{p}(x), \textup{ord}_\mathfrak{p}(y) \} \]
    follows that $\norm{\phi_\mathfrak{p}} = 1$.
\end{example}

\begin{example}{valuation}
    For an \tref{AA:irreducible-element}{irreducible} polynomial $P$ in the polynomial ring $F[X]$ over an arbitrary field $F$, we have the number $\textup{ord}_P(f) \in \ZZ_{\ge 0}$ of factors $P$ occurring in the factorization of a non-zero polynomial $f \in F[X]$, which is well-defined as $F[X]$ is a \tref{AA:unique-factorization-domain}{unique factorization domain}. It yields the valuation on the field of fractions $F(X)$,
    \[ \phi_P : F(X) \to \RR_{\ge 0}, \quad f \mapsto c^{\textup{ord}_P(f)} ,  \]
    for some fixed $c \in (0, 1)$. From
    \[ \textup{ord}_P(f + g) \ge \min \{ \textup{ord}_P(f), \textup{ord}_P(g) \} \]
    follows that $\norm{\phi_P} = 1$.
\end{example}

\begin{example}{valuation}
    If $K = \FF_q$ is a finite field, then every valuation $\phi$ is \textit{trivial}: every non-zero $x \in K$ has finite order, so $\phi(x)^n = \phi(x^n) = \phi(1) = 1$ for some $n \ge 1$, so $\phi(x) = 1$.
\end{example}

\begin{topic}{archimedean-valuation}{(non-)archimedean valuation}
    A \tref{valuation}{valuation} $\phi$ on a \tref{AA:field}{field} $K$ is called \textbf{non-archimedean} if its norm $\norm{\phi}$ equals $1$. Otherwise, the valuation is called \textbf{archimedean}.
\end{topic}

\begin{example}{archimedean-valuation}
    \begin{itemize}
        \item Let $K$ be a number field, and $\sigma : K \to \CC$ an embedding. The valuation coming from the absolute value,
        \[ \phi_\sigma : K \to \RR_{\ge 0}, \quad x \mapsto |\sigma(x)| , \]
        has norm $2$, and is thus archimedean.
        \item Let $K$ be a \tref{number-field}{number field} and $\mathfrak{p}$ a prime ideal of the \tref{ring-of-integers}{ring of integers} $\mathcal{O}_K$. The valuation
        \[ \phi_\mathfrak{p} : K \to \RR_{\ge 0}, \quad x \mapsto c^{\textup{ord}_\mathfrak{p}(x)} , \]
        for some fixed $c \in (0, 1)$, is non-archimedean as $\textup{ord}_\mathfrak{p}(x + y) \ge \min \{ \textup{ord}_\mathfrak{p}(x), \textup{ord}_\mathfrak{p}(y) \}$ for all $x, y \in K$.
        
        In fact, every non-trivial non-archimedean valuation $\phi$ on $K$ is of this form. Namely, let $A = \{ x \in K \mid \phi(x) \le 1 \}$ be the valuation ring of $\phi$, and $\mathfrak{m} = \{ x \in K \mid \phi(x) < 1 \}$ its maximal ideal. Then $\mathcal{O}_K \subset A$ as any $x \in \mathcal{O}_K$ satisfies an equation $x^n = \sum_{i = 0}^{n - 1} a_i x^i$ with $a_i \in \ZZ \subset A$, so
        \[ \phi(x^n) \le \max_{1 \le i \le n - 1} \phi(a_i x^i) \le \max_{1 \le i \le n - 1} \phi(x)^i , \]
        implying $\phi(x) \le 1$. We obtain a prime ideal $\mathfrak{p} = \mathfrak{m} \cap \mathcal{O}_K$ of $\mathcal{O}_K$, and the localization $\mathcal{O}_{K, \mathfrak{p}}$ is a discrete valuation ring. Picking a uniformizer $\pi \in \mathcal{O}_K$ with $\operatorname{ord}_\mathfrak{p}(\pi) = 1$, we can write any $x \in K^*$ as $x = u \pi^{\operatorname{ord}_\mathfrak{p}}(x)$ with $u \in \mathcal{O}_{K, \mathfrak{p}}^*$. Hence, $\phi(x) = c^{\operatorname{ord}_\mathfrak{p}(x)}$ with $c = \phi(\pi) \in (0, 1)$.
    \end{itemize}
\end{example}

\begin{example}{archimedean-valuation}
    A valuation $\phi : K \to \RR_{\ge 0}$ is non-archimedean if and only if it is bounded on the subring $Z = \{ n \cdot 1 : n \in \ZZ \} \subset K$. Namely, if $\phi$ is non-archimedean, then $\norm{\phi} = 1$, so $\phi(\pm n \cdot 1) \le \phi(1) = 1$ for all $n \in \ZZ_{\ge 0}$, and hence $\phi$ is bounded on $Z$ by $1$. Conversely, suppose $\phi$ is bounded on $Z$ by $M > 0$. After replacing $\phi$ by a suitable power of $\phi$ if necessary, we can assume $\phi$ satisfies the triangle inequality. Now, taking $n$-th roots of both sides of
    \[ \phi(x + y)^n = \phi \left( \sum_{i = 0}^{n} x^i y^{n - i} \right) \le (n + 1) M \max \{ \phi(x), \phi(y) \}^n , \]
    and letting $n$ go to infinity, we find that $\phi$ is non-archimedean.
    
    In particular, this shows that any valuation on a field of characteristic $p > 0$ is non-archimedean, as $Z = \FF_p$ is finite.
\end{example}

\begin{topic}{place}{place}
    Two \tref{valuation}{valuations} $\phi$ and $\psi$ on a \tref{AA:field}{field} $K$ are said to be \textit{equivalent} if $\phi = \psi^r$ for some constant $r > 0$. A \textbf{place} of $K$ is an equivalence class of non-trivial valuations on $K$.
    
    Places are also known as \textbf{prime divisors}, or simply \textbf{primes}, of $K$. Places corresponding to \tref{archimedean-valuation}{archimedian} (resp. non-archimedean) valuations are called \textbf{infinite primes} (resp. \textbf{finite primes}). This terminology comes from the fact that finite primes on a \tref{number-field}{number field} $K$ precisely correspond to the non-zero prime ideals $\mathfrak{p}$ of the \tref{NT:ring-of-integers}{ring of integers} $\mathcal{O}_K$ of $K$, and infinite primes on $K$ correspond to embeddings $\sigma : K \to \CC$, up to complex conjugation.
\end{topic}

% \begin{topic}{complete-valued-field}{complete valued field}
    
% \end{topic}

\begin{topic}{hensel-lifting-lemma}{Hensel's lifting lemma}
    Let $K$ be a \tref{AA:field}{field}, complete with respect to a \tref{archimedean-valuation}{non-archimedean} \tref{valuation}{valuation}, and $A$ the valuation ring of $K$.
    Suppose that $f \in A[X]$ is a polynomial that factors over the residue field $k = A / \mathfrak{m}$ as
    \[ \overline{f} = \overline{g} \cdot \overline{h} \in k[X] \]
    with $\overline{g}, \overline{h} \in k[X]$ non-zero and coprime. Then \textbf{Hensel's lifting lemma} states that there exists a factorization $f = g \cdot h$ in $A[X]$ such that $\deg(g) = \deg(\overline{g})$ and $g, h$ have reduction $\overline{g}, \overline{h}$ in $k[X]$.
\end{topic}

\begin{example}{hensel-lifting-lemma}
    Consider $f = 2X^2 + X + 2 \in \ZZ_2[X]$, whose reduction $\overline{f} = X \in \FF_2[X]$ can be factored as $\overline{g} \cdot \overline{h}$ with $\overline{g} = X$ and $\overline{h} = 1$. The proof of Hensel's lemma is constructive, in the sense that we can compute successive approximations for $g$ and $h$.
    \[ g_0 = X + 2, \quad h_0 = 1 \quad (f \equiv g_0 h_0 \mod 2^1) \]
    \[ g_1 = X + 10, \quad h_1 = 2X - 3 \quad (f \equiv g_1 h_1 \mod 2^4) \]
    \[ g_2 = X + 4810, \quad h_2 = 2X - 9619 \quad (f \equiv g_2 h_2 \mod 2^{10}) \]
    \[ g_3 = X + 82462974405415300810, \quad h_3 = 2X - 164925948810830601619 \quad (f \equiv g_3 h_3 \mod 2^{18}) \]
    In general, we are guaranteed to have $f \equiv g_i h_i \mod 2^{2^i}$.
\end{example}

\begin{example}{hensel-lifting-lemma}
    Let $f \in A[x]$ be a polynomial such that the reduction $\overline{f} = f \mod \mathfrak{m} \in k[x]$ has a simple zero $\overline{\alpha} \in k$. Then we can write $\overline{f} = \overline{g} \cdot \overline{h}$ with $\overline{g} = x - \overline{\alpha}$ coprime to $\overline{h}$. Now by Hensel's lemma, there exists a root $\alpha \in A$ of $f$, which is a lift of $\overline{\alpha}$.
    
    In particular, consider the \tref{p-adic-numbers}{$p$-adic rationals} $\QQ_p$ for some prime $p$. The polynomial $f = x^p - x$ splits as
    \[ x^p - x = \prod_{a \in \FF_p} (x - a) \in \FF_p[x] \]
    over the residue field $\FF_p$. As a consequence, every $a \in \FF_p^*$ lifts to a $(p - 1)$-th root of unity in $\QQ_p$ (in fact, in $\ZZ_p$).
\end{example}

\begin{topic}{local-field}{local field}
    A \textbf{local field} is a \tref{AA:field}{field} $K$ together with a non-trivial \tref{valuation}{valuation} $\phi : K \to \RR_{\ge 0}$ such that the induced topology on $K$ is \tref{TO:locally-compact-space}{locally compact}.
\end{topic}

\begin{example}{local-field}
    Let $K$ be a \tref{number-field}{number field}, and $\phi : K \to \RR_{\ge 0}$ a valuation corresponding to a prime $\mathfrak{p}$ of the \tref{ring-of-integers}{ring of integers} $\mathcal{O}_K$ of $K$. Then the completion $K_\phi$ of $K$ with respect to $\phi$ is a local field.
\end{example}

\begin{example}{local-field}
    Any local field $K$ is \tref{TO:complete-metric-space}{complete} with respect to the metric induced by $\phi$, and either
    \begin{itemize}
        \item $K$ is \tref{archimedean-valuation}{archimedean}, and topologically isomorphic to $\RR$ or $\CC$,
        \item $K$ is non-archimedean, its valuation is discrete and its residue field is finite.
    \end{itemize}
\end{example}

\begin{topic}{abhyankar-lemma}{Abhyankar's lemma}
    Let $K$ be a \tref{AA:field}{field} with discrete \tref{valuation}{valuation} $\phi$, and let $L$ and $E$ be two extensions of $K$ that are contained in some finite extension $M = LE$ of $K$. Let $\psi$ be an extension of $\phi$ to $M$, and $\psi_L$ and $\psi_E$ the restrictions of $\psi$ to $L$ and $E$. \textbf{Abhyankar' lemma} states that if $\psi_L/\phi$ is tamely ramified and $e(\psi_L/\phi)$ divides $e(\psi_E/\phi)$, then $\psi$ is unramified over $\phi_E$.
\end{topic}

\begin{topic}{krasner-lemma}{Krasner's lemma}
    Let $K$ be a \tref{AA:field}{field}, complete with respect to a \tref{archimedean-valuation}{non-archimedean valuation} $\phi$, and let $\psi$ be the unique extension of $\phi$ to the \tref{AA:algebraic-closure}{algebraic closure} $\overline{K}$ of $K$.
    \textbf{Krasner's lemma} states that for every separable element $\alpha \in \overline{K}$ and element $\beta \in \overline{K}$ such that
    \[ \psi(\alpha - \beta) < \psi(\alpha - \alpha') \]
    for every $K$-conjugate $\alpha' \ne \alpha$ of $\alpha$, there is an inclusion $K(\alpha) \subset K(\beta)$.
\end{topic}

\begin{example}{krasner-lemma}
    \begin{proof}
        Let $\sigma$ be any automorphism of $\overline{K} / K(\beta)$ and let $\alpha' = \sigma(\alpha)$. If $\alpha' \ne \alpha$, then
        \[ \psi(\alpha - \beta) = \psi(\sigma(\alpha - \beta)) = \psi(\alpha' - \beta) = \psi(\alpha' - \alpha + \alpha - \beta) = \max \{ \psi(\alpha - \alpha'), \psi(\alpha - \beta) = \psi(\alpha - \alpha') \} , \]
        where in the fourth equality we used that $\psi(\alpha - \beta) < \psi(\alpha' - \alpha)$. However, this contradicts our assumption, so it follows that $\sigma(\alpha) = \alpha$, and thus $\alpha \in K(\beta)$.
    \end{proof}
\end{example}

\begin{example}{krasner-lemma}
    Krasner's lemma can be used to show that \tref{AA:splitting-field}{splitting fields} are `locally constant' in the following sense. Let $K(\alpha) / K$ be a \tref{AA:galois-extension}{Galois extension} of degree $n$, let $f \in K[x]$ be the \tref{AA:minimal-polynomial}{minimal polynomial} of $\alpha$ over $K$, and let $g \in K[x]$ be a polynomial of degree less than $n$. We claim that $K(\alpha)$ is the splitting field of $f + tg$ for all $t \in K$ with $\psi(t)$ sufficiently small.
    
    To see this, let $\alpha_1, \ldots, \alpha_n$ be the Galois conjugates of $\alpha$. Since they are distinct, there exists some $\gamma > 0$ such that $\gamma < \psi(\alpha_i - \alpha_j)$ for all $i \ne j$. As the roots of a polynomial vary continuously with its coefficients, for sufficiently small $\psi(t)$ we can assume $f + tg$ has roots $\beta_1, \ldots, \beta_n$ with $\psi(\alpha_i - \beta_i) < \gamma$. But now $\psi(\alpha_i - \beta_i) < \gamma < \psi(\alpha_i - \alpha_j)$ for all $j \ne i$, so Krasner's lemma implies $K(\alpha_i) \subset K(\beta_i)$. On the other hand, $[K(\beta_i) : K] \le \deg(f + tg) = \deg(f) = [K(\alpha_i) : K]$, so that $K(\beta_i) = K(\alpha_i)$. We conclude that for sufficiently small $\psi(t)$, the splitting field of $f + tg$ is $K(\alpha)$. Moreover, since $[K(\beta)_i : K] = \deg(f + tg)$, this shows $f + tg$ is irreducible for sufficiently small $\psi(t)$.
\end{example}

\begin{topic}{ramification-group}{ramification group}
    Let $L/K$ be a \tref{AA:galois-extension}{Galois extension} of \tref{archimedean-valuation}{non-archimedean} \tref{local-field}{local fields}, and denote by $\psi$ the valuation on $L$.
    For any $i \ge 0$, the \textbf{$i$-th ramification group} of $L/K$ is
    \[ \begin{aligned}
        G_i &= \{ \sigma \in \operatorname{Gal}(L/K) \mid \psi(x - \sigma(x)) < \psi(\pi_L^i) \textup{ for all } x \in A_L \} \\
        &= \ker(G \to \operatorname{Aut}(A_L/\mathfrak{p}_L^{i + 1})) ,
    \end{aligned} \]
    where $A_L = \{ x \in L \mid \psi(x) \le 0 \}$ denotes the valuation ring of $L$, and $\mathfrak{p}_L = \{ x \in L \mid \psi(x) < 0 \}$ its maximal ideal, with uniformizer $\pi_L \in \mathfrak{p}_L$.
    
    The ramification groups form a decreasing filtration
    \[ \operatorname{Gal}(L/K) \supset G_0 \supset G_1 \supset \cdots \supset \{ 1 \} . \]
    The corresponding sequence of fields $V_i = L^{G_i}$ are known for $i \ge 0$ as the \textbf{ramification fields} of $L/K$.
\end{topic}

\begin{topic}{decomposition-group}{decomposition group}
    Let $K$ be a \tref{AA:field}{field} with \tref{valuation}{valuation} $\phi$, and let $L/K$ be a finite \tref{AA:galois-extension}{Galois extension} with valuation $\psi$ extending $\phi$. The \textbf{decomposition group} of $\psi$ in $L/K$ is
    \[ G_\psi = \{ \sigma \in \operatorname{Gal}(L/K) \mid \psi(\sigma(x)) = \psi(x) \textup{ for all } x \in L \} . \]
    The \textbf{decomposition field} of $\psi$ in $L/K$ is the corresponding invariant field
    \[ Z_\psi = \{ x \in L \mid \sigma(x) = x \textup{ for all } \sigma \in G_\psi \} . \]
\end{topic}

\begin{example}{decomposition-group}
    Suppose the valuation $\psi$ on $L$ is either archimedean or discrete. Then $Z_\psi$ is the largest subfield $K \subset E \subset L$ for which the ramification index and residue class degree equal
    \[ e(\psi|_E / \phi) = f(\psi|_E / \phi) = 1 . \]
\end{example}

\begin{topic}{inertia-group}{inertia group}
    Let $K$ be a \tref{AA:field}{field} with \tref{valuation}{valuation} $\phi$, and let $L/K$ be a finite \tref{AA:galois-extension}{Galois extension} with valuation $\psi$ extending $\phi$. The \textbf{inertia group} of $\psi$ in $L/K$ is
    \[ \begin{aligned}
        I_\psi &= \{ \sigma \in \operatorname{Gal}(L/K) \mid \psi(\sigma(x) - x) < 1 \textup{ for all } x \in L \} \\
        &= \ker (\operatorname{Gal}(L/K) \to \operatorname{Aut}(A_L / \mathfrak{p}_L)) ,
    \end{aligned} \]
    where $A_L = \{ x \in L \mid \psi(x) \le 0 \}$ denotes the valuation ring of $L$, and $\mathfrak{p}_L = \{ x \in L \mid \psi(x) < 0 \}$ its maximal ideal, with uniformizer $\pi_L \in \mathfrak{p}_L$.
    
    The \textbf{inertia field} of $\psi$ in $L/K$ is the corresponding invariant field
    \[ T_\psi = \{ x \in L \mid \sigma(x) = x \textup{ for all } \sigma \in I_\psi \} . \]
\end{topic}

\begin{topic}{valuation-topology}{valuation topology}
    Let $K$ be a \tref{AA:field}{field} with \tref{valuation}{valuation} $\phi : K \to \RR_{\ge 0}$. The \textbf{valuation topology} on $K$ is the \tref{TO:topological-space}{topology} $\mathcal{T}_\phi$ on $K$ for which a \tref{TO:basis}{basis} is given by the open balls
    \[ U_\varepsilon(x) = \{ y \in K \mid \phi(x - y) < \varepsilon \} \]
    for all $x \in K$ and $\varepsilon > 0$.
\end{topic}

\begin{example}{valuation-topology}
    The topology $\mathcal{T}_\phi$ is \tref{TO:discrete-topology}{discrete} if and only if $\phi$ is trivial. Namely, if $\phi$ is trivial, then $U_{1/2}(x) = \{ x \}$ is open for all $x \in K$. Conversely, if $\mathcal{T}_\phi$ is discrete, then in particular $\{ 0 \}$ is open. Hence, $\{ 0 \} = U_\varepsilon(0)$ for some $\varepsilon > 0$, so $\phi(x) \ge \varepsilon$ for all $x \ne 0$. Now, if there exists any $x \in K$ with $\phi(x) \ne 1$, then $\phi(x^k) = \phi(x)^k < \varepsilon$ for some sufficient $k \in \ZZ$, yielding a contradiction. Hence $\phi$ is trivial.
\end{example}

\begin{example}{valuation-topology}
    For $\phi$ a \tref{archimedean-valuation}{non-archimedean valuation} on $K$, the valuation topology can be counterintuitive. Namely, for any $z \in K$, $\varepsilon > 0$ and $x, y \in U_\varepsilon(z)$, we have
    \[ \phi(x - y) = \phi(x - z + z - y) \le \max \{ \phi(x - z), \phi(z - y) \} < \varepsilon , \]
    which shows that every point in the open ball is a center, that is, $U_\varepsilon(z) = U_\varepsilon(x) = U_\varepsilon(y)$.
\end{example}

\begin{example}{valuation-topology}
    Given two non-trivial valuations $\phi$ and $\psi$ on $K$, there is an equivalence
    \[ \mathcal{T}_\psi \subset \mathcal{T}_\phi \iff  \phi = \psi^r \textup{ for some } r > 0 . \]
    Indeed, $(\Leftarrow)$ is clear from the definition, so we focus on $(\Rightarrow)$.
    As $\phi$ is non-trivial, there exists an element $y \in K$ with $0 < \phi(y) < 1$. We claim
    \[ \phi(x) < 1 \iff \psi(x) < 1 \textup{ for all } x \in K . \tag{$*$} \]
    Here, $(\Rightarrow)$ follows as the inequality $\phi(x) < 1$ amounts to saying the sequence $(x^n)_n$ converges to $0$ in $\mathcal{T}_\phi$. For $(\Leftarrow)$, take $x \in K$ with $\psi(x) < 1$. If $\phi(x) > 1$, then $x^{-1}$ would violate $(\Rightarrow)$ of $(*)$, and if $\phi(x) = 1$ then $yx^{-n}$ would violate $(\Rightarrow)$ of $(*)$ for sufficiently large $n$, so indeed $\phi(x) < 1$ as well.
    Now, for any $x \in K^*$, let $\alpha, \beta \in \RR$ be such that $\phi(x) = \phi(y)^\alpha$ and $\psi(x) = \psi(y)^\beta$. Then for any $a, b \in \ZZ$ with $b > 0$ we have
    \[ \alpha > \frac{a}{b} \iff \phi(y)^\alpha = \phi(x) < \phi(y)^{a/b} \iff \phi(x^b y^{-a}) < 1 \overset{(*)}{\iff} \psi(x^b y^{-a}) < 1 \iff \beta > \frac{a}{b} , \]
    which shows $\alpha = \beta$, so $r = \log \phi(x) / \log \psi(x) = \log \phi(y) / \log \psi(y)$ is independent of $x$, and we conclude $\phi = \psi^r$.
\end{example}
