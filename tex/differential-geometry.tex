\begin{topic}{topological-manifold}{topological manifold}
    A \textbf{topological manifold of dimension $n$} is a \tref{TO:second-countable}{second countable} \tref{TO:hausdorff-space}{Hausdorff} \tref{TO:topological-space}{topological space} in which every point has an open neighborhood that is \tref{TO:homeomorphism}{homeomorphic} to an open subset of $\RR^n$.
\end{topic}

\begin{topic}{atlas}{atlas}
    Let $M$ be a \tref{topological-manifold}{topological manifold}. A \textbf{chart} on $M$ is a pair $(U, \varphi)$ where $U$ is an open subset of $M$ and $\varphi : U \to \tilde{U}$ is a \tref{TO:homeomorphism}{homeomorphism} from $U$ to an open subset $\tilde{U} \subset \RR^n$. An \textbf{atlas} for $M$ is a collection of charts $\{ (U_\alpha, \varphi_\alpha) \}$ such that the $U_\alpha$ cover $M$.
    
    An atlas is \textbf{smooth} if any two charts $(U, \varphi)$ and $(V, \psi)$ are \textit{smoothly compatible}, that is, either $U \cap V = \varnothing$ or the \textit{transition map}
    \[ \psi \circ \varphi^{-1} : \varphi(U \cap V) \to \psi(U \cap V) \]
    is \textit{smooth}: all its continuous partial derivatives exist.
    
    A smooth atlas is \textbf{maximal} if it is not contained in any strictly larger smooth atlas.
\end{topic}

\begin{topic}{smooth-structure}{smooth structure}
    A \textbf{smooth structure} on a \tref{topological-manifold}{topological manifold} $M$ is a \tref{atlas}{maximal smooth atlas} for $M$.
\end{topic}

\begin{topic}{smooth-manifold}{smooth manifold}
    A \textbf{smooth manifold} is \tref{topological-manifold}{topological manifold} $M$ together with a \tref{smooth-structure}{smooth structure} on $M$.
\end{topic}

\begin{topic}{smooth-map}{smooth map}
    A map $f : M \to N$ between \tref{smooth-manifold}{smooth manifolds} is \textbf{smooth} if for all \tref{atlas}{charts} $(U, \varphi)$ of $M$ and $(V, \psi)$ of $N$, the composition
    \[ \psi \circ f \circ \varphi^{-1} : \varphi(f^{-1}(V) \cap U) \to \psi(V) \]
    is \textit{smooth}, i.e. all its continuous partial derivatives exist.
\end{topic}

\begin{topic}{diffeomorphism}{diffeomorphism}
    A \textbf{diffeomorphism} is a \tref{smooth-map}{smooth map} $f : M \to N$ between \tref{smooth-manifold}{smooth manifolds} that has a smooth inverse $f^{-1} : N \to M$.
    
    A \textbf{local diffeomorphism} is a smooth map $f : M \to N$ for which every point $p \in M$ has a neighborhood $U \subset M$ such that $f(U)$ is open in $N$ and $f : U \to f(U)$ is a diffeomorphism.
    
    The manifolds $M$ and $N$ are called \textbf{diffeomorphic} if there exists diffeomorphism between them.
\end{topic}

\begin{topic}{submersion}{submersion}
    A \tref{smooth-map}{smooth map} $f : M \to N$ between \tref{smooth-manifold}{smooth manifolds} is a \textbf{submersion} if the derivative $df_p : T_p M \to T_{f(p)} N$ is surjective for all $p \in M$.
\end{topic}

\begin{topic}{immersion}{immersion}
    A \tref{smooth-map}{smooth map} $f : M \to N$ between \tref{smooth-manifold}{smooth manifolds} is an \textbf{immersion} if the derivative $df_p : T_p M \to T_{f(p)} N$ is injective for all $p \in M$.
\end{topic}

\begin{topic}{embedding}{embedding}
    A \tref{smooth-map}{smooth map} $f : M \to N$ between \tref{smooth-manifold}{smooth manifolds} is an \textbf{embedding} if it is an \tref{immersion}{immersion} and a \tref{TO:homeomorphism}{homeomorphism} onto its image.
\end{topic}

\begin{topic}{tangent-space}{tangent space}
    Let $M$ be a \tref{smooth-manifold}{smooth manifold} and $p$ a point of $M$. A \textbf{derivation at $p$}, or a \textbf{tangent vector at $p$}, is a linear map $v : C^\infty(M) \to \RR$ satisfying
    \[ v(fg) = f(p) v(g) + g(p) v(f) \]
    for all $f, g \in C^\infty(M)$. The \textbf{tangent space} to $M$ at $p$ is the set of all derivations at $p$, denoted $T_p M$. It is a real \tref{LA:vector-space}{vector space} of the same dimension as the dimension of $M$.
\end{topic}

\begin{topic}{tangent-bundle}{tangent bundle}
    Let $M$ be an $n$-dimensional \tref{smooth-manifold}{smooth manifold}. The \textbf{tangent bundle} of $M$, denoted $TM$, is defined as the disjoint union of the \tref{tangent-space}{tangent spaces} at all points of $M$,
    \[ TM = \bigsqcup_{p \in M} T_p M . \]
    It naturally has the structure of a $2n$-dimensional smooth manifold such that the projection map
    \[ \pi : TM \to M, \quad (p, v) \mapsto p \]
    is \tref{smooth-map}{smooth}.
\end{topic}

\begin{topic}{vector-field}{vector field}
    Let $M$ be a \tref{smooth-manifold}{smooth manifold}. A \textbf{vector field} on $M$ is a section of the \tref{tangent-bundle}{tangent bundle} $\pi : TM \to M$, i.e. a map $X : M \to TM$ such that $X(p) \in T_p M$ for all $p \in M$.
\end{topic}

\begin{topic}{cotangent-space}{cotangent space}
    Let $M$ be a \tref{smooth-manifold}{smooth manifold}. The \textbf{cotangent space} to $M$ at a point $p \in M$, denoted $T^*_p M$ is the \tref{LA:dual-vector-space}{dual} of the \tref{tangent-space}{tangent space} $T_p M$,
    \[ T^*_p M = (T_p M)^* . \]
    Elements of $\xi \in T^*_p M$ are called \textbf{tangent covectors} at $p$.
\end{topic}

\begin{topic}{cotangent-bundle}{cotangent bundle}
    Let $M$ be an $n$-dimensional \tref{smooth-manifold}{smooth manifold}. The \textbf{cotangent bundle} of $M$, denoted $T^*M$, is defined as the disjoint union of the \tref{cotangent-space}{cotangent spaces} at all points of $M$,
    \[ T^*M = \bigsqcup_{p \in M} T^*_p M . \]
    It naturally has the structure of a $2n$-dimensional smooth manifold such that the projection map
    \[ \pi : T^*M \to M, \quad (p, \xi) \mapsto p \]
    is \tref{smooth-map}{smooth}.
\end{topic}

\begin{topic}{vector-bundle}{vector bundle}
    Let $M$ be a \tref{smooth-manifold}{smooth manifold}. A \textbf{vector bundle of rank $k$} over $M$ is a smooth manifold $E$ together with a surjective \tref{smooth-map}{smooth map} $\pi : E \to M$ such that
    \begin{itemize}
        \item for each $p \in M$, the \textit{fiber} $E_p = \pi^{-1}(p)$ has the structure of a real \tref{LA:vector-space}{vector space},
        \item for each $p \in M$, there exists a neighborhood $U \subset M$ of $p$ and a \tref{diffeomorphism}{diffeomorphism} $\Phi : \pi^{-1}(U) \to U \times \RR^k$ such that
        \[ \begin{tikzcd} \pi^{-1}(U) \arrow{rr}{\Phi} \arrow[swap]{dr}{\pi} && U \times \RR^k \arrow{dl}{\pi_U} \\ & U & \end{tikzcd} \]
        commutes, and the restriction of $\Phi|_{E_p} : E_p \to \{ p \} \times \RR^k \simeq \RR^k$ is a linear isomorphism.
    \end{itemize}
\end{topic}

\begin{topic}{differential-form}{differential form}
    Let $M$ be a \tref{smooth-manifold}{smooth manifold}. A \textbf{smooth differential $k$-form} is a smooth section of the $k$-th exterior power of the \tref{cotangent-bundle}{cotangent bundle} $\wedge^k T^* M$. The set of all smooth differential $k$-forms is a \tref{LA:vector-space}{vector space} denoted by $\Omega^k(M)$.
\end{topic}

\begin{topic}{exterior-derivative}{exterior derivative}
    Let $M$ be a \tref{smooth-manifold}{smooth manifold}. The \textbf{exterior product} is the unique linear map $d : \Omega^k(M) \to \Omega^{k - 1}(M)$ between \tref{differential-form}{differential forms} satisfying
    \begin{itemize}
        \item for any smooth function $f$ (that is, a $0$-form), $df$ is the \textit{differential} of $f$, that is, $df(X) = X(f)$ for any \tref{vector-field}{vector field} $X$,
        \item $d(df) = 0$ for all smooth functions $f$,
        \item $d(\omega \wedge \eta) = d \omega \wedge \eta + (-1)^k (\omega \wedge d\eta)$ for all $k$-forms $\omega$ and $p$-forms $\eta$.
    \end{itemize}
    It can be shown that $d(d\omega) = 0$ for all $k$-forms $\omega$, which is commonly expressed as $d^2 = 0$.
    
    In local coordinates, if $\omega = \sum_I \omega_I dx^I$ (multi-index notation) then
    \[ d \omega = \sum_I \sum_j \frac{\partial \omega_I}{\partial x^j} dx^j \wedge dx^I . \]
\end{topic}

\begin{example}{exterior-derivative}
    Let $\omega = x^2 \; dy \wedge dz + yz \; dx \wedge dz$ be a $2$-form on $\RR^3$. Then
    \[ \begin{aligned}
        d\omega
            &= 2x \; dx \wedge dy \wedge dz + z \; dy \wedge dx \wedge dz + y \; dz \wedge dx \wedge dz \\
            &= (2x - z) \; dx \wedge dy \wedge dz .
    \end{aligned} \]
\end{example}

\begin{topic}{interior-product}{interior product}
    Let $M$ be a \tref{smooth-manifold}{smooth manifold}. The \textbf{interior product} is the \textit{contraction} of a \tref{differential-form}{differential form} with a \tref{vector-field}{vector field}. That is, if $X$ is a vector field on $M$, then
    \[ \iota_X : \Omega^k(M) \to \Omega^{k - 1} \]
    is the map defined by
    \[ (\iota_X \omega)(Y_1, \ldots, Y_{k - 1}) = \omega(X, Y_1, \ldots, Y_{k - 1}) \]
    for any vector fields $Y_1, \ldots, Y_{k - 1}$.
\end{topic}

\begin{topic}{lie-derivative}{Lie derivative}
    Let $M$ be a \tref{smooth-manifold}{smooth manifold}. The \textbf{Lie derivative} of a \tref{tensor-field}{tensor field} $T$ with respect to a \tref{vector-field}{vector field} $X$ is the tensor field $\mathcal{L}_X T$ (of the same type as $T$), given by
    \[ (\mathcal{L}_X T)_p = \frac{d}{dt}\Big|_{t = 0} \left( \left(\varphi^t_X\right)^* T \right)_p \]
    for all points $p \in M$. Here $\varphi^t_X$ denotes the flow of $X$.
\end{topic}

\begin{example}{lie-derivative}
    When $f$ is a smooth function, we have
    \[ \begin{aligned}
        (\mathcal{L}_X f)(p)
            &= \frac{d}{dt}\Big|_{t = 0} \left(\left(\varphi^t_X\right)^* f \right)(p) \\
            &= \frac{d}{dt}\Big|_{t = 0} f \left(\varphi^t_X(p)\right) \\
            &= df_p \left( \frac{d}{dt}\Big|_{t = 0} \varphi^t_X(p) \right) \\
            &= df_p(X_p) \\
            &= X(f)(p) .
    \end{aligned} \]
    
\end{example}

\begin{topic}{lie-bracket-vector-fields}{Lie bracket of vector fields}
    Let $M$ be a \tref{smooth-manifold}{smooth manifold}. The \textbf{Lie bracket} of two \tref{vector-field}{vector fields} $X, Y$ is defined as the vector field $[X, Y]$ given by
    \[ [X, Y](f) = X(Y(f)) - Y(X(f)) \]
    for all smooth functions $f$ on $M$.
\end{topic}

\begin{topic}{cartan-formula}{Cartan formula}
    The \textbf{Cartan formula} relates the \tref{interior-product}{interior product}, \tref{exterior-derivative}{exterior derivative} and \tref{lie-derivative}{Lie derivative} via
    \[ \mathcal{L}_X \omega = d(\iota_X \omega) + \iota_X d \omega . \]
\end{topic}

\begin{topic}{closed-form}{closed form}
    Let $M$ be a \tref{smooth-manifold}{smooth manifold}. A \tref{differential-form}{differential form} $\omega$ is \textbf{closed} if its \tref{exterior-derivative}{exterior derivative} is zero, that is, if $dw = 0$.
\end{topic}

\begin{topic}{exact-form}{exact form}
    Let $M$ be a \tref{smooth-manifold}{smooth manifold}. A \tref{differential-form}{differential form} $\omega$ is \textbf{exact} if it is the \tref{exterior-derivative}{exterior derivative} of another differential form $\eta$, that is, if $\omega = d \eta$.
\end{topic}

\begin{topic}{tensor-field}{tensor field}
    Let $M$ be a \tref{smooth-manifold}{smooth manifold}. A \textbf{tensor bundle} on $M$ is any \tref{vector-bundle}{vector bundle} of the form
    \[ \underbrace{TM \otimes \cdots TM}_{p \text{ times}} \otimes \underbrace{T^*M \otimes \cdots T^*M}_{q \text{ times}} . \]
    A \textbf{tensor field} of \textit{type} $(p, q)$ is a section of this bundle.
\end{topic}

\begin{topic}{parallelizable-manifold}{parallelizable manifold}
    A \tref{smooth-manifold}{smooth manifold} $M$ is \textbf{parallelizable} if its \tref{tangent-bundle}{tangent bundle} $TM$ is trivial, that is, $TM$ is \tref{diffeomorphism}{diffeomorphic} to $M \times \RR^n$.
\end{topic}

\begin{topic}{inverse-function-theorem}{inverse function theorem}
    The \textbf{inverse function theorem} states that if $f : M \to N$ is a \tref{smooth-map}{smooth map} between \tref{smooth-manifold}{smooth manifolds}, whose derivative $df_p : T_p M \to T_{f(p)} N$ at some point $p \in M$ is an isomorphism, then $f$ is a \tref{diffeomorphism}{local diffeomorphism} at $p$.
\end{topic}

\begin{topic}{partition-of-unity}{partition of unity}
    Let $M$ be a \tref{smooth-manifold}{smooth manifold}. A \textbf{partition of unity} on $M$ is a set of smooth functions $\{ f_\alpha : M \to [0, 1] \}$ so that each point $p \in M$ has a neighborhood where only finitely many $f_\alpha$ are nonzero and $\sum_\alpha f_\alpha(p) = 1$.
    
    It is a theorem that for any open cover $\{ U_\alpha \}$ of $M$, there exists a partition of unity $\{ f_\alpha \}$ such that the support of $f_\alpha$ is contained in $U_\alpha$.
\end{topic}

\begin{topic}{whitney-embedding-theorem}{Whitney embedding theorem}
    The \textbf{Whiteney embedding theorem} states that any \tref{smooth-manifold}{smooth $n$-dimensional manifold} can be \tref{embedding}{embedded} in $\RR^{2n}$, for all $n > 0$.
\end{topic}

\begin{example}{whitney-embedding-theorem}
    The dimension $2n$ is a sharp lower bound: projective space $\RR P^n$ cannot be embedded in $\RR^{2n - 1}$ whenever $n$ is a power of $2$.
\end{example}

\begin{topic}{riemannian-manifold}{Riemannian manifold}
    Let $M$ be a \tref{smooth-manifold}{smooth manifold}. A \textbf{Riemannian metric} on $M$ is a smooth symmetric \tref{tensor-field}{$(0, 2)$-tensor field} $g$ that is positive definite at each point $p \in M$. The pair $(M, g)$ is called a \textbf{Riemannian manifold}.
\end{topic}

\begin{topic}{almost-complex-structure}{almost complex manifold}
    Let $M$ be a \tref{smooth-manifold}{smooth manifold}. An \textbf{almost complex structure} on $M$ is a smooth \tref{tensor-field}{tensor field} $J \in \Gamma(TM \otimes T*M)$ satisfying $J^2 = -1$ when viewed as a bundle isomorphism $J : TM \to TM$. The pair $(M, J)$ is called an \textbf{almost complex manifold}.
\end{topic}

\begin{topic}{kahler-manifold}{Kähler manifold}
    A \textbf{Kähler manifold} is a \tref{smooth-manifold}{smooth manifold} $M$ with three compatible structures: a complex structure $J$, a \tref{riemannian-manifold}{Riemannian structure} $g$, and a \tref{symplectic-manifold}{symplectic structure} $\omega$. Compatibility means that
    \[ g(X, Y) = \omega(X, JY) \]
    for all \tref{vector-field}{vector fields} $X, Y$ on $M$.
    
    Note that two-out-of-three structures determines the other structure.
\end{topic}

\begin{topic}{connection}{connection}
    Let $\pi : E \to M$ be a \tref{vector-bundle}{vector bundle} over a \tref{smooth-manifold}{smooth manifold} $M$. A \textbf{connetion} on $E$ is a linear differential operator
    \[ \nabla : \Gamma(E) \to \Gamma(T^*M \otimes E) \]
    satisfying the \textit{Leibniz rule}
    \[ \nabla(fs) = df \otimes s + f \nabla s \]
    for all smooth functions $f$ and sections $s \in \Gamma(E)$.
\end{topic}

\begin{topic}{curvature-connection}{curvature of a connection}
    Let $\nabla$ be a \tref{connection}{connection} on a \tref{vector-bundle}{vector bundle} $\pi : E \to M$. The map $\nabla^2 : \Omega^0(M; E) \to \Omega^2(M; E)$ is $C^\infty(M)$-linear and hence induced by a section $F_\nabla \in \Omega^2(M; \text{End}(E))$, called the \textbf{curvature} of $\nabla$.
\end{topic}

\begin{topic}{curvature-form}{curvature form}
    Let $\nabla$ be a \tref{connection}{connection} on a \tref{vector-bundle}{vector bundle} $\pi : E \to M$. The \textbf{curvature form} of $E$ is the $2$-form $k(E) = \frac{1}{2 \pi i} \text{tr } F_\nabla \in \Omega^2(M)$, where $F_\nabla \in \Omega^2(M; \text{End}(M))$ is the \tref{curvature-connection}{curvature} of $\nabla$. It is a \tref{closed-form}{closed form} and the corresponding cohomology class $\kappa(E) = [k(E)] \in H_\text{dR}^2(M)$ is called the \textbf{curvature class} of $E$.
\end{topic}

% ---

\begin{topic}{de-rham-isomorphism}{de Rham isomorphism}
    The \textbf{de Rham isomorphism} is the isomorphism given by
    \[ H_{\textup{dR}}^k(X) \xrightarrow{\sim} H^k(X; \RR) : [ \omega ] \mapsto \int_{(-)} \omega . \]
    
    The exterior product endows the direct sum of these groups with a ring structure. A further result of the theorem is that the two cohomology rings are isomorphic (as graded rings), where the analogous product on singular cohomology is the cup product.
\end{topic}

\begin{topic}{poincare-duality}{Poincaré duality}
    Let $X$ be a compact oriented $n$-dimensional manifold. \textbf{Poincaré duality} states that there is a canonical isomorphism
    \[ H^k(X; \ZZ) \xrightarrow{\sim} H_{n - k}(X; \ZZ) \]
    which sends $\omega$ to the cap product $[X] \frown \omega$, where $[X] \in H_n(X; \ZZ)$ denotes the fundamental class of $X$. That is, $\omega$ is sent to
    \[ (\eta \mapsto (\omega \smile \eta)([X])) \in \Hom(H^{n - k}(X; \ZZ), \ZZ) = H_{n - k}(X; \ZZ) . \]
\end{topic}

\begin{topic}{fundamental-class}{fundamental class}
    The \textbf{fundamental class} of a closed orientable $n$-dimensional manifold $M$ is a homology class $[M] \in H_n(M; \ZZ)$ which is a generator of the homology group corresponding to the orientation of $M$.
\end{topic}

\begin{topic}{lie-group}{Lie group}
    A \textbf{Lie group} is a \tref{GT:group}{group} $G$ which is also a finite-dimensional \tref{smooth-manifold}{smooth manifold}, such that the multiplication map $G \times G \to G : (x, y) \mapsto xy$ and the inversion map $G \to G : x \mapsto x^{-1}$ are \tref{smooth-map}{smooth}.
\end{topic}
