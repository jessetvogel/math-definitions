\begin{topic}{de-rham-isomorphism}{de Rham isomorphism}
    The \textbf{de Rham isomorphism} is the isomorphism given by
    \[ H_{\textup{dR}}^k(X) \xrightarrow{\sim} H^k(X; \RR) : [ \omega ] \mapsto \int_{(-)} \omega . \]
    
    The exterior product endows the direct sum of these groups with a ring structure. A further result of the theorem is that the two cohomology rings are isomorphic (as graded rings), where the analogous product on singular cohomology is the cup product.
\end{topic}

\begin{topic}{poincare-duality}{Poincaré duality}
    Let $X$ be a compact oriented $n$-dimensional manifold. \textbf{Poincaré duality} states that there is a canonical isomorphism
    \[ H^k(X; \ZZ) \xrightarrow{\sim} H_{n - k}(X; \ZZ) \]
    which sends $\omega$ to the cap product $[X] \frown \omega$, where $[X] \in H_n(X; \ZZ)$ denotes the fundamental class of $X$. That is, $\omega$ is sent to
    \[ (\eta \mapsto (\omega \smile \eta)([X])) \in \Hom(H^{n - k}(X; \ZZ), \ZZ) = H_{n - k}(X; \ZZ) . \]
\end{topic}
