\begin{topic}{field-extension}{field extension}
    A \textbf{field extension} is an inclusion of fields $k \hookrightarrow \ell$. We also say that $\ell$ is a field \textit{over} $k$, and write $k \subset \ell$ or $\ell / k$.
    
    Note that any morphism of fields $k \to \ell$ is injective, and hence a field extension, as the only ideals of a field are $(0)$ and $(1)$, but $1$ does not lie in the kernel.
\end{topic}

\begin{topic}{algebraic-transcendental}{algebraic/transcendental}
    Let $\ell / k$ be a \tref{field-extension}{field extension} and take an element $\alpha \in \ell$. If there exists a polynomial $f \in k[x]$ with $f(\alpha) = 0$, we call $\alpha$ \textbf{algebraic} over $k$. Otherwise, $\alpha$ is called \textbf{transcendental} over $k$.
\end{topic}

\begin{topic}{minimal-polynomial}{minimal-polynomial}
    Let $\ell / k$ be a \tref{field-extension}{field extension} and take an \tref{algebraic-transcendental}{algebraic} element $\alpha \in \ell$. The \textbf{minimal polynomial} of $\alpha$ over $k$ is the unique monic polynomial $f \in k[x]$ of minimal degree such that $f(\alpha) = 0$.
\end{topic}

\begin{topic}{prime-field}{prime field}
    The \textbf{prime field} of a field $k$ is its smallest subfield. When $\text{char}(k) = 0$, this is $\QQ$ and when $\text{char}(k) = p > 0$, this is $\mathbb{F}_p$.
\end{topic}

\begin{topic}{frobenius-morphism}{frobenius morphism}
    Let $k$ be a field of characteristic $p > 0$. The map $F : k \to k : x \mapsto x^p$ is a homomorphism (use binomial theorem) called the \textbf{Frobenius morphism}.
    
    Note that $F$ is injective as it is a field morphism. So when $k$ is finite, $F$ will also be surjective, and hence an automorphism of $k$. 
\end{topic}

\begin{topic}{splitting-field}{splitting field}
    Let $k$ be a field, and let $f \in k[x]$ be a non-constant polynomial. A \textbf{splitting field} $\ell$ of $f$ over $k$ is a \tref{field-extension}{field extension} such that
    \begin{itemize}
        \item $f$ splits in $\ell[x]$ as a product of linear factors,
        \item if $\alpha_1, \ldots, \alpha_s$ are the zeros of $f$ in $\ell$, then $\ell = k(\alpha_1, \ldots, \alpha_s)$.
    \end{itemize}
    A splitting field always exists, and it is unique up to unique isomorphism over $k$.
\end{topic}

\begin{topic}{separable}{separable}
    Let $k$ be a field, and let $f \in k[x]$ be a non-constant polynomial. Then $f$ is called \textbf{separable} if the roots of $f$ in any field extension $k \subset \ell$ are distinct.
    
    In particular, it suffices to verify this for a \tref{splitting-field}{splitting field} of $f$ over $k$.
\end{topic}
