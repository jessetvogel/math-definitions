\begin{topic}{group}{group}
    A \textbf{group} is a set $G$ together with an operation $G \times G \to G$ (the \textit{group law}) written as $(x, y) \mapsto xy$, and an element $1 \in G$ (the \textit{unit}), satisfying
    \begin{itemize}
        \item (\textit{associativity}) $(xy)z = x(yz)$ for all $x, y, z \in G$,
        \item (\textit{unit element}) $1 \cdot x = x \cdot 1 = x$ for all $x \in G$,
        \item (\textit{inverses}) for all $x \in G$ there exists an $x^{-1} \in G$ such that $x x^{-1} = x^{-1} x = 1$, called the \textit{inverse} of $x$.
    \end{itemize}
\end{topic}

\begin{topic}{subgroup}{subgroup}
    A \textbf{subgroup} $H$ of a \tref{group}{group} $G$ is a subset $H \subset G$ which, with the same group law and unit, is itself a group.
\end{topic}

\begin{topic}{abelian-group}{abelian group}
    A \tref{group}{group} $G$ is called \textbf{abelian} if $xy = yx$ for all $x, y \in G$.
\end{topic}

\begin{topic}{order}{order}
    Let $G$ be a \tref{group}{group}. The \textbf{order} of an element $x \in G$, denoted $\text{ord}(x)$, is the least positive integer $n$ such that $x^n = 1$. If no such $n$ exists, then $\text{ord}(x) = \infty$.
\end{topic}

\begin{topic}{cyclic-group}{cyclic group}
    A \textbf{cyclic group} is a \tref{group}{group} $G$ generated by a single element $x \in G$, that is $G = \{ x^n : x \in \ZZ \}$.
\end{topic}

\begin{topic}{group-homomorphism}{group homomorphism}
    Let $G$ and $H$ be two \tref{group}{groups}. A \textbf{homomorphism} from $G$ to $H$ is a map $f : G \to H$ satisfying $f(xy) = f(x) f(y)$ for all $x, y \in G$.
\end{topic}

\begin{topic}{kernel}{kernel}
    Let $f : G \to H$ be a \tref{group-homomorphism}{group homomorphism}. The \textbf{kernel} of $f$, denoted $\ker f$, is defined as
    \[ \ker f = \{ x \in G : f(x) = 0 \} . \]
    It is a \tref{normal-subgroup}{normal} \tref{subgroup}{subgroup} of $G$.
\end{topic}

\begin{topic}{group-center}{group center}
    The \textbf{center} of a \tref{group}{group} $G$ is the subgroup
    \[ Z(G) = \{ x \in G : xy = yx \text{ for all } y \in G \} . \]
\end{topic}

\begin{topic}{symmetric-group}{symmetric group}
    Let $\Sigma$ be a set. The \textbf{symmetric group} on $\Sigma$ is the \tref{group}{group} $S_\Sigma$ of all bijections $\Sigma \to \Sigma$. The group law is given by composition, and the unit is the identity map.
    
    When $\Sigma = \{ 1, 2, \ldots, n \}$ for some integer $n \ge 1$, one writes $S_n$ for the symmetry group. Its elements are called \textbf{permutations}.
\end{topic}

\begin{topic}{cayleys-theorem}{Cayley's theorem}
    \textbf{Cayley's theorem} states that every \tref{group}{group} $G$ is isomorphic to a \tref{subgroup}{subgroup} of the \tref{symmetric-group}{symmetric group} $S_G$. In particular, $G$ is isomorphic to the image of the morphism
    \[ \varphi : G \to S_G, \qquad x \mapsto (y \mapsto xy) . \]
\end{topic}

\begin{topic}{cyclic-permutation}{cyclic permutation}
    A \tref{symmetric-group}{permutation} $\sigma \in S_n$ is a \textbf{cyclic permutation}, or \textbf{cycle}, of length $k$ if there exist $k$ distinct integers $1 \le a_1, \ldots, a_k \le n$ with $\sigma(a_i) = a_{i + 1}$ for $1 \le i < k$ and $\sigma(a_k) = a_1$, and $\sigma(x) = x$ for $x \not\in \{ a_1, \ldots, a_k \}$. This is denoted by
    \[ \sigma = (a_1 \;\; a_2 \;\; \ldots \;\; a_k) . \]
    A cycle of length $2$ is also called a \textbf{transposition}.
\end{topic}

\begin{topic}{permutation-sign}{permutation sign}
    The \textbf{sign} of a \tref{symmetric-group}{permutation} $\sigma \in S_n$ is defined as
    \[ \text{sign}(\sigma) = \prod_{1 \le i < j \le n} \frac{\sigma(j) - \sigma(i)}{j - i} \in \{ +1, -1 \} . \]
    We call $\sigma$ \textit{even} if $\text{sign}(\sigma) = 1$ and \textit{odd} if $\text{sign}(\sigma) = -1$.
    
    This defines a \tref{group-homomorphism}{group homomorphism}
    \[ \text{sign} : S_n \to \{ +1, -1 \} . \]
\end{topic}

\begin{topic}{alternating-group}{alternating group}
    For $n \ge 1$, the \textbf{alternating group} $A_n$ is the subgroup of the \tref{symmetric-group}{symmetric group} $S_n$ of all \tref{permutation-sign}{even} permutations.
\end{topic}

\begin{topic}{normal-subgroup}{normal subgroup}
    A \tref{subgroup}{subgroup} $N$ of a \tref{group}{group} $G$ is called \textbf{normal} if $N = g N g^{-1}$ for all $g \in G$.
\end{topic}

\begin{example}{normal-subgroup}
    The \tref{kernel}{kernel} of a morphism $f : G \to H$ is always a normal subgroup. Indeed, for any $x \in \ker f$ and $g \in G$ we have
    \[ f(gxg^{-1}) = f(g) f(x) f(g)^{-1} = f(g) f(g)^{-1} = 1 , \]
    so $gxg^{-1} \in \ker f$, and thus $g (\ker f) g^{-1} \subset \ker f$. The other inclusion is shown completely similarly.
    
    Conversely, any normal subgroup $N \subset G$ is the kernel of the quotient map $\pi : G \to G / N$.
\end{example}

\begin{topic}{central-subgroup}{central subgroup}
    A \tref{subgroup}{subgroup} $H$ of a \tref{group}{group} $G$ is called \textbf{central} if $H$ lies in the \tref{group-center}{center} of $G$.
\end{topic}

\begin{topic}{quotient-group}{quotient group}
    Let $G$ be a \tref{group}{group} and $N$ a \tref{normal-subgroup}{normal subgroup}. The \textbf{quotient group} $G/N$ is the group of cosets
    \[ G/N = \{ gN : g \in G \} \]
    with group law $(gN)(hN) = (gh)N$ and unit $N$.
\end{topic}

\begin{topic}{simple-group}{simple group}
    A \tref{group}{group} $G$ is called \textbf{simple} if its only \tref{normal-subgroup}{normal subgroups} are $\{ 1 \}$ and $G$.
\end{topic}

\begin{topic}{torsion-subgroup}{torsion subgroup}
    The \textbf{torsion subgroup} of a \tref{group}{group} $G$ is the subgroup of elements of finite \tref{order}{order}.
    \[ G_{\text{tor}} = \{ x \in G : \text{ord}(x) < \infty \} \]
\end{topic}

\begin{topic}{normalizer}{normalizer}
    Let $H$ be a \tref{subgroup}{subgroup} of a \tref{group}{group} $G$. The \textbf{normalizer} of $H$ is the subgroup
    \[ N_H = \{ x \in G : x H x^{-1} = H \} . \]
\end{topic}

\begin{topic}{group-action}{group action}
    Let $G$ be a \tref{group}{group} and $X$ a set. An \textbf{action} of $G$ on $X$ is map
    \[ G \times X \to X : (g, x) \mapsto g \cdot x \]
    satisfying $1 \cdot x = x$ and $g \cdot (h \cdot x) = (gh) \cdot x$ for all $g, h \in G$ and $x \in X$.
\end{topic}

\begin{topic}{centralizer}{centralizer}
    Let $G$ be a \tref{group}{group}. The \textbf{centralizer} of an element $g \in G$ is the subgroup
    \[ G_g = \{ h \in G : h x h^{-1} = g \} . \]
\end{topic}

\begin{topic}{stabilizer}{stabilizer}
    Let $G$ be a \tref{group}{group} \tref{group-action}{acting} on a set $X$. Then the \textbf{stabilizer} of $x \in X$ is the \tref{subgroup}{subgroup}
    \[ G_x = \{ g \in G : g \cdot x = x \} . \]
\end{topic}

\begin{topic}{orbit}{orbit}
    Let $G$ be a \tref{group}{group} \tref{group-action}{acting} on a set $X$. The \textbf{orbit} of an element $x \in X$ is the set
    \[ Gx = \{ g \cdot x : g \in G \} \subset X . \]
\end{topic}

\begin{topic}{solvable-group}{solvable group}
    A \tref{group}{group} $G$ is \textbf{solvable} if there exist \tref{subgroup}{subgroups} $H_1, H_2, \ldots, H_r$
    \[ G = H_0 \supset H_1 \supset \cdots \supset H_r = \{ 1 \} \]
    where $H_{i + 1}$ is \tref{normal-subgroup}{normal} in $H_i$ and $H_i/H_{i + 1}$ is abelian.
\end{topic}

\begin{example}{solvable-group}
    The group $S_3$ is solvable, since we have the sequence $S_3 \supset A_3 \supset \{ 1 \}$, and $S_3 / A_3 \simeq \ZZ/2\ZZ$ and $A_3/\{ 1 \} \simeq \ZZ/3\ZZ$.
\end{example}

\begin{topic}{commutator-subgroup}{commutator subgroup}
    The \textbf{commutator subgroup} of a \tref{group}{group} $G$ is the subgroup of $G$ generated by all elements of the form $ghg^{-1}h^{-1}$ with $g, h \in G$.
\end{topic}

\begin{topic}{conjugation}{conjugation}
    Let $G$ be a \tref{group}{group}. Two elements $x, y \in G$ are called \textbf{conjugate} if there exists some $g \in G$ such that $g x g^{-1} = y$.
\end{topic}

% // Borel subgroup
% Dihedral group
% Free (abelian) group
% Subgroup index
% Lagrange's theorem
% Sylow p-group
