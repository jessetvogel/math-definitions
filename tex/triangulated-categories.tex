\begin{topic}{triangulated-category}{triangulated category}
    Let $\mathcal{D}$ be an \tref{additive-category}{additive category}. The structure of a \textbf{triangulated category} on $\mathcal{D}$ is given by an additive equivalence
    \[ [1] : \mathcal{D} \to \mathcal{D} \qquad \text{(the shift functor)} \]
    and a set of \textbf{distinguished triangles}
    \[ A \to B \to C \to A[1] \]
    such that
    \begin{enumerate}
        \item \begin{itemize}
            \item Any $A \to A \to 0 \to A[1]$ is distinguished.
            \item Any triangle isomorphic to a distinguished triangle is distinguished.
            \item Any morphism $f : A \to B$ can be completed to a distinguished triangle $A \overset{f}{\to} B \to C \to A[1]$.
        \end{itemize}
        \item A triangle
        \[ A \xrightarrow{f} B \xrightarrow{g} C \xrightarrow{h} A[1] \]
        is distinguished if and only if
        \[ B \xrightarrow{g} C \xrightarrow{h} A[1] \xrightarrow{-f[1]} B[1] \]
        is.
        \item Given a commutative diagram of distinguished triangles with vertical $f$ and $g$,
        \[ \begin{tikzcd}
            A \arrow{r} \arrow{d}{f} & B \arrow{r} \arrow{d}{g} & C \arrow{r} \arrow[dashed]{d}{h} & A[1] \arrow{d}{f[1]} \\
            A' \arrow{r} & B' \arrow{r} & C' \arrow{r} & A'[1]
        \end{tikzcd} \]
        there exists an $h : C \to C'$ completing the diagram. Note: this $h$ may be non-unique.
        \item The \textit{octahedron axiom}.
    \end{enumerate}
\end{topic}

\begin{topic}{exact-functor-triangulated}{exact functor of triangulated categories}
    An \tref{additive-functor}{additive functor} between \tref{triangulated-category}{triangulated categories} $F : \mathcal{D} \to \mathcal{D}'$ is \textbf{exact} if
    \begin{enumerate}
        \item There exists a natural isomorphism $F \circ [1]_\mathcal{D} \xrightarrow{\sim} [1]_{\mathcal{D}'} \circ F$.
        \item $F$ maps distinguished triangles to distinguished triangles.
    \end{enumerate}
\end{topic}

\begin{topic}{triangulated-subcategory}{triangulated subcategory}
    Let $\mathcal{D}$ be a \tref{triangulated-category}{triangulated category}. Then a subcategory $\mathcal{D}' \subset \mathcal{D}$ is a \textbf{triangulated subcategory} if it admits the structure of a triangulated category such that the inclusion is \tref{exact-functor-triangulated}{exact}.
\end{topic}

\begin{topic}{admissible-subcategory}{admissible subcategory}
    Let $\mathcal{D}$ be a \tref{triangulated-category}{triangulated category}. A full \tref{triangulated-subcategory}{triangulated subcategory} $\mathcal{D}' \subset \mathcal{D}'$ is \textbf{admissible} if the inclusion has a right adjoint $\pi : \mathcal{D} \to \mathcal{D}'$.
    
    Equivalently, $\mathcal{D}'$ is admissible if for all $A \in \mathcal{D}$ there exists a distinguished triangle
    \[ B \to A \to C \to B[1] \]
    with $B \in \mathcal{D}'$ and $C \in {\mathcal{D}'}^\perp$ (the \tref{orthogonal-complement}{orthogonal complement}). Indeed, one may use $B = \pi(A)$.
\end{topic}

\begin{topic}{orthogonal-complement}{orthogonal complement}
    Let $\mathcal{D}$ be a \tref{triangulated-category}{triangulated category}. The (right) \textbf{orthogonal complement} of a subcategory $\mathcal{D}' \subset \mathcal{D}$ is the full subcategory ${\mathcal{D}'}^\perp$ of all objects $A \in \mathcal{D}$ such that $\Hom(B, A) = 0$ for all $B \in \mathcal{D}'$.
\end{topic}

\begin{topic}{exceptional-sequence}{exceptional sequence}
    Let $\mathcal{D}$ be a $k$-linear \tref{triangulated-category}{triangulated category}. An object $E \in \mathcal{D}$ is called \textbf{exceptional} if
    \[ \Hom(E, E[\ell]) = \left\{\begin{array}{cl} k & \text{ if } \ell = 0, \\ 0 & \text{ otherwise.} \end{array}\right. \]
    An \textbf{exceptional sequence} is a sequence $E_1, \ldots, E_n$ of exceptional objects such that
    \[ \Hom(E_j, E_i[\ell]) = 0 \qquad \text{for $i < j$ and all $\ell$}. \]
\end{topic}

\begin{topic}{semi-orthogonal-decomposition}{semi-orthogonal decomposition}
    Let $\mathcal{D}$ be a \tref{triangulated-category}{triangulated category}. A sequence of full \tref{admissible-subcategory}{admissible} \tref{triangulated-subcategory}{triangulated subcategories} $\mathcal{D}_1, \mathcal{D}_2, \ldots, \mathcal{D}_n \subset \mathcal{D}$ is \textbf{semi-orthogonal} if
    \[ \mathcal{D}_i \subset \mathcal{D}_j^\perp \qquad \text{for all $i < j$}. \]
    Such a sequence defines a \textbf{semi-orthogonal decomposition} of $\mathcal{D}$ if $\mathcal{D}$ is generated by the $\mathcal{D}_i$. In this case one writes
    \[ \mathcal{D} = \langle \mathcal{D}_1, \mathcal{D}_2, \ldots, \mathcal{D}_n \rangle . \]
\end{topic}
