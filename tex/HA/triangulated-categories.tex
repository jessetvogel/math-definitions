\begin{topic}{triangulated-category}{triangulated category}
    Let $\mathcal{D}$ be an \tref{additive-category}{additive category}. The structure of a \textbf{triangulated category} on $\mathcal{D}$ is given by an additive equivalence
    \[ [1] : \mathcal{D} \to \mathcal{D} \qquad \text{(the shift functor)} \]
    and a set of \textbf{distinguished triangles}
    \[ A \to B \to C \to A[1] \]
    such that
    \begin{enumerate}[(1)]
        \item \begin{itemize}
            \item Any $A \to A \to 0 \to A[1]$ is distinguished.
            \item Any triangle isomorphic to a distinguished triangle is distinguished.
            \item Any morphism $f : A \to B$ can be completed to a distinguished triangle $A \overset{f}{\to} B \to C \to A[1]$.
        \end{itemize}
        \item A triangle
        \[ A \xrightarrow{f} B \xrightarrow{g} C \xrightarrow{h} A[1] \]
        is distinguished if and only if
        \[ B \xrightarrow{g} C \xrightarrow{h} A[1] \xrightarrow{-f[1]} B[1] \]
        is.
        \item Given a commutative diagram of distinguished triangles with vertical $f$ and $g$,
        \[ \begin{tikzcd}
            A \arrow{r} \arrow{d}{f} & B \arrow{r} \arrow{d}{g} & C \arrow{r} \arrow[dashed]{d}{h} & A[1] \arrow{d}{f[1]} \\
            A' \arrow{r} & B' \arrow{r} & C' \arrow{r} & A'[1]
        \end{tikzcd} \]
        there exists an $h : C \to C'$ completing the diagram. Note: this $h$ may be non-unique.
        \item (\textit{octahedron axiom}) Given exact triangles
        \[ X \xrightarrow{u} Y \xrightarrow{j} Z' \xrightarrow{k} X[1] \]
        \[ Y \xrightarrow{v} Z \xrightarrow{\ell} X' \xrightarrow{i} Y[1] \]
        \[ X \xrightarrow{vu} Z \xrightarrow{m} Y' \xrightarrow{n} X[1] , \]
        there exists an exact triangle
        \[ Z' \xrightarrow{f} Y' \xrightarrow{g} X' \xrightarrow{h} Z'[1] , \]
        such that 
        \[ \ell = gm, \quad k = nf, \quad h = j[1]i, \quad ig = u[1]n \quad \text{ and } \quad fj = mv . \]
        \[ \begin{tikzcd}[row sep=4em, column sep=5em] & Y' \arrow{rd}{g} & \\ Z' \arrow{ur}{f} \arrow[d, swap,"k","{[}1{]}"'] \ar[from=ddr, crossing over]{}{j} & & X' \arrow[ddl, swap, "i", "{[}1{]}"'] \arrow[ll, "h"',"{[}1{]}"] \\ X \arrow[swap]{dr}{u} \arrow[crossing over]{rr}{vu} \arrow[swap, from=uur, crossing over, "n", "{[}1{]}"'] & & Z \arrow[swap]{u}{\ell} \arrow[crossing over, swap]{uul}{m} \\ & Y \arrow[swap]{ur}{v} & \end{tikzcd} \]
    \end{enumerate}
\end{topic}

\begin{topic}{exact-functor-triangulated}{exact functor of triangulated categories}
    An \tref{additive-functor}{additive functor} between \tref{triangulated-category}{triangulated categories} $F : \mathcal{D} \to \mathcal{D}'$ is \textbf{exact} if
    \begin{itemize}
        \item there exists a natural isomorphism $F \circ [1]_\mathcal{D} \xrightarrow{\sim} [1]_{\mathcal{D}'} \circ F$,
        \item $F$ maps distinguished triangles to distinguished triangles.
    \end{itemize}
\end{topic}

\begin{topic}{triangulated-subcategory}{triangulated subcategory}
    Let $\mathcal{T}$ be a \tref{triangulated-category}{triangulated category}. Then a subcategory $\mathcal{S} \subset \mathcal{T}$ is a \textbf{triangulated subcategory} if it admits the structure of a triangulated category such that the inclusion is \tref{exact-functor-triangulated}{exact}.
\end{topic}

\begin{topic}{admissible-subcategory}{admissible subcategory}
    Let $\mathcal{T}$ be a \tref{triangulated-category}{triangulated category}. A full \tref{triangulated-subcategory}{triangulated subcategory} $\mathcal{S} \subset \mathcal{T}$ is \textbf{right admissible} (resp. \textbf{left admissible}) if the inclusion $i : \mathcal{S} \to \mathcal{T}$ has a right (resp. left) adjoint $\pi : \mathcal{T} \to \mathcal{S}$. If it is both left and right admissible, it is called \textbf{admissible}.
\end{topic}

\begin{example}{admissible-subcategory}
    The following are equivalent.
    \begin{enumerate}[(i)]
        \item The full triangulated subcategory $\mathcal{S} \subset \mathcal{T}$ is right admissible.
        \item For all $X \in \mathcal{T}$ there exists a distinguished triangle
        \[ A \to X \to B \to A[1] \]
        with $A \in \mathcal{S}$ and $B \in {\mathcal{S}}^\perp$, the \tref{orthogonal-complement}{orthogonal complement}.
    \end{enumerate}
    \begin{proof}
        $(i \Rightarrow ii)$ Let $\pi : \mathcal{T} \to \mathcal{S}$ be a right adjoint and take $A = \pi(X)$. The transpose of $\id_{\pi(X)}$ under the adjunction is a morphism $A \to X$, and fits into some exact triangle $A \to X \to B \to A[1]$. Now for any $D \in \mathcal{S}$ we have
        \[ \Hom_\mathcal{T}(D, A) = \Hom_{\mathcal{S}'}(D, \pi(X)) \simeq \Hom_\mathcal{T}(i(D), X) = \Hom_\mathcal{T}(D, X) , \]
        so from the long exact sequence
        \[ \cdots \to \Hom_\mathcal{T}(D, A) \to \Hom_\mathcal{T}(D, X) \to \Hom_\mathcal{T}(D, B) \to \Hom_\mathcal{T}(D, A[1]) \to \cdots \]
        follows that $\Hom_\mathcal{T}(D, B) = 0$, that is, $B \in \mathcal{S}^\perp$.
        $(ii \Rightarrow i)$ For each $X \in \mathcal{T}$, choose an exact triangle $A_X \to X \to B_X \to A_X[1]$ as described. Define a functor $\pi : \mathcal{T} \to \mathcal{S}$ by $\pi(X) = A_X$. For a morphism $f : X \to Y$, note that $\Hom_\mathcal{T}(A_X, A_Y) \simeq \Hom_\mathcal{T}(A_X, Y)$ since $B_Y \in \mathcal{S}^\perp$ (again use the long exact sequence). Hence, there is a unique morphism $f' : A_X \to A_Y$ making the obvious square commute, and we set $\pi(f) = f'$. One can check that this indeed gives a functor, and that $\pi$ is right adjoint to $i$.
    \end{proof}
\end{example}

\begin{topic}{orthogonal-complement}{orthogonal complement}
    Let $\mathcal{T}$ be a \tref{triangulated-category}{triangulated category}. The (right) \textbf{orthogonal complement} of a subcategory $\mathcal{S} \subset \mathcal{T}$ is the full subcategory $\mathcal{S}^\perp$ of all objects $A \in \mathcal{T}$ such that $\Hom_\mathcal{T}(B, A) = 0$ for all $B \in \mathcal{S}$.
\end{topic}

\begin{topic}{spanning-class}{spanning class}
    Let $\mathcal{T}$ be a \tref{triangulated-category}{triangulated category}. A collection $\Omega$ of objects in $\mathcal{T}$ is a \textbf{spanning class} (or spans $\mathcal{T}$) if for all $B \in \mathcal{T}$ we have
    \begin{itemize}
        \item $\Hom(A, B[i]) = 0$ for all $A \in \Omega$ and $i \in \ZZ$ implies $B \simeq 0$,
        \item $\Hom(B[i], A) = 0$ for all $A \in \Omega$ and $i \in \ZZ$ implies $B \simeq 0$.
    \end{itemize}
\end{topic}

\begin{topic}{exceptional-sequence}{exceptional sequence}
    Let $\mathcal{D}$ be a $k$-linear \tref{triangulated-category}{triangulated category}. An object $E \in \mathcal{D}$ is called \textbf{exceptional} if
    \[ \Hom_\mathcal{D}(E, E[\ell]) = \left\{\begin{array}{cl} k & \text{ if } \ell = 0, \\ 0 & \text{ otherwise.} \end{array}\right. \]
    An \textbf{exceptional sequence} is a sequence $E_1, \ldots, E_n$ of exceptional objects such that
    \[ \Hom_\mathcal{D}(E_i, E_j[\ell]) = 0 \qquad \textup{for $i > j$ and all $\ell$}. \]
    Such an exceptional sequence is \textbf{strong} if moreover
    \[ \Hom_\mathcal{D}(E_i, E_j[\ell]) = 0 \qquad \textup{for all $i, j$ and $\ell \ne 0$}. \]
    If $\mathcal{D}$ is a \tref{triangulated-category}{triangulated category}, such an exceptional sequence is \textbf{full} if the objects $E_1, \ldots, E_n$ generate $\mathcal{D}$.
\end{topic}

% \begin{example}{exceptional-sequence}
%     Let $X$ be a connected smooth projective variety, and $\mathcal{D} = \textbf{D}^\textup{b}(X)$. Then any line bundle $\mathcal{L}$ over $X$ is exceptional, since
%     \[ \Hom_{\mathcal{O}_X}(\mathcal{L}, \mathcal{L}) = k \]
%     and
%     \[ \textup{Ext}^i(\mathcal{L}, \mathcal{L}) = 0 \]
%     for $i > 0$ since $\mathcal{L}$ is locally free.
% \end{example}

\begin{example}{exceptional-sequence}
    For $\mathcal{D} = \textbf{D}^\textup{b}(\textbf{FinVect}_k)$, the bounded derived category of finite vector spaces over $k$, the one-dimensional vector space $k$ is exceptional, and moreover, it generates $\mathcal{D}$ as a triangulated category.

    For general $\mathcal{D}$, the subcategory $\langle E \rangle$ generated by an exceptional object $E \in \mathcal{D}$ is equivalent to $\textbf{D}^\textup{b}(\textbf{FinVect}_k)$. For example, every object in $\langle E \rangle$ is isomorphic to a direct sum of shifts of $E$.
\end{example}

\begin{topic}{semi-orthogonal-decomposition}{semi-orthogonal decomposition}
    Let $\mathcal{D}$ be a \tref{triangulated-category}{triangulated category}. A sequence of full \tref{admissible-subcategory}{admissible} \tref{triangulated-subcategory}{triangulated subcategories} $\mathcal{D}_1, \mathcal{D}_2, \ldots, \mathcal{D}_n \subset \mathcal{D}$ is \textbf{semi-orthogonal} if
    \[ \mathcal{D}_i \subset \mathcal{D}_j^\perp \qquad \text{for all $i < j$}. \]
    Such a sequence defines a \textbf{semi-orthogonal decomposition} of $\mathcal{D}$ if $\mathcal{D}$ is generated by the $\mathcal{D}_i$. In this case one writes
    \[ \mathcal{D} = \langle \mathcal{D}_1, \mathcal{D}_2, \ldots, \mathcal{D}_n \rangle . \]
\end{topic}

\begin{example}{semi-orthogonal-decomposition}
    Given an \tref{exceptional-sequence}{exceptional sequence} $E_1, \ldots, E_n$ in $\mathcal{D}$, the subcategories
    \[ \mathcal{D}_1 = \langle E_1 \rangle, \ldots, \mathcal{D}_n = \langle E_n \rangle , \]
    form a semi-orthogonal sequence, and if the $E_i$ generate $\mathcal{D}$, then they form a semi-orthogonal decomposition.
\end{example}

\begin{example}{semi-orthogonal-decomposition}
    The \tref{derived-category}{derived category} $\textbf{D}^\textup{b}(\PP^n)$ of coherent sheaves on projective space has a semi-orthogonal decomposition
    \[ \textbf{D}^\textup{b}(\PP^n) = \langle \mathcal{O}(-n), \mathcal{O}(-n + 1), \ldots, \mathcal{O}(-1), \mathcal{O} \rangle . \]
    Indeed, the sequence $\mathcal{O}(-n), \ldots, \mathcal{O}$ is exceptional since
    \[ \textup{Ext}^r(\mathcal{O}(i), \mathcal{O}(j)) \simeq \textup{Ext}^r(\mathcal{O}, \mathcal{O}(j - i)) \simeq H^r(\PP^n, \mathcal{O}(j - i)) = \left\{ \begin{array}{cl}
         k & \textup{ if } r = 0, i \le j , \\
         k & \textup{ if } r = n, i - j < n + 1 , \\
         0 & \textup{ otherwise}.
    \end{array} \right. \]
    Furthermore, any coherent sheaf on $\PP^n$ has a finite resolution by direct sums of sheaves $\mathcal{O}(i)$, and combined with the Koszul complex (and tensor products with $\mathcal{O}(i)$ thereof),
    \[ 0 \to \mathcal{O} \to \cdots \to \mathcal{O}(i)^{\binom{n + 1}{n + 1 - i}} \to \cdots \to \mathcal{O}(n)^{n + 1} \to \mathcal{O}(n + 1) \to 0 , \]
    it follows that the sheaves $\mathcal{O}(-n), \ldots, \mathcal{O}$ generate $\textbf{D}^\textup{b}(\PP^n)$.
\end{example}

\begin{topic}{thick-subcategory}{thick subcategory}
    Let $\mathcal{T}$ be a \tref{triangulated-category}{triangulated category}. A \textbf{thick subcategory} $\mathcal{S}$ is a \tref{CT:full-subcategory}{full} \tref{triangulated-subcategory}{triangulated subcategory} which is closed under taking summands.
\end{topic}

\begin{topic}{verdier-quotient}{Verdier quotient}
    Let $\mathcal{T}$ be a \tref{triangulated-category}{triangulated category} and $\mathcal{S} \subset \mathcal{T}$ a \tref{triangulated-subcategory}{triangulated subcategory}. The \textbf{Verdier quotient} $\mathcal{T}/\mathcal{S}$ is a triangulated category equipped with an \tref{additive-functor}{additive} and \tref{exact-functor-triangulated}{exact} quotient functor
    \[ Q : \mathcal{T} \to \mathcal{T}/\mathcal{S} , \]
    which is universal among all functors $\mathcal{T} \to \mathcal{D}$ which send objects of $\mathcal{S}$ to objects isomorphic to zero.
    \[ \begin{tikzcd} \mathcal{T} \arrow{r}{Q} \arrow{rd} & \mathcal{T} / \mathcal{S} \arrow[dashed]{d} \\ & \mathcal{D} \end{tikzcd} \]
\end{topic}

\begin{example}{verdier-quotient}
    Let $\mathcal{S} \subset \mathcal{T}$ be a \tref{admissible-subcategory}{right admissible subcategory}. We will show that the quotient $\mathcal{T}/\mathcal{S}$ is equivalent to the \tref{orthogonal-complement}{orthogonal complement} $S^\perp$. Since $\mathcal{S}$ is right admissible, any object $X$ of $\mathcal{T}$ fits in an exact triangle
    \[ \pi(X) \to X \to Q(X) \to \pi(X)[1] , \]
    where $\pi : \mathcal{T} \to \mathcal{S}$ denotes a \tref{CT:adjunction}{right adjoint} to the inclusion $i : \mathcal{S} \to \mathcal{T}$, and $Q(X)$ lies in $\mathcal{S}^\perp$. Moreover, for any morphism $f : X \to Y$ we obtain an exact sequence
    \[ 0 = \Hom(\pi(X)[1], Q(Y)) \to \Hom(Q(X), Q(Y)) \to \Hom(X, Q(Y)) \to \Hom(\pi(X), Q(Y)) = 0 \]
    which shows there is a unique morphism $Q(f) : Q(X) \to Q(Y)$ making the natural square commute.
    Now by uniqueness of $Q(f)$, we obtain a functor
    \[ Q : \mathcal{T} \to S^\perp , \]
    which is easily seen to be additive and exact.
    
    Universality is easily verified: any other additive and exact $F : \mathcal{T} \to \mathcal{D}$ that sends objects of $\mathcal{S}$ to zero, yields $\mathcal{S}^\perp \to \mathcal{D}$ by the restricting $F$ to $S^\perp$. And indeed, applying $F$ to the above exact triangle gives $F(X) \simeq F(Q(X))$.
\end{example}

\begin{topic}{tensor-triangulated-category}{tensor triangulated category}
    A \textbf{tensor triangulated category} is a \tref{triangulated-category}{triangulated category} $\mathcal{T}$ with a \tref{CT:symmetric-monoidal-category}{symmetric monoidal structure} $\otimes : \mathcal{T} \times \mathcal{T} \to \mathcal{T}$ such that $X \otimes (-)$ and $(-) \otimes X$ are \tref{exact-functor-triangulated}{exact functors} for all $X$ in $\mathcal{T}$.
\end{topic}
