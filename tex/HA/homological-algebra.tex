% Abelian categories
\begin{topic}{additive-category}{additive category}
    An \textbf{additive category} is a category $\mathcal{A}$ such that
    \begin{itemize}
        \item every $\Hom_\mathcal{A}(A, B)$ is an abelian group, and composition is bilinear,
        \item there is a \tref{CT:zero-object}{zero object} $0$ which is both \tref{CT:terminal-object}{terminal} and \tref{CT:initial-object}{initial},
        \item finite direct sums and finite direct products exist, and coincide.
    \end{itemize}
\end{topic}

\begin{topic}{abelian-category}{abelian category}
    An \textbf{abelian category} is an \tref{additive-category}{additive category} category $\mathcal{A}$ in which
    \begin{itemize}
        \item all kernels and cokernels exist,
        \item every monomoprhism is the kernel of some morphism, and every epimoprhism is the cokernel of some morphism.
    \end{itemize}
\end{topic}

\begin{example}{abelian-category}
    For a \tref{AA:ring}{ring} $R$, the category of left (or right) \tref{AA:module}{$R$-modules} is an abelian category.
    
    In particular, the category of abelian groups (that is, $\ZZ$-modules) $\textbf{Ab}$ is an abelian category.
\end{example}

\begin{topic}{additive-functor}{additive functor}
    Let $\mathcal{A}$ and $\mathcal{B}$ be \tref{additive-category}{additive categories}. A functor $F : \mathcal{A} \to \mathcal{B}$ is called \textbf{additive} if for all $A, B$ in $\mathcal{A}$ the map
    \[ \Hom_\mathcal{A}(A, B) \to \Hom_\mathcal{B}(FA, FB) \]
    is a group morphism.
    
    It follows that an additive functor preserves finite direct sums and sends $0$ to $0$.
\end{topic}

\begin{topic}{exact-sequence}{(short) exact sequence}
    Let $\mathcal{A}$ be an \tref{abelian-category}{abelian category}. An \textbf{exact sequence} is a sequence of objects and morphisms
    \[ \cdots \rightarrow A_{i - 1} \xrightarrow{f_{i - 1}} A_i \xrightarrow{f_i} A_{i + 1} \rightarrow \cdots \]
    such that $\ker f_i = \im f_{i - 1}$ for all $i$.
    A \textbf{short exact sequence} is an exact sequence of the form
    \[ 0 \to A \xrightarrow{f} B \xrightarrow{g} C \to 0 . \]
    Concretely, this means that $f$ is injective, $g$ is surjective, and $\ker g = \im f$.
\end{topic}

\begin{example}{exact-sequence}
    Any exact sequence
    \[ \cdots \rightarrow A_{i - 1} \xrightarrow{f_{i - 1}} A_i \xrightarrow{f_i} A_{i + 1} \rightarrow \cdots \]
    can be split up into short exact sequences: let $B_i = \ker f_i = \im f_{i + 1}$, then we obtain short exact sequences
    \[ 0 \to B_i \to A_i \to B_{i + 1} \to 0 \]
    for each $i$.
\end{example}

\begin{topic}{exact-functor}{exact functor}
    Let $F : \mathcal{A} \to \mathcal{B}$ be an \tref{additive-functor}{additive functor} between \tref{abelian-category}{abelian categories}. Then $F$ is called \textbf{left exact} if for all short exact sequences
    \[ 0 \to M \to N \to L \to 0 \]
     in $\mathcal{A}$, the sequence
    \[ 0 \to FM \to FN \to FL \]
    is exact. Similarly, $F$ is called \textbf{right exact} if for all such short exact sequences
    \[ FM \to FN \to FL \to 0 \]
    is exact. Finally, $F$ is called \textbf{exact} if it is both left and right exact.
\end{topic}

\begin{topic}{injective-object}{injective object}
    Let $\mathcal{A}$ be an \tref{abelian-category}{abelian category}. An object $I$ of $\mathcal{A}$ is called \textbf{injective} if the functor
    \[ \Hom_\mathcal{A}(-, I) : \mathcal{A}^\text{op} \to \textbf{Ab} \]
    is \tref{exact-functor}{exact}. Since it is already left exact for any object $I$, this is equivalent to the condition that for all monomorphisms $f : M \to N$ and morphisms $g : M \to I$ there exists a morphism $h : N \to I$ such that $hf = g$.
    \[ \begin{tikzcd} M \arrow[hookrightarrow]{r}{f} \arrow[swap]{d}{g} & N \arrow[dashed]{ld}{\exists h} \\ I & \end{tikzcd} \]
\end{topic}

\begin{topic}{projective-object}{projective object}
    Let $\mathcal{A}$ be an \tref{abelian-category}{abelian category}. An object $P$ of $\mathcal{A}$ is called \textbf{projective} if the functor
    \[ \Hom_\mathcal{A}(P, -) : \mathcal{A} \to \textbf{Ab} \]
    is \tref{exact-functor}{exact}. Since it is already left exact for any object $P$, this is equivalent to the condition that for all epimorphisms $f : N \to M$ and morphisms $g : P \to M$ there exists a morphism $h : P \to N$ such that $fh = g$.
    \[ \begin{tikzcd} & N \arrow[twoheadrightarrow]{d}{f} \\ P \arrow[swap]{r}{g} \arrow[dashed]{ur}{\exists h} & M \end{tikzcd} \]
\end{topic}

% Complexes
\begin{topic}{chain-complex}{(co)chain complex}
    Let $\mathcal{A}$ be an \tref{abelian-category}{abelian category}. A \textbf{chain complex} $A_\bdot$ in $\mathcal{A}$ is a sequence of objects $A_i$ in $\mathcal{A}$, for $i \in \ZZ$, together with morphisms $d_i : A_i \to A_{i - 1}$ such that $d_{i - 1} \circ d_i = 0$ for all $i$.
    \[ \cdots \xleftarrow{d_{i - 1}} A_{i - 1} \xleftarrow{d_i} A_i \xleftarrow{d_{i + 1}} A_{i + 1} \xleftarrow{d_{i + 2}} \cdots \]
    Dually, a \textbf{cochain complex} $A^\bdot$ in $\mathcal{A}$ is a sequence of objects $A^i$ in $\mathcal{A}$, for $i \in \ZZ$, together with morphisms $d^i : A^i \to A^{i + 1}$ such that $d^{i + 1} \circ d^i = 0$ for all $i$.
    \[ \cdots \xrightarrow{d^{i - 2}} A^{i - 1} \xrightarrow{d^{i - 1}} A^i \xrightarrow{d^i} A^{i + 1} \xrightarrow{d^{i + 1}} \cdots \]
    
    A \textbf{morphism of chain complexes} or \textbf{chain map} $f : A_\bdot \to B_\bdot$ is given by a collection of morphisms $f_i : A_i \to B_i$ for each $i \in \ZZ$ such that $d^B_i \circ f_i = f_{i - 1} \circ d^A_i$. Dually, there are \textbf{cochain maps}.
    
    If the objects of $\mathcal{A}$ are sets, then elements of $A_n$ are called \textit{$n$-chains}, elements of $\ker d_n$ are called \textit{$n$-cycles}, and elements of $\im d_{n - 1}$ are called \textit{$n$-boundaries}. Similarly, elements of $A^n$ are called \textit{$n$-cochains}, elements of $\ker d^n$ are called \textit{$n$-cocycles}, and elements of $\im d^{n - 1}$ are called \textit{$n$-coboundaries}. Note that any $n$-(co)boundary is an $n$-(co)cycle.
\end{topic}

\begin{topic}{homology-object}{(co)homology object}
    Let $\mathcal{A}$ be an \tref{abelian-category}{abelian category}. The \textbf{$i$-th homology object} of a \tref{chain-complex}{chain complex} $A_\bdot$ is the quotient
    \[ H_i(A_\bdot) = \ker d_i / \im d_{i + 1} . \]
    Dually, the \textbf{$i$-th cohomology object} of a cochain complex $A^\bdot$ is the quotient
    \[ H^i(A^\bdot) = \ker d^i / \im d^{i - 1} . \]
\end{topic}

\begin{topic}{long-exact-sequence-homology}{long exact sequence in homology}
    Let $\mathcal{A}$ be an \tref{abelian-category}{abelian category}, and let
    \[ 0 \rightarrow A^\bdot \rightarrow B^\bdot \rightarrow C^\bdot \rightarrow 0 \]
    be a short exact sequence of \tref{chain-complex}{complexes} in $\mathcal{A}$. Then there is an induced long exact sequence
    \[ \cdots \rightarrow H^i(A^\bdot) \rightarrow H^i(B^\bdot) \rightarrow H^i(C^\bdot) \rightarrow H^{i + 1}(A^\bdot) \rightarrow \cdots . \]
\end{topic}

\begin{topic}{chain-homotopy}{chain homotopy}
    Let $\mathcal{A}$ be an \tref{abelian-category}{abelian category}, and let $f, g : A^\bdot \to B^\bdot$ be morphisms of \tref{chain-complex}{chain complexes} in $\mathcal{A}$. A \textbf{chain homotopy} from $f$ to $g$ is a collection of morphisms $k^i : A^i \to B^{i - 1}$ such that
    \[ f^i - g^i = d_B^{i - 1} k^i + k^{i + 1} d_A^i . \]
    If a homotopy exists from $f$ to $g$, we say that $f$ and $g$ are \textbf{homotopic}. This gives an equivalence relation on the set of morphisms from $A^\bdot$ to $B^\bdot$.
\end{topic}

\begin{example}{chain-homotopy}
    Homotopic maps $f, g : A^\bdot \to B^\bdot$ yield the same map on the level of homology. Namely, if $f - g = dk + kd$, then for any $x \in H^i(A^\bdot)$ we have
    \[ f(x) - g(x) = d_B^{i - 1}(k^i(x)) + k^{i + 1}(d_A^i(x)) = 0 , \]
    since $d_B^{i - 1}(k^i(x)) = 0 \in H^{i}(B^\bdot)$ and $d_A^i(x) = 0 \in H^{i + 1}(A^\bdot)$, since $x \in \ker d_A^i$.
\end{example}

\begin{topic}{chain-homotopy-equivalence}{chain homotopy equivalence}
    Let $\mathcal{A}$ be an \tref{abelian-category}{abelian category}. A morphism of \tref{chain-complex}{chain complexes} $f : M^\bdot \to N^\bdot$ is called a \textbf{chain homotopy equivalence} if there exists a morphism $g : N^\bdot \to M^\bdot$ such that $fg$ and $gf$ are \tref{chain-homotopy}{homotopic} to the identity morphism.
\end{topic}

\begin{example}{chain-homotopy-equivalence}
    Consider $R = \ZZ \times \ZZ$ and the $R$-module $M = \ZZ \times \{ 0 \}$. Then
    \[ 0 \to M \xrightarrow{\id} M \to 0 \qquad \textup{ and } \qquad  0 \to \{ 0 \} \times \ZZ \hookrightarrow \ZZ \times \ZZ \twoheadrightarrow M \to 0 \]
    are two projective resolutions of $M$ as an $R$-module, so they should be related by a homotopy equivalence. Calling the resolutions $A^\bdot$ and $B^\bdot$, respectively, define chain maps $f : A^\bdot \to B^\bdot$ and $g : B^\bdot \to A^\bdot$ as
    \[ \begin{tikzcd}
        0 \arrow{r} & 0 \arrow[bend left=20]{d} \arrow{r} & M \arrow[bend left=20]{d}{i} \arrow{r} & 0 \\
        0 \arrow{r} & \{ 0 \} \times \ZZ \arrow[bend left=20]{u} \arrow[hookrightarrow]{r} & \ZZ \times \ZZ \arrow[bend left=20]{u}{r} \arrow{r} & 0
    \end{tikzcd} \]
    with $i(0, b) = (0, b)$ and $r(a, b) = (0, b)$. Indeed, $gf = \id$ and $fg - \id = dk + kd$ for
    \[ k^0 : \ZZ \times \ZZ \to \{ 0 \} \times \ZZ, \quad (a, b) \mapsto (0, b) . \]
    This shows that $f$ and $g$ are homotopy equivalences. In particular, the compositions $fg$ and $gf$ are \tref{quasi-isomorphism}{quasi-isomorphisms}.
\end{example}

\begin{topic}{quasi-isomorphism}{quasi-isomorphism}
    Let $\mathcal{A}$ be an \tref{abelian-category}{abelian-category}. A morphism $f : M^\bdot \to N^\bdot$ of \tref{chain-complex}{chain complexes} in $\mathcal{A}$ is called a \textbf{quasi-isomorphism} if the induced maps on homology
    \[ H^i(f) : H^i(M^\bdot) \to H^i(N^\bdot) \]
    are all isomorphisms.
\end{topic}

\begin{example}{quasi-isomorphism}
    Quasi-isomorphisms are generally not invertible. For example, the morphism of complexes
    \[ \begin{tikzcd} \cdots \arrow{r} & 0 \arrow{d} \arrow{r} & \ZZ \arrow{r}{\cdot 2} \arrow{d} & \ZZ \arrow{r} \arrow{d} & 0 \arrow{r} \arrow{d} & \cdots \\ \cdots \arrow{r} & 0 \arrow{r} & 0 \arrow{r} & \ZZ/2\ZZ \arrow{r} & 0 \arrow{r} & \cdots  \end{tikzcd} \]
    is a quasi-isomorphism, but not invertible.
    
    Also there are complexes with isomorphic homology, but which are not quasi-isomorphic. For example,
    \[ 0 \to \CC[x, y] \oplus \CC[x, y] \overset{\varphi}{\longrightarrow} \CC[x, y] \to 0 \qquad \text{ and } \qquad 0 \to \CC[x, y] \overset{0}{\longrightarrow} \CC \to 0 \]
    with $\varphi(f, g) = xf + yg$.
\end{example}

\begin{topic}{injective-resolution}{injective resolution}
    Let $\mathcal{A}$ be an \tref{abelian-category}{abelian category} and $M$ an object in $\mathcal{A}$. An \textbf{injective resolution} of $M$ is an exact sequence
    \[ 0 \to M \to I^0 \to I^1 \to \cdots \]
    where all $I^i$ are \tref{injective-object}{injective objects}. Equivalently, this can be seen as a \tref{quasi-isomorphism}{quasi-isomorphism} $M[0] \to I^\bdot$ in $\text{Comp}(\mathcal{A})$.
\end{topic}

\begin{example}{injective-resolution}
    Let $A, B$ be objects of $\mathcal{A}$, and $f : A \to B$ a morphism. Let $0 \to A \to I_A^\bdot$ and $0 \to B \to I_B^\bdot$ be injective resolutions of $A$ and $B$, respectively. Then the map $f : A \to B$ can be lifted to a map $f^\bdot : I_A^\bdot \to I_B^\bdot$ such that $H^0(f^\bdot) = f$. Moreover, any two such liftings of $f$ are homotopic. In particular, for $A = B$ and $f = \id_A$, this shows that an injective resolution of $A$ is unique up to homotopy equivalence. Moreover, there is a bijection
    \[ \Hom_{\textbf{K}(\mathcal{A})}(I_A^\bdot, I_B^\bdot) \simeq \Hom_{\mathcal{A}}(A, B) , \]
    where $\textbf{K}(\mathcal{A})$ is the \tref{homotopy-category}{homotopy category} of $\mathcal{A}$.
    \begin{proof}
        First we construct one such lift $f^\bdot$. Since $A \to I_A^0$ is injective, the map $d_B \circ f : A \to I_B^0$ lifts to some $f^0 : I_A^0 \to I_B^0$ by injectivity of $I_B^0$. Suppose by induction that we have defined $f^k$ for $k \le n$ such that $d_B \circ f^k = f^{k + 1} \circ d_A$ for all $k < n$. Note that $d_B \circ f^n$ sends $\ker d_A^n$ to $\im (d_B \circ f^n \circ d_A^{n - 1}) = \im (d_B \circ d_B \circ f^{n - 1}) = 0$. Hence $d_B \circ f^n$ descends to a map $d_A(I_A^n) \to I_B^{n + 1}$, and by injectivity of $I_B^{n + 1}$ this lifts to a map $f^{n + 1} : I_A^{n + 1} \to I_B^{n + 1}$. By construction this map satisfies $f^{n + 1} \circ d_A = d_B \circ f^n$. Indeed, as $i_B \circ f = f^0 \circ i_A$, we have $H^0(f^\bdot) = f$.
    
        To show two such liftings $f_1^\bdot$ and $f_2^\bdot$ are homotopic, we can equivalently show that $f_1^\bdot - f_2^\bdot$ is homotopic to the zero map $I_A^\bdot \to I_B^\bdot$. Note that $f_1^\bdot - f_2^\bdot$ lifts the zero map $A \to B$, so it suffices to take $f = 0$ and show that the obtained map $f^\bdot$ is of the form $f^n = d_B \circ h^n + h^{n + 1} \circ d_A$.
    
        Since $f^0(A) = 0$, the map $f^0$ descends to a map $d(I_A^0) \to I_B^0$, which lifts to a map $h^1 : I_A^1 \to I_B^0$ by injectivity of $I_B^0$. Again by induction, assume that we have maps $h^k : I_A^k \to I_B^{k - 1}$ for all $k < n$, such that $f^k = d_B \circ h^k + h^{k + 1} \circ d_A$ for all $k < n$. Then, note that $f^n - d_B h^n$ sends $\ker d_A^n$ to
        \[ \im ((f^n - d_B \circ h^n) \circ d_A) = \im (d_B \circ f^{n - 1} - d_B \circ (f^{n - 1} - d_B \circ h^{n - 1})) = 0 , \]
        so $f^n - d_B \circ h^n$ descends to a map $d(I_A^n) \to I_B^n$ and by injectivity of $I_B^n$ lifts to a map $h^{n + 1} : I_A^{n + 1} \to I_B^n$, which satisfies $f^n = d_B \circ h^n + h^{n + 1} \circ d_A$ by construction.
    \end{proof}
\end{example}

\begin{topic}{projective-resolution}{projective resolution}
    Let $\mathcal{A}$ be an \tref{abelian-category}{abelian category} and $M$ an object in $\mathcal{A}$. A \textbf{projective resolution} of $M$ is an exact sequence
    \[ \cdots \to P^1 \to P^0 \to M \to 0 \]
    where all $P^i$ are \tref{projective-object}{projective objects}. Equivalently, this can be seen as a \tref{quasi-isomorphism}{quasi-isomorphism} $M[0] \to P^\bdot$ in $\text{Comp}(\mathcal{A})$.
\end{topic}

% Derived functors
\begin{topic}{right-derived-functors}{right derived functors}
    Let $\mathcal{A}$ and $\mathcal{B}$ be \tref{abelian-category}{abelian categories}, assume that $\mathcal{A}$ has \tref{enough-injectives}{enough injectives}, and let $F : \mathcal{A} \to \mathcal{B}$ be a \tref{exact-functor}{left exact} functor. Take $M$ to be an object of $\mathcal{A}$ and pick an \tref{injective-resolution}{injective resolution} $M[0] \to I^\bdot$. Then we define
    \[ \text{R}^i F M = H^i(F(I^\bdot)) . \]
    This is well-defined (independent of the chosen resolution) and yields additive functors $\text{R}^i F : \mathcal{A} \to \mathcal{B}$, called the \textbf{right derived functors} of $F$. Note that $R^0 F \simeq F$.
    
    For each short exact sequence $0 \to A \to B \to C \to 0$ in $\mathcal{A}$ there is an associated long exact sequence
    \[ 0 \to FA \to FB \to FC \to \text{R}^1FA \to \text{R}^1FB \to \text{R}^1FC \to \text{R}^2FA \to \cdots \]
\end{topic}

\begin{topic}{left-derived-functors}{left derived functors}
    Let $\mathcal{A}$ and $\mathcal{B}$ be \tref{abelian-category}{abelian categories}, assume that $\mathcal{A}$ has \textit{enough projectives}, and let $F : \mathcal{A} \to \mathcal{B}$ be a \tref{exact-functor}{right exact} functor. Take $M$ to be an object of $\mathcal{A}$ and pick a \tref{projective-resolution}{projective resolution} $P^\bdot \to M[0]$. Then we define
    \[ \text{L}_i F M = H^i(F(P^\bdot)) . \]
    This is well-defined (independent of the chosen resolution) and yields additive functors $\text{L}_i F : \mathcal{A} \to \mathcal{B}$, called the \textbf{left derived functors} of $F$. Note that $\text{L}_0 F \simeq F$.
    
    For each short exact sequence $0 \to A \to B \to C \to 0$ in $\mathcal{A}$ there is an associated long exact sequence
    \[ \cdots \to \text{L}_2FC \to \text{L}_1FA \to \text{L}_1FB \to \text{L}_1FC \to FA \to FB \to FC \to 0 \]
\end{topic}

\begin{topic}{acyclic-object}{acyclic object}
    Let $F : \mathcal{A} \to \mathcal{B}$ be a \tref{exact-functor}{left exact functor} between \tref{abelian-category}{abelian categories}. An object $A$ of $\mathcal{A}$ is called \textbf{acyclic with respect to $F$}, or \textbf{$F$-acyclic}, if
    \[ \textup{R}^iF(A) = 0 \quad \textup{for all } i > 0, \]
    where $\textup{R}^i F$ are the \tref{right-derived-functors}{right derived functors} of $F$.
    
    Similarly, if $F : \mathcal{A} \to \mathcal{B}$ is a right exact functor, an object $A$ of $\mathcal{A}$ is \textbf{$F$-acyclic} if $\textup{L}_i F(A) = 0$ for all $i > 0$, where $\textup{L}_i F$ denote the \tref{left-derived-functors}{left derived functors} of $F$.
\end{topic}

\begin{example}{acyclic-object}
    Given a left (resp. right) exact functor $F : \mathcal{A} \to \mathcal{B}$ and an object $A$ of $\mathcal{A}$, an \textit{acyclic resolution} of $A$ is a resolution $0 \to A \to I^0 \to I^1 \to \cdots$ (resp. $\cdots \to P^1 \to P^0 \to A \to 0$) of $A$ where the $I^i$ (resp. $P^i$) are $F$-acyclic objects. The left (resp. right) derived functors can be computed as
    \[ \textup{R}^i F(A) = H^i(F(I^\bdot)) \qquad \left(\textup{resp. } \textup{L}_i F(A) = H^i(F(P^\bdot)) \right) . \]
\end{example}

\begin{example}{acyclic-object}
    \begin{itemize}
        \item Given a \tref{AA:ring}{commutative ring} $R$, the objects which are acyclic with respect to all functors $(-) \otimes_R M$, are the \tref{AA:flat-module}{flat modules} over $R$.
        \item The objects which are acyclic with respect to all functors $\Hom_\mathcal{A}(A, -)$, are the \tref{injective-object}{injective objects}. Moreover, an injective object is acyclic with respect to any left exact functor $F$.
        \item The objects which are acyclic with respect to all functors $\Hom_\mathcal{A}(-, A)$, are the \tref{projective-object}{projective objects}. Moreover, a projective object is acyclic with respect to any right exact functor $F$.
    \end{itemize}
\end{example}

% Homological lemmas
\begin{topic}{five-lemma}{five lemma}
    Let $\mathcal{A}$ be an \tref{abelian-category}{abelian category}, and consider the following commutative diagram.
    \[ \begin{tikzcd} A \arrow{r}{f} \arrow{d}{a} & B \arrow{r}{g} \arrow{d}{b} & C \arrow{r}{h} \arrow{d}{c} & D \arrow{r}{i} \arrow{d}{d} & E \arrow{d}{e} \\ A' \arrow{r}{f'} & B' \arrow{r}{g'} & C' \arrow{r}{h'} & D \arrow{r}{i'} & E' \end{tikzcd} \]
    Suppose that both rows are exact. The \textbf{five lemma} states that
    \begin{itemize}
        \item if $b$ and $d$ are injective and $a$ is surjective, then $c$ is injective.
        \item if $b$ and $d$ are surjective and $e$ is injective, then $c$ is surjective.
        \item if $b$ and $d$ are isomorphisms, $a$ is surjective and $e$ is injective, then $c$ is an isomorphism.
    \end{itemize}
\end{topic}

\begin{example}{five-lemma}
    \begin{proof}
        \begin{itemize}
            \item Take any $\gamma \in \ker c$. Then $dh(\gamma) = h'c(\gamma) = 0$, so $h(\gamma) = 0$ by injectivity of $d$, and thus we can write $\gamma = g(\beta)$ for some $\beta \in B$. Now, $g'b(\beta) = cg(\beta) = c(\gamma) = 0$, so $b(\beta) \in \ker g' = \im f'$, and thus we can write $b(\beta) = f'(\alpha')$ for some $\alpha' \in A'$. Since $a$ is surjective, we have $\alpha' = a(\alpha)$ for some $\alpha \in A$. Now $b(f(\alpha)) = f'a(\alpha) = f(\alpha') = b(\beta)$, so $f(\alpha) = \beta$ by injectivity of $b$. Therefore, $\gamma = gf(\alpha) = 0$, and thus $c$ is injective.
            \item The second statement follows from the first by duality.
            \item The third statement follows from the first and the second.
        \end{itemize}
    \end{proof}
\end{example}

\begin{topic}{snake-lemma}{snake lemma}
    Let $\mathcal{A}$ be an \tref{abelian-category}{abelian category}, and consider the following commutative diagram.
    \[ \begin{tikzcd} & A \arrow{d}{a} \arrow{r}{f} & B \arrow{d}{b} \arrow{r}{g} & C \arrow{d}{c} \arrow{r} & 0 \\ 0 \arrow{r} & A' \arrow{r}{f'} & B' \arrow{r}{g'} & C' \end{tikzcd} \]
    Suppose that the rows are exact. The \textbf{snake lemma} states that there is an exact sequence of kernels and cokernels
    \[ \ker a \rightarrow \ker b \rightarrow \ker c \xrightarrow{d} \coker a \rightarrow \coker b \rightarrow \coker c \]
    where $d$ is known as the \textit{connecting homomorphism}. Furthermore, if $f$ is \tref{CT:monomorphism}{mono} then so is $\ker a \to \ker b$, and if $g'$ is \tref{CT:epimorphism}{epi} then so is $\coker b \to \coker c$.
\end{topic}

\begin{topic}{spectral-sequence}{spectral sequence}
    Let $\mathcal{A}$ be an \tref{abelian-category}{abelian category}. A \textbf{spectral sequence} in $\mathcal{A}$ is a collection of objects
    \[ (E_r^{p, q}, E^n) \qquad n, p, q, r \in \ZZ, r \ge 1 \]
    and morphisms
    \[ d_r^{p, q} : E_r^{p, q} \to E_r^{p + r, q - r + 1} \]
    satisfying
    \begin{itemize}
        \item $d_r^2 = 0$ (i.e. the $E_r^{p + \bdot r, q - \bdot r + \bdot}$ are complexes),
        \item there are isomorphisms (which are part of the data)
        \[ E_{r + 1}^{p, q} \simeq H^0(E_r^{p + \bdot r, q - \bdot r + \bdot}), \]
        \item for any $(p, q)$ there exists an $r_0$ such that $d_r^{p, q} = d_r^{p - r, q + r - 1} = 0$ for all $r \ge r_0$. In particular, $E_r^{p, q} \simeq E_{r_0}^{p, q}$ for all $r \ge r_0$, and this object is denoted by $E_\infty^{p, q}$.
        \item There is a decreasing filtration
        \[ 0 \subset \cdots \subset F^{p + 1} E^n \subset F^p E^n \subset \cdots E^n , \]
        with $\cap_p F^p E^n = 0$ and $\cup_p F^p E^n = E^n$, and isomorphisms
        \[ E_\infty^{p, q} \simeq F^p E^{p + q} / F^{p + 1} E^{p + q} . \]
    \end{itemize}
    
    In some sense, the objects $E_r^{p, q}$ converge towards subquotients of a certain filtration of $E^n$. Usually, the objects of one layer of some fixed $r$ are given. Then one writes
    \[ E_r^{p, q} \Rightarrow E^{p + q} . \]
\end{topic}

\begin{topic}{serre-subcategory}{Serre subcategory}
    A non-empty \tref{CT:full-subcategory}{full subcategory} $\mathcal{B}$ of an \tref{abelian-category}{abelian category} $\mathcal{A}$ is a \textbf{Serre subcategory} if for any exact sequence in $\mathcal{A}$,
    \[ 0 \to A \to B \to C \to 0 , \]
    $B$ is in $\mathcal{B}$ if and only if $A$ and $C$ are in $\mathcal{B}$. In words, $\mathcal{B}$ is closed under taking subobjects, quotient objects and extensions.
\end{topic}

\begin{topic}{grothendieck-group}{Grothendieck group}
    The \textbf{Grothendieck group} $\text{K}_0(\mathcal{A})$ of an \tref{abelian-category}{abelian category} $\mathcal{A}$ is defined as the free abelian group on isomorphism classes of objects in $\mathcal{A}$, modulo the relations $[B] = [A] + [C]$ for every short exact sequence
    \[ 0 \to A \to B \to C \to 0 \]
    in $\mathcal{A}$.
\end{topic}

\begin{example}{grothendieck-group}
    Let $\mathcal{A} = \textbf{Vect}_k$ be the category of finite-dimensional vector spaces over a field $k$. Since such vector spaces are determined up to isomorphism by their dimension, and $\dim_k V \oplus W = \dim_k V + \dim_k W$, we find that $\text{K}_0(\textbf{Vect}_k) = \ZZ$.
\end{example}

\begin{example}{grothendieck-group}
    Let $\mathcal{A} = \text{Coh}(\PP^n)$ be the category of coherent sheaves on projective space $\PP^n$. Note that the twisting sheaves $\mathcal{O}(d)$ must generate the Grothendieck group $\text{K}_0(\mathcal{A})$, as any coherent sheaf has a finite resolution by direct sums of twisting sheaves. The exact Koszul complex
    \[ 0 \to \mathcal{O} \to \bigoplus^{n + 1} \mathcal{O}(1) \to \bigoplus^{\binom{n + 1}{2}} \mathcal{O}(2) \to \cdots \to \bigoplus^{n + 1} \mathcal{O}(n) \to \mathcal{O}(n + 1) \to 0 \]
    now shows that
    \[ \text{K}_0(\text{Coh}(\PP^n)) = \ZZ[T] / (1 - T)^{n + 1} \]
    (injectivity can be shown using euler-characteristics). Moreover, the tensor product of coherent sheaves corresponds to the multiplication in the ring.
\end{example}

\begin{topic}{ext-functors}{Ext functors}
    Let $\mathcal{A}$ be an \tref{abelian-category}{abelian category} and $A$ an object of $\mathcal{A}$. The functor
    \[ \Hom_{\mathcal{A}}(A, -) : \mathcal{A} \to \textbf{Ab} \]
    is \tref{exact-functor}{left exact}, so (if $\mathcal{A}$ has \tref{enough-injectives}{enough injectives}) it has \tref{right-derived-functors}{right derived functors}, which are known as the \textbf{Ext functors}
    \[ \text{Ext}^i_\mathcal{A}(A, B) = \text{R}^i \Hom_\mathcal{A}(A, B) . \]
\end{topic}

\begin{example}{ext-functors}
    Let $A$ be a $\ZZ$-module and $m > 0$ an integer. From the free resolution of $\ZZ/m\ZZ$,
    \[ 0 \to \ZZ \xrightarrow{\cdot m} \ZZ \to \ZZ / m \ZZ \to 0 , \]
    follows that
    \[ \textup{Ext}^n_\ZZ(\ZZ/m\ZZ, A) = \left\{ \begin{array}{cl} \{ a \in A : ma = 0 \} & \textup{ for } n = 0 , \\ A/mA & \textup{ for } n = 1 . \end{array} \right. \]
\end{example}

\begin{example}{ext-functors}
    For $n > 0$, elements of the Ext group $\text{Ext}^n(A, B)$ correspond to extensions
    \[ 0 \to B \to E_0 \to \cdots \to E_{n - 1} \to A \to 0 . \]
    Consider the case $n = 1$. Let $P^\bdot \to A$ (call the differential maps $\pi_i$) be a projective resolution of $A$, so that $\text{Ext}^1(A, B) = H^1(\Hom(P^\bdot, B))$. This is the group of maps $g : P^1 \to B$ such that $g \circ \pi_2 = 0$, modulo all maps of the form $f \circ \pi_0$ for $f : A \to B$. Now we can construct the pushout
    \[ E_0 = P^0 \oplus B / \{ \pi_1(q) = g(q) \text{ for } q \in P^1 \} . \]
    Note that $g$ descends to $\overline{g} : P^0 \to B$, and that the maps $B \to E_0$, $b \mapsto (b, 0)$ and $E_0 \to A, (b, p) \mapsto \pi_0(p)$ are well-defined (the latter as $\pi_0 \pi_1 = 0$), and respectively injective and surjective. It is not hard to prove exactness in the middle, so we get an extension
    \[ 0 \to B \to E_0 \to A \to 0 . \]
    In particular, the trivial element in $\text{Ext}^1(A, B)$ yields the trivial extension
    \[ 0 \to A \to A \oplus B \to B \to 0 . \]
\end{example}

\begin{topic}{tor-functors}{Tor functors}
    For a commutative \tref{AA:ring}{ring} $R$ and an $R$-module $B$, the \tref{AA:tensor-product}{tensor product} gives a functor
    \[ (-) \otimes_R B : R\textbf{-Mod} \to R\textbf{-Mod} \]
    which is \tref{exact-functor}{right exact}, so it has \tref{left-derived-functors}{left derived functors}, which are known as the \textbf{Tor functors}
    \[ \text{Tor}^R_i(A, B) = A \overset{\text{L}_i}{\otimes} B . \]
\end{topic}

\begin{example}{tor-functors}
    Let $A$ be a $\ZZ$-module and $m > 0$ an integer. From the free resolution of $\ZZ/m\ZZ$,
    \[ 0 \to \ZZ \xrightarrow{\cdot m} \ZZ \to \ZZ / m \ZZ \to 0 , \]
    follows that
    \[ \textup{Tor}_n^\ZZ(\ZZ/m\ZZ, A) = \left\{ \begin{array}{cl} A/mA & \textup{ for } n = 0 , \\ \{ a \in A : ma = 0 \} & \textup{ for } n = 1 . \end{array} \right. \]
\end{example}

\begin{topic}{mapping-complex}{mapping complex}
    Let $C_\bdot$ and $D_\bdot$ be \tref{chain-complex}{chain complexes} with values in an \tref{additive-category}{additive category} $\mathcal{A}$. The \textbf{mapping complex} associated to $C_\bdot$ and $D_\bdot$ is the chain complex $[C, D] _\bdot$ given by
    \[ [C, D]_n = \prod_{k \in \ZZ} \Hom_\mathcal{A}(C_k, D_{k + n}) \]
    and with differentials
    \[ \partial_n : [C, D]_n \to [C, D]_{n - 1}, \quad (f_k)_{k \in \ZZ} \mapsto \left(\partial_D \circ f_k - (-1)^n f_{k - 1} \circ \partial_C \right)_{k \in \ZZ} . \]
\end{topic}

\begin{topic}{composition-series}{composition series}
    A \textbf{composition series} of an object $A$ in an \tref{abelian-category}{abelian category} $\mathcal{A}$ is a sequence of subobjects
    \[ 0 = A_0 \subsetneq A_1 \subsetneq \cdots \subsetneq A_n = A \]
    such that the quotients $A_i/A_{i - 1}$, known as the \textbf{composition factors}, are simple for all $1 \le i \le n$. If $A$ has a composition series, the integer $n$ only depends on $A$ and is called the \textit{length} of $A$.
\end{topic}

\begin{example}{composition-series}
    The \textbf{Jordan--Hölder theorem} states that if an object $A$ has two composition series
    \[ 0 = A_0 \subsetneq A_1 \subsetneq \cdots \subsetneq A_n = A \qquad \text{and} \qquad 0 = B_0 \subsetneq B_1 \subsetneq \cdots \subsetneq B_m = A , \]
    then $m = n$ and the composition factors $A_i/A_{i - 1}$ are isomorphic to $B_i/B_{i - 1}$ up to permutation.
    \begin{proof}
        Let $j$ be the smallest index such that $A_1 \subset B_j$. Then the map $A_1 \to B_j/B_{j - 1}$ must be an isomorphism. Namely, it is surjective as its image is non-zero (since $A_1 \nsubseteq B_{j - 1}$) and $B_j/B_{j - 1}$ is simple. Furthermore, it is injective since its kernel is a proper subobject of $A_1$, and thus zero since $A_1$ is simple. Now, the quotient $A/A_1 = A_n/A_1 = B_m/A_1$ has the two filtrations
        \[ 0 \subsetneq A_2/A_1 \subsetneq A_3/A_1 \subsetneq \cdots \subsetneq A_n/A_1 \]
        and
        \[ 0 \subsetneq (B_1 + A_1)/A_1 \subsetneq \cdots \subsetneq (B_{j - 1} + A_1)/A_1 = B_j/A_1 \subsetneq B_{j + 1}/A_1 \subsetneq \cdots \subsetneq B_m/A_1 \]
        and the result follows from induction.    
    \end{proof}
\end{example}

\begin{topic}{yoneda-product}{Yoneda product}
    Let $\mathcal{A}$ be an \tref{abelian-category}{abelian category} with \tref{enough-injectives}{enough injectives}. The \textbf{Yoneda product} is the pairing between \tref{ext-functors}{Ext groups}
    \[ \text{Ext}_\mathcal{A}^m(B, C) \otimes \text{Ext}_\mathcal{A}^n(A, B) \to \text{Ext}_\mathcal{A}^{m + n}(A, C) \]
    induced by $\Hom_\mathcal{A}(B, C) \otimes \Hom_\mathcal{A}(A, B) \to \Hom_\mathcal{A}(A, C)$, $f \otimes g \mapsto f \circ g$.
    
    In terms of extensions
    \[ \xi : 0 \to B \to E_0 \to \cdots \to E_{n - 1} \to A \to 0 \]
    \[ \eta : 0 \to C \to F_0 \to \cdots \to F_{m - 1} \to B \to 0 \]
    the Yoneda cup product is given by
    \[ \xi \smile \eta : 0 \to C \to F_0 \to \cdots \to F_{m - 1} \to E_0 \to \cdots \to E_{n - 1} \to A \to 0 \]
    in $\text{Ext}_\mathcal{A}^{m + n}(A, C)$.
\end{topic}

\begin{example}{yoneda-product}
    For $n = 0$ and $m = 1$, we have that the product of $f : A \to B$ in $\text{Ext}^0(A, B) = \Hom(A, B)$ and $\eta : 0 \to C \to F_0 \to B \to 0$ in $\text{Ext}^1(B, C)$ is the \textit{pullback}
    \[ 0 \to C \to F_0 \oplus_B A \to A \to 0 . \]
    Similarly, for $n = 1$ and $m = 0$, the product of $0 \to B \to E_0 \to A \to 0$ in $\text{Ext}^1(A, B)$ and $g : B \to C$ in $\text{Ext}^0(B, C) = \Hom(B, C)$ is the \textit{pushout}
    \[ 0 \to C \to C \oplus_B E_0 \to A \to 0 . \]
    It is not hard to verify these sequences are indeed exact. 
\end{example}

\begin{example}{yoneda-product}
    In terms of the \tref{derived-category}{derived category} $\textbf{D}(\mathcal{A})$, the Yoneda product can easily be described. From the isomorphism
    \[ \textup{Ext}^i_\mathcal{A}(A, B) \simeq \Hom_{\textbf{D}(\mathcal{A})}(A, B[i]), \]
    we find that the Yoneda product is simply given by composition in the derived category
    \[ \Hom_{\textbf{D}(\mathcal{A})}(B, C[i]) \otimes \Hom_{\textbf{D}(\mathcal{A})}(A, B[j]) \to \Hom_{\textbf{D}(\mathcal{A})}(A, C[i + j]), \quad f \otimes g \mapsto f[j] \circ g . \]
\end{example}

\begin{topic}{homological-dimension}{(global) homological dimension}
    Let $\mathcal{A}$ be an \tref{abelian-category}{abelian category}, and $A$ an object in $\mathcal{A}$. The \textbf{homological dimension} of $A$ is the least number $n$ such that there exists a \tref{projective-resolution}{projective resolution}
    \[ 0 \to P^n \to \cdots \to P^1 \to P^0 \to A . \]
    If such a resolution does not exist, the homological dimension of $A$ is $\infty$.
    
    The \textbf{global homological dimension} of the category $\mathcal{A}$ is the supremum of the homological dimension of all objects $A$ of $\mathcal{A}$.
\end{topic}

\begin{example}{homological-dimension}
    By \tref{AA:hilbert-syzygy-theorem}{Hilbert's syzygy theorem}, any $k[x_1, \ldots, x_n]$-module has a projective resolution of length $\le n$. Since this upper bound is attained for the \tref{UN:koszul-complex}{Koszul complex}, the global homological dimension of $k[x_1, \ldots, x_n]\text{-}\textbf{Mod}$ is $n$.
\end{example}

\begin{example}{homological-dimension}
    Knowing the global homological dimension $n$ of a category $\mathcal{A}$ is useful, as it follows that for any \tref{exact-functor}{right exact functor} $F : \mathcal{A} \to \mathcal{B}$ the \tref{left-derived-functors}{left derived functors} $\text{L}^i F A$ are zero for all $A$ in $\mathcal{A}$ and $i > n$.
    
    In particular, for $F = \Hom_\mathcal{A}(-, B)$ we find $\text{Ext}_\mathcal{A}^i(A, B) = 0$ for all $A, B$ in $\mathcal{A}$ and $i > n$.
\end{example}

\begin{topic}{grothendieck-spectral-sequence}{Grothendieck spectral sequence}
    Let $F : \mathcal{A} \to \mathcal{B}$ and $G : \mathcal{B} \to \mathcal{C}$ be \tref{exact-functor}{left exact functors} between \tref{abelian-category}{abelian categories}, such that $\mathcal{A}$ and $\mathcal{B}$ have \tref{enough-injectives}{enough injectives}, and $F$ maps \tref{injective-object}{injective objects} to \tref{acyclic-object}{$G$-acyclic objects}. Then there is a \tref{spectral-sequence}{spectral sequence}
    \[ E_2^{p, q} = (R^p G \circ R^q F)(A) \Rightarrow R^{p + q}(G \circ F)(A) , \]
    called the \textbf{Grothendieck spectral sequence}.
\end{topic}

\begin{topic}{leray-spectral-sequence}{Leray spectral sequence}
    Let $X$ and $Y$ be \tref{TO:topological-space}{topological spaces} and $f : X \to Y$ a continuous map. For a \tref{AG:sheaf}{sheaf} $\mathcal{F}$ on $X$, there exists a \tref{spectral-sequence}{spectral sequence}
    \[ E_2^{p, q} = H^p(Y, R^q f_* \mathcal{F}) \Rightarrow H^{p + q}(X, \mathcal{F}) \]
    called the \textbf{Leray spectral sequence}. It is a special case of the \tref{grothendieck-spectral-sequence}{Grothendieck spectral sequence}.
\end{topic}

\begin{topic}{exact-category}{exact category}
    An \textbf{exact category} is an \tref{additive-category}{additive category} $\mathcal{A}$ with a class $E$ of sequences
    \[ A \to B \to C \]
    (referred to as `\textit{short exact sequences}') such that
    \begin{itemize}
        \item $E$ is closed under isomorphisms and contains all canonical sequences of the form
        \[ A \to A \oplus C \to C , \]
        \item if $B \to C$ occurs as the second arrow of a sequence in $E$ (called an \textit{admissible epimorphism}) and $D \to C$ is any morphism, then the pullback exists, and the projection to $D$ is an admissible epimorphism as well. Dually, if $A \to B$ occurs as the first arrow of a sequence in $E$ (called an \textit{admissible monomorphism}) and $A \to D$ is any morphism, then the pushout exists, and the inclusion from $A$ is also an admissible monomorphism,
        \item admissible monomorphisms are kernels of their corresponding admissible epimorphisms, and dually admissible epimorphisms are cokernels of their corresponding admissible monomorphisms. The composition of two admissible monomorphisms is admissible, and similarly for admissible epimorphisms.
    \end{itemize}
\end{topic}

\begin{topic}{frobenius-category}{Frobenius category}
    A \textbf{Frobenius category} is an \tref{exact-category}{exact category} with enough \tref{injective-object}{injectives} and enough \tref{projective-object}{projectives}, such that the classes of injectives and projectives coincides.
\end{topic}

\begin{topic}{mapping-cone}{mapping cone}
    Let $\mathcal{A}$ be an \tref{abelian-category}{abelian category}, and $f : A^\bdot \to B^\bdot$ a morphism between \tref{chain-complex}{cochain complexes} in $\mathcal{A}$. The \textbf{mapping cone} of $f$ is the complex $C(f)$ in $\mathcal{A}$ given by
    \[ C(f)^i = A^{i + 1} \oplus B^i , \]
    with differential
    \[ d_{C(f)}^i = \begin{pmatrix} -d_A^{i + 1} & 0 \\ f^{i + 1} & d_B^i \end{pmatrix} . \]    
\end{topic}

\begin{example}{mapping-cone}
    Let $f : X \to Y$ be a \tref{TO:continuous-map}{continuous map} between \tref{TO:topological-space}{topological spaces}. Then the mapping cone $C(f_\bdot)$ of the induced morphism $f_\bdot : C_\bdot(X; \ZZ) \to C_\bdot(Y; \ZZ)$ between \tref{AT:singular-homology}{singular complexes} of $X$ and $Y$ is \tref{AT:homotopy-equivalence}{homotopy equivalent} to the singular complex of the \tref{mapping-cone}{mapping cone} of $f$.
\end{example}

\begin{topic}{enough-injectives}{enough injectives}
    An \tref{abelian-category}{abelian category} $\mathcal{A}$ is said to have \textbf{enough injectives} if for every object $A$ in $\mathcal{A}$ there exists an \tref{injective-object}{injective object} $I$ and an injective map $A \to I$.
\end{topic}

\begin{example}{enough-injectives}
    If $\mathcal{A}$ has enough injectives, then any object $A$ of $\mathcal{A}$ has an \tref{injective-resolution}{injective resolution}. Such a resolution can be constructed inductively as follows.
    
    Start with an injective object $I^0$ and an injective map $A \to I^0$. Then let $C^0 = \coker (A \to I^0)$, and pick an injective object $I^1$ with injective map $C^0 \to I^1$, and take $I^0 \to I^1$ the composition via $C^0$. Note, since $C^0 \to I^1$ is injective, that $\ker (I^0 \to I^1) = \ker(I^0 \to C^0) = \im (A \to I^0)$ by construction. Inductively, for any $i \ge 0$, define $C^{i + 1} = \coker(I^i \to I^{i + 1})$ and pick an injective object $I^{i + 1}$ with injective map $C^{i + 1} \to I^{i + 2}$, and let $I^{i + 1} \to I^{i + 2}$ be the composition via $C^{i + 1}$. Again, by construction $\ker (I^{i + 1} \to I^{i + 2}) = \ker (I^{i + 1} \to C^{i + 1}) = \im (I^i \to I^{i + 1})$, which shows exactness at $I^{i + 1}$.
    \[ \begin{tikzcd}
        0 \arrow{r} & A \arrow{r} & I^0 \arrow[twoheadrightarrow]{d} \arrow{r} & I^1 \arrow[twoheadrightarrow]{d} \arrow{r} & I^2 \arrow[twoheadrightarrow]{d} \arrow{r} & \cdots \\
        & & C^0 \arrow[hookrightarrow]{ur} & C^1 \arrow[hookrightarrow]{ur} & C^2 \arrow[hookrightarrow]{ur} & \cdots
    \end{tikzcd} \]
\end{example}

\begin{example}{enough-injectives}
    \begin{itemize}
        \item The category $\textbf{Ab}$ of \tref{GT:abelian-group}{abelian groups} has enough injectives.
        \item For each \tref{AA:ring}{ring} $R$, the category of (left) \tref{AA:module}{$R$-modules} has enough injectives.
        \item The category of \tref{AG:sheaf}{sheaves} on a topological space $X$ has enough injectives.
    \end{itemize}
\end{example}

\begin{topic}{delta-functor}{delta-functor}
    Let $\mathcal{A}$ and $\mathcal{B}$ be \tref{abelian-category}{abelian categories}. A \textbf{homological $\delta$-functor} from $\mathcal{A}$ to $\mathcal{B}$ is a collection of \tref{additive-functor}{additive functors} $T_n : \mathcal{A} \to \mathcal{B}$ for $n \ge 0$, together with morphisms
    \[ \delta_n : T_n(C) \to T_{n - 1}(A) \]
    for each \tref{exact-sequence}{short exact sequence} $0 \to A \to B \to C \to 0$ in $\mathcal{A}$, satisfying the following conditions.
    \begin{itemize}
        \item For each short exact sequence $0 \to A \to B \to C \to 0$ in $\mathcal{A}$, there is a long exact sequence
        \[ \cdots \to T_{n + 1}(C) \xrightarrow{\delta_{n + 1}} T_n(A) \to T_n(B) \to T_n(C) \xrightarrow{\delta} T_{n - 1}(A) \to \cdots \to T_0(C) \to 0 . \]
        \item For each morphism of short exact sequences from $0 \to A' \to B' \to C' \to 0$ to $0 \to A \to B \to C \to 0$, the diagram
        \[ \begin{tikzcd} T_n(C') \arrow{r}{\delta_n} \arrow{d} & T_{n - 1}(A') \arrow{d} \\ T_n(C) \arrow{r}{\delta_n} & T_{n - 1}(A) \end{tikzcd} \]
        commutes.
    \end{itemize}
    Similarly, a \textbf{cohomological $\delta$-functor} is a collection of additive functors $T^n : \mathcal{A} \to \mathcal{B}$ for $n \ge 0$, together with morphisms $\delta^n : T^n(C) \to T^{n + 1}(A)$ for each short exact sequence $0 \to A \to B \to C \to 0$, satisfying the dual conditions.
\end{topic}

\begin{example}{delta-functor}
    \begin{itemize}
        \item \tref{homology-object}{Homology} gives a homological $\delta$-functor $H_*(-)$ from \tref{chain-complex}{chain complexes} $\textbf{Ch}_{\ge 0}(\mathcal{A})$ to $\mathcal{A}$.
        \item Similarly, cohomology gives a cohomological $\delta$-functor $H^*(-)$ from cochain complexes $\textbf{Ch}^{\ge 0}(\mathcal{A})$ to $\mathcal{A}$.
    \end{itemize}
\end{example}

\begin{topic}{group-cohomology}{group cohomology}
    Let $G$ be a \tref{GT:group}{group}, and $M$ an \tref{GT:abelian-group}{abelian group} with an action of $G$. Let $C^\bdot(G, M)$ be the \tref{chain-complex}{cochain complex} given by
    \[ C^n(G, M) = \{ \textup{functions } f : G^n \to M \} \]
    with differentials $d^n : C^n(G, M) \to C^{n + 1}(G, M)$ given by
    \[ \begin{aligned} (d^n f)(g_1, \ldots, g_{n + 1}) &= g_1 f(g_2, \ldots, g_{n + 1}) + \sum_{i = 1}^{n} (-1)^i f(g_1, \ldots, g_{i - 1}, g_i g_{i + 1}, \ldots, g_{n + 1}) \\ &\qquad + (-1)^{n + 1} f(g_1, \ldots, g_n) . \end{aligned} \]
    The \textbf{group cohomology} of $G$ with coefficients in $M$ are the \tref{homology-object}{cohomology groups}
    \[ H^i(G, M) = H^i(C^\bdot(G, M)) . \]
\end{topic}

\begin{example}{group-cohomology}
    Consider the cohomology groups in low degree.
    \begin{itemize}
        \item (\textit{degree $0$}) For any $m \in M$ we have $(d^0(m))(g) = gm - m$, so
        \[ H^0(G, M) = \ker d^0 = \{ m \in M : gm - m = 0 \textup{ for all } g \in G \} = M^G \]
        is the $G$-invariant submodule of $M$.
        \item (\textit{degree $1$}) Functions $f : G \to M$ such that $f(g_1 g_2) = f(g_1) + g_1 f(g_2)$ for all $g_1, g_2 \in G$ are called \textit{crossed homomorphisms}. Functions $f : G \to M$ of the form $f(g) = gm - m$ for some $m \in M$ are called \textit{principal homomorphisms}. Note that $\ker d^1$ and $\im d^0$ are precisely the sets of crossed and principal homomorphisms, respectively. Therefore, $H^1(G, M)$ is the set of crossed homomorphisms modulo the set of principal homomorphisms.
        
        In particular, if $G$ acts trivially on $M$, every principal homomorphism is trivial, and crossed homomorphisms correspond to group homomorphisms $G \to M$, so that $H^1(G, M) \simeq \Hom(G, M)$.
    \end{itemize}
\end{example}

\begin{example}{group-cohomology}
    Alternatively, group cohomology can be described in terms of \tref{ext-functors}{Ext functors},
    \[ H^n(G, M) = \textup{Ext}^n_{\ZZ[G]}(\ZZ, M) , \]
    which might make sense since $H^0(G, M) = M^G = \Hom_{\ZZ[G]}(\ZZ, M)$.
    
    To compute these Ext groups, we take a free resolution of $\ZZ$ as a $\ZZ[G]$-module,
    \[ \cdots \xrightarrow{d_3} F_2 \xrightarrow{d_2} F_1 \xrightarrow{d_1} F_0 \xrightarrow{d_0} \ZZ \to 0 , \]
    with $F_n = \ZZ[G^{n + 1}]$ and differentials given by
    \[ d_n : F_n \to F_{n - 1}, \quad (g_0, \ldots, g_n) \mapsto \sum_{i = 0}^{n} (-1)^i (g_0, \ldots, \widehat{g}_i, \ldots, g_n) . \]
    Applying $\Hom_{\ZZ[G]}(-, M)$ now gives the (co)complex
    \[ \cdots \xleftarrow{\partial^{n + 1}} \Hom_{\ZZ[G]}(F_{n + 1}, M) \xleftarrow{\partial^n} \Hom_{\ZZ[G]}(F_n, M) \xleftarrow{\partial^{n - 1}} \cdots \xleftarrow{\partial^0} \Hom_{\ZZ[G]}(F_0, M) \leftarrow 0 \]
    with differentials
    \[ (\partial^n \varphi)(g_0, \ldots, g_{n + 1}) = \sum_{i = 0}^{n + 1} (-1)^i \varphi(g_0, \ldots, \widehat{g}_i, \ldots, g_{n + 1}) . \]
    Indeed, this complex is isomorphic to $C^\bdot(G, M)$ via the isomorphisms
    \[ \Psi_n : \Hom_{\ZZ[G]}(F_n, M) \xrightarrow{\sim} C^n(G, M), \quad \Psi_n(\varphi)(g_1, \ldots, g_n) = \varphi(1, g_1, g_1 g_2, g_1 g_2 g_3, \ldots, g_1 \cdots g_n ) , \]
    whose inverses are given by
    \[ \Phi_n : C^n(G, M) \xrightarrow{\sim} \Hom_{\ZZ[G]}(F_n, M), \quad \Phi_n(f)(g_0, \ldots, g_n) = g_0 f(g_0^{-1} g_1, g_1^{-1} g_2, \ldots, g_{n - 1}^{-1} g_n) . \]
    In particular the cohomology groups agree, and thus group cohomology is given by
    \[ H^n(G, M) \simeq \textup{R}^i \Hom_{\ZZ[G]}(\ZZ, M) = \textup{Ext}^n_{\ZZ[G]}(\ZZ, M) . \]
\end{example}

\begin{topic}{contractible-chain-complex}{contractible chain complex}
    A \tref{chain-complex}{chain complex} $C^\bdot$ is \textbf{contractible} if it is \tref{chain-homotopy-equivalence}{homotopy equivalent} to the zero complex. That is, there exist a collection of morphisms $k^i : C^i \to C^{i - 1}$, with $i \in \ZZ$, such that
    \[ d^{i - 1} k^i + k^{i + 1} d^i = \id_{C^i} . \]
\end{topic}

\begin{example}{contractible-chain-complex}
    The complex
    \[ \cdots \to 0 \to \ZZ \xrightarrow{\id} \ZZ \to 0 \to \cdots \]
    is contractible, which follows from taking $k^0 : \ZZ \to \ZZ$ to be the identity.
    
    Note that any contractible complex has zero \tref{homology-object}{homology}, since homotopy equivalences preserve homology. However, not every complex with zero homology is contractible. Namely, the complex
    \[ \cdots \to 0 \to \ZZ \xrightarrow{\cdot 2} \ZZ \to \ZZ / 2\ZZ \to 0 \to \cdots \]
    has zero homology, but is not contractible since the sequence does not split.
\end{example}

\begin{topic}{baer-sum}{Baer sum}
    Let $\mathcal{A}$ be an \tref{abelian-category}{abelian category}, and $0 \to A \xrightarrow{f} E \xrightarrow{g} B \to 0$ and $0 \to A \xrightarrow{f'} E' \xrightarrow{g'} B \to 0$ be two \tref{exact-sequence}{short exact sequences}. The \textbf{Baer sum} of these sequences is the sequence
    \[ 0 \to A \to \coker \left(A \xrightarrow{(f, -f')} E \oplus_B E'\right) \to B \to 0 , \]
    where the maps are given by $a \mapsto (f(a), 0) = (0, f'(a))$ and $(e, e') \mapsto g(e) = g'(e')$.
    
    Viewing the sequences as extensions, the Baer sum is given by addition in $\textup{Ext}_\mathcal{A}^1(B, A)$.
\end{topic}

\begin{topic}{linear-category}{linear category}
    Let $k$ be a \tref{AA:ring}{commutative ring}. A \textbf{linear category} $\mathcal{A}$ is a \tref{CT:category}{category} \tref{CT:enriched-category}{enriched} over the category of \tref{AA:module}{$k$-modules}. Concretely, all morphism sets are $k$-modules, and composition
    \[ \Hom_\mathcal{A}(A, B) \times \Hom_\mathcal{A}(B, C) \to \Hom_\mathcal{A}(A, C) \]
    is $k$-bilinear.
\end{topic}

\begin{topic}{grothendieck-category}{Grothendieck category}
    A \textbf{Grothendieck category} is an \tref{abelian-category}{abelian category} $\mathcal{A}$ such that
    \begin{itemize}
        \item arbitrary direct sums exist in $\mathcal{A}$,
        \item \tref{CT:direct-limit}{direct limits} of short exact sequences in $\mathcal{A}$ are again short exact,
        \item $\mathcal{A}$ has a \textit{generator}, that is, there exists an object $G$ in $\mathcal{A}$ such that the \tref{CT:functor}{functor} $\Hom_\mathcal{A}(G, -) : \mathcal{A} \to \textbf{Set}$ is \tref{CT:faithful-functor}{faithful}.
    \end{itemize}
\end{topic}

\begin{topic}{effaceable-functor}{effaceable functor}
    Let $F : \mathcal{A} \to \mathcal{B}$ be an \tref{additive-functor}{additive functor} between \tref{abelian-category}{abelian categories}. Then $F$ is \textbf{effaceable} if for every object $A$ of $\mathcal{A}$ there exists a \tref{CT:monomorphism}{monomorphism} $m : A \to A'$ such that $F(m) = 0$.
\end{topic}

\begin{example}{effaceable-functor}
    If $\mathcal{A}$ has \tref{enough-injectives}{enough injectives}, then $F$ is effaceable if and only if $F(I) = 0$ for all \tref{injective-object}{injective objects} $I$ of $\mathcal{A}$.
    \begin{proof}
        $(\Leftarrow)$ As $\mathcal{A}$ has enough injectives, for any $A$ of $\mathcal{A}$, there exists a monomorphism $m : A \to I$ with $I$ injective. Since $F(I) = 0$, also $F(m) = 0$.
        $(\Rightarrow)$ Let $I$ be an injective object of $\mathcal{A}$. Since $F$ is effaceable, there exists a monomorphism $m : I \to J$, and by injectivity of $I$, there exists a map $h : J \to I$ such that $hm = \id_I$. Now $F(\id_I) = F(h) \circ F(m) = F(h) \circ 0 = 0$, so $F(I) = 0$.
    \end{proof}
\end{example}

\begin{topic}{finite-abelian-category}{(locally) finite abelian category}
    Let $k$ be a \tref{AA:field}{field}, and let $\mathcal{A}$ be a \tref{linear-category}{$k$-linear category}. Then $\mathcal{A}$ is \textbf{locally finite} if
    \begin{itemize}
        \item the vector space $\Hom_\mathcal{A}(A, B)$ is finite-dimensional for all objects $A$ and $B$ of $\mathcal{A}$,
        \item every object of $\mathcal{A}$ has finite \tref{composition-series}{length}.
    \end{itemize}
    If moreover
    \begin{itemize}
        \item $\mathcal{A}$ contains finitely many \tref{simple-object}{simple objects} (up to isomorphism),
    \end{itemize}
    then $\mathcal{A}$ is \textbf{finite}.
\end{topic}

\begin{example}{finite-abelian-category}
    A theorem by Deligne states that for every finite abelian category $\mathcal{A}$ over $k$, there is a finite-dimensional $k$-algebra $A$ and a $k$-linear \tref{CT:equivalence-of-categories}{equivalence of categories}
    \[ \mathcal{A} \simeq A\textup{-}\textbf{Mod}_\textup{fd} , \]
    between $\mathcal{A}$ and the category of $A$-modules which are finite-dimensional as vector spaces over $k$. Such $A$ is uniquely determined up to \tref{AA:morita-equivalence}{Morita equivalence}.
\end{example}

\begin{topic}{simple-object}{(semi)simple object}
    Let $\mathcal{A}$ be an \tref{abelian-category}{abelian category}. An object $A$ of $\mathcal{A}$ is \textbf{simple} if it is non-zero and its only subobjects are $0$ and itself. An object of $\mathcal{A}$ is \textbf{semisimple} if it is a direct sum of simple objects.
\end{topic}

\begin{topic}{deligne-tensor-product}{Deligne tensor product}
    Let $\mathcal{A}$ and $\mathcal{B}$ be \tref{finite-abelian-category}{finite} \tref{abelian-category}{abelian categories} over a field $k$. The \textbf{Deligne tensor product} of $\mathcal{A}$ and $\mathcal{B}$ is a finite abelian category $\mathcal{A} \boxtimes \mathcal{B}$ together with a bifunctor $\boxtimes : \mathcal{A} \times \mathcal{B} \to \mathcal{A} \boxtimes \mathcal{B}$ that is \tref{exact-functor}{right exact} in both variables, which is universal in the sense that for every bifunctor $F : \mathcal{A} \times \mathcal{B} \to \mathcal{C}$ which is right exact in both variables, there exists a unique right exact functor $G : \mathcal{A} \boxtimes \mathcal{B} \to \mathcal{C}$ and a unique \tref{CT:natural-transformation}{natural} isomorphism that $F \simeq G \circ \boxtimes$.
    \[ \begin{tikzcd}
        \mathcal{A} \times \mathcal{B} \arrow{r}{\boxtimes} \arrow[swap]{dr}{F} & \mathcal{A} \boxtimes \mathcal{B} \arrow[dashed]{d}{G} \\
        & \mathcal{C}
    \end{tikzcd} \]
\end{topic}

\begin{example}{deligne-tensor-product}
    Let $A$ and $B$ be two finite-dimensional algebras over $k$, and write $A\textup{-}\textbf{Mod}_\textup{fd}$ for the finite abelian category of $A$-modules which are finite-dimensional as vector spaces over $k$. Then
    \[ A\textup{-}\textbf{Mod}_\textup{fd} \boxtimes B\textup{-}\textbf{Mod}_\textup{fd} \simeq (A \otimes_k B)\textup{-}\textbf{Mod}_\textup{fd} . \]
    Indeed, we can construct $G$ as follows: for any finite-dimensional $(A \otimes_k B)$-module $M$, take a presentation
    \[ (A \otimes_k B)^{\oplus n_1} \to (A \otimes_k B)^{\oplus n_0} \to M \to 0 , \]
    and put $G(M) = \coker\left( F(A, B)^{\oplus n_1} \to F(A, B)^{\oplus n_0} \right)$. This is well-defined by right exactness of $F$, and it is easy to see that $G$ is right exact and that $F \simeq G \circ \boxtimes$.
    
    In fact, this is the prototype example since every finite abelian category is equivalent to $A\textup{-}\textbf{Mod}_\textup{fd}$ for some finite-dimensional $k$-algebra $A$.
\end{example}

\begin{topic}{double-complex}{double-complex}
    Let $\mathcal{A}$ be an \tref{abelian-category}{abelian category}. A \textbf{double complex} $A_{\bdot, \bdot}$ in $\mathcal{A}$ is a sequence of objects $A_{i, j}$ in $\mathcal{A}$, for $i, j \in \ZZ$, together with horizontal and vertical morphisms $d^h_{i, j} : A_{i, j} \to A_{i - 1, j}$ and $d^v_{i, j} : A_{i, j} \to A_{i, j - 1}$, satisfying
    \[ (d^h)^2 = (d^v)^2 = d^h d^v - d^v d^h = 0 . \]
    \[ \begin{tikzcd}
        & \vdots \arrow{d} & \vdots \arrow{d} & \\
        \cdots \arrow{r} & A_{i, j} \arrow{r}{d^h_{i, j}} \arrow[swap]{d}{d^v_{i, j}} & A_{i - 1, j} \arrow{r} \arrow{d}{d^v_{i - 1, j}} & \cdots \\
        \cdots \arrow{r} & A_{i, j - 1} \arrow[swap]{r}{d^h_{i, j - 1}} \arrow{d} & A_{i - 1, j} \arrow{r} \arrow{d} & \cdots \\
         & \vdots & \vdots & 
    \end{tikzcd} \]
\end{topic}

\begin{topic}{total-complex}{total complex}
    Let $\mathcal{A}$ be an \tref{abelian-category}{abelian category} and $(A_{\bdot, \bdot}, d^h, d^v)$ a \tref{double-complex}{double complex} in $\mathcal{A}$. The \textbf{total complex} of $A_{\bdot, \bdot}$ is the \tref{chain-complex}{chain complex} $(A_\bdot, d)$ with
    \[ A_n = \bigoplus_{i + j = n} A_{i, j} \]
    and
    \[ d_n = \bigoplus_{i + j = n} d^v_{i, j} + (-1)^j d^h_{i, j} : A_n \to A_{n - 1} . \]
\end{topic}

\begin{topic}{kunneth-formula}{Künneth formula}
    Let $C_\bdot$ and $D_\bdot$ be \tref{chain-complex}{chain complexes} of $k$-modules, for some \tref{AA:ring}{commutative ring} $k$. If for all $n \in \ZZ$, the $k$-modules $C_n$, $D_n$, $\ker(d_n^C)$ and $\ker(d_n^D)$ are all \tref{AA:flat-module}{flat}, there is a \tref{exact-sequence}{short exact sequence}
    \[ 0 \to \bigoplus_{p + q = n} H_p(C_\bdot) \otimes H_q(D_\bdot) \to H_n(C_\bdot \otimes D_\bdot) \to \bigoplus_{p + q = n - 1} \textup{Tor}_1^k(H_p(C_\bdot), H_q(D_\bdot)) \to 0 , \]
    known as the \textbf{Künneth formula}. In particular, if $H_n(C_\bdot)$ or $H_n(D_\bdot)$ are \tref{AA:projective-module}{projective} $k$-modules for all $n$, there is an isomorphism
    \[ \bigoplus_{p + q = n} H_p(C_\bdot) \otimes H_q(D_\bdot) \simeq H_n(C_\bdot \otimes D_\bdot) . \]
\end{topic}
