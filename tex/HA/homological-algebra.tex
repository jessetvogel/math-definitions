% Abelian categories
\begin{topic}{additive-category}{additive category}
    An \textbf{additive category} is a category $\mathcal{A}$ such that
    \begin{itemize}
        \item every $\Hom_\mathcal{A}(A, B)$ is an abelian group, and composition is bilinear,
        \item there is a \tref{CT:zero-object}{zero object} $0$ which is both \tref{CT:terminal-object}{terminal} and \tref{CT:initial-object}{initial},
        \item direct sums and direct products exist, and coincide.
    \end{itemize}
\end{topic}

\begin{topic}{abelian-category}{abelian category}
    An \textbf{abelian category} is an \tref{additive-category}{additive category} category $\mathcal{A}$ in which
    \begin{itemize}
        \item all kernels and cokernels exist,
        \item every monomoprhism is the kernel of some morphism, and every epimoprhism is the cokernel of some morphism.
    \end{itemize}
\end{topic}

\begin{example}{abelian-category}
    For a \tref{CA:ring}{ring} $R$, the category of left (or right) \tref{CA:module}{$R$-modules} is an abelian category.
    
    In particular, the category of abelian groups (that is, $\ZZ$-modules) $\textbf{Ab}$ is an abelian category.
\end{example}

\begin{topic}{additive-functor}{additive functor}
    Let $\mathcal{A}$ and $\mathcal{B}$ be \tref{additive-category}{additive categories}. A functor $F : \mathcal{A} \to \mathcal{B}$ is called \textbf{additive} if for all $A, B$ in $\mathcal{A}$ the map
    \[ \Hom_\mathcal{A}(A, B) \to \Hom_\mathcal{B}(FA, FB) \]
    is a group morphism.
    
    It follows that an additive functor preserves finite direct sums and sends $0$ to $0$.
\end{topic}

\begin{topic}{exact-sequence}{(short) exact sequence}
    Let $\mathcal{A}$ be an \tref{abelian-category}{abelian category}. An \textbf{exact sequence} is a sequence of objects and morphisms
    \[ \cdots \rightarrow A_{i - 1} \xrightarrow{f_{i - 1}} A_i \xrightarrow{f_i} A_{i + 1} \rightarrow \cdots \]
    such that $\ker f_i = \im f_{i - 1}$ for all $i$.
    A \textbf{short exact sequence} is an exact sequence of the form
    \[ 0 \to A \xrightarrow{f} B \xrightarrow{g} C \to 0 . \]
    Concretely, this means that $f$ is injective, $g$ is surjective, and $\ker g = \im f$.
\end{topic}

\begin{example}{exact-sequence}
    Any exact sequence
    \[ \cdots \rightarrow A_{i - 1} \xrightarrow{f_{i - 1}} A_i \xrightarrow{f_i} A_{i + 1} \rightarrow \cdots \]
    can be split up into short exact sequences: let $B_i = \ker f_i = \im f_{i + 1}$, then we obtain short exact sequences
    \[ 0 \to B_i \to A_i \to B_{i + 1} \to 0 \]
    for each $i$.
\end{example}

\begin{topic}{exact-functor}{exact functor}
    Let $F : \mathcal{A} \to \mathcal{B}$ be an \tref{additive-functor}{additive functor} between \tref{abelian-category}{abelian categories}. Then $F$ is called \textbf{left exact} if for all short exact sequences
    \[ 0 \to M \to N \to L \to 0 \]
     in $\mathcal{A}$, the sequence
    \[ 0 \to FM \to FN \to FL \]
    is exact. Similarly, $F$ is called \textbf{right exact} if for all such short exact sequences
    \[ FM \to FN \to FL \to 0 \]
    is exact. Finally, $F$ is called \textbf{exact} if it is both left and right exact.
\end{topic}

\begin{topic}{injective-object}{injective object}
    Let $\mathcal{A}$ be an \tref{abelian-category}{abelian category}. An object $I$ of $\mathcal{A}$ is called \textbf{injective} if the functor
    \[ \Hom_\mathcal{A}(-, I) : \mathcal{A}^\text{op} \to \textbf{Ab} \]
    is \tref{exact-functor}{exact}. Since it is already left exact for any object $I$, this is equivalent to the condition that for all monomorphisms $f : M \to N$ and morphisms $g : M \to I$ there exists a morphism $h : N \to I$ such that $hf = g$.
    \[ \begin{tikzcd} M \arrow[hookrightarrow]{r}{f} \arrow[swap]{d}{g} & N \arrow[dashed]{ld}{\exists h} \\ I & \end{tikzcd} \]
\end{topic}

\begin{topic}{projective-object}{projective object}
    Let $\mathcal{A}$ be an \tref{abelian-category}{abelian category}. An object $P$ of $\mathcal{A}$ is called \textbf{projective} if the functor
    \[ \Hom_\mathcal{A}(P, -) : \mathcal{A} \to \textbf{Ab} \]
    is \tref{exact-functor}{exact}. Since it is already right exact for any object $P$, this is equivalent to the condition that for all epimorphisms $f : N \to M$ and morphisms $g : P \to M$ there exists a morphism $h : P \to N$ such that $fh = g$.
    \[ \begin{tikzcd} & N \arrow[twoheadrightarrow]{d}{f} \\ P \arrow[swap]{r}{g} \arrow[dashed]{ur}{\exists h} & M \end{tikzcd} \]
\end{topic}

% Complexes
\begin{topic}{chain-complex}{chain complex}
    Let $\mathcal{A}$ be an \tref{abelian-category}{abelian category}. A \textbf{chain complex} $A_\bdot$ in $\mathcal{A}$ is a sequence of objects $A_i$ in $\mathcal{A}$, for $i \in \ZZ$, together with morphisms $d_i : A_i \to A_{i - 1}$ such that $d_{i - 1} \circ d_i = 0$ for all $i$.
    \[ \cdots \xleftarrow{d_{i - 1}} A_{i - 1} \xleftarrow{d_i} A_i \xleftarrow{d_{i + 1}} A_{i + 1} \xleftarrow{d_{i + 2}} \cdots \]
    Dually, a \textbf{cochain complex} $A^\bdot$ in $\mathcal{A}$ is a sequence of objects $A^i$ in $\mathcal{A}$, for $i \in \ZZ$, together with morphisms $d^i : A^i \to A^{i + 1}$ such that $d^{i + 1} \circ d^i = 0$ for all $i$.
    \[ \cdots \xrightarrow{d^{i - 2}} A^{i - 1} \xrightarrow{d^{i - 1}} A^i \xrightarrow{d^i} A^{i + 1} \xrightarrow{d^{i + 1}} \cdots \]
    
    A \textbf{morphism of chain complexes} or \textbf{chain map} $f : A_\bdot \to B_\bdot$ is given by a collection of morphisms $f_i : A_i \to B_i$ for each $i \in \ZZ$ such that $d^B_i \circ f_i = f_{i - 1} \circ d^A_i$. Dually, there are \textbf{cochain maps}.
    
    If the objects of $\mathcal{A}$ are sets, then elements of $A_n$ are called \textit{$n$-chains}, elements of $\ker d_n$ are called \textit{$n$-cycles}, and elements of $\im d_{n - 1}$ are called \textit{$n$-boundaries}. Similarly, elements of $A^n$ are called \textit{$n$-cochains}, elements of $\ker d^n$ are called \textit{$n$-cocycles}, and elements of $\im d^{n - 1}$ are called \textit{$n$-coboundaries}. Note that any $n$-(co)boundary is an $n$-(co)cycle.
\end{topic}

\begin{topic}{homology-object}{(co)homology object}
    Let $\mathcal{A}$ be an \tref{abelian-category}{abelian category}. The \textbf{$i$-th homology object} of a \tref{chain-complex}{chain complex} $A_\bdot$ is the quotient
    \[ H_i(A_\bdot) = \ker d_i / \im d_{i + 1} . \]
    Dually, the \textbf{$i$-th cohomology object} of a cochain complex $A^\bdot$ is the quotient
    \[ H^i(A^\bdot) = \ker d^i / \im d^{i - 1} . \]
\end{topic}

\begin{topic}{long-exact-sequence-homology}{long exact sequence in homology}
    Let $\mathcal{A}$ be an \tref{abelian-category}{abelian category}, and let
    \[ 0 \rightarrow A^\bdot \rightarrow B^\bdot \rightarrow C^\bdot \rightarrow 0 \]
    be a short exact sequence of \tref{chain-complex}{complexes} in $\mathcal{A}$. Then there is an induced long exact sequence
    \[ \cdots \rightarrow H^i(A^\bdot) \rightarrow H^i(B^\bdot) \rightarrow H^i(C^\bdot) \rightarrow H^{i + 1}(A^\bdot) \rightarrow \cdots . \]
\end{topic}

\begin{topic}{chain-homotopy}{chain homotopy}
    Let $\mathcal{A}$ be an \tref{abelian-category}{abelian category}, and let $f, g : A^\bdot \to B^\bdot$ be morphisms of \tref{chain-complex}{chain complexes} in $\mathcal{A}$. A \textbf{chain homotopy} from $f$ to $g$ is a collection of morphisms $k^i : A^i \to B^{i - 1}$ such that $f - g = dk + kd$.
    
    If a homotopy exists from $f$ to $g$, we say that $f$ and $g$ are \textbf{homotopic}. This gives an equivalence relation on the set of morphisms from $A^\bdot$ to $B^\bdot$.
\end{topic}

\begin{topic}{chain-homotopy-equivalence}{chain homotopy equivalence}
    Let $\mathcal{A}$ be an \tref{abelian-category}{abelian category}. A morphism of \tref{chain-complex}{chain complexes} $f : M^\bdot \to N^\bdot$ is called a \textbf{chain homotopy equivalence} if there exists a morphism $g : N^\bdot \to M^\bdot$ such that $fg$ and $gf$ are \tref{chain-homotopy}{homotopic} to the identity morphism.
\end{topic}

\begin{topic}{quasi-isomorphism}{quasi-isomorphism}
    Let $\mathcal{A}$ be an \tref{abelian-category}{abelian-category}. A morphism $f : M^\bdot \to N^\bdot$ of \tref{chain-complex}{chain complexes} in $\mathcal{A}$ is called a \textbf{quasi-isomorphism} if the induced maps on homology
    \[ H^i(f) : H^i(M^\bdot) \to H^i(N^\bdot) \]
    are all isomorphisms.
\end{topic}

\begin{example}{quasi-isomorphism}
    Quasi-isomorphisms are generally not invertible. For example, the morphism of complexes
    \[ \begin{tikzcd} \cdots \arrow{r} & 0 \arrow{d} \arrow{r} & \ZZ \arrow{r}{\cdot 2} \arrow{d} & \ZZ \arrow{r} \arrow{d} & 0 \arrow{r} \arrow{d} & \cdots \\ \cdots \arrow{r} & 0 \arrow{r} & 0 \arrow{r} & \ZZ/2\ZZ \arrow{r} & 0 \arrow{r} & \cdots  \end{tikzcd} \]
    is a quasi-isomorphism, but not invertible.
    
    Also there are complexes with isomorphic homology, but which are not quasi-isomorphic. For example,
    \[ 0 \to \CC[x, y] \oplus \CC[x, y] \overset{\varphi}{\longrightarrow} \CC[x, y] \to 0 \qquad \text{ and } \qquad 0 \to \CC[x, y] \overset{0}{\longrightarrow} \CC \to 0 \]
    with $\varphi(f, g) = xf + yg$.
\end{example}

\begin{topic}{injective-resolution}{injective resolution}
    Let $\mathcal{A}$ be an \tref{abelian-category}{abelian category} and $M$ an object in $\mathcal{A}$. An \textbf{injective resolution} of $M$ is an exact sequence
    \[ 0 \to M \to I^0 \to I^1 \to \cdots \]
    where all $I^i$ are \tref{injective-object}{injective objects}. Equivalently, this can be seen as a \tref{quasi-isomorphism}{quasi-isomorphism} $M[0] \to I^\bdot$ in $\text{Comp}(\mathcal{A})$.
\end{topic}

\begin{topic}{projective-resolution}{projective resolution}
    Let $\mathcal{A}$ be an \tref{abelian-category}{abelian category} and $M$ an object in $\mathcal{A}$. A \textbf{projective resolution} of $M$ is an exact sequence
    \[ \cdots \to P^1 \to P^0 \to M \to 0 \]
    where all $P^i$ are \tref{projective-object}{projective objects}. Equivalently, this can be seen as a \tref{quasi-isomorphism}{quasi-isomorphism} $M[0] \to P^\bdot$ in $\text{Comp}(\mathcal{A})$.
\end{topic}

% Derived functors
\begin{topic}{right-derived-functors}{right derived functors}
    Let $\mathcal{A}$ and $\mathcal{B}$ be \tref{abelian-category}{abelian categories}, assume that $\mathcal{A}$ has \textit{enough injectives}, and let $F : \mathcal{A} \to \mathcal{B}$ be a \tref{exact-functor}{left exact} functor. Take $M$ to be an object of $\mathcal{A}$ and pick an \tref{injective-resolution}{injective resolution} $M[0] \to I^\bdot$. Then we define
    \[ \text{R}^i F M = H^i(F(I^\bdot)) . \]
    This is well-defined (independent of the chosen resolution) and yields additive functors $\text{R}^i F : \mathcal{A} \to \mathcal{B}$, called the \textbf{right derived functors} of $F$. Note that $R^0 F \simeq F$.
    
    For each short exact sequence $0 \to A \to B \to C \to 0$ in $\mathcal{A}$ there is an associated long exact sequence
    \[ 0 \to FA \to FB \to FC \to \text{R}^1FA \to \text{R}^1FB \to \text{R}^1FC \to \text{R}^2FA \to \cdots \]
\end{topic}

\begin{topic}{left-derived-functors}{left derived functors}
    Let $\mathcal{A}$ and $\mathcal{B}$ be \tref{abelian-category}{abelian categories}, assume that $\mathcal{A}$ has \textit{enough projectives}, and let $F : \mathcal{A} \to \mathcal{B}$ be a \tref{exact-functor}{right exact} functor. Take $M$ to be an object of $\mathcal{A}$ and pick a \tref{projective-resolution}{projective resolution} $P^\bdot \to M[0]$. Then we define
    \[ \text{L}_i F M = H^i(F(P^\bdot)) . \]
    This is well-defined (independent of the chosen resolution) and yields additive functors $\text{L}_i F : \mathcal{A} \to \mathcal{B}$, called the \textbf{left derived functors} of $F$. Note that $\text{L}_0 F \simeq F$.
    
    For each short exact sequence $0 \to A \to B \to C \to 0$ in $\mathcal{A}$ there is an associated long exact sequence
    \[ \cdots \to \text{L}_2FC \to \text{L}_1FA \to \text{L}_1FB \to \text{L}_1FC \to FA \to FB \to FC \to 0 \]
\end{topic}

\begin{topic}{acyclic-object}{acyclic object}
    Let $F : \mathcal{A} \to \mathcal{B}$ be an \tref{additive-functor}{additive functor} between \tref{abelian-category}{abelian categories}. An object $A$ of $\mathcal{A}$ is called \textbf{acyclic with respect to $F$}, or \textbf{$F$-acyclic}, if
    \[ \text{R}^iF(A) = 0 \quad \text{for all } i > 0, \]
    where $\text{R}^i F$ are the \tref{right-derived-functors}{right derived functors} of $F$.
\end{topic}

% Homological lemmas
\begin{topic}{five-lemma}{five lemma}
    Let $\mathcal{A}$ be an \tref{abelian-category}{abelian category}, and consider the following commutative diagram.
    \[ \begin{tikzcd} A \arrow{r}{f} \arrow{d}{a} & B \arrow{r}{g} \arrow{d}{b} & C \arrow{r}{h} \arrow{d}{c} & D \arrow{r}{i} \arrow{d}{d} & E \arrow{d}{e} \\ A' \arrow{r}{f'} & B' \arrow{r}{g'} & C' \arrow{r}{h'} & D \arrow{r}{i'} & E' \end{tikzcd} \]
    Suppose that both rows are exact. The \textbf{five lemma} states that
    \begin{itemize}
        \item if $b$ and $d$ are injective and $a$ is surjective, then $c$ is injective.
        \item if $b$ and $d$ are surjective and $e$ is injective, then $c$ is surjective.
        \item if $b$ and $d$ are isomorphisms, $a$ is surjective and $e$ is injective, then $c$ is an isomorphism.
    \end{itemize}
\end{topic}

\begin{topic}{snake-lemma}{snake lemma}
    Let $\mathcal{A}$ be an \tref{abelian-category}{abelian category}, and consider the following commutative diagram.
    \[ \begin{tikzcd} & A \arrow{d}{a} \arrow{r}{f} & B \arrow{d}{b} \arrow{r}{g} & C \arrow{d}{c} \arrow{r} & 0 \\ 0 \arrow{r} & A' \arrow{r}{f'} & B' \arrow{r}{g'} & C' \end{tikzcd} \]
    Suppose that the rows are exact. The \textbf{snake lemma} states that there is an exact sequence of kernels and cokernels
    \[ \ker a \rightarrow \ker b \rightarrow \ker c \xrightarrow{d} \coker a \rightarrow \coker b \rightarrow \coker c \]
    where $d$ is known as the \textit{connecting homomorphism}. Furthermore, if $f$ is \tref{CT:monomorphism}{mono} then so is $\ker a \to \ker b$, and if $g'$ is \tref{CT:epimorphism}{epi} then so is $\coker b \to \coker c$.
\end{topic}

\begin{topic}{spectral-sequence}{spectral sequence}
    Let $\mathcal{A}$ be an \tref{abelian-category}{abelian category}. A \textbf{spectral sequence} in $\mathcal{A}$ is a collection of objects
    \[ (E_r^{p, q}, E^n) \qquad n, p, q, r \in \ZZ, r \ge 1 \]
    and morphisms
    \[ d_r^{p, q} : E_r^{p, q} \to E_r^{p + r, q - r + 1} \]
    satisfying
    \begin{itemize}
        \item $d_r^2 = 0$ (i.e. the $E_r^{p + \bdot r, q - \bdot r + \bdot}$ are complexes),
        \item there are isomorphisms (which are part of the data)
        \[ E_{r + 1}^{p, q} \simeq H^0(E_r^{p + \bdot r, q - \bdot r + \bdot}), \]
        \item for any $(p, q)$ there exists an $r_0$ such that $d_r^{p, q} = d_r^{p - r, q + r - 1} = 0$ for all $r \ge r_0$. In particular, $E_r^{p, q} \simeq E_{r_0}^{p, q}$ for all $r \ge r_0$, and this object is denoted by $E_\infty^{p, q}$.
        \item There is a decreasing filtration
        \[ 0 \subset \cdots \subset F^{p + 1} E^n \subset F^p E^n \subset \cdots E^n , \]
        with $\cap_p F^p E^n = 0$ and $\cup_p F^p E^n = E^n$, and isomorphisms
        \[ E_\infty^{p, q} \simeq F^p E^{p + q} / F^{p + 1} E^{p + q} . \]
    \end{itemize}
    
    In some sense, the objects $E_r^{p, q}$ converge towards subquotients of a certain filtration of $E^n$. Usually, the objects of one layer of some fixed $r$ are given. Then one writes
    \[ E_r^{p, q} \Rightarrow E^{p + q} . \]
\end{topic}

\begin{topic}{serre-subcategory}{Serre subcategory}
    A non-empty \tref{CT:full-subcategory}{full subcategory} $\mathcal{B}$ of an \tref{abelian-category}{abelian category} $\mathcal{A}$ is a \textbf{Serre subcategory} if for any exact sequence in $\mathcal{A}$,
    \[ 0 \to A \to B \to C \to 0 , \]
    $B$ is in $\mathcal{B}$ if and only if $A$ and $C$ are in $\mathcal{B}$. In words, $\mathcal{B}$ is closed under taking subobjects, quotient objects and extensions.
\end{topic}

\begin{topic}{grothendieck-group}{Grothendieck group}
    The \textbf{Grothendieck group} $\text{K}_0(\mathcal{A})$ of an \tref{abelian-category}{abelian category} $\mathcal{A}$ is defined as the free abelian group on isomorphism classes of objects in $\mathcal{A}$, modulo the relations $[B] = [A] + [C]$ for every short exact sequence
    \[ 0 \to A \to B \to C \to 0 \]
    in $\mathcal{A}$.
\end{topic}

\begin{example}{grothendieck-group}
    Let $\mathcal{A} = \textbf{Vect}_k$ be the category of finite-dimensional vector spaces over a field $k$. Since such vector spaces are determined up to isomorphism by their dimension, and $\dim_k V \oplus W = \dim_k V + \dim_k W$, we find that $\text{K}_0(\textbf{Vect}_k) = \ZZ$.
\end{example}

\begin{example}{grothendieck-group}
    Let $\mathcal{A} = \text{Coh}(\PP^n)$ be the category of coherent sheaves on projective space $\PP^n$. Note that the twisting sheaves $\mathcal{O}(d)$ must generate the Grothendieck group $\text{K}_0(\mathcal{A})$, as any coherent sheaf has a finite resolution by direct sums of twisting sheaves. The exact Koszul complex
    \[ 0 \to \mathcal{O} \to \bigoplus^{n + 1} \mathcal{O}(1) \to \bigoplus^{\binom{n + 1}{2}} \mathcal{O}(2) \to \cdots \to \bigoplus^{n + 1} \mathcal{O}(n) \to \mathcal{O}(n + 1) \to 0 \]
    now shows that
    \[ \text{K}_0(\text{Coh}(\PP^n)) = \ZZ[T] / (1 - T)^{n + 1} \]
    (injectivity can be shown using euler-characteristics). Moreover, the tensor product of coherent sheaves corresponds to the multiplication in the ring.
\end{example}

\begin{topic}{ext-functors}{Ext functors}
    Let $\mathcal{A}$ be an \tref{abelian-category}{abelian category} and $A$ an object of $\mathcal{A}$. The functor
    \[ \Hom_{\mathcal{A}}(A, -) : \mathcal{A} \to \textbf{Ab} \]
    is \tref{exact-functor}{left exact}, so (if $\mathcal{A}$ has enough injectives) it has \tref{right-derived-functors}{right derived functors}, which are known as the \textbf{Ext functors}
    \[ \text{Ext}^i_\mathcal{A}(A, B) = \text{R}^i \Hom_\mathcal{A}(A, B) . \]
\end{topic}

\begin{example}{ext-functors}
    For $n > 0$, elements of the Ext group $\text{Ext}^n(A, B)$ correspond to extensions
    \[ 0 \to B \to E_0 \to \cdots \to E_{n - 1} \to A \to 0 . \]
    Consider the case $n = 1$. Let $P^\bdot \to A$ (call the differential maps $\pi_i$) be a projective resolution of $A$, so that $\text{Ext}^1(A, B) = H^1(\Hom(P^\bdot, B))$. This is the group of maps $g : P^1 \to B$ such that $g \circ \pi_2 = 0$, modulo all maps of the form $f \circ \pi_0$ for $f : A \to B$. Now we can construct the pushout
    \[ E_0 = P^0 \oplus B / \{ \pi_1(q) = g(q) \text{ for } q \in P^1 \} . \]
    Note that $g$ descends to $\overline{g} : P^0 \to B$, and that the maps $B \to E_0$, $b \mapsto (b, 0)$ and $E_0 \to A, (b, p) \mapsto \pi_0(p)$ are well-defined (the latter as $\pi_0 \pi_1 = 0$), and respectively injective and surjective. It is not hard to prove exactness in the middle, so we get an extension
    \[ 0 \to B \to E_0 \to A \to 0 . \]
    In particular, the trivial element in $\text{Ext}^1(A, B)$ yields the trivial extension
    \[ 0 \to A \to A \oplus B \to B \to 0 . \]
\end{example}

\begin{topic}{tor-functors}{Tor functors}
    For a commutative \tref{CA:ring}{ring} $R$ and an $R$-module $B$, the \tref{CA:tensor-product}{tensor product} gives a functor
    \[ (-) \otimes_R B : R\textbf{-Mod} \to R\textbf{-Mod} \]
    which is \tref{exact-functor}{right exact}, so it has \tref{left-derived-functors}{left derived functors}, which are known as the \textbf{Tor functors}
    \[ \text{Tor}^R_i(A, B) = A \overset{\text{L}_i}{\otimes} B . \]
\end{topic}

\begin{topic}{mapping-complex}{mapping complex}
    Let $C_\bdot$ and $D_\bdot$ be \tref{chain-complex}{chain complexes} with values in an \tref{additive-category}{additive category} $\mathcal{A}$. The \textbf{mapping complex} associated to $C_\bdot$ and $D_\bdot$ is the chain complex $[C, D] _\bdot$ given by
    \[ [C, D]_n = \prod_{k \in \ZZ} \Hom_\mathcal{A}(C_k, D_{k + n}) \]
    and with differentials
    \[ \partial_n : [C, D]_n \to [C, D]_{n - 1}, \quad (f_k)_{k \in \ZZ} \mapsto \left(\partial_D \circ f_k - (-1)^n f_{k - 1} \circ \partial_C \right)_{k \in \ZZ} . \]
\end{topic}

\begin{topic}{composition-series}{composition series}
    A \textbf{composition series} of an object $A$ in an \tref{abelian-category}{abelian category} $\mathcal{A}$ is a sequence of subobjects
    \[ 0 = A_0 \subsetneq A_1 \subsetneq \cdots \subsetneq A_n = A \]
    such that the quotients $A_i/A_{i - 1}$, known as the \textbf{composition factors}, are simple for all $1 \le i \le n$. If $A$ has a composition series, the integer $n$ only depends on $A$ and is called the \textit{length} of $A$.
\end{topic}

\begin{example}{composition-series}
    The \textbf{Jordan--Hölder theorem} states that if an object $A$ has two composition series
    \[ 0 = A_0 \subsetneq A_1 \subsetneq \cdots \subsetneq A_n = A \qquad \text{and} \qquad 0 = B_0 \subsetneq B_1 \subsetneq \cdots \subsetneq B_m = A , \]
    then $m = n$ and the composition factors $A_i/A_{i - 1}$ are isomorphic to $B_i/B_{i - 1}$ up to permutation.
    
    To prove this, let $j$ be the smallest index such that $A_1 \subset B_j$. Then the map $A_1 \to B_j/B_{j - 1}$ must be an isomorphism. Namely, it is surjective as its image is non-zero (since $A_1 \nsubseteq B_{j - 1}$) and $B_j/B_{j - 1}$ is simple. Furthermore, it is injective since its kernel is a proper subobject of $A_1$, and thus zero since $A_1$ is simple. Now, the quotient $A/A_1 = A_n/A_1 = B_m/A_1$ has the two filtrations
    \[ 0 \subsetneq A_2/A_1 \subsetneq A_3/A_1 \subsetneq \cdots \subsetneq A_n/A_1 \]
    and
    \[ 0 \subsetneq (B_1 + A_1)/A_1 \subsetneq \cdots \subsetneq (B_{j - 1} + A_1)/A_1 = B_j/A_1 \subsetneq B_{j + 1}/A_1 \subsetneq \cdots \subsetneq B_m/A_1 \]
    and the result follows from induction.
\end{example}

\begin{topic}{yoneda-product}{Yoneda product}
    The \textbf{Yoneda product} is the pairing between \tref{ext-functors}{Ext groups}
    \[ \text{Ext}^n(A, B) \otimes \text{Ext}^m(B, C) \to \text{Ext}^{m + n}(A, C) \]
    induced by $\Hom(A, B) \otimes \Hom(B, C) \to \Hom(A, C)$, $f \otimes g \mapsto g \circ f$.
    
    Explicitly, given two extensions
    \[ \xi : 0 \to B \to E_0 \to \cdots \to E_{n - 1} \to A \to 0 \]
    \[ \eta : 0 \to C \to F_0 \to \cdots \to F_{m - 1} \to B \to 0 \]
    the Yoneda cup product is given by
    \[ \xi \smile \eta : 0 \to C \to F_0 \to \cdots \to F_{m - 1} \to E_0 \to \cdots \to A \to 0 \]
    in $\text{Ext}^{m + n}(A, C)$.
\end{topic}

\begin{example}{yoneda-product}
    For $n = 0$ and $m = 1$ we have that the product of $f : A \to B$ (in $\text{Ext}^0(A, B) = \Hom(A, B)$) and $\eta : 0 \to C \to F_0 \to B \to 0$ (in $\text{Ext}^1(A, B)$) is
    \[ 0 \to C \to F_0 \oplus_B A \to A \to 0 . \]
    It is not hard to verify this sequence is indeed exact.
\end{example}

\begin{topic}{homological-dimension}{(global) homological dimension}
    Let $\mathcal{A}$ be an \tref{abelian-category}{abelian category}, and $A$ an object in $\mathcal{A}$. The \textbf{homological dimension} of $A$ is the least number $n$ such that there exists a \tref{projective-resolution}{projective resolution}
    \[ 0 \to P^n \to \cdots \to P^1 \to P^0 \to A . \]
    If such a resolution does not exist, the homological dimension of $A$ is $\infty$.
    
    The \textbf{global homological dimension} of the category $\mathcal{A}$ is the supremum of the homological dimension of all objects $A$ of $\mathcal{A}$.
\end{topic}

\begin{example}{homological-dimension}
    By \tref{CA:hilbert-syzygy-theorem}{Hilbert's syzygy theorem}, any $k[x_1, \ldots, x_n]$-module has a projective resolution of length $\le n$. Since this upper bound is attained for the \tref{UN:koszul-complex}{Koszul complex}, the global homological dimension of $k[x_1, \ldots, x_n]\text{-}\textbf{Mod}$ is $n$.
\end{example}

\begin{example}{homological-dimension}
    Knowing the global homological dimension $n$ of a category $\mathcal{A}$ is useful, as it follows that for any \tref{exact-functor}{right exact functor} $F : \mathcal{A} \to \mathcal{B}$ the \tref{left-derived-functors}{left derived functors} $\text{L}^i F A$ are zero for all $A$ in $\mathcal{A}$ and $i > n$.
    
    In particular, for $F = \Hom_\mathcal{A}(-, B)$ we find $\text{Ext}_\mathcal{A}^i(A, B) = 0$ for all $A, B$ in $\mathcal{A}$ and $i > n$.
\end{example}

\begin{topic}{grothendieck-spectral-sequence}{Grothendieck spectral sequence}
    Let $F : \mathcal{A} \to \mathcal{B}$ and $G : \mathcal{B} \to \mathcal{C}$ be \tref{exact-functor}{left exact functors} between \tref{abelian-category}{abelian categories}, such that $\mathcal{A}$ and $\mathcal{B}$ have enough injectives, and $F$ maps \tref{injective-object}{injective objects} to \tref{acyclic-object}{$G$-acyclic objects}. Then there is a \tref{spectral-sequence}{spectral sequence}
    \[ E_2^{p, q} = (R^p G \circ R^q F)(A) \Rightarrow R^{p + q}(G \circ F)(A) , \]
    called the \textbf{Grothendieck spectral sequence}.
\end{topic}

\begin{topic}{leray-spectral-sequence}{Leray spectral sequence}
    Let $X$ and $Y$ be \tref{TO:topological-space}{topological spaces} and $f : X \to Y$ a continuous map. For a \tref{AG:sheaf}{sheaf} $\mathcal{F}$ on $X$, there exists a \tref{spectral-sequence}{spectral sequence}
    \[ E_2^{p, q} = H^p(Y, R^q f_* \mathcal{F}) \Rightarrow H^{p + q}(X, \mathcal{F}) \]
    called the \textbf{Leray spectral sequence}. It is a special case of the \tref{grothendieck-spectral-sequence}{Grothendieck spectral sequence}.
\end{topic}

\begin{topic}{exact-category}{exact category}
    An \textbf{exact category} is an \tref{additive-category}{additive category} $\mathcal{A}$ with a class $E$ of sequences
    \[ A \to B \to C \]
    (referred to as `\textit{short exact sequences}') such that
    \begin{itemize}
        \item $E$ is closed under isomorphisms and contains all canonical sequences of the form
        \[ A \to A \oplus C \to C , \]
        \item if $B \to C$ occurs as the second arrow of a sequence in $E$ (called an \textit{admissible epimorphism}) and $D \to C$ is any morphism, then the pullback exists, and the projection to $D$ is an admissible epimorphism as well. Dually, if $A \to B$ occurs as the first arrow of a sequence in $E$ (called an \textit{admissible monomorphism}) and $A \to D$ is any morphism, then the pushout exists, and the inclusion from $A$ is also an admissible monomorphism,
        \item admissible monomorphisms are kernels of their corresponding admissible epimorphisms, and dually admissible epimorphisms are cokernels of their corresponding admissible monomorphisms. The composition of two admissible monomorphisms is admissible, and similarly for admissible epimorphisms.
    \end{itemize}
\end{topic}

\begin{topic}{frobenius-category}{Frobenius category}
    A \textbf{Frobenius category} is an \tref{exact-category}{exact category} with enough \tref{injective-object}{injectives} and enough \tref{projective-object}{projectives}, such that the classes of injectives and projectives coincides.
\end{topic}

\begin{topic}{mapping-cone}{mapping cone}
    Let $\mathcal{A}$ be an \tref{abelian-category}{abelian category}, and $f : A^\bdot \to B^\bdot$ a morphism between \tref{chain-complex}{cochain complexes} in $\mathcal{A}$. The \textbf{mapping cone} of $f$ is the complex $C(f)$ in $\mathcal{A}$ given by
    \[ C(f)^i = A^{i + 1} \oplus B^i , \]
    with differential
    \[ d_{C(f)}^i = \begin{pmatrix} -d_A^{i + 1} & 0 \\ f^{i + 1} & d_B^i \end{pmatrix} . \]    
\end{topic}
