% \begin{topic}{chain-complex}{chain complex}
%     A \textbf{chain complex} $C_\bullet$ is a sequence of abelian groups $C_n$, for $n \in \ZZ$, together with group morphisms $d_n : C_n \to C_{n - 1}$ such that $d_{n - 1} \circ d_n = 0$ for all $n$.
%     \[ \cdots \xrightarrow{d_{n + 2}} C_{n + 1} \xrightarrow{d_{n + 1}} C_n \xrightarrow{d_n} C_{n - 1} \xrightarrow{d_{n - 1}} \cdots \]
%     Elements of $C_n$ are called \textit{$n$-chains}. Elements of $\ker d_n$ are called $n$-cycles. Elements of $\im d_{n + 1}$ are called $n$-boundaries. Note that any $n$-boundary is an $n$-cycle.
% \end{topic}

% \begin{topic}{cochain-complex}{cochain complex}
%     A \textbf{cochain complex} $C^\bullet$ is a sequence of abelian groups $C^n$, for $n \in \ZZ$, together with group morphisms $d^n : C^n \to C^{n + 1}$ such that $d^{n + 1} \circ d^n = 0$ for all $n$.
%     \[ \cdots \xrightarrow{d^{n - 2}} C_{n - 1} \xrightarrow{d_{n - 1}} C_n \xrightarrow{d_n} C_{n + 1} \xrightarrow{d_{n + 1}} \cdots \]
%     Elements of $C_n$ are called \textit{$n$-cochains}. Elements of $\ker d_n$ are called $n$-cocycles. Elements of $\im d_{n - 1}$ are called $n$-coboundaries. Note that any $n$-coboundary is an $n$-cocycle.
% \end{topic}

% \begin{topic}{chain-map}{chain map}
%     Let $C_\bullet$ and $D_\bullet$ be \tref{chain-complex}{chain complexes}. A morphism between chain complexes $f : C_\bullet \to D_\bullet$, that is a \textbf{chain map}, is given by a sequence of group morphisms $f_n : C_n \to D_n$ such that the following squares commute for all $n \in \ZZ$:
%     \[ \begin{tikzcd} C_n \arrow{r}{f_n} \arrow[swap]{d}{d_n} & D_n \arrow{d}{d_n} \\ C_{n - 1} \arrow[swap]{r}{f_{n - 1}} & D_{n - 1} \end{tikzcd} \]
    
%     Note that chain maps induce group morphisms on the \tref{homology-group}{homology groups}, $H_n(C_\bullet) \to H_n(D_\bullet)$.
% \end{topic}

\begin{topic}{homology-group}{homology group}
    Let $C_\bullet$ be a \tref{HA:chain-complex}{chain complex}. The quotient
    \[ H_n(C_\bullet) = \ker d_n / \im d_{n + 1} \]
    is called the \textbf{$n$-th homology group} of $C_\bullet$.
\end{topic}

\begin{topic}{long-exact-sequence-homology}{long exact sequence in homology}
    Let
    \[ 0 \rightarrow C' \xrightarrow{i} C \xrightarrow{p} \overline{C} \rightarrow 0 \]
    be a short exact sequence of \tref{HA:chain-complex}{chain complexes}. Then there is an induced long exact sequence
    \[ \cdots \rightarrow H_n(C') \xrightarrow{i_*} H_n(C) \xrightarrow{p_*} H_n(\overline{C}) \xrightarrow{\delta} H_{n - 1}(C') \rightarrow \cdots , \]
    where $\delta$ is called the \textit{connecting homomorphism}: for any $\alpha = [ p(c) ] \in H_n(\overline{C})$ with $c \in C_n$, we have $\delta(\alpha) = [ d_n(c) ] \in H_{n - 1}(C')$.
\end{topic}

\begin{topic}{singular-homology}{singular (co)homology}
    Let $X$ be a topological space and $A$ an abelian group. A \textbf{singular $n$-simplex} in $X$ is a continous map
    \[ \sigma : \Delta^n \to X , \]
    where 
    \[ \Delta^n = \left\{ (x_0, \ldots, x_n) \in \RR^{n + 1} : \text{all } x_i \ge 0 \text{ and } \sum_{i = 0}^n x_i = 1 \right\} \]
    denotes the \textit{standard $n$-simplex}. The set of all singular $n$-simplices in $X$ is denoted $\mathcal{S}(X)_n$.
    
    The \textbf{singular chain complex} of $X$ with coefficients in $A$ is the \tref{HA:chain-complex}{chain complex} given by
    \[ C_n(X; A) = A[\mathcal{S}(X)_n] , \]
    ($C_n(X; A) = 0$ for $n < 0$) and the differentials $d_n : C_n(X; A) \to C_{n - 1}(X; A)$ are induced from
    \[ \mathcal{S}(X)_n \to \mathcal{S}(X)_{n - 1} : \sigma \mapsto \sum_{i = 1}^{n} (-1)^i \; \sigma \circ \delta_i , \]
    where $\delta_i : \Delta^{n - 1} \to \Delta^n$ maps to the $i$-th face.
    
    The \tref{homology-group}{homology groups} of $C_\bullet(X; A)$ are called the \textbf{singular homology groups}, and denoted
    \[ H_n(X; A) . \]
    
    Dually, there is the \textbf{singular cochain complex}
    \[ C^n(X; A) = \Hom_\ZZ(C_n(X; \ZZ), A) \]
    The cohomology groups of $C^\bullet(X; A)$ are called the \textbf{singular cohomology groups}, and denoted
    \[ H^n(X; A) . \]
\end{topic}

\begin{topic}{fundamental-group}{fundamental group}
    Let $X$ be a topological space, and take a point $x_0 \in X$. The \textbf{fundamental group} of $X$ w.r.t. the basepoint $x_0$ is given by
    \[ \pi_1(X, x_0) = \{ \text{continuous } f : [0, 1] \to X : f(0) = x_0 = f(1) \} / \sim{} , \]
    where $f \sim{} g$ if the paths are homotopic.
    
    The group multiplication is given by concatenating paths, and inversion is given by inverting paths. In general, the fundamental group is not abelian.
\end{topic}

\begin{topic}{relative-homology}{relative homology}
    Let $X$ be a topological space, $X' \subset X$ a subspace, and $A$ an abelian group. The \textbf{relative chain complex} of $(X, X')$ is defined as the quotient complex
    \[ C_n(X, X'; A) = C_n(X; A) / C_n(X'; A) .  \]
    Its \tref{homology-group}{homology groups}
    \[ H_n(X, X'; A) = H_n(C_\bullet(X, X; A)) \]
    are called \textbf{relative homology groups} of $(X, X')$.
    
    From the exact sequence
    \[ 0 \to C_\bdot(X'; A) \to C_\bdot(X; A) \to C_\bdot(X, X'; A) \to 0 \]
    follows the long exact sequence in homology
    \[ \cdots \to H_n(X; A) \to H_n(X, X'; A) \xrightarrow{\delta} H_{n - 1}(X'; A) \to \cdots \to H_0(X, X'; A) \to 0 . \]
\end{topic}

\begin{topic}{excision-theorem}{excision theorem}
    Let $X$ be a topological space, $X' \subset X$ a subpace, and $Y \subset X'$ a subspace such that $\text{closure}(Y) \subset \text{interior}(X')$. The \textbf{excision theorem} states that the inclusion $X \backslash Y \hookrightarrow X$ induces isomorphisms of relative homology groups
    \[ H_n(X \backslash Y, X' \backslash Y; A) \xrightarrow{\sim} H_n(X, X'; A) \]
    for all $n \in \ZZ$ and coefficient groups $A$.
\end{topic}

\begin{topic}{mayer-vietoris-sequence}{Mayer–Vietoris sequence}
    Let $X$ be a topological space, and $U, V \subset X$ two subsets with $X = \textup{interior}(U) \cup \textup{interior}(V)$. Then there is a long exact sequence of homology groups (with coefficients in $A$)
    \[ \cdots \longrightarrow H_n(U \cap V; A) \overset{i}{\longrightarrow} H_n(U; A) \oplus H_n(V; A) \overset{p}{\longrightarrow} H_n(X; A) \overset{\partial}{\longrightarrow} H_{n - 1}(U \cap V; A) \longrightarrow \cdots \]
    where $i$ is induced by the inclusions $U \cap V \hookrightarrow U$ and $U \cap V \hookrightarrow V$, and $p = i_*^U - i_*^V$. The map $\delta$ is a connecting morphism. This sequence is called the \textbf{Mayer–Vietoris sequence}.
    
    Dually, there is one for cohomology
    \[ \cdots \longrightarrow H^n(X; A) \overset{r}{\longrightarrow} H^n(U; A) \oplus H^n(V; A) \overset{\Delta}{\longrightarrow} H^n(U \cap V; A) \overset{\partial}{\longrightarrow} H^{n + 1}(X; A) \longrightarrow \cdots \]
\end{topic}

\begin{topic}{deformation-retract}{(strong) deformation retract}
    A \textbf{deformation retraction} of a \tref{TO:topological-space}{topological space} $X$ onto a subspace $A$ is a continuous map
    \[ F : X \times [0, 1] \to X \]
    such that 
    \[ F(x, 0) = x, \quad F(x, 1) \in A \quad \text{ and } \quad F(a, 1) = a \]
    for all $x \in X$ and $a \in A$.
    It is a \textbf{strong deformation retraction} if moreover
    \[ F(a, t) = a \qquad \text{ for all } t \in [0, 1] . \]
\end{topic}

\begin{example}{deformation-retract}
    The $n$-sphere $S^n$ is a strong deformation retract of $\RR^{n + 1} - \{ 0 \}$, given by the map
    \[  F(x, t) = (1 - t) x + \frac{t x}{\norm{x}} . \]
\end{example}

\begin{topic}{cup-product}{cup product}
    The \textbf{cup product} is a product on the \tref{singular-homology}{singular cohomology} of a \tref{TO:topological-space}{topological space} $X$. Given a commutative \tref{CA:ring}{ring} $R$, it is the product
    \[ (-) \smile (-) : C^p(X; R) \times C^q(X; R) \to C^{p + q}(X; R) \]
    given by
    \[ (f \smile g)(\sigma) = f(\sigma \circ i_p^\text{front}) \cdot g(\sigma \circ i_q^\text{back}) \]
    where $i_p^\text{front} : \Delta^{p} \to \Delta^{p + q}$ sends vertex $i$ to vertex $i$, and $i_q^\text{back}$ sends vertex $i$ to vertex $n - q + i$.
    
    The cup product descents to a product on the cohomology groups
    \[ (-) \smile (-) : H^p(X; R) \times H^q(X; R) \to H^{p + q}(X; R), \qquad ([f], [g]) \mapsto [f \smile g] , \]
    making $H^*(X; R)$ into a graded commutative $R$-algebra (that is, $[f] \smile [g] = (-1)^{pq} [g] \smile [f]$).
    
    The cup product satisfies the following properties:
    \begin{itemize}
        \item (\textit{Leibniz rule}) $d(f \smile g) = df \smile g + (-1)^p f \smile dg$,
        \item (\textit{functorial}) for every map $\varphi : X \to Y$, $\varphi^*(f \smile g) = \varphi^* f \smile \varphi^* g$,
        \item (\textit{associative}) $f \smile (g \smile h) = (f \smile g) \smile h$,
        \item (\textit{unital}) $f \smile 1 = f = 1 \smile f$.
    \end{itemize}
\end{topic}

\begin{topic}{cap-product}{cap product}
    Let $X$ be a \tref{TO:topological-space}{topological space}. Given a commutative \tref{CA:ring}{ring} $R$, the \textbf{cap product} is the following bilinear map on \tref{singular-homology}{singular (co)homology}
    \[ (-) \frown (-) : C^q(X; R) \times C_{p + q}(X; R) \to C_p(X; R) \]
    induced by sending $f : \mathcal{S}(X)_q \to R$ and $\sigma \in \mathcal{S}(X)_{p + q}$ to
    \[ f \frown \sigma = (-1)^{pq} f(\sigma \circ i_q^\text{back}) \cdot (\sigma \circ i_p^\text{front})  \]
    where $i_p^\text{front} : \Delta^{p} \to \Delta^{p + q}$ sends vertex $i$ to vertex $i$, and $i_q^\text{back}$ sends vertex $i$ to vertex $n - q + i$.
    
    The cup product descents to a product on the cohomology groups
    \[ (-) \frown (-) : H^q(X; R) \times H_{p + q}(X; R) \to H_q(X; R), \qquad ([f], [g]) \mapsto [f \frown g] . \]
\end{topic}

\begin{topic}{weak-homotopy-equivalence}{weak-homotopy-equivalence}
    A \textbf{weak homotopy equivalence} is a continuous map $f : X \to Y$ if it induces bijections
    \[ f_* : \pi_n(X, x) \to \pi_n(Y, f(x)) \]
    for all $n \ge 0$ and all basepoints $x \in X$.
\end{topic}

\begin{topic}{null-homotopic}{null homotopic}
    A continuous map $f : X \to Y$ between \tref{TO:topological-space}{topological spaces} is \textbf{null-homotopic} if it is homotopic to a constant map.
\end{topic}

\begin{topic}{homotopy-groups}{homotopy groups}
    For any $n \ge 0$, the \textbf{$n$-th homotopy group} of a \tref{TO:topological-space}{topological space} $X$ with basepoint $x_0 \in X$ is the set
    \[ \pi_n(X, x_0) = [(S^n, s_0), (X, x_0)]_* \]
    of basepoint preserving homotopy classes of based continuous maps $S^n \to X$.
    
    For $n \ge 1$ it has a group structure.
\end{topic}

\begin{topic}{cohomology-compact-support}{cohomology with compact support}
    Let $X$ be a \tref{TO:topological-space}{topological space}. The \textbf{cohomology of $X$ with compact support} $H_c^n(X; R)$ is given by the cohomology of the following subcomplex of the singular cohomology complex
    \[ C_c^n(X; R) = \big\{ f : \mathcal{S}(X)_n \to R : \text{ there is a compact $K \subset X$ with $f(\sigma) = 0$ for all $\sigma \in \mathcal{S}(X \backslash K)_n$} \big\} . \]
    
    Alternatively,
    \[ H_c^n(X; R) = \varinjlim H^n(X, X \backslash K; R) \]
    where the direct limit is taken over all compact $K \subset X$ w.r.t. inclusions.
\end{topic}
