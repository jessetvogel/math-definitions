\begin{topic}{topological-quantum-field-theory}{Topological Quantum Field Theory (TQFT)}
    Let $(\textbf{Bd}_n, \sqcup, \varnothing)$ be the \tref{CT:monoidal-category}{monoidal category} of (diffeomorphism classes of) $n$-dimensional (\tref{DG:orientable-manifold}{oriented}) \tref{DG:bordism}{bordisms}, and let $k$ be a \tref{AA:ring}{commutative ring}. An $n$-dimensional \textbf{Topological Quantum Field Theory (TQFT)} over $k$ is a \tref{CT:monoidal-functor}{monoidal functor}
    \[ Z \colon \textbf{Bd}_n \to \textbf{Mod}_k , \]
    where the monoidal structure on $\textbf{Mod}_k$ is given by the \tref{AA:tensor-product}{tensor product}.
\end{topic}

% \begin{example}{topological-quantum-field-theory}
%     Let $Z \colon \textbf{Bd}_1 \to \textbf{Mod}_k$ be a $1$-dimensional TQFT. Since the objects of $\textbf{Bd}_1$ consist of finitely many points, the value of $Z$ on objects is completely determined by $Z(\star)$, as $Z(k \textup{ points}) = Z(\star)^{\otimes k}$. 
% \end{example}

\begin{example}{topological-quantum-field-theory}
    Let $Z \colon \textbf{Bd}_n \to \textbf{Mod}_k$ be an $n$-dimensional TQFT over a field $k$. For any closed oriented $(n - 1)$-dimensional manifold $M$, let $\overline{M}$ be the manifold with reversed orientation. Consider the bordisms
    \[ U_M \colon M \sqcup \overline{M} \to \varnothing \quad \textup{ and } \quad U_M^\dag \colon \varnothing \to \overline{M} \sqcup M \]
    given by the cylinder $M \times [0, 1]$.
    Note that $Z(U_M)(1) = \sum_{i = 1}^{m} v_i \otimes \overline{v}_i$ for some $v_i \in Z(M)$ and $\overline{v}_i \in Z(\overline{M})$, and moreover we can pick such $v_i$ linearly independent and $\overline{v}_i$ linearly independent.
    Now since $(U_M \sqcup \id_M) \circ (\id_M \sqcup U_M^\dag) = \id_M$, it follows that
    \[ v = \sum_{i = 1}^{m} Z(U_M^\dag)(v \otimes \overline{v}_i) v_i \]
    for all $v \in Z(M)$. In particular, this implies $Z(M)$ is finite-dimensional and $\{ v_1, \ldots, v_m \}$ is a basis for $Z(M)$.
    Completely analogous, switching the roles of $M$ and $\overline{M}$, from the equality $(\id_{\overline{M}} \sqcup U_M) \circ (U_M^\dag \sqcup \id_{\overline{M}}) = \id_{\overline{M}}$ we find that
    \[ \overline{v} = \sum_{i = 1}^{m} Z(U_M^\dag)(v_i \otimes \overline{v}) \overline{v}_i \]
    for all $\overline{v} \in Z(\overline{M})$, so $Z(\overline{M})$ is finite-dimensional as well, and $\{ \overline{v}_1, \ldots, \overline{v}_m \}$ is a basis for $Z(\overline{M})$. Moreover, this shows that $Z(\overline{M})$ can be identified as the \tref{LA:dual-vector-space}{dual} to $Z(M)$, with $\{ \overline{v}_1, \ldots, \overline{v}_m \}$ as the dual basis of $\{ v_1, \ldots, v_m \}$ with respect to the non-degenerate pairing
    \[ Z(M) \otimes_k Z(\overline{M}) \to k, \quad v \otimes \overline{v} \mapsto Z(U_M^\dagger)(v \otimes \overline{v}) . \]
\end{example}

\begin{topic}{path-ordered-exponential}{path-ordered exponential}
    Let $A$ be a real or complex \tref{AA:algebra}{algebra} and $a \colon \RR \to A$ a \tref{TO:continuous-map}{continuous function}. The \textbf{path-ordered exponential} of $a$ is the function $\textup{P} \exp [a](t) \colon \RR \to A$ given by
    \[ \textup{P} \exp [a] (t) = \sum_{n = 0}^\infty \int_0^t \int_0^{t'_n} \int_0^{t'_{n - 1}} \cdots \int_0^{t'_2} a(t'_n) \cdots a(t'_1) d t'_1 \cdots d t'_{n - 2} d t'_{n - 1} d t'_n . \]
\end{topic}

\begin{example}{path-ordered-exponential}
    The path-ordered exponential is the unique solution to the initial value problem of finding an $f \colon \RR \to A$ satisfying
    \[ \frac{d}{dt} f(t) = a(t) f(t) \quad \textup{ with } \quad f(0) = 1 . \]
\end{example}

\begin{example}{path-ordered-exponential}
    When $A$ is commutative or $a$ is constant, the path-ordered exponential reduces to the ordinary exponential
    \[ \textup{P} \exp[a](t) = \exp \left( \int_0^t a(s) ds \right) . \]
\end{example}