\subsection{Algebraic curves}

\begin{definition}
    A \keyword{curve} is a projective variety of dimension one.
\end{definition}

\begin{proposition}
    Let $P$ be a point on a curve $C$. Then $\mathcal{O}_{C, P}$ is a discrete valuation ring.
\end{proposition}
\begin{proof}
    \todo{proof}
\end{proof}

\begin{definition}
    The \keyword{valuation} on $\mathcal{O}_{C, P}$ is given by
    \begin{align*}
        \text{ord}_P : \mathcal{O}_{C, P} &\to \{ 1, 2, 3, \ldots \} \cup \{ \infty \} \\
        f &\mapsto \sup \left\{ n \in \ZZ : f \in \mathfrak{m}_P^n \right\}
    \end{align*}
    Using $\text{ord}_P(f / g) = \text{ord}_P(f) - \text{ord}_P(g)$ \todo{well-defined?} we can extend this to a map
    \[ \text{ord}_P : \overline{k}(C) \to \ZZ \cup \{ \infty \} . \]
    A \keyword{uniformizer} for $C$ at $P$ is a function $t \in \overline{k}(C)$ such that $\text{ord}_P(t) = 1$, i.e. a generator for $\mathfrak{m}_P$.
    
    For $f \in \overline{k}(C)$, we refer to $\text{ord}_P(f)$ as the \keyword{order} of $f$ at $P$. When $\text{ord}_P(f) > 0$, $f$ has a zero at $P$. When $\text{ord}_P(f) < 0$, $f$ has a pole at $P$. If $\text{ord}_P(f) \ge 0$, $f$ is \keyword{regular} at $P$ and we can evaluate $f(P)$.
\end{definition}

\todo{existence of uniformizers?}

\begin{proposition}
    For every $f \in \overline{k}(C)$, there are only finitely many points at $C$ where $f$ has a zero or a pole. If $f$ has no poles or zeros, then $f$ is constant.
\end{proposition}

\begin{proof}
    \todo{..}
\end{proof}

\begin{example}
    Consider the curve
    \[ C : Y^2 Z = X^3 + X Z^2 . \]
    Take the point $P = (0 : 0 : 1)$. The maximal ideal $\mathfrak{m}_P$ has the property that $\mathfrak{m}_P / \mathfrak{m}_P^2$ is generated by $Y$, so we have
    \[
        \text{ord}_P(Y) = 1, \qquad
        \text{ord}_P(X) = \text{ord}_P(Y^2 / (X^2 + Z^2))
        % = \text{ord}_P(Y^2) - \text{ord}_P(X^2 + Z^2)
        = 2 - 0 = 2 . \]
\end{example}




\begin{theorem}
    Let $\varphi : C_1 \to C_2$ be a morphism of curves. Then $\varphi$ is either constant or surjective.
\end{theorem}

\begin{proof}
    \todo{?}
\end{proof}

\begin{theorem}
    Equivalence of categories
    \[ \{ \text{smooth curves over $k$ with non-constant maps} \} \simeq \\ \{ \text{finitely generated extensions $K/k$ of transcendence degree one s.t. $K \cap \overline{k} = k$, with $k$-linear maps} \} \]
\end{theorem}

% \begin{example}[Hyperelliptic curves]
%     Assume that $\char(k) \ne 2$. Let $f \in k[x]$ be a polynomial without double roots. Then consider the curve which is the homogenization of
%     \[ C : y^2 = f(x) . \]
%     Indeed this curve is nonsingular, because if it were singular at a point $P = (x_0, y_0)$ ...
% \end{example}




\subsubsection{Divisors}

\begin{definition}
    The \keyword{divisor group} of a curve $C$ over $k$, denoted $\text{Div}(C)$, is the free abelian group generated by the points of $C$. So a divisor $D \in \text{Div}(C)$ looks like
    \[ D = \sum_{P \in C} n_P , \]
    where finitely many $n_P$ are nonzero.
    The \keyword{degree} of a divisor $D$ defined by
    \[ \text{deg}(D) = \sum_{P \in C} n_P . \]
    The \keyword{divisors of degree zero} form a subgroup of $\text{Div}(C)$, which we denote by $\text{Div}^0(C)$.
\end{definition}

\begin{definition}
    Assume that $C$ is a smooth curve, and let 
\end{definition}
 



\begin{example}
    On $\PP^1$, every divisor of degree $0$ is principal. Namely, take $D = \sum_{P \in \PP^1} n_P$ and define
    \[ f = \prod_{P \in \PP^1} (P_X X - P_Y Y)^{n_P} \]
    where we write $P = (P_X : P_Y)$. Since $\sum n_P = 0$ we have $f \in k(\PP^1)$. This shows $\text{Pic}(\PP^1) \simeq \ZZ$. The converse is also true: if $\text{Pic}(C) \simeq \ZZ$, then $C \simeq \PP^1$.
\end{example}


Exact sequence
\[ \begin{tikzcd} 1 \arrow{r} & \overline{k}^\times \arrow{r} & \overline{k}(C) \arrow{r}{\text{div}} & \text{Div}^0(C) \arrow{r} & \text{Pic}^0(C) \arrow{r} & 0 \end{tikzcd} \]

The number theoretic analogue of this for a number field $K$ would be
\[ \begin{tikzcd} 1 \arrow{r} & \begin{array}{c} \text{units} \\ \text{of $K$} \end{array} \arrow{r} & K^\times \arrow{r} & \begin{array}{c} \text{fractional} \\ \text{ideals of $K$} \end{array} \arrow{r} & \begin{array}{c} \text{ideal class} \\ \text{group of $K$} \end{array} \arrow{r} & 1 \end{tikzcd} \]


