\begin{topic}{module}{module}
    Let $R$ be a \tref{ring}{commutative ring}. An \textbf{$R$-module} is an abelian group $M$ with an action of $R$, that is a map
    \[ R \times M \to M, \qquad (a, m) \mapsto a \cdot m, \]
    satisfying
    \begin{itemize}
        \item $a \cdot (m + m') = a \cdot m + a \cdot m'$,
        \item $(a + b) \cdot m = a \cdot m + b \cdot m$,
        \item $a \cdot (bm) = (ab) \cdot m$,
        \item $1 \cdot m = m$,
    \end{itemize}
    for all $a, b \in R$ and $m, m' \in M$.
\end{topic}

\begin{topic}{free-module}{free module}
    Let $R$ be a \tref{ring}{commutative ring}. An $R$-module $M$ is \textbf{free} if it is isomorphic to
    \[ \bigoplus_{i \in I} R , \]
    for some indexing set $I$.
\end{topic}

\begin{topic}{cyclic-module}{cyclic module}
    Let $R$ be a \tref{ring}{commutative ring}. An $R$-module $M$ is \textbf{cyclic} if it can be generated by one element, that is $M = Rm$ for some $m \in M$.
\end{topic}

\begin{topic}{projective-module}{projective module}
    Let $R$ be a \tref{ring}{commutative ring}. An $R$-module $P$ is called \textbf{projective} if for every morphism $g : P \to M$ and surjective morphism $f : N \to M$ of $R$-modules, there exists a morphism $h : P \to N$ of $R$-modules such that $fh = g$. We do not require this map to be unique.
    \[ \begin{tikzcd} & N \arrow[twoheadrightarrow]{d}{f} \\ P \arrow[swap]{r}{g} \arrow[dashed]{ur}{\exists h} & M \end{tikzcd} \]
\end{topic}

\begin{example}{projective-module}
    \tref{free-module}{Free modules} are projective.
    
    More generally, a \tref{finitely-generated-module}{finitely generated} $R$-module $P$ is projective if and only if $P$ is finitely presented and the localization $P_\mathfrak{m}$ is a free $R_\mathfrak{m}$-module for all \tref{maximal-ideal}{maximal ideals} $\mathfrak{m}$ of $R$.
\end{example}

\begin{topic}{injective-module}{injective module}
    Let $R$ be a \tref{ring}{commutative ring}. An $R$-module $I$ is called \textbf{injective} if for every morphism $g : M \to I$ and injective morphism $f : M \to N$ of $R$-modules, there exists a morphism $h : N \to Q$ of $R$-modules such that $hf = g$. We do not require this map to be unique.
    \[ \begin{tikzcd} M \arrow[hookrightarrow]{r}{f} \arrow[swap]{d}{g} & N \arrow[dashed]{ld}{\exists h} \\ I & \end{tikzcd} \]
\end{topic}

\begin{topic}{flat-module}{flat module}
    Let $R$ be a \tref{ring}{commutative ring}. An $R$-module $M$ is \textbf{flat} if $(-) \otimes_R M$ is exact. Since the tensor product is already right-exact, this is equivalent to saying $(-) \otimes_R M$ sends injective morphisms to injective morphisms.
\end{topic}

\begin{example}{flat-module}
    Any \tref{free-module}{free module} $\bigoplus_{i \in I} R$ is flat over $R$, because if $M \to N$ is injective, then so is $\bigoplus_{i \in I} M \to \bigoplus_{i \in I} N$.
    In particular, $k$-vector spaces are flat over $k$.
    
    More generally, since direct summands of flat modules are again flat, it follows that \tref{projective-module}{projective modules} are flat, since a projective module is a direct summand of a free module.
\end{example}

\begin{example}{flat-module}
    Take $R = \ZZ$ and $M = \ZZ/2\ZZ$. Then $M$ is not flat, because tensoring
    \[ 0 \rightarrow \ZZ \xrightarrow{\cdot 2} \ZZ \rightarrow \ZZ/2\ZZ \to 0 \]
    with $\ZZ/2\ZZ$ gives
    \[ 0 \rightarrow \ZZ/2\ZZ \xrightarrow{0} \ZZ/2\ZZ \xrightarrow{\id} \ZZ/2\ZZ \to 0 , \]
    which is not exact on the left.
\end{example}

\begin{example}{flat-module}
    Let $f : A \to B$ be a \tref{ring-morphism}{morphism of rings}, and suppose that $x \in A$ is not a \tref{zero-divisor}{zero-divisor} while $f(x) \in B$ is. Then $B$ cannot be flat as an $A$-module, since multiplication by $x$ is an injective map $A \to A$, while multiplication by $f(x)$ is not an injective map $B \to B$.
\end{example}

\begin{topic}{faithful-module}{faithful module}
     Let $R$ be a \tref{ring}{commutative ring}. An $R$-module $M$ is \textbf{faithful} if the \tref{annihilator}{annihilator} $\text{Ann}(M) = 0$.
\end{topic}

\begin{topic}{finitely-generated-module}{finitely generated module}
    Let $R$ be a \tref{ring}{commutative ring}. An $R$-module $M$ is \textbf{finitely generated} if there exists a finite set $m_1, m_2, \ldots, m_n \in M$ (\textit{generators}) such that
    \[ M = R m_1 + R m_2 + \cdots + R m_n . \]
\end{topic}

\begin{topic}{nakayamas-lemma}{Nakayama's lemma}
    \textbf{Nakayama's lemma} states that if $M$ is a \tref{finitely-generated-module}{finitely generated module} over $R$, and $I \subset R$ an \tref{ideal}{ideal} such that $IM = M$, then there exists an $r \in R$ with $r \equiv 1 \; (\text{mod } I)$ such that $rM = 0$.
    
    The following corollary is also known as Nakayama's lemma: if $M$ is a finitely generated module over $R$, and $I \subset R$ an ideal contained in the \tref{jacobson-radical}{Jacobson radical} $\mathfrak{J}_R$ of $R$ such that $IM = M$, then $M = 0$.
    
    (This follows from the above by observing that $r \equiv 1 \; (\text{mod } \mathfrak{J}_R)$ must be a unit, so $M = r^{-1} r M = 0$.)
\end{topic}

\begin{topic}{faithfully-flat-module}{faithfully flat module}
    Let $R$ be a \tref{ring}{commutative ring}. An $R$-module $M$ is \textbf{faithfully flat} if for all sequences
    \[ 0 \to A \to B \to C \to 0 \]
    of $R$-modules, the sequence is exact if and only if
    \[ 0 \to A \otimes_R M \to B \otimes_R M \to C \otimes_R M \to 0 \]
    is exact.
\end{topic}

\begin{topic}{simple-module}{(semi)simple module}
    Let $R$ be a \tref{ring}{ring}. A (left or right) $R$-module $M$ is \textbf{simple} if it is non-zero and has no proper submodules. It is \textbf{semisimple} if it is isomorphic to the direct sum of simple modules.
\end{topic}

\begin{topic}{graded-module}{graded module}
    A \textbf{graded module} $M$ over a \tref{graded-ring}{graded ring} $R$ is an $R$-\tref{module}{module}
    \[ M = \bigoplus_{i \ge 0} M_i \]
    such that
    \[ R_i M_j \subset M_{i + j} \]
    for all $i, j \ge 0$.
\end{topic}

\begin{topic}{length-module}{length of a module}
    The \textbf{length} of a \tref{module}{module} $M$ over a \tref{ring}{ring} $R$ is the maximum length of chains of submodules
    \[ M_0 \subsetneq M_1 \subsetneq \cdots \subsetneq M_n = M . \]
    It may be infinite.
\end{topic}

\begin{topic}{depth-module}{depth of a module}
    Let $R$ be a \tref{ring}{commutative ring}, $I \subset R$ an \tref{ideal}{ideal} and $M$ a finitely generated \tref{module}{$R$-module} such that $IM \subsetneq M$. Then the $I$-depth of $M$ is defined as
    \[ \text{depth}_I(M) = \min \{ i \in \ZZ : \text{Ext}^i(R/I, M) \ne 0 \} . \]
    When $R$ is a \tref{local-ring}{local ring}, one usually takes $I$ equal to be the maximal ideal $\mathfrak{m}$.
\end{topic}

\begin{topic}{bimodule}{bimodule}
    Let $R$ and $S$ be two \tref{ring}{rings}. An \textbf{$R$-$S$-bimodule} is an \tref{GT:abelian-group}{abelian group} $M$ such that $M$ is both a left $R$-module and a right $S$-module, and such that $(rm)s = r(ms)$ for all $r \in R$, $s \in S$ and $m \in M$.
    
    An $R$-bimodule is an $R$-$R$-bimodule.
\end{topic}

\begin{example}{bimodule}
    Given any field $k$, the ring of $m \times n$ matrices $\text{Mat}_{m \times n}(k)$ is an $R$-$S$-bimodule where $R = \text{Mat}_{m \times m}(k)$ and $S = \text{Mat}_{n \times n}(k)$.
\end{example}

\begin{topic}{torsionless-module}{torsionless module}
    A \tref{module}{module} $M$ over a \tref{ring}{ring} $R$ is \textbf{torsionless} if for every $m \in M$ there exists some $R$-module morphism $f : M \to R$ with $f(m) \ne 0$.
\end{topic}

\begin{topic}{reflexive-module}{reflexive module}
    A \tref{module}{module} $M$ over a \tref{ring}{ring} $R$ is \textbf{reflexive} if the map
    \[ M \to \Hom_R(\Hom_R(M, R), R), \quad m \mapsto (f \mapsto f(m)) \]
    is a bijection.
\end{topic}

\begin{topic}{external-tensor-product}{external tensor product}
    Let $R$ and $S$ be \tref{algebra}{algebras} over a field $k$. If $M$ is an \tref{module}{$R$-module} and $N$ an $S$-module, then the \textbf{external tensor product} of $M$ and $N$ is the $R \otimes_k S$-module
    \[ M \boxtimes N = M|_k \otimes_k N|_k . \]
\end{topic}
