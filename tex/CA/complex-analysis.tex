\begin{topic}{complex-differentiable}{complex differentiable}
    A function $f : U \to \CC$, with $U \subset \CC$ an open subset, is \textbf{complex differentiable} at $z \in U$ if the limit
    \[ f'(z) = \lim_{n \to \infty} \frac{f(z_n) - f(z)}{z_n - z} \]
    exists and is the same for any sequence $(z_n)_{n \in \NN}$ in $U$ converging to $z$.
\end{topic}

\begin{topic}{holomorphic-function}{holomorphic function}
    A function $f : U \to \CC$, with $U \subset \CC$ an open subset, is \textbf{holomorphic} if it is \tref{complex-differentiable}{complex differentiable} for all $z \in U$.
    
    A function $f : U \to \CC^n$, with $U \subset \CC^m$ an open subset, is \textbf{holomorphic} if every component is holomorphic as a function in each variable separately, with the other variables fixed.
    
    A map $f : X \to Y$ between \tref{DG:complex-manifold}{complex manifolds} is \textbf{holomorphic} if for every point $x \in X$, there is a \tref{DG:atlas}{chart} $(U, \varphi)$ on $X$ with $x \in U$ and a chart $(V, \psi)$ on $Y$ with $f(U) \subset V$, such that the composition
    \[ \varphi(U) \xrightarrow{\varphi^{-1}} U \xrightarrow{f} V \xrightarrow{\psi} \psi(V) \subset \CC^n \]
    is holomorphic.
\end{topic}

\begin{example}{holomorphic-function}
    The function $f(z) = |z|^2$ is not holomorphic on $\CC$ as
    \[ \lim_{n \to \infty} \frac{|1 + \tfrac{1}{n}|^2 - |1|^2}{(1 + \tfrac{1}{n}) - 1} = 2 \ne 0 = \lim_{n \to \infty} \frac{|1 + \tfrac{i}{n}|^2 - |1|^2}{(1 + \tfrac{i}{n}) - 1} . \]
    In fact, while $f$ is complex differentiable at $0$, it is not holomorphic in any neighborhood of $0$.
\end{example}

\begin{topic}{cauchy-riemann-equations}{Cauchy--Riemann equations}
    The \textbf{Cauchy--Riemann equations} state that a function $f : \CC \to \CC$ given by
    \[ f(x + iy) = u(x, y) + i v(x, y) \]
    where $u, v : \RR^2 \to \RR$ have continuous first partial derivatives, is \tref{holomorphic-function}{holomorphic} if and only if
    \[ \frac{\partial u}{\partial x} = \frac{\partial v}{\partial y} \quad \textup{ and } \quad \frac{\partial u}{\partial y} = - \frac{\partial v}{\partial x} . \]
\end{topic}

\begin{example}{cauchy-riemann-equations}
    Consider the function $f(z) = z^2$, which can be written as $f(x + iy) = u(x, y) + iv(x, y)$ with $u(x, y) = x^2 - y^2$ and $v(x, y) = 2xy$. Then we see that
    \[ \frac{\partial u}{\partial x} = 2x = \frac{\partial v}{\partial y} \quad \textup{ and } \quad \frac{\partial u}{\partial y} = -2y = -\frac{\partial v}{\partial x} , \]
    so $f$ is holomorphic.
\end{example}

\begin{topic}{cauchy-integral-theorem}{Cauchy's integral theorem}
    Let $f : U \to \CC$ be a \tref{holomorphic-function}{holomorphic function}, with $U \subset \CC$ a \tref{TO:simply-connected-space}{simply connected} open subset. Then \textbf{Cauchy's integral theorem} states that
    \[ \oint_\gamma f(z) dz = 0 \]
    for any closed loop $\gamma$ in $U$.
\end{topic}

\begin{topic}{cauchy-integral-formula}{Cauchy's integral formula}
    Let $f : U \to \CC$ be a \tref{holomorphic-function}{holomorphic function}, with $U \subset \CC$ an open subset containing a closed disk $D = \{ z \in \CC \mid |z - a| \le r \}$ for some $a \in U$ and $r > 0$. Then \textbf{Cauchy's integral formula} states that
    \[ f(a) = \frac{1}{2 \pi i} \oint_\gamma \frac{f(z)}{z - a} dz , \]
    where $\gamma$ is the curve going counterclockwise along the boundary of $D$.
\end{topic}

\begin{topic}{entire-function}{entire function}
    An \textbf{entire function} is a function $f : \CC \to \CC$ which is \tref{holomorphic-function}{holomorphic} on the whole complex plane.
\end{topic}

\begin{topic}{julia-set}{Julia set}
    Let $f : \CC \to \CC$ be a polynomial function. The \textbf{filled Julia set} for $f$ is the subset of $\CC$ given by
    \[  \{ z \in \CC \mid \{ f^n(z) : n \in \NN \} \textup{ is bounded} \} \]
    and the \textbf{Julia set} for $f$ is the \tref{TO:boundary}{boundary} of the filled Julia set.
\end{topic}

\begin{example}{julia-set}
    \begin{itemize}
        \item The filled Julia set for $f(z) = z^2$ is the unit disk, whose Julia set is the unit circle.
    
        \item The filled Julia set for $f(z) = z^2 + c$ with $c = -0.4 + 0.6i$ is shown below.
        \img{julia-set.png}
    \end{itemize}
\end{example}

\begin{topic}{mandelbrot-set}{Mandelbrot set}
    The Mandelbrot set is the set of complex numbers $c \in \CC$ for which the sequence $(z_n)_{n \in \NN}$ given by
    \[ z_0 = 0 \quad \textup{ and } \quad z_{n + 1} = z_n^2 + c \]
    is bounded.
    \img{mandelbrot-set.png}
\end{topic}

\begin{topic}{modular-form}{modular form}
    Let $\Gamma \subset \textup{SL}_2(\ZZ)$ be a \tref{GT:subgroup}{subgroup} of finite \tref{GT:index-subgroup}{index} and $k$ an integer.  A \textbf{modular form} of level $\Gamma$ and weight $k$ is a \tref{holomorphic-function}{holomorphic function} $f : \mathbb{H} \to \CC$ on the upper half-plane $\mathbb{H} = \{ z \in \CC \mid \operatorname{Im}(z) > 0 \}$ satisfying
    \begin{itemize}
        \item $f\left(\frac{az + b}{cz + d}\right) = (cz + d)^k f(z)$ for all $z \in \mathbb{H}$ and $\begin{pmatrix} a & b \\ c & d \end{pmatrix} \in \Gamma$,
        \item the function $(cz + d)^{-k} f\left(\frac{az + b}{cz + d}\right)$ is bounded as $\operatorname{Im}(z) \to \infty$, for all $\begin{pmatrix} a & b \\ c & d \end{pmatrix} \in \textup{SL}_2(\ZZ)$.
    \end{itemize}
\end{topic}

\begin{example}{modular-form}
    For $k \ge 4$ even, consider the \textit{Eisenstein series}
    \[ E_k(z) = \sum_{(0, 0) \ne (a, b) \in \ZZ^2} \frac{1}{(a + bz)^k} . \]
    The conditions on $k$ are to ensure the series converges. One easily verifies that
    \[ E_k(z + 1) = E_k(z) \quad \textup{ and } \quad E_k(- z^{-1}) = z^k E_k(z) . \]
    So, since $\textup{SL}_2(\ZZ)$ is generated by $S = \begin{pmatrix} 0 & -1 \\ 1 & 0 \end{pmatrix}$ and $T = \begin{pmatrix} 1 & 1 \\ 0 & 1 \end{pmatrix}$, we conclude that $E_k$ is a modular form (of level $\textup{SL}_2(\ZZ)$) of weight $k$.
\end{example}

\begin{topic}{meromorphic-function}{meromorphic function}
    Let $U \subset \CC^m$ be a connected open subset. The field of \textbf{meromorphic functions} $\mathcal{M}(U)$ is the \tref{AA:field-of-fractions}{field of fractions} of the \tref{AA:domain}{domain} of \tref{holomorphic-function}{holomorphic functions} on $U$.
\end{topic}

\begin{topic}{picard-theorem}{Picard theorem}
    \textbf{Picard's little theorem} states that, for any non-constant \tref{entire-function}{entire function} $f : \CC \to \CC$, the set of values $f$ attains is either the whole of $\CC$ or $\CC$ minus one point.

    \textbf{Picard's great theorem} states that, for any analytic function $f : \CC \to \CC$ that has an essential singularity at a point $z$, the function $f$ attains, on any punctured neighborhood of $z$, all possible complex values, with at most one exception, infinitely often.
\end{topic}
