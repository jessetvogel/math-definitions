\begin{topic}{ring}{ring}
    A \textbf{ring} is an abelian group $R$ with an operation called multiplication and an element $1 \in R$ satisfying
    \begin{itemize}
        \item (\textit{associativity}) $a(bc) = (ab)c$,
        \item (\textit{distributivity}) $a(b + c) = ab + ac$ and $(a + b)c = ac + bc$,
        \item (\textit{unit}) $1 \cdot a = a \cdot 1 = a$,
    \end{itemize}
    for all $a, b, c \in R$.
    
    A ring $R$ is called \textbf{commutative} if moreover $ab = ba$ for all $a, b \in R$.
\end{topic}

\begin{topic}{ring-morphism}{ring morphism}
    A map $f : R \to S$ between \tref{ring}{rings} is a \textbf{ring morphism} if
    \begin{itemize}
        \item $f(1) = 1$,
        \item $f(a + b) = f(a) + f(b)$,
        \item $f(ab) = f(a) f(b)$,
    \end{itemize}
    for all $a, b \in R$.
\end{topic}

\begin{topic}{unit}{unit}
    Let $R$ be a \tref{ring}{ring}. An element $a \in R$ is called a \textbf{unit} if there exists some $b \in R$ such that $ab = 1$. This is denoted $b = a^{-1}$.
\end{topic}

\begin{topic}{irreducible-element}{irreducible element}
    Let $R$ be a \tref{domain}{domain}. An element $a \in R$ is called a \textbf{irreducible} if $a$ is not a \tref{unit}{unit}, and for all $b, c \in R$ such that $bc = a$ either $b$ or $c$ is a unit.
\end{topic}

\begin{topic}{zero-divisor}{zero-divisor}
    Let $R$ be a \tref{ring}{ring}. An element $a \in R$ is called a \textbf{zero-divisor} if $a \ne 0$ and $ab = 0$ for some $b \ne 0$.
\end{topic}

\begin{topic}{nilpotent-element}{nilpotent element}
    Let $R$ be a \tref{ring}{ring}. An element $a \in R$ is \textbf{nilpotent} if $a^n = 0$ for some positive integer $n$.
\end{topic}

\begin{topic}{domain}{domain}
    A non-zero \tref{ring}{commutative ring} $R$ is called a \textbf{domain} if it has no zero-divisors, that is, $ab = 0$ implies $a = 0$ or $b = 0$ for all $a, b \in R$.
\end{topic}

\begin{topic}{field}{field}
    A \tref{ring}{commutative ring} $R$ is called a \textbf{field} if all non-zero elements are units, that is, for all non-zero $a \in R$ there exists a $b \in R$ such that $ab = 1$. 
\end{topic}

\begin{topic}{ideal}{ideal}
    Let $R$ be a \tref{ring}{ring}. An \textbf{ideal} of $R$ is a subset $I \subset R$ such that
    \begin{itemize}
        \item $I$ is a subgroup of $R$ under addition,
        \item (\textit{left-ideal}) $ra \in I$  for all $r \in R$ and $a \in I$,
        \item (\textit{right-ideal}) $ar \in I$ for all $r \in R$ and $a \in I$.
    \end{itemize}
\end{topic}

\begin{topic}{coprime-ideals}{coprime ideals}
    Two \tref{ideal}{ideals} $I, J$ of a \tref{ring}{commutative ring} $R$ are \textbf{coprime} if $I + J = R$.
\end{topic}

\begin{topic}{radical-ideal}{radical ideal}
    Let $R$ be a \tref{ring}{ring}. The \textbf{radical} of an ideal $I \subset R$ is given by
    \[ \sqrt{I} = \{ x \in R : x^n \in I \} . \]
    An ideal $I$ is called \textbf{radical} if $I = \sqrt{I}$.
\end{topic}

\begin{topic}{principal-ideal}{principal ideal}
    A \textbf{principal ideal} is an \tref{ideal}{ideal} $I \subset R$ generated by a single element, that is, of the form
    \[ I = (a) = \{ r a : r \in R \} \]
    for some $a \in R$.
\end{topic}

\begin{topic}{prime-ideal}{prime ideal}
    Let $R$ be a \tref{ring}{ring}. An \tref{ideal}{ideal} $I \subset R$ is called \textbf{prime} if $I \ne R$ and
    \[ ab \in I \implies a \in I \text{ or } b \in I . \]
    
    Equivalently, $I \subset R$ is a prime ideal iff the quotient ring $R / I$ is a \tref{domain}{domain}.
\end{topic}

\begin{topic}{maximal-ideal}{maximal ideal}
    Let $R$ be a \tref{ring}{ring}. An \tref{ideal}{ideal} $I \subset R$ is called \textbf{maximal} if $I \ne R$ and
    \[ I \subset J \subsetneq R \text{ for some ideal } J \subset R \implies I = J . \]
    
    Equivalently, $I \subset R$ is a maximal ideal iff the quotient ring $R / I$ is a \tref{field}{field}.
    
    Every maximal ideal is \tref{prime-ideal}{prime}.
\end{topic}

\begin{topic}{irreducible-ideal}{irreducible ideal}
    Let $R$ be a \tref{ring}{ring}. An \tref{ideal}{ideal} $I \subset R$ is called \textbf{irreducible} if
    \[ I = J_1 \cap J_2 \quad \implies \quad I = J_1 \text{ or } I = J_2 . \]
\end{topic}

\begin{topic}{quotient-ring}{quotient ring}
    Given a \tref{ring}{commutative ring} $R$ and \tref{ideal}{ideal} $I$, the \textbf{quotient ring} $R/I$ is the ring
    \[ R/I = \{ a + I : a \in R \} \]
    where addition and multiplication are given by
    \[ (a + I) + (b + I) = (a + b) + I \quad \text{ and } \quad (a + I) (b + I) = (ab) + I . \]
    It has the universal property that each morphism of rings $f : R \to S$ with $f(I) = 0$ uniquely extends to a morphism $R/I \to S$.
    \[ \begin{tikzcd} R \arrow{r}{f} \arrow[swap]{d}{\pi} & S \\ R/I \arrow[dashed]{ur} & \end{tikzcd} \]
\end{topic}

\begin{topic}{local-ring}{local ring}
    A \textbf{local ring} is a \tref{ring}{ring} $R$ with exactly one \tref{maximal-ideal}{maximal ideal}. The maximal is often denoted by $\mathfrak{m}$.
    
    The quotient $k = R/\mathfrak{m}$ is called the \textbf{residue field}.
\end{topic}

\begin{topic}{local-morphism}{local morphism}
    A \textbf{local morphism} of \tref{local-ring}{local rings} $f : R \to S$ is a ring morphism such that $f(\mathfrak{m}_R) \subset \mathfrak{m}_S$.
\end{topic}

\begin{topic}{finite-type}{finite type}
    A ring morphism $R \to S$ is said to be of \textbf{finite type} if $S$ is isomorphic to a quotient of $R[x_1, \ldots, x_n]$ for some integer $n$.
    
    In this case, $S$ is also said to be a \textbf{finitely generated $R$-algebra}.
\end{topic}

\begin{topic}{finite-presentation}{finite presentation}
    A ring morphism $R \to S$ is said to be of \textbf{finite presentation} if $S$ is isomorphic to $R[x_1, \ldots, x_n] / (f_1, \ldots, f_m)$ for some integer $n$ and some $f_i \in R[x_1, \ldots, x_n]$.
    
    In this case, $S$ is also said to be a \textbf{finitely presented $R$-algebra}.
\end{topic}

\begin{topic}{krull-dimension}{Krull dimension}
    The \textbf{Krull dimension} of a commutative ring $R$ is the supremum of the lengths of all chains of \tref{prime-ideal}{prime ideals}, where a chain of the form
    \[ \mathfrak{p}_0 \subsetneq \mathfrak{p}_1 \subsetneq \cdots \subsetneq \mathfrak{p}_n \]
    has length $n$.
\end{topic}

\begin{topic}{group-ring}{group ring}
    Let $R$ be a \tref{ring}{ring} and $G$ a group. The \textbf{group ring} $R[G]$ of $G$ over $R$ is defined as
    \[ R[G] = \bigoplus_{g \in G} R , \]
    where multiplication is induced by
    \[ (a \cdot g) \cdot (b \cdot h) = (ab) \cdot gh . \]
    % where multiplication is given by
    % \[ (a_g)_{g \in G} \cdot (b_g)_{g \in G} = \left(\sum_{h \in G} a_h b_{h^{-1}g}\right)_{g \in G} \]
\end{topic}

\begin{topic}{chinese-remainder-theorem}{Chinese remainder theorem}
    Let $R$ be a \tref{ring}{commutative ring} and suppose $I, J$ are coprime ideals of $R$, i.e. $I + J = R$. Then the \textbf{Chinese remainder theorem} states that $I \cap J = I \cdot J$ and that there is an isomorphism of rings
    \[ R / (I \cdot J) \xrightarrow{\sim} (R / I) \times (R/J), \qquad a \text{ mod } (I \cdot J) \mapsto (a \text{ mod } I, b \text{ mod } J) . \]
    
    In particular, when $R = \ZZ$ with $I = (n)$ and $J = (m)$ for $n, m$ relatively prime, there is
    \[ \ZZ / nm \ZZ \simeq (\ZZ/n\ZZ) \times (\ZZ/m\ZZ), \qquad a \text{ mod } nm \mapsto (a \text{ mod } n, a \text{ mod } m) . \]
\end{topic}

\begin{topic}{idempotent}{idempotent}
    An element $e \in R$ of a \tref{ring}{ring} $R$ is called \textbf{idempotent} if $e^2 = e$.
\end{topic}

\begin{topic}{dual-numbers}{dual numbers}
    Let $R$ be a \tref{ring}{commutative ring}. The \textbf{ring of dual numbers} over $R$ is the quotient ring
    \[ R[\varepsilon] / (\varepsilon^2) . \]
\end{topic}

\begin{topic}{noetherian-ring}{noetherian ring}
    A \tref{ring}{commutative ring} $R$ is called \textbf{noetherian} if it satisfies the \textit{ascending chain condition}: for any increasing sequence of ideals
    \[ I_1 \subset I_2 \subset I_3 \subset \cdots \]
    there exists some $N \in \NN$ such that $I_n = I_N$ for any $n \ge N$.
    
    Equivalently, $R$ is noetherian if all its ideals are finitely generated.
\end{topic}

\begin{example}{noetherian-ring}
    The following are all noetherian rings.
    \begin{itemize}
        \item Any \tref{field}{field}: their only ideal is $(0)$.
        \item Any \tref{principal-ideal-domain}{PID}: every ideal is generated by a single element.
        \item If $R$ is noetherian, then so is $R[x]$: this is \textit{Hilbert's basis theorem}.
    \end{itemize}
\end{example}

\begin{example}{noetherian-ring}
    The ring $k[x_1, x_2, x_3, \ldots]$ is not noetherian. Namely, the sequence of ideals
    \[ (x_1) \subset (x_1, x_2) \subset (x_1, x_2, x_3) \subset \ldots \]
    is ascending, but does not stabilize.
\end{example}

\begin{topic}{localization}{localization}
    Let $R$ be a \tref{ring}{commutative ring}, and $S \subset R$ a \textit{multiplicative set} (that is, $1 \in S$ and $xy \in S$ for all $x, y \in S$). Then the \textbf{localization} of $R$ w.r.t. $S$ is the ring
    \[ S^{-1} R = R \times S / \sim{} \quad \text{ where } (r_1, s_1) \sim{} (r_2, s_2) \text{ if } t(r_1 s_2 - r_2 s_1) = 0 \text{ for some } t \in S . \]
\end{topic}

\begin{example}{localization}
    If $R$ is a \tref{domain}{domain}, then $S = R - \{ 0 \}$ is a multiplicative set, and $S^{-1} R$ is the \tref{field-of-fractions}{field of fractions} of $R$.
\end{example}

\begin{example}{localization}
    For any commutative ring $R$ and $f \in R$, the set
    \[ S = \{ 1, f, f^2, \ldots \} \]
    is a multiplicative set, and the localization is
    \[ R_f := S^{-1} R = \left\{ \frac{a}{f^n} : a \in R, \; n \ge 0 \right\} . \]
\end{example}

\begin{example}{localization}
    For any commutative ring $R$ and \tref{prime-ideal}{prime ideal} $\mathfrak{p}$, the set
    \[ S = \{ f \in R : f \not\in \mathfrak{p} \} \]
    is a multiplicative set, and the localization
    \[ R_\mathfrak{p} := S^{-1} R = \left\{ \frac{f}{g} : f, g \in R, \; g \not\in \mathfrak{p} \right\} \]
    is a \tref{local-ring}{local ring} with maximal ideal $\mathfrak{p} R_\mathfrak{p}$.
\end{example}

\begin{topic}{total-quotient-ring}{total quotient ring}
    The \textbf{total quotient ring} of a \tref{ring}{commutative ring} $R$ is the \tref{localization}{localization} $S^{-1} R$, with $S$ the set of elements of $R$ which are not \tref{zero-divisor}{zero-divisors}.
    
    When $R$ is a \tref{domain}{domain}, this construction gives the \tref{field-of-fractions}{field of fractions} of $R$.
\end{topic}

\begin{topic}{regular-ring}{regular (local) ring}
    A \textbf{regular local ring} is a commutative \tref{noetherian-ring}{noetherian} \tref{local-ring}{local} ring such that the minimal number of generators of its maximal ideal is equal to its \tref{krull-dimension}{Krull dimension}. Equivalently, a local ring $R$ with maximal ideal $\mathfrak{m}$ and residue field $k = R / \mathfrak{m}$ is regular iff $\dim_k(\mathfrak{m} / \mathfrak{m}^2) = \dim R$.
    
    A \textbf{regular ring} is a commutative noetherian ring, such that the \tref{localization}{localization} at every \tref{prime-ideal}{prime ideal} is a regular local ring.
\end{topic}

\begin{example}{regular-ring}
    Every field is a regular local ring: their Krull dimension is zero, and their maximal ideal is $(0)$.
\end{example}

\begin{example}{regular-ring}
    The local ring $R = k[x]/(x^2)$ is not a regular local ring. Its only prime ideal is its maximal ideal $\mathfrak{m} = (x)/(x^2)$, so the Krull dimension is zero, but the minimal number of generators of $\mathfrak{m}$ is one.
\end{example}

\begin{topic}{principal-ideal-domain}{principal ideal domain (PID)}
    A \textbf{principal ideal domain} (PID) is a \tref{domain}{domain} $R$ in which every \tref{ideal}{ideal} is \tref{principal-ideal}{principal}. That is, every ideal $I \subset R$ is of the form $I = (x)$ for some element $x \in R$.
\end{topic}

\begin{topic}{unique-factorization-domain}{unique factorization domain (UFD)}
    A \textbf{unique factorization domain} (UFD) is a \tref{domain}{domain} $R$ for which every non-zero $x \in R$ can be written as the product of a \tref{unit}{unit} and a finite number of \tref{irreducible-element}{irreducible} elements:
    \[ a = u \cdot p_1 \cdot p_2 \cdot \cdots \cdot p_k \qquad u \in R^\times, \; k \ge 0, \; p_i \in R \text{ irreducible}. \]
\end{topic}

\begin{topic}{euclidean-ring}{Euclidean ring}
    A \textbf{Euclidean ring} is a \tref{domain}{domain} $R$ for which there exists a function
    \[ g : R^\times \to \ZZ_{\ge 0} \]
    such that for all $a, b \in R$ with $b \ne 0$, there exists $q, r \in R$ with $a = qb + r$ and either $r = 0$ or $g(r) < g(b)$.
    
    That is, a ring in which one can perform division with remainder. The function $g$ is used to say that the `remainder' $r$ is `smaller' than the element $b$ one divides by.
    
    In particular, one can find the \textit{gcd} of elements by means of the \textit{Euclidean algorithm}.
\end{topic}

\begin{topic}{field-of-fractions}{field of fractions}
    The \textbf{field of fractions} of a \tref{domain}{domain} $R$ is the \tref{field}{field} given by
    \[ K = \left\{ (a, b) : a, b \in R, b \ne 0 \right\} / \sim{} \]
    where $(a, b) \sim{} (c, d)$ iff $ad = bc$. Indeed $(a, b)$ represents the fraction $\frac{a}{b}$. Addition and multiplication are given by
    \[ \frac{a}{b} + \frac{c}{d} = \frac{ad + bc}{bd} \quad \text{and} \quad \frac{a}{b} \cdot \frac{c}{d} = \frac{ac}{bd} . \]
\end{topic}

\begin{example}{field-of-fractions}
    The field of fractions of $\ZZ$ is $\QQ$.
\end{example}

\begin{topic}{valuation-ring}{valuation ring}
    A \textbf{valuation ring} is a \tref{domain}{domain} $R$ such that for every element $x$ of its \tref{field-of-fractions}{field of fractions}, at least one of $x$ or $x^{-1}$ belongs to $R$.
\end{topic}

\begin{topic}{discrete-valuation-ring}{discrete valuation ring}
    A domain \tref{domain}{domain} $R$ is a \textbf{discrete valuation ring} if there is a \textit{discrete valuation} $v$ of its \tref{field-of-fractions}{field of fractions} $K$, such that $R$ is the \textit{valuation ring} of $v$. This means there is a homomorphism
    \[ v : K^\times \to \ZZ \]
    such that
    \[ R = \{ x \in K : v(x) \ge 0 \} . \]
    
    In particular, $R$ is \tref{local-ring}{local} with maximal ideal $\mathfrak{m} = \{ x \in K : v(x) > 0 \}$. Indeed, any element $x \in R$ with $v(x) = 0$ is a unit in $R$.
\end{topic}

\begin{example}{discrete-valuation-ring}
    Let $K = \QQ$ and fix a prime number $p$. Any non-zero $x \in \QQ$ can be written uniquely as $x = p^m y$ where $k \in \ZZ$ and both the numerator and denominator of $y$ are coprime to $p$. Now define the discrete valuation $v_p(x) = m$, then the valuation ring is the local ring $\ZZ_{(p)}$.
\end{example}

% \begin{topic}{dedekind-domain}{Dedekind domain}
%     A \textbf{Dedekind domain} is a \tref{domain}{domain} in which every non-zero proper ideal factors into a product of prime ideals.
% \end{topic}

\begin{topic}{monic-polynomial}{monic polynomial}
    A polynomial $f$ is called \textbf{monic} if its leading coefficient is $1$.
\end{topic}

\begin{topic}{integral-element}{integral element}
    Let $A$ be a \tref{ring}{commutative ring}, and $B$ an $A$-algebra. An element $x \in B$ is said to be \textbf{integral} over $A$ if $x$ is a root of a monic polynomial with coefficients in $A$. That is,
    \[ x^n + a_1 x^{n - 1} + \cdots + a_n = 0 \]
    for some $a_i \in A$.
\end{topic}

\begin{topic}{integral-closure}{integral closure}
    Let $A$ be a \tref{ring}{commutative ring}, and $B$ an $A$-algebra. The \textbf{integral closure} of $A$ in $B$ is the subring of $B$ of elements which are \tref{integral-element}{integral} over $A$.
    
    If the integral closure is equal to $A$, then $A$ is said to be \textbf{integrally closed} in $B$. If the integral closure is equal to $B$, the ring $B$ is said to be \textbf{integral} over $A$.
\end{topic}

\begin{topic}{artin-ring}{artin ring}
    A \tref{ring}{commutative ring} $R$ is called \textbf{artin} if it satisfies the \textit{descending chain condition}: for any decreasing sequence of ideals
    \[ I_1 \supset I_2 \supset I_3 \supset \cdots \]
    there exists some $N \in \NN$ such that $I_n = I_N$ for any $n \ge N$.
\end{topic}

\begin{topic}{fractional-ideal}{fractional ideal}
    Let $R$ be a \tref{domain}{domain} and $K$ its \tref{field-of-fractions}{field of fractions}. An $R$-submodule $M$ of $K$ is a \textbf{fractional ideal} of $R$ if $xM \subset R$ for some $x \ne 0$ in $R$.
\end{topic}

\begin{topic}{invertible-ideal}{invertible ideal}
    Let $R$ be a \tref{domain}{domain} and $K$ its \tref{field-of-fractions}{field of fractions}. An $R$-submodule $M$ of $K$ is an \textbf{invertible ideal} if there exists a submodule $N$ of $K$ such that $MN = R$. This module $N$ is then unique and equal to
    \[ N = (R : M) := \{ x \in K : xM \subset R \} . \]
    
    The invertible ideals form a group with respect to multiplication, whose unit element is $R = (1)$.
\end{topic}

\begin{topic}{syzygy-module}{syzygy module}
    Let $g_1, g_2, \ldots, g_k$ be generators of \tref{module}{module} $M$ over a \tref{ring}{commutative ring} $R$. The \textbf{syzygy module} is the $R$-module consisting of all relations between the generators, i.e. the kernel of
    \[ \bigoplus_{i = 1}^{k} R \to M, \qquad (r_1, \ldots, r_k) \mapsto r_1 g_1 + \cdots r_k g_k . \]
    Inductively, one can define the $n$-th syzygy module for any $n \ge 1$ after choosing generators.
\end{topic}

% Hilbert's three theorems
\begin{topic}{hilbert-basis-theorem}{Hilbert's basis theorem}
    \textbf{Hilbert's basis theorem} states that if a \tref{ring}{commutative ring} $R$ is \tref{noetherian-ring}{noetherian}, then so is the polynomial ring $R[x]$.
\end{topic}

% \begin{topic}{hilbert-nullstellensatz}{Hilbert's Nullstellensatz}
%     \textbf{Hilbert's Nullstellensatz} states that there is a bijection
%     \[ \left\{ \begin{array}{c} \text{radical ideals of} \\ k[x_1, \ldots, x_n] \end{array} \right\} \leftrightarrow \left\{ \begin{array}{c} \text{closed subsets} \\ \text{of $\AA^n$} \end{array} \right\} \]
%     \[ I \mapsto Z(I) \]
%     \[ I \mapsfrom Z(I) \]
% \end{topic}

\begin{topic}{hilbert-syzygy-theorem}{Hilbert's syzygy theorem}
    \textbf{Hilbert's syzygy theorem} states that if $M$ is a finitely generated module over a polynomial ring $k[x_1, \ldots, x_n]$, then the $n$-th \tref{syzygy-module}{syzygy module} $M$ is always \tref{free-module}{free}.
    
    In particular, this implies that there exists a free resolution
    \[ 0 \to F_k \to F_{k - 1} \to \cdots \to F_0 \to M \to 0 \]
    of length $k \le n$.
\end{topic}

\begin{topic}{primary-ideal}{primary ideal}
    Let $R$ be a \tref{ring}{commutative ring}. An \tref{ideal}{ideal} $\mathfrak{q}$ of $R$ is \textbf{primary} if $\mathfrak{q} \ne R$ and if
    \[ xy \in \mathfrak{q} \implies \text{ either } x \in \mathfrak{q} \text{ or } y^n \in \mathfrak{q} \text{ for some } n > 0 . \]
    In other words,
    \[ \mathfrak{q} \text{ is primary } \iff A / \mathfrak{q} \ne 0 \text{ and every zero-divisor in } A / \mathfrak{q} \text{ is nilpotent} . \]
\end{topic}

\begin{topic}{primary-decomposition}{primary decomposition}
    Let $R$ be a \tref{ring}{commutative ring}. A \textbf{primary decomposition} of an ideal $I \subset R$ is an expression of $I$ as a finite intersection of \tref{primary-ideal}{primary ideals},
    \[ I = \bigcap_{i = 1}^{n} \mathfrak{q}_i . \]
    
    The primary decomposition is said to be \textbf{minimal} if (i) the $\mathfrak{p}_i = r(\mathfrak{q}_i)$ are all distinct, and (ii) no $\mathfrak{q}_i$ contains the intersection of the other primary ideals.
\end{topic}

\begin{topic}{annihilator}{annihilator}
    Let $R$ be a \tref{ring}{commutative ring} and $M$ an $R$-module. The \textbf{annihilator} of $M$ is the subring
    \[ \text{Ann}(M) = \{ x \in R : xM = 0 \} \]
\end{topic}

\begin{topic}{nilradical}{nilradical}
     The \textbf{nilradical} of a \tref{ring}{commutative ring} $R$ is the ideal of \tref{nilpotent-element}{nilpotent} elements
     \[ \mathfrak{N}_R = \{ x \in R : x \text{ is nilpotent } \} . \]
     It is equal to the intersection of all prime ideals of $R$,
     \[ \mathfrak{N}_R = \bigcap_{\mathfrak{p} \subset R \text{ prime}} \mathfrak{p} . \]
\end{topic}

\begin{topic}{jacobson-radical}{Jacobson radical}
     The \textbf{Jacobson radical} of a \tref{ring}{commutative ring} $R$ is the ideal given by the intersection of all \tref{maximal-ideal}{maximal} ideals,
     \[ \mathfrak{J}_R = \bigcap_{\mathfrak{m} \subset R \text{ maximal}} \mathfrak{m} . \]
     It can also be characterized by
     \[ x \in \mathfrak{J}_R \iff 1 - xy \text{ is a unit for all } y \in R . \]
\end{topic}

\begin{topic}{regular-sequence}{regular sequence}
    Let $R$ be a \tref{ring}{commutative ring}, and $M$ an $R$-module. An $M$-\textbf{regular sequence} is a sequence
    \[ r_1, r_2, \ldots, r_d \in R \]
    such that $r_i$ is not a \textit{zero-divisor} on $M/(r_1, \ldots, r_{i - 1})$. That is, if $r_i m = 0$ for some $m \in M / (r_1, \ldots, r_{i - 1})$, then $m = 0$.
    
    When $M = R$, such a sequence is simply called a \textbf{regular sequence}.
\end{topic}

\begin{example}{regular-sequence}
    Consider $R = k[x, y]$ and the sequence $(xy, x^2)$. This sequence is not regular, since $x^2 \cdot y = 0$ in $\QQ[x, y] / (xy)$, although $y \ne 0$.
\end{example}

\begin{example}{regular-sequence}
    Consider $R = k[x, y, z]$ and the sequence $(x, y(1 - x), z(1 - x))$. This sequence is regular since
    \[ y(1 - x) = y \in k[x, y, z]/(x) = k[y, z] \quad \text{ and } \quad z(1 - x) = z \in k[x, y, z]/(x, y - xy) = k[z] \]
    are both non-zero-divisors.
    
    However, note that the order matters as $(y(1 - x), z(1 - x), x)$ is not a regular sequence. Namely, $z(1 - x) \cdot y = 0 \in k[x, y, z] / (y(1 - x))$ even though $y \ne 0$.
\end{example}

\begin{topic}{hilbert-series}{Hilbert series}
    Let $S = \bigoplus_{i \ge 0} S_i$ be a finitely generated graded commutative algebra over a field $k$, with $S_0 = k$. The \textbf{Hilbert series} of $S$ is defined as
    \[ \text{H}_S(t) = \sum_{i = 0}^{\infty} \dim_k S_i \cdot t^i . \]
\end{topic}

\begin{example}{hilbert-series}
    Let $R = k[x_1, \ldots, x_n]$ and $I = (f)$ for some homogenous polynomial of degree $d$. Then we have an exact sequence (a free resolution)
    \[ 0 \to R(-d) \xrightarrow{\cdot f} R \to R / I \to 0 , \]
    which implies that
    \[ \text{H}_{R/I}(t) = \text{H}_R(t) - \text{H}_{R(-d)}(t) = \text{H}_R(t) (1 - t^d) = \frac{1 - t^d}{(1 - t)^n}. \]
\end{example}

\begin{topic}{absolutely-flat-ring}{absolutely flat ring}
    A \tref{ring}{commutative ring} $R$ is called \textbf{absolutely flat} if every $R$-module is \tref{flat-module}{flat}.
\end{topic}

\begin{topic}{standard-smooth}{standard smooth}
    Let $A$ be \tref{ring}{commutative ring}, and $B = A[x_1, \ldots, x_n] / (f_1, \ldots, f_c)$ a finitely presented $A$-algebra. Then $B$ is a \textbf{standard smooth} over $A$ if the determinant
    \[ \det \begin{pmatrix}
        \frac{\partial f_1}{\partial x_1} & \frac{\partial f_1}{\partial x_2} & \cdots & \frac{\partial f_1}{\partial x_c} \\
        \frac{\partial f_2}{\partial x_1} & \frac{\partial f_2}{\partial x_2} & \cdots & \frac{\partial f_2}{\partial x_c} \\ 
        \vdots & \vdots & \ddots & \vdots \\ 
        \frac{\partial f_c}{\partial x_1} & \frac{\partial f_c}{\partial x_2} & \cdots & \frac{\partial f_c}{\partial x_c}
    \end{pmatrix} \]
    maps to an invertible element in $B$. Note that this definition is dependent on the presentation of $B$!
\end{topic}

\begin{example}{standard-smooth}
    Take $A = k$ a field and $B = k[x, y] / (xy)$. Then $\partial (xy) / \partial x = y$ which is not invertible in $B$, so $k[x, y] / (xy)$ is not standard smooth over $k$.
    
    Taking $C = k[x, y] / (xy - 1)$, we see that $\partial (xy - 1) / \partial x = y$, which is invertible in $C$. Hence $k[x, y] / (xy - 1)$ is standard smooth over $k$.
\end{example}

\begin{topic}{etale-algebra}{étale algebra}
    Let $A$ be a \tref{ring}{commutative ring} and $B$ an $A$-algebra. Then $B$ is an \textbf{étale algebra} over $A$ if $B$ is a \tref{flat-module}{flat} $A$-module and $\Omega_{B/A} = 0$.
\end{topic}

\begin{topic}{fitting-ideal}{Fitting ideal}
    Let $R$ be a \tref{ring}{commutative ring} and $M$ an $R$-\tref{module}{module} generated by elements $m_1, \ldots, m_n \in M$, with relations
    \[ a_{i1} m_1 + \cdots a_{in} m_n = 0 \text{ for } i = 1, 2, \ldots, \ell . \]
    Then the \textbf{$i$-th Fitting ideal} $\text{Fitt}_i(M)$ of $M$ is generated by the minors (determinants of submatrices) of order $n - i$ of the matrix $a_{ij}$. It can be shown that this does not depend on the choice of generators or relations. One has the inclusions
    \[ \text{Fitt}_0(M) \subset \text{Fitt}_1(M) \subset \text{Fitt}_2(M) \subset \cdots \]
    Intuitively, the $i$-th fitting ideal (or actually the quotient $R / \text{Fitt}_i(M)$) measures the obstruction for $M$ to be generated by $i$ elements.
    
    Sometimes the \textbf{Fitting ideal} of $M$ is defined as the first non-zero fitting ideal.
\end{topic}

\begin{example}{fitting-ideal}
    If $M$ is \tref{free-module}{free} of rank $n$, the matrix $a_{ij}$ will be of size $0 \times n$. Hence $\text{Fitt}_i(M) = 0$ for $i < n$ as there are no submatrices of size $n - i > 0$, and $\text{Fitt}_i(M) = R$ for $i \ge n$ as the determinant of a $0 \times 0$ matrix is one.
\end{example}

\begin{example}{fitting-ideal}
    Consider $M = \ZZ / p \ZZ \times \ZZ / p^2 \ZZ$ as $\ZZ$-module. It can be generated by $m_1 = (1, 0)$ and $m_2 = (0, 1)$ with relations $p m_1 = 0$ and $p^2 m_2 = 0$, so the corresponding matrix is
    \[ a_{ij} = \begin{pmatrix} p & 0 \\ 0 & p^2 \end{pmatrix} \]
    and thus
    \[ \text{Fitt}_0(M) = (p^3), \quad \text{Fitt}_1(M) = (p, p^2) = (p) \quad \text{ and } \quad \text{Fitt}_{\ge 2}(M) = \ZZ . \]
\end{example}

\begin{example}{fitting-ideal}
    Consider the finitely generated abelian group $M = \ZZ^r \times \ZZ / d_1 \ZZ \times \cdots \times \ZZ / d_k \ZZ$ with $d_1 | d_2 | \cdots | d_k$. Using the natural generators, the relation matrix is
    \[ a_{ij} = \begin{pmatrix} \textbf{0}_{r \times r} & & &  \\  & d_1 & \\ & & \ddots & \\ & & & d_k \end{pmatrix} \]
    and thus
    \[ \begin{array}{rcl}
         \text{Fitt}_{0 \le i \le r}(M) &=& (0) , \\
        \text{Fitt}_{r + 1}(M) &=& (d_1 d_2\cdots d_k) , \\
        \text{Fitt}_{r + 2}(M) &=& (d_1 d_2\cdots d_{k - 1}) , \\
        & \vdots & \\
        \text{Fitt}_{r + k - 1}(M) &=& (d_1) , \\
        \text{Fitt}_{\ge r + k}(M) &=& \ZZ .
    \end{array} \]
\end{example}

\begin{example}{fitting-ideal}
    Consider the scheme $X = \Spec k[x_1, \ldots, x_n] / I$ with $I = (f_1, \ldots, f_k)$ for some polynomials $f_i$. The module of differentials
    \[ \Omega_{X/k} = \left(\bigoplus_{i = 1}^{n} k[x_1, \ldots, x_n] / I \cdot dx_i \right) / (df_1, df_2, \ldots, df_k) \]
    is generated by $dx_1, \ldots, dx_n$ and relations
    \[ df_i = \sum_{j = 1}^{n} \frac{\partial f_i}{\partial x_j} dx_j = 0 \quad \text{ for } i = 1, 2, \ldots, k , \]
    so the corresponding matrix is the Jacobian matrix
    \[ J = \begin{pmatrix}
        \frac{\partial f_1}{\partial x_1} & \frac{\partial f_1}{\partial x_2} & \cdots & \frac{\partial f_1}{\partial x_n} \\
        \frac{\partial f_2}{\partial x_1} & \frac{\partial f_2}{\partial x_2} & \cdots & \frac{\partial f_2}{\partial x_n} \\
        \vdots & \vdots & \ddots & \vdots \\
        \frac{\partial f_k}{\partial x_1} & \frac{\partial f_k}{\partial x_2} & \cdots & \frac{\partial f_k}{\partial x_n}
    \end{pmatrix} . \]
    Now $X$ is smooth over $k$ if and only if $\Omega_{X/k}$ is locally free of rank $n = \dim X$, which is equivalent to the Fitting ideal $\text{Fitt}_{\dim X}(\Omega_{X/k})$ generating the unit ideal in the localization of each prime ideal of $k[x_1, \ldots, x_n] / I$, which is equivalent to the Fitting ideal $\text{Fitt}_{\dim X}(\Omega_{X/k})$ generating the unit ideal in $k[x_1, \ldots, x_n] / I$ itself.
\end{example}

\begin{topic}{gcd}{greatest common divisor (GCD)}
    Let $R$ be a \tref{ring}{commutative ring}. A \textbf{greatest common divisor} of two elements $a, b \in R$ is an element $d \in R$ such that $d | a, b$ and for any other $d' \in R$ with $d' | a, b$ one has $d' | d$. It is denoted $d = \text{gcd}(a, b)$.
\end{topic}

\begin{topic}{lcm}{least common multiple (LCM)}
    Let $R$ be a \tref{ring}{commutative ring}. A \textbf{least common multiple} of two elements $a, b \in R$ is an element $m \in R$ such that $a, b | m$ and for any other $m' \in R$ with $a, b | m$ one has $m | m'$. It is denoted $m = \text{lcm}(a, b)$.
\end{topic}

\begin{topic}{gcd-domain}{GCD domain}
    A \textbf{GCD domain} is a \tref{domain}{domain} $R$ in which any two elements $a, b$ have a \tref{gcd}{greatest common divisor}.
\end{topic}

\begin{topic}{projective-dimension}{projective dimension}
    The \textbf{projective dimension} of a \tref{module}{module} $M$ over a \tref{ring}{commutative ring} $R$ is the minimal length of a projective resolution of $M$.
    \[ \cdots \to P_n \to \cdots \to P_2 \to P_1 \to P_0 \to M \to 0 \]
    It may be infinite.
\end{topic}

\begin{topic}{injective-dimension}{injective dimension}
    The \textbf{injective dimension} of a \tref{module}{module} $M$ over a \tref{ring}{commutative ring} $R$ is the minimal length of a injective resolution of $M$.
    \[ 0 \to M \to I^0 \to I^1 \to I^2 \to \cdots \]
    It may be infinite.
\end{topic}

% \begin{example}{projective-dimension}
%     Consider $R = k[x, y] / (xy)$ and $M = k$ as an $R$-module, where $x$ and $y$ act by multiplication by zero.
% \end{example}

\begin{topic}{graded-ring}{graded ring}
    A \textbf{graded ring} is a \tref{ring}{ring} $R$ that is decomposed as a direct sum
    \[ R = \bigoplus_{i \ge 0} R_i \]
    of additive groups, such that
    \[ R_i R_j \subset R_{i + j} \]
    for all $i, j \ge 0$.
\end{topic}

\begin{example}{graded-ring}
    The polynomial ring $R = k[x_1, \ldots, x_n]$ is a graded ring, with
    \[ R_i = \{ f \in R : f \text{ is homogeneous of degree $i$} \} . \]
\end{example}

\begin{topic}{morita-equivalence}{Morita equivalence}
    Two \tref{ring}{rings} $R$ and $S$ are said to be \textbf{Morita equivalent} if there is an \tref{CT:equivalence-of-categories}{equivalence of categories} between the category of (left) $R$-modules, and (left) $S$-modules.
\end{topic}

\begin{example}{morita-equivalence}
    Any ring $R$ is Morita equivalent to the ring $\text{M}_n(R)$ of $n \times n$ matrices with elements in $R$, for any $n > 0$. Indeed, take
    \[ R\text{-Mod} \to \text{M}_n(R)\text{-Mod}, \quad M \to R^{n \times 1} \otimes_R M \simeq M^n , \]
    where $\text{M}_n(R)$ acts on $M^n$ by matrix multiplication on the left. Inversely, take
    \[ \text{M}_n(R)\text{-Mod} \to R\text{-Mod}, \quad N \to R^{1 \times n} \otimes_{\text{M}_n(R)} N . \]
    Indeed these are inverse to each other since
    \[ R^{1 \times n} \otimes_{\text{M}_n(R)} R^{n \times 1} \simeq R \quad \text{and} \quad R^{n \times 1} \otimes_R R^{1 \times n} \simeq \text{M}_n(R) . \]
\end{example}

\begin{topic}{gorenstein-ring}{Gorenstein ring}
    A \textbf{Gorenstein local ring} is a commutative \tref{noetherian-ring}{noetherian} \tref{local-ring}{local} \tref{ring}{ring} $R$ with finite \tref{injective-dimension}{injective dimension} as an $R$-module.
    
    A \textbf{Gorenstein ring} is a commutative noetherian ring $R$ such that the localization $R_\mathfrak{p}$ is Gorenstein for each \tref{prime-ideal}{prime ideal} $\mathfrak{p}$.
\end{topic}

\begin{topic}{cohen-macaulay}{Cohen--Macaulay}
    A \tref{finitely-generated-module}{finitely generated} \tref{module}{module} $M$ over a commutative \tref{noetherian-ring}{noetherian} \tref{local-ring}{local} \tref{ring}{ring} $R$ is \textbf{Cohen--Macaulay} if its \tref{depth-module}{depth} $\text{depth}_{\mathfrak{m}}(M)$ equals its \tref{krull-dimension}{dimension} $\dim_R(M) := \dim(R/\text{Ann}_R(M))$.
    
    More generally, for a commutative noetherian ring $R$, a finitely generated $R$-module $M$ is \textbf{Cohen--Macaulay} if the localization $M_\mathfrak{m}$ is Cohen--Macaulay over $R_\mathfrak{m}$ for each maximal ideal $\mathfrak{m}$ of $R$.
    
    A commutative noetherian ring $R$ is \textbf{Cohen--Macaulay} if it is so as a module over itself.
\end{topic}

\begin{topic}{homogeneous-ideal}{homogeneous ideal}
    A \textbf{homogeneous ideal} in a \tref{graded-ring}{graded ring} is an \tref{ideal}{ideal} generated by homogeneous elements.
\end{topic}

\begin{example}{homogeneous-ideal}
    In $S = k[x, y, z]$, the ideals $(x, y, z)$ and $(x^2, y + z)$ are homogeneous, while $(x + y^2)$ is not.
\end{example}

\begin{topic}{irrelevant-ideal}{irrelevant ideal}
    The \textbf{irrelevant ideal} of a \tref{graded-ring}{graded ring} is the \tref{ideal}{ideal} generated by the homogeneous elements of degree greater than zero. More generally, a \tref{homogeneous-ideal}{homogeneous ideal} of a graded ring is called \textbf{irrelevant} if its \tref{radical-ideal}{radical} contains the irrelevant ideal.
\end{topic}

\begin{example}{irrelevant-ideal}
    The irrelevant ideal of the graded ring $S = k[x_1, \ldots, x_n]$ is the ideal $S_+ = (x_1, \ldots, x_n)$.
\end{example}

\begin{topic}{algebra}{algebra}
    Let $A$ be a \tref{ring}{commutative ring}. An \textbf{$A$-algebra} is ring $B$ with a \tref{ring-morphism}{ring morphism} $f : A \to B$.
\end{topic}

\begin{example}{algebra}
    Let $k$ be a field, and $B = \text{Mat}_{n}(k)$ the ring of $n \times n$ matrices over $k$. Then $B$ is a $k$-algebra under the morphism $i : k \to B$ which sends $a \mapsto a \cdot I$.
\end{example}

\begin{topic}{derivation}{derivation}
    Let $A$ be a \tref{ring}{commutative ring}, $B$ an \tref{algebra}{$A$-algebra}, and $M$ a \tref{module}{$B$-module}. An \textbf{$A$-derivation} is a map $d : B \to M$ such that
    \begin{itemize}
        \item $d(b + b') = db + db'$ for all $b, b' \in B$,
        \item $d(bb') = bdb' + b'db$ for all $b, b' \in B$,
        \item $da = 0$ for all $a \in A$.
    \end{itemize}
    The set of $A$-derivations $d : B \to M$ is often denoted $\text{Der}_A(B, M)$.
\end{topic}

\begin{topic}{height-ideal}{height ideal}
    The \textbf{height} of a \tref{prime-ideal}{prime ideal} $\mathfrak{p}$ in a \tref{ring}{commutative-ring} $R$ is the supremum of lengths of chains of prime ideals, where a chain
    \[ \mathfrak{p}_0 \subsetneq \mathfrak{p}_1 \subsetneq \cdots \subsetneq \mathfrak{p}_n = \mathfrak{p} \]
    has length $n$.
\end{topic}

\begin{topic}{associated-graded-ring}{associated graded ring}
    The \textbf{associated graded ring} of a \tref{ring}{commutative-ring} $R$ and an \tref{ideal}{ideal} $\mathfrak{a}$ is the \tref{graded-ring}{graded ring}
    \[ G_\mathfrak{a}(R) = \bigoplus_{n = 0}^{\infty} \mathfrak{a}^n / \mathfrak{a}^{n + 1} . \]
\end{topic}

\begin{topic}{completion}{completion}
    Let $R$ be a \tref{ring}{commutative ring} and $M$ an \tref{module}{$R$-module}. Each \tref{ideal}{ideal} $\mathfrak{a} \subset R$ determines a \tref{TO:topological-space}{topology} on $M$ called the \textbf{$\mathfrak{a}$-adic topology}: a subset $U \subset M$ is \textit{open} if and only if for each $x \in U$ there exists a positive integer $n$ such that $x + \mathfrak{a}^n M \subset U$.
    
    The \textbf{completion} of $M$ with respect to $\mathfrak{a}$ is the \tref{CT:inverse-limit}{inverse limit}
    \[ \widehat{M} = \varprojlim_{n \ge 1} M / \mathfrak{a}^n M = \left\{ (x_n)_{n \ge 1} \in \prod_{n \ge 1} M / \mathfrak{a}^n M : x_m = x_n \text{ mod } \mathfrak{a}^n M \text{ for } m \le n \right\} \]
    with the \tref{TO:subspace-topology}{subspace} \tref{TO:product-topology}{product topology}.
    
    When $R = M$, the completion $\widehat{R}$ has a ring structure and is called the \textbf{completion} of $R$, with repsect to $\mathfrak{a}$.
\end{topic}

\begin{example}{completion}
    When $R = k[x_1, \ldots, x_n]$ and $\mathfrak{a} = (x_1, \ldots, x_n)$, the completion is the power series ring
    \[ \widehat{R} = k\llbracket x_1, \ldots, x_n \rrbracket . \]
\end{example}
