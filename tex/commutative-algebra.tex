\begin{topic}{ring}{ring}
    Definition \textbf{ring}
\end{topic}

\begin{topic}{ideal}{ideal}
    Let $R$ be a \tref{ring}{ring}. An \textbf{ideal} of $R$ is a subset $I \subset R$ such that
    \begin{itemize}
        \item $I$ is a subgroup of $R$ under addition,
        \item $ra \in I$ (left-ideal) for all $r \in R$ and $a \in I$,
        \item $ar \in I$ (right-ideal) for all $r \in R$ and $a \in I$,
    \end{itemize}
\end{topic}

\begin{topic}{principal-ideal}{principal ideal}
    A \textbf{principal ideal} is an \tref{ideal}{ideal} $I \subset R$ generated by a single element, that is, of the form
    \[ I = (a) = \{ r a : r \in R \} \]
    for some $a \in R$.
\end{topic}

\begin{topic}{unit}{unit}
    Let $R$ be a \tref{ring}{ring}. An element $a \in R$ is called a \textbf{unit} if there exists some $b \in R$ such that $ab = 1$. We write $b = a^{-1}$.
\end{topic}

\begin{topic}{zero-divisor}{zero-divisor}
    Let $R$ be a \tref{ring}{ring}. An element $a \in R$ is called a \textbf{zero-divisor} if $a \ne 0$ and $ab = 0$ for some $b \ne 0$.
\end{topic}

\begin{topic}{nilpotent}{nilpotent}
    Let $R$ be a \tref{ring}{ring}. An element $a \in R$ is \textbf{nilpotent} if $a^n = 0$ for some positive integer $n$.
\end{topic}

\begin{topic}{domain}{domain}
    A \tref{ring}{ring} $R$ is called a \textbf{domain} if it has no zero-divisors, that is, $ab = 0$ implies $a = 0$ or $b = 0$ for all $a, b \in R$.
\end{topic}

\begin{topic}{field}{field}
    A \tref{ring}{ring} $R$ is called a \textbf{field} if all non-zero elements are units, that is, for all non-zero $a \in R$ there exists a $b \in R$ such that $ab = 1$. 
\end{topic}

\begin{topic}{prime-ideal}{prime ideal}
    Let $R$ be a \tref{ring}{ring}. An \tref{ideal}{ideal} $I \subset R$ is called \textbf{prime} if $I \ne R$ and whenever $ab \in I$ we have $a \in I$ or $b \in I$.
    
    Equivalently, $I \subset R$ is a prime ideal iff the quotient ring $R / I$ is a \tref{domain}{domain}.
\end{topic}

\begin{topic}{maximal-ideal}{maximal ideal}
    Let $R$ be a \tref{ring}{ring}. An \tref{ideal}{ideal} $I \subset R$ is called \textbf{maximal} if $I \ne R$ and whenever $I \subset J \subsetneq R$ for some ideal $J \subset R$ we have $I = J$.
    
    Equivalently, $I \subset R$ is a maximal ideal iff the quotient ring $R / I$ is a \tref{field}{field}.
    
    Every maximal ideal is \tref{prime-ideal}{prime}.
\end{topic}

\begin{topic}{local-ring}{local ring}
    A \textbf{local ring} is a \tref{ring}{ring} $R$ with exactly one \tref{maximal-ideal}{maximal ideal}. The maximal is often denoted by $\mathfrak{m}$.
    
    The quotient $k = R/\mathfrak{m}$ is called the \textbf{residue field}.
\end{topic}

\begin{topic}{local-morphism}{local morphism}
    A \textbf{local morphism} of \tref{local-ring}{local rings} $f : R \to S$ is a ring morphism such that $f(\mathfrak{m}_R) \subset \mathfrak{m}_S$.
\end{topic}

\begin{topic}{finite-type}{finite type}
    A ring morphism $R \to S$ is said to be of \textbf{finite presentation} if $S$ is isomorphic to a quotient of $R[x_1, \ldots, x_n]$ for some integer $n$.
    
    In this case, $S$ is also said to be a \textbf{finitely generated $R$-algebra}.
\end{topic}

\begin{topic}{finite-presentation}{finite presentation}
    A ring morphism $R \to S$ is said to be of \textbf{finite presentation} if $S$ is isomorphic to $R[x_1, \ldots, x_n] / (f_1, \ldots, f_m)$ for some integer $n$ and some $f_i \in R[x_1, \ldots, x_n]$.
    
    In this case, $S$ is also said to be a \textbf{finitely presented $R$-algebra}.
\end{topic}

\begin{topic}{krull-dimension}{Krull dimension}
    The \textbf{Krull dimension} of a commutative ring $R$ is the supremum of the lengths of all chains of \tref{prime-ideal}{prime ideals}, where a chain of the form
    \[ \mathfrak{p}_0 \subsetneq \mathfrak{p}_1 \subsetneq \cdots \subsetneq \mathfrak{p}_n \]
    has length $n$.
\end{topic}
