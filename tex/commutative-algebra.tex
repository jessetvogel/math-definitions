\begin{topic}{ring}{ring}
    A \textbf{ring} is an abelian group $R$ with an operation called multiplication and an element $1 \in R$ satisfying
    \begin{itemize}
        \item (\textit{associativity}) $a(bc) = (ab)c$,
        \item (\textit{distributivity}) $a(b + c) = ab + ac$ and $(a + b)c = ac + bc$,
        \item (\textit{unit}) $1 \cdot a = a \cdot 1 = a$,
    \end{itemize}
    for all $a, b, c \in R$.
    
    A ring $R$ is called commutative if moreover $ab = ba$ for all $a, b \in R$.
\end{topic}

\begin{topic}{ring-morphism}{ring morphism}
    A map $f : R \to S$ between \tref{ring}{rings} is a \textbf{ring morphism} if
    \begin{itemize}
        \item $f(1) = 1$,
        \item $f(a + b) = f(a) + f(b)$,
        \item $f(ab) = f(a) f(b)$,
    \end{itemize}
    for all $a, b \in R$.
\end{topic}

\begin{topic}{unit}{unit}
    Let $R$ be a \tref{ring}{ring}. An element $a \in R$ is called a \textbf{unit} if there exists some $b \in R$ such that $ab = 1$. We write $b = a^{-1}$.
\end{topic}

\begin{topic}{irreducible}{irreducible}
    Let $R$ be a \tref{domain}{domain}. An element $a \in R$ is called a \textbf{irreducible} if $a$ is not a \tref{unit}{unit}, and for all $b, c \in R$ such that $bc = a$ either $b$ or $c$ is a unit.
\end{topic}

\begin{topic}{zero-divisor}{zero-divisor}
    Let $R$ be a \tref{ring}{ring}. An element $a \in R$ is called a \textbf{zero-divisor} if $a \ne 0$ and $ab = 0$ for some $b \ne 0$.
\end{topic}

\begin{topic}{nilpotent}{nilpotent}
    Let $R$ be a \tref{ring}{ring}. An element $a \in R$ is \textbf{nilpotent} if $a^n = 0$ for some positive integer $n$.
\end{topic}

\begin{topic}{domain}{domain}
    A non-zero commutative \tref{ring}{ring} $R$ is called a \textbf{domain} if it has no zero-divisors, that is, $ab = 0$ implies $a = 0$ or $b = 0$ for all $a, b \in R$.
\end{topic}

\begin{topic}{field}{field}
    A commutative \tref{ring}{ring} $R$ is called a \textbf{field} if all non-zero elements are units, that is, for all non-zero $a \in R$ there exists a $b \in R$ such that $ab = 1$. 
\end{topic}

\begin{topic}{ideal}{ideal}
    Let $R$ be a \tref{ring}{ring}. An \textbf{ideal} of $R$ is a subset $I \subset R$ such that
    \begin{itemize}
        \item $I$ is a subgroup of $R$ under addition,
        \item (\textit{left-ideal}) $ra \in I$  for all $r \in R$ and $a \in I$,
        \item (\textit{right-ideal}) $ar \in I$ for all $r \in R$ and $a \in I$.
    \end{itemize}
\end{topic}

\begin{topic}{principal-ideal}{principal ideal}
    A \textbf{principal ideal} is an \tref{ideal}{ideal} $I \subset R$ generated by a single element, that is, of the form
    \[ I = (a) = \{ r a : r \in R \} \]
    for some $a \in R$.
\end{topic}

\begin{topic}{prime-ideal}{prime ideal}
    Let $R$ be a \tref{ring}{ring}. An \tref{ideal}{ideal} $I \subset R$ is called \textbf{prime} if $I \ne R$ and whenever $ab \in I$ we have $a \in I$ or $b \in I$.
    
    Equivalently, $I \subset R$ is a prime ideal iff the quotient ring $R / I$ is a \tref{domain}{domain}.
\end{topic}

\begin{topic}{maximal-ideal}{maximal ideal}
    Let $R$ be a \tref{ring}{ring}. An \tref{ideal}{ideal} $I \subset R$ is called \textbf{maximal} if $I \ne R$ and whenever $I \subset J \subsetneq R$ for some ideal $J \subset R$ we have $I = J$.
    
    Equivalently, $I \subset R$ is a maximal ideal iff the quotient ring $R / I$ is a \tref{field}{field}.
    
    Every maximal ideal is \tref{prime-ideal}{prime}.
\end{topic}

\begin{topic}{local-ring}{local ring}
    A \textbf{local ring} is a \tref{ring}{ring} $R$ with exactly one \tref{maximal-ideal}{maximal ideal}. The maximal is often denoted by $\mathfrak{m}$.
    
    The quotient $k = R/\mathfrak{m}$ is called the \textbf{residue field}.
\end{topic}

\begin{topic}{local-morphism}{local morphism}
    A \textbf{local morphism} of \tref{local-ring}{local rings} $f : R \to S$ is a ring morphism such that $f(\mathfrak{m}_R) \subset \mathfrak{m}_S$.
\end{topic}

\begin{topic}{finite-type}{finite type}
    A ring morphism $R \to S$ is said to be of \textbf{finite presentation} if $S$ is isomorphic to a quotient of $R[x_1, \ldots, x_n]$ for some integer $n$.
    
    In this case, $S$ is also said to be a \textbf{finitely generated $R$-algebra}.
\end{topic}

\begin{topic}{finite-presentation}{finite presentation}
    A ring morphism $R \to S$ is said to be of \textbf{finite presentation} if $S$ is isomorphic to $R[x_1, \ldots, x_n] / (f_1, \ldots, f_m)$ for some integer $n$ and some $f_i \in R[x_1, \ldots, x_n]$.
    
    In this case, $S$ is also said to be a \textbf{finitely presented $R$-algebra}.
\end{topic}

\begin{topic}{krull-dimension}{Krull dimension}
    The \textbf{Krull dimension} of a commutative ring $R$ is the supremum of the lengths of all chains of \tref{prime-ideal}{prime ideals}, where a chain of the form
    \[ \mathfrak{p}_0 \subsetneq \mathfrak{p}_1 \subsetneq \cdots \subsetneq \mathfrak{p}_n \]
    has length $n$.
\end{topic}

\begin{topic}{group-ring}{group ring}
    Let $R$ be a \tref{ring}{ring} and $G$ a group. The \textbf{group ring} $R[G]$ of $G$ over $R$ is defined as
    \[ R[G] = \bigoplus_{g \in G} R , \]
    where multiplication is induced by
    \[ (a \cdot g) \cdot (b \cdot h) = (ab) \cdot gh . \]
    % where multiplication is given by
    % \[ (a_g)_{g \in G} \cdot (b_g)_{g \in G} = \left(\sum_{h \in G} a_h b_{h^{-1}g}\right)_{g \in G} \]
\end{topic}

\begin{topic}{chinese-remainder-theorem}{Chinese remainder theorem}
    Let $R$ be a commutative \tref{ring}{ring} and suppose $I, J$ are coprime ideals of $R$, i.e. $I + J = R$. Then the \textbf{Chinese remainder theorem} states that $I \cap J = I \cdot J$ and that there is an isomorphism of rings
    \[ R / (I \cdot J) \xrightarrow{\sim} (R / I) \times (R/J), \qquad a \text{ mod } (I \cdot J) \mapsto (a \text{ mod } I, b \text{ mod } J) . \]
    
    In particular, when $R = \ZZ$ with $I = (n)$ and $J = (m)$ for $n, m$ relatively prime, there is
    \[ \ZZ / nm \ZZ \simeq (\ZZ/n\ZZ) \times (\ZZ/m\ZZ), \qquad a \text{ mod } nm \mapsto (a \text{ mod } n, a \text{ mod } m) . \]
\end{topic}

\begin{topic}{idempotent}{idempotent}
    An element $e \in R$ of a \tref{ring}{ring} $R$ is called \textbf{idempotent} if $e^2 = e$.
\end{topic}

\begin{topic}{dual-numbers}{dual numbers}
    Let $R$ be a commutative \tref{ring}{ring}. The \textbf{ring of dual numbers} over $R$ is the quotient ring
    \[ R[\varepsilon] / (\varepsilon^2) . \]
\end{topic}

\begin{topic}{noetherian-ring}{noetherian ring}
    A commutative \tref{ring}{ring} $R$ is called \textbf{noetherian} if it satisfies the \textit{ascending chain condition}: for any increasing sequence of ideals
    \[ I_1 \subset I_2 \subset I_3 \subset \cdots \]
    there exists some $N \in \NN$ such that $I_n = I_N$ for any $n \ge N$.
    
    Equivalently, $R$ is noetherian if all its ideals are finitely generated.
\end{topic}

\begin{example}{noetherian-ring}
    The following are all noetherian rings.
    \begin{itemize}
        \item Any \tref{field}{field}: their only ideal is $(0)$.
        \item Any \tref{principal-ideal-domain}{PID}: every ideal is generated by a single element.
        \item If $R$ is noetherian, then so is $R[x]$: this is \textit{Hilbert's basis theorem}.
    \end{itemize}
\end{example}

\begin{example}{noetherian-ring}
    The ring $k[x_1, x_2, x_3, \ldots]$ is not noetherian. Namely, the sequence of ideals
    \[ (x_1) \subset (x_1, x_2) \subset (x_1, x_2, x_3) \subset \ldots \]
    is ascending, but does not stabilize.
\end{example}

\begin{topic}{localization}{localization}
    Let $R$ be a commutative \tref{ring}{ring}, and $S \subset R$ a \textit{multiplicative set} (that is, $1 \in S$ and $xy \in S$ for all $x, y \in S$). Then the \textbf{localization} of $R$ w.r.t. $S$ is the ring
    \[ S^{-1} R = R \times S / \sim{} \quad \text{ where } (r_1, s_1) \sim{} (r_2, s_2) \text{ if } t(r_1 s_2 - r_2 s_1) = 0 \text{ for some } t \in S . \]
\end{topic}

\begin{topic}{regular-ring}{regular (local) ring}
    A \textbf{regular local ring} is a commutative \tref{noetherian-ring}{noetherian} \tref{local-ring}{local} ring such that the minimal number of generators of its maximal ideal is equal to its \tref{krull-dimension}{Krull dimension}.
    
    A \textbf{regular ring} is a commutative noetherian ring, such that the \tref{localization}{localization} at every \tref{prime-ideal}{prime ideal} is a regular local ring.
\end{topic}

\begin{example}{regular-ring}
    Every field is a regular local ring: their Krull dimension is zero, and their maximal ideal is $(0)$.
\end{example}

\begin{example}{regular-ring}
    The local ring $R = k[x]/(x^2)$ is not a regular local ring. Its only prime ideal is its maximal ideal $\mathfrak{m} = (x)/(x^2)$, so the Krull dimension is zero, but the minimal number of generators of $\mathfrak{m}$ is one.
\end{example}

\begin{topic}{principal-ideal-domain}{principal ideal domain (PID)}
    A \textbf{principal ideal domain} (PID) is a \tref{domain}{domain} $R$ in which every \tref{ideal}{ideal} is \tref{principal-ideal}{principal}. That is, every ideal $I \subset R$ is of the form $I = (x)$ for some element $x \in R$.
\end{topic}

\begin{topic}{unique-factorization-domain}{unique factorization domain (UFD)}
    A \textbf{unique factorization domain} (UFD) is a \tref{domain}{domain} $R$ for which every non-zero $x \in R$ can be written as the product of a \tref{unit}{unit} and a finite number of \tref{irreducible}{irreducible} elements:
    \[ a = u \cdot p_1 \cdot p_2 \cdot \cdots \cdot p_k \qquad u \in R^\times, \; k \ge 0, \; p_i \in R \text{ irreducible}. \]
\end{topic}

\begin{topic}{euclidean-ring}{Euclidean ring}
    A \textbf{Euclidean ring} is a \tref{domain}{domain} $R$ for which there exists a function
    \[ g : R^\times \to \ZZ_{\ge 0} \]
    such that for all $a, b \in R$ with $b \ne 0$, there exists $q, r \in R$ with $a = qb + r$ and either $r = 0$ or $g(r) < g(b)$.
    
    That is, a ring in which one can perform division with remainder. The function $g$ is used to say that the 'remainder' $r$ is 'smaller' than the element $b$ one divides by.
    
    In particular, one can find the \textit{gcd} of elements by means of the \textit{Euclidean algorithm}.
\end{topic}
