\begin{topic}{ring}{ring}
    Definition \textbf{ring}
\end{topic}

\begin{topic}{ideal}{ideal}
    Let $R$ be a \tref{ring}{ring}. An \textbf{ideal} of $R$ is a subset $I \subset R$ such that
    \begin{itemize}
        \item $I$ is a subgroup of $R$ under addition,
        \item $ra \in I$ (left-ideal) for all $r \in R$ and $a \in I$,
        \item $ar \in I$ (right-ideal) for all $r \in R$ and $a \in I$,
    \end{itemize}
\end{topic}

\begin{topic}{principal-ideal}{principal ideal}
    A \textbf{principal ideal} is an \tref{ideal}{ideal} $I \subset R$ generated by a single element, that is, of the form
    \[ I = (a) = \{ r a : r \in R \} \]
    for some $a \in R$.
\end{topic}

\begin{topic}{unit}{unit}
    Let $R$ be a \tref{ring}{ring}. An element $a \in R$ is called a \textbf{unit} if there exists some $b \in R$ such that $ab = 1$. We write $b = a^{-1}$.
\end{topic}

\begin{topic}{zero-divisor}{zero-divisor}
    Let $R$ be a \tref{ring}{ring}. An element $a \in R$ is called a \textbf{zero-divisor} if $a \ne 0$ and $ab = 0$ for some $b \ne 0$.
\end{topic}

\begin{topic}{domain}{domain}
    A \tref{ring}{ring} $R$ is called a \textbf{domain} if it has no zero-divisors, that is, $ab = 0$ implies $a = 0$ or $b = 0$ for all $a, b \in R$.
\end{topic}

\begin{topic}{field}{field}
    A \tref{ring}{ring} $R$ is called a \textbf{field} if all non-zero elements are units, that is, for all non-zero $a \in R$ there exists a $b \in R$ such that $ab = 1$. 
\end{topic}

\begin{topic}{prime-ideal}{prime ideal}
    Let $R$ be a \tref{ring}{ring}. An \tref{ideal}{ideal} $I \subset R$ is called \textbf{prime} if $I \ne R$ and whenever $ab \in I$ we have $a \in I$ or $b \in I$.
    
    Equivalently, $I \subset R$ is a prime ideal iff the quotient ring $R / I$ is a \tref{domain}{domain}.
\end{topic}

\begin{topic}{maximal-ideal}{maximal ideal}
    Let $R$ be a \tref{ring}{ring}. An \tref{ideal}{ideal} $I \subset R$ is called \textbf{maximal} if $I \ne R$ and whenever $I \subset J \subsetneq R$ for some ideal $J \subset R$ we have $I = J$.
    
    Equivalently, $I \subset R$ is a maximal ideal iff the quotient ring $R / I$ is a \tref{field}{field}.
    
    Every maximal ideal is \tref{prime-ideal}{prime}.
\end{topic}



\begin{topic}{module}{module}
    Let $R$ be a ring. An $R$-module is ...
\end{topic}

\begin{topic}{projective-module}{projective module}
    Let $R$ be a ring. An $R$-module $P$ is called \textbf{projective} if for every morphism $g : P \to M$ and surjective morphism $f : N \to M$ of $R$-modules, there exists a morphism $h : P \to N$ of $R$-modules such that $fh = g$. We do not require this map to be unique.
    \[ \begin{tikzcd} & N \arrow[twoheadrightarrow]{d}{f} \\ P \arrow[swap]{r}{g} \arrow[dashed]{ur}{\exists h} & M \end{tikzcd} \]
\end{topic}

\begin{topic}{injective-module}{injective module}
    Let $R$ be a ring. An $R$-module $I$ is called \textbf{injective} if for every morphism $g : M \to I$ and injective morphism $f : M \to N$ of $R$-modules, there exists a morphism $h : N \to Q$ of $R$-modules such that $hf = g$. We do not require this map to be unique.
    \[ \begin{tikzcd} M \arrow[hookrightarrow]{r}{f} \arrow[swap]{d}{g} & N \arrow[dashed]{ld}{\exists h} \\ I & \end{tikzcd} \]
\end{topic}
