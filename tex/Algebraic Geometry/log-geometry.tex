\begin{topic}{log-scheme}{Log scheme}
    A \textbf{log scheme} $(X, \mathcal{M})$ consists of a \tref{scheme}{scheme} $X$ together with a \tref{sheaf}{sheaf} of \tref{AA:monoid}{commutative monoids} $\mathcal{M}$ on $X$, and a monoid morphism $\alpha \colon \mathcal{M} \to \mathcal{O}_X$ such that $\alpha \colon \alpha^{-1} \mathcal{O}_X^* \to \mathcal{O}_X^*$ is an isomorphism.
    
    A \textit{morphism of log schemes} $(X, \mathcal{M}) \to (Y, \mathcal{N})$ is a morphism of schemes $f \colon X \to Y$ together with a map $h \colon f^{-1} \mathcal{N} \to \mathcal{M}$ such that
    \[ \svg \begin{tikzcd} f^{-1} \mathcal{N} \arrow{r}{h} \arrow[swap]{d}{f^{-1} \beta} & \mathcal{M} \arrow{d}{\alpha} \\ f^{-1} \mathcal{O}_Y \arrow{r}{f^\#} & \mathcal{O}_X \end{tikzcd} \]
    commutes. Such a morphism is called \textit{strict} if the induced map $f^* \mathcal{N} \to \mathcal{M}$ is an isomorphism.
\end{topic}

\begin{example}{log-scheme}
    \begin{itemize}
        \item For any scheme $X$, taking $\mathcal{M} = \mathcal{O}_X^*$ gives the \textit{trivial log structure} on $X$. It is the initial object in the category of log structures over $X$.
        \item For any scheme $X$, taking $\mathcal{M} = \mathcal{O}_X$ gives the final object in the category of log structures over $X$.
        \item For a scheme $X$ and divisor $D \subset X$, the \textit{divisorial log structure} is given by
        \[ \mathcal{M}(U) = \{ f \in \mathcal{O}_X(U) : f \text{ is invertible outside $D$} \} \]
        for any open subset $U \subset X$. 
    \end{itemize}
\end{example}

\begin{example}{log-scheme}
    Let $X = \Spec k[x, y]$. The map $\NN^2 \to k[x, y] : (a, b) \mapsto x^a y^b$ induces a sheaf of monoids
    \[ \mathcal{M}' = \NN^2 \oplus \mathcal{O}_X^* \to \mathcal{O}_X , \quad (a, b, f) \mapsto x^a y^b f , \]
    which is not quite a log structure. The \textit{associated} log structure is
    \[ \mathcal{M} = \mathcal{M}' \oplus_{\alpha^{-1} \mathcal{O}_X^*} \mathcal{O}_X^* . \]
    More generally, given a commutative ring $R$ and commutative monoid $P$, there exists a canonical log structure on the scheme $X = \Spec(R[P])$. Namely, the natural map $P \to R[P]$ induces a morphism of sheaves $\alpha \colon \underline{P} \to \mathcal{O}_X$, where $\underline{P}$ denotes the \tref{constant-sheaf}{constant sheaf} with values in $P$. The associated log structure is
    \[ \mathcal{P} = \underline{P} \oplus_{\alpha^{-1} \mathcal{O}_X^*} \mathcal{O}_X^* . \]
\end{example}

\begin{topic}{fine-log-scheme}{fine log scheme}
    A \tref{log-scheme}{log scheme} $(X, \mathcal{M})$ is \textbf{fine} if it is \textit{coherent} and \textit{integral}. That is, étale locally on $X$, there exists a finitely generated integral monoid $P$ and a morphism $P_X \to \mathcal{O}_X$ (where $P_X$ denotes the constant sheaf) whose associated log structure is isomorphic to $\mathcal{M}$.
\end{topic}

\begin{topic}{log-smooth-morphism}{log smooth morphism}
    Let $f \colon (X, \mathcal{M}) \to (Y, \mathcal{N})$ be a morphism of \tref{fine-log-scheme}{fine} \tref{log-scheme}{log schemes}. Then $f$ is \textbf{formally log smooth} if for any commutative diagram
    \[ \svg \begin{tikzcd} (T', \mathcal{L}') \arrow{r} \arrow[swap]{d}{i} & (X, \mathcal{M}) \arrow{d}{f} \\ (T, \mathcal{L}) \arrow{r} & (Y, \mathcal{N}) \end{tikzcd} \]
    with $i$ an strict closed immersion (i.e. $T' \to T$ a \tref{closed-immersion}{closed immersion} and $i^* \mathcal{L} \to \mathcal{L}'$ an isomorphism), and $T' \subset T$ defined by an ideal $I$ with $I^2 = 0$, there exists étale locally on $T$ a morphism $g \colon (T, \mathcal{L}) \to (X, \mathcal{M})$ making the diagram commute.
    
    The morphism $f$ is \textbf{log smooth} if it is formally log smooth and the underlying morphism of schemes is \tref{finite-presentation}{locally of finite presentation}.
\end{topic}

\begin{topic}{log-etale-morphism}{log étale morphism}
    Let $f \colon (X, \mathcal{M}) \to (Y, \mathcal{N})$ be a morphism of \tref{fine-log-scheme}{fine} \tref{log-scheme}{log-schemes}. Then $f$ is \textbf{formally log étale} if for any commutative diagram
    \[ \svg \begin{tikzcd} (T', \mathcal{L}') \arrow{r} \arrow[swap]{d}{i} & (X, \mathcal{M}) \arrow{d}{f} \\ (T, \mathcal{L}) \arrow{r} & (Y, \mathcal{N}) \end{tikzcd} \]
    with $i$ an strict closed immersion (i.e. $T' \to T$ a \tref{closed-immersion}{closed immersion} and $i^* \mathcal{L} \to \mathcal{L}'$ an isomorphism), and $T' \subset T$ defined by an ideal $I$ with $I^2 = 0$, there exists a unique morphism $g \colon (T, \mathcal{L}) \to (X, \mathcal{M})$ making the diagram commute.
    
    The morphism $f$ is \textbf{log étale} if it is formally log étale and the underlying morphism of schemes is \tref{finite-presentation}{locally of finite presentation}.
\end{topic}

\begin{example}{log-etale-morphism}
    Let $k$ be a commutative ring. Let $P$ and $Q$ be finitely generated integral commutative monoids, take $X = \Spec k[P]$ and $Y = \Spec k[Q]$ with their natural log structures, and $f \colon X \to Y$ a morphism of log schemes induced by a morphism of monoids $\varphi \colon Q \to P$. Then, it can be proven that $f$ is log étale if and only if $\ker \varphi^\textup{gp}$ and $\coker \varphi^\textup{gp}$ are finite groups whose order is invertible in $k$.
    
    Considering $\AA^1_k = \Spec k[\NN]$ and $\AA^2_k = \Spec k[\NN^2]$, it follows that
    \begin{itemize}
        \item the map $g \colon \AA^1_k \to \AA^1_k$, $x \mapsto x^{17}$ induced by $\varphi \colon \NN \to \NN$, $n \mapsto 17n$ is log étale if and only if $17$ is invertible in $k$, since $\ker(\varphi^\textup{gp} \colon \ZZ \to \ZZ) = 0$ and $\coker(\varphi^\textup{gp} \colon \ZZ \to \ZZ) = \ZZ / 17 \ZZ$ are finite. Note that $g$ is not étale as a morphism of schemes, as $f$ is ramified at the origin.
        \item the map $h \colon \AA^2_k \to \AA^1_k$, $(x, y) \mapsto xy$ induced by $\varphi \colon \NN \to \NN^2$, $n \mapsto (n, n)$ is not log étale since $\coker(\varphi^\textup{gp} \colon \ZZ \to \ZZ^2) \isom \ZZ$ is infinite.
    \end{itemize}
    % Note that $f$ is log étale (provided $17 \in k^*$), and induces the morphism on the \tref{log-differentials}{log differentials}
    % \[ \begin{aligned}
    %     \omega^1_{\AA^1_k/k} &\to \omega^1_{\AA^1_k/k} \\
    %     \frac{dy}{y} \mapsto \frac{d(x^17)}{x^17} = 17 \frac{dx}{x}
    % \end{aligned} \]
    % is an isomorphism.
\end{example}

\begin{topic}{log-differentials}{log differentials}
    Let $f \colon (X, \mathcal{M}) \to (Y, \mathcal{N})$ be a morphism of \tref{log-scheme}{log schemes}. The $\mathcal{O}_X$-module $\omega^1_{X/Y}$ of \textbf{log differentials} is the quotient of
    \[ \Omega_{X/Y}^1 \oplus (\mathcal{O}_X \otimes_\ZZ \mathcal{M}^\textup{gp}) \]
    by the following relations of local sections:
    \begin{itemize}
        \item $(d \alpha(a), 0) = (0, \alpha(a) \otimes a)$ for all $a \in M$,
        \item $(0, 1 \otimes a) = 0$ for all $a \in \im(f^{-1} \mathcal{N} \to \mathcal{M})$.
    \end{itemize}
    
    Think of $\omega^1_{X/Y}$ as an extension of the usual sheaf of differentials $\Omega_{X/Y}^1$, where elements of the form $d \log(a) = \frac{da}{a}$ (represented by $(0, 1 \otimes a)$) are added for all $a \in \mathcal{M}^\textup{gp}$, up to the image of $f^{-1} \mathcal{N}$. In fact, we have a morphism
    \[ d \log \colon \mathcal{M} \to \omega_{X/Y}^1, \quad f \mapsto d \log(f) . \]
\end{topic}

\begin{example}{log-differentials}
    Let $k$ be a field with characteristic $\ne 2$. Consider the map $\AA^1_k \to \AA^1_k$ given by $t = s^2$. The usual sheaf of differentials $\Omega^1_{\AA^1_k}$ is given by the free $k[x]$-module with generator $d x$, and the above map induces
    \[ \Omega^1_{\AA^1_k} \to \Omega^1_{\AA^1_k}, \qquad d t \mapsto 2 s d s \]
    The idea is that this map is not very nice in the sense that the cokernel has support at $s = 0$, and that we `fix' this using log differentials.
    
    Take the divisor $D = \{ 0 \}$, and consider the log structure $\mathcal{M}$ on $\AA^1_k$ given by
    \[ \mathcal{M} = \{ f \in \mathcal{O}_X : f \text{ is invertible outside } D \} . \]
    Note that $x \in \mathcal{M}$, so we obtain $d \log(x) = \frac{d x}{x} \in \omega^1_{\AA^1_k}$. In fact, one can show that $\omega^1_{\AA^1_k}$ is given by the free $k[x]$-module generated by $\frac{d x}{x}$. Now the map from before is given by
    \[ \omega^1_{\AA^1_k} \to \omega^1_{\AA^1_k} \qquad  \qquad \frac{d t}{t} \mapsto 2 \frac{d s} s , \]
    which is now an isomorphism.
\end{example}

\begin{example}{log-differentials}
    Let $k$ be a commutative ring. Let $P$ and $Q$ commutative monoids, take $X = \Spec k[P]$ and $Y = \Spec k[Q]$ with their natural log structure, and let $f \colon X \to Y$ be a morphism of log schemes induced by a morphism of monoids $Q \to P$. Then the sheaf of log differentials $\omega^1_{X/Y}$ is given by
    \[ \begin{aligned}
        \omega^1_{X/Y} &\isom \mathcal{O}_X \otimes_\ZZ (P^\textup{gp} / \operatorname{im} Q^\textup{gp}) , \\
        a \cdot d \log b  &\mapsfrom a \otimes b
    \end{aligned} \]
\end{example}

\begin{topic}{characteristic-monoid-sheaf}{characteristic monoid sheaf}
    Let $(X, \mathcal{M})$ be a \tref{log-scheme}{log scheme}. The \textbf{characteristic monoid sheaf} is the quotient sheaf
    \[ \overline{\mathcal{M}} = \mathcal{M} / \alpha^{-1} \mathcal{O}_X^* . \]
\end{topic}

\begin{example}{characteristic-monoid-sheaf}
    Consider the log structure on $X = \Spec k[x, y]$ associated to the map $\NN^2 \to k[x, y] : (a, b) \mapsto x^a y^b$. The characteristic monoid sheaf $\overline{M} = \mathcal{M} / \alpha^{-1} \mathcal{O}_X^*$ is a \tref{skyscraper-sheaf}{skyscraper sheaf} at the origin with value $\NN^2$.
\end{example}
