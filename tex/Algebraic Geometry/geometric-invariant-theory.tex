\begin{topic}{categorical-quotient}{categorical quotient}
    Let $G$ be a \tref{group-scheme}{group scheme} acting on a \tref{scheme}{scheme} $X$ via $\sigma \colon G \times X \to X$. A \textbf{categorical quotient} for the action is a morphism $\pi \colon X \to Y$ such that
    \[ \svg \begin{tikzcd} G \times_S X \arrow[shift left=0.25em]{r}{\sigma} \arrow[swap,shift right=0.25em]{r}{\pi_X} & X \arrow{r}{\pi} & Y \end{tikzcd} \]
    is a coequalizer diagram. That is, the diagram commutes and for every $f \colon X \to Z$ with $f \circ \sigma = f \circ \pi_X$ there exists a unique $g \colon Y \to Z$ such that $f = g \circ \pi$. Note that a categorical quotient (if it exists) is unique up to unique isomorphism.
\end{topic}

\begin{topic}{geometric-quotient}{geometric quotient}
    Let $G$ be a \tref{group-scheme}{group scheme} acting on a \tref{scheme}{scheme} $X$ via $\sigma \colon G \times X \to X$. A \textbf{geometric quotient} for the action is a morphism $\pi \colon X \to Y$ satisfying
    \begin{enumerate}[label=(\roman*)]
        \item $\pi \circ \sigma = \pi \circ \pi_X$,
        \item $\pi$ is surjective, and the image of $\Psi = (\sigma, \pi_X) \colon G \times X \to X \times X$ is $X \times_Y X$ (or equivalently, the geometric fibers of $\pi$ are precisely the orbits of the geometric points of $X$),
        \item $\pi$ is \textit{submersive}, i.e. a subset $U \subset Y$ is open iff $\pi^{-1}(U) \subset X$ is open (i.e. $Y$ has a \tref{TO:quotient-topology}{quotient topology}),
        \item the structure sheaf $\mathcal{O}_Y$ is the subsheaf of $\pi_* \mathcal{O}_X$ consisting of invariant functions, i.e. if $\alpha \in \pi_* \mathcal{O}_X(U) = \mathcal{O}_X(\pi^{-1}(U))$ then $\alpha \in \mathcal{O}_Y(U)$ if and only if
        \[ \svg \begin{tikzcd} G \times \pi^{-1}(U) \arrow{r}{\sigma} \arrow[swap]{d}{\pi_2} & \pi^{-1}(U) \arrow{d}{\alpha} \\ \pi^{-1}(U) \arrow{r}{\alpha} & \AA^1 \end{tikzcd} \]
        commutes.
    \end{enumerate}
\end{topic}

\begin{example}{geometric-quotient}
    Let $X = \AA^n - \{ 0 \}$ and $G = \GG_m$ act on $X$ by scaling the coordinates. Then the map $\pi \colon \AA^n - \{ 0 \} \to \PP^{n - 1}$ is a geometric quotient.
    
    If we instead take $X = \AA^n$ with $G = \GG_m$ acting on $X$ by scaling the coordinates, then the map $\pi' \colon \AA^n \to \{ \star \}$ is a \tref{categorical-quotient}{categorical quotient}, but not a geometric quotient: the only geometric fiber of $\pi'$ is the whole of $X$, but this is not the orbit of any geometric point of $X$. To be more precise, the image of $\Psi : (g, x) \mapsto (g \cdot x, x)$ is not the whole of $\AA^n \times \AA^n$.
\end{example}

\begin{example}{geometric-quotient}
    A geometric quotient $\pi \colon X \to Y$ is a \tref{categorical-quotient}{categorical quotient}. Namely, let $f \colon X \to Z$ be an $S$-morphism with $f \circ \sigma = f \circ p_2$. The goal is to construct a morphism $g \colon Y \to Z$ with $f = g \circ \pi$. Let $\{ V_i \}$ be an affine open covering of $Z$. Then $f^{-1}(V_i)$ is an invariant open subset of $X$ for each $i$, so by condition (ii), $f^{-1}(V_i) = \pi^{-1}(U_i)$ for \textit{some} subset $U_i \subset Y$. But then $U_i$ is open by (iii), and since $\pi$ is surjective, $\{ U_i \}$ is an open cover of $Y$. Now note that any morphism $g \colon Y \to Z$ with $f = g \circ \pi$ must satisfy $g(U_i) \subset V_i$. Hence it must be defined by a set of (compatible) morphisms $h_i \colon \mathcal{O}_Z(V_i) \to \mathcal{O}_Y(U_i)$ such that
    \[ \svg \begin{tikzcd} \mathcal{O}_Z(V_i) \arrow{r}{h_i} \arrow{d}{f^*} & \mathcal{O}_Y(U_i) \arrow{d}{\pi^*} \\ \mathcal{O}_X(f^{-1}(V_i)) \arrow[equal]{r} & \mathcal{O}_X(\pi^{-1}(U_i)) \end{tikzcd} \]
    commutes. Since $\pi^*$ is injective by (iv), the $h_i$ are unique if they exist. Well, for any $\alpha \in \mathcal{O}_Z(V_i)$ we have that $\pi^*(\alpha)$ is an invariant element of $\mathcal{O}_X(\pi^{-1}(U_i))$ in the sense of (iv), hence it lies in the subring $\pi^*(\mathcal{O}_Y(U_i))$, so such $h_i$ exist.
    It remains to check that the $h_i$ agree on overlaps, so that they indeed glue to the desired $g \colon Y \to Z$.
\end{example}

\begin{topic}{affine-git-quotient}{affine GIT quotient}
    Let $G$ be a \tref{reductive-algebraic-group}{reductive} \tref{algebraic-group}{algebraic group} acting on an \tref{affine-scheme}{affine scheme} $X = \Spec R$. Then there is a dual action of $G$ on $R$ given by $\widehat{\sigma} \colon R \to R \otimes \mathcal{O}_G(G)$, yielding the \textit{ring of invariants}
    \[ R^G := \{ r \in R \mid \widehat{\sigma}(r) = r \otimes 1 \} . \]
    The \textbf{affine GIT quotient} of $X$ by $G$ is the morphism
    \[ \pi \colon X \to X \sslash G := \Spec\left(R^G\right) \]
    induced by the inclusion $R^G \to R$.
\end{topic}

\begin{example}{affine-git-quotient}
    \begin{itemize}
        \item For $X = \Spec k[x]$ and $G = \ZZ/2\ZZ$ acting on $X$ by $x \mapsto -x$, we have $k[x]^{\ZZ/2\ZZ} = k[x^2]$, so the affine GIT quotient is $\Spec k[x] \to \Spec k[y]$ given by $y \mapsto x^2$.
        \item For $X = \Spec k[\lambda, \lambda^{-1}, (\lambda \pm 1)^{-1}]$ and $G = \ZZ/2\ZZ$ acting on $X$ by $\lambda \mapsto -\lambda^{-1}$, we have $R^{\ZZ/2\ZZ} = k[\lambda + \lambda^{-1}, (\lambda + \lambda^{-1} \pm 2)^{-1}]$, so the affine GIT quotient is $\CC \setminus \{ 0, \pm 1 \} \to \CC \setminus \{ \pm 2 \}$ given by $\lambda \mapsto \lambda + \lambda^{-1}$.
        \item For $X = \AA^n = \Spec k[x_1, \ldots, x_n]$ and $G = \GG_m$ acting on $X$ by $x_i \mapsto x_i \otimes t$, we have $k[x]^{\GG_m} = k$, so the affine GIT quotient is $\AA^n \to \Spec k$.
    \end{itemize}
\end{example}

\begin{topic}{projective-git-quotient}{projective GIT quotient}
    Let $G$ be a \tref{reductive-algebraic-group}{reductive} \tref{algebraic-group}{algebraic group} acting on an \tref{algebraic-scheme}{algebraic scheme} $X$ over a field $k$, and let $\mathcal{L}$ be a \tref{equivariant-sheaf}{$G$-linearized} \tref{invertible-sheaf}{invertible sheaf} on $X$. The \textbf{projective GIT quotient} of $X$ by $G$ with respect to $\mathcal{L}$ is the morphism
    \[ \pi \colon X^\textup{ss}(\mathcal{L}) \to X \sslash_\mathcal{L} G := \text{Proj}\left(\bigoplus_{n \ge 0} H^0(X, \mathcal{L}^n)^G \right) , \]
    where
    \[ X^\textup{ss}(\mathcal{L}) = \left\{ x \in X \mid \exists s \in H^0(X, \mathcal{L}^n)^G \text{ for some $n \ge 0$ such that $s(x) \ne 0$ and $X_s$ is affine} \right\} \]
    denotes the \textit{semistable locus} of $X$. It is a uniform \tref{categorical-quotient}{categorical quotient}.
    
    Furthermore, there exists an open subset $U \subset X \sslash_\mathcal{L} G$ such that $\pi^{-1}(U)$ equals the set of \textit{stable points} \[ X^\textup{s}(\mathcal{L}) = \left\{ x \in X : \begin{array}{c} \exists s \in H^0(X, \mathcal{L}^n)^G \text{ for some $n \ge 0$ such that $s(x) \ne 0$,} \\ \text{$X_s$ is affine and the action of $G$ on $X_s$ is closed} \end{array} \right\} \]
    and the restriction $\pi|_{X^\textup{s}(\mathcal{L})} \colon X^\textup{s}(\mathcal{L}) \to U$ is a uniform \tref{geometric-quotient}{geometric quotient}.
\end{topic}

% \begin{topic}{projective-git-quotient}
%     Let $X = \AA^n_k$ and $G = \GG_m$ acting on $X$ by
%     \[ t \cdot (x_1, \cdots, x_n) = (t x_1, \ldots, t x_n) . \]
%     Take $\mathcal{L}$ by $(x_1, \ldots, x_n)$ with linearization $$
    
    
% \end{topic}

\begin{example}{projective-git-quotient}
    Note that the \tref{affine-git-quotient}{affine GIT quotient} can be seen as a particular (trivial) case of the projective GIT quotient. Given $X = \Spec R$ with an action of $G$, one can take the line bundle $\mathcal{L} = \mathcal{O}_X$ with linearization $\phi \colon \sigma^* \mathcal{O}_X \to p_2^* \mathcal{O}_X$ the identity. Then $H^0(X, \mathcal{O}_X^n)^G = R^G$ for any $n \ge 0$, and so
    \[ X \sslash_{\mathcal{L}} G = \operatorname{Proj} \left( \bigoplus_{n \ge 0} R^G \right) = \operatorname{Proj} \left( R^G[x] \right) = \PP^0_{\Spec \left( R^G \right) } = \Spec\left( R^G \right) , \]
    which is the affine GIT quotient. Note that $X^\textup{ss} = X$ as $1 \in R = H^0(X, \mathcal{L})$ is an invariant section which does not vanish on any point.
    
    In general, we have $H^0(X, \mathcal{L}^0)^G = \mathcal{O}_X(X)^G = R^G$ and hence there is always an induced map
    \[ X \sslash_{\mathcal{L}} G \to \Spec\left( R^G \right) \]
    to the affine GIT quotient.
\end{example}

\begin{topic}{luna-stratification}{Luna stratification}
    Let $G$ be a \tref{reductive-algebraic-group}{reductive algebraic group} over an algebraically closed field $k$, acting on an \tref{affine-scheme}{affine} \tref{variety}{variety} $X$, and let $\pi \colon X \to X \sslash G = \Spec(\mathcal{O}_X(X)^G)$ be the \tref{affine-git-quotient}{GIT quotient}.
    Every fiber $\pi^{-1}(p)$, for $p \in X \sslash G$, has a unique closed orbit, so choose $x_p \in \pi^{-1}(p)$ such that this closed orbit equals $G \cdot x_p$. For any conjugacy class $(H)$ of reductive subgroups $H \subset G$, let 
    \[ (X \sslash G)_{(H)} = \{ p \in X \sslash G \mid \textup{Stab}(x_p) \in (H) \} , \]
    which is independent of the choice of the $x_p$. The \textbf{Luna stratification} of $X \sslash G$ is the stratification
    \[ X \sslash G = \bigsqcup_{(H)} (X \sslash G)_{(H)} . \]
\end{topic}


\begin{example}{luna-stratification}
    Let $G = \textup{SL}_2(\CC)$, and consider $G$ acting on itself by conjugation. The GIT quotient is given by the trace map
    \[ \pi \colon G \to G \sslash G \isom \AA^1, \quad A \mapsto \tr(A) . \]
    The fiber $\pi^{-1}(2)$ consists of the orbits of $I = \left(\begin{smallmatrix} 1 & 0 \\ 0 & 1 \end{smallmatrix}\right)$ and $J = \left(\begin{smallmatrix} 1 & 1 \\ 0 & 1 \end{smallmatrix} \right)$, so the unique closed orbit of $\pi^{-1}(2)$ is $\left\{ I \right\}$, as the closure of $G \cdot J$ contains $\lim_{a \to 0} \left(\begin{smallmatrix} 1 & a \\ 0 & 1 \end{smallmatrix}\right) = I \not\in G \cdot J$. Similarly, the unique closed orbit of the fiber $\pi^{-1}(-2)$ is $\left\{ -I \right\}$. In both cases, the corresponding stabilizer is equal to $G$.
    
    For $t \ne \pm 2$, the fiber $\pi^{-1}(t)$ is precisely the orbit of $\left(\begin{smallmatrix} \lambda & 0 \\ 0 & \lambda^{-1} \end{smallmatrix}\right)$, with $\lambda \in \CC \setminus \{ 0, \pm 1 \}$ such that $\lambda + \lambda^{-1} = t$, and the corresponding conjugacy class of stabilizers is $(D)$, where $D \subset G$ is the subgroup of diagonal matrices.
    
    Therefore, the Luna stratification of $G \sslash G \isom \AA^1$ is given by
    \[ \AA^1 = (G \sslash G)_{(G)} \sqcup (G \sslash G)_{(D)}  = \{ \pm 2 \} \sqcup (\AA^1 \setminus \{ \pm 2 \}) . \]
\end{example}

\begin{topic}{s-equivalence}{S-equivalence}
    Let $G$ be an \tref{algebraic-group}{algebraic group} acting on a \tref{variety}{variety} $X$. Two points $x, y \in X$ are called \textbf{$S$-equivalent} if the closures of the orbits of $x$ and $y$ intersect.
\end{topic}

\begin{example}{s-equivalence}
    Let $G = \GG_m$ act on $X = \Spec k[x, y]$ via $\alpha \cdot (x, y) = (\alpha x, \alpha^{-1} y)$. The orbits of $X$ are given by
    \[ \{ (0, 0) \}, \quad \{ (0, y) : y \ne 0  \}, \quad \{ (x, 0) : x \ne 0  \}, \quad \textup{ and } \quad \{ (x, y) : xy = a \}  \]
    for any $a \ne 0$. Note that the origin is contained in the closures of the second and third orbit. Hence, the points of the first three orbits are all $S$-equivalent.
    
    The \tref{affine-git-quotient}{GIT quotient} is given by
    \[ X \sslash \GG_m = \Spec k[x, y]^{\GG_m} = \Spec k[z] , \]
    with $z = xy$, and we see that $S$-equivalent points are identified under the quotient.
\end{example}

% \begin{topic}{hilbert-mumford-criterion}{Hilbert--Mumford criterion}
%     Let $G$ be a \tref{reductive-algebraic-group}{reductive algebraic group} over a field $k$ acting linearly on a $k$-vector space $V$. 
    
%     The \textbf{Hilbert--Mumford criterion} states that
%     \begin{enumerate}[label=(\roman*)]
%         \item $x \in V$ is \tref{projective-git-quotient}{semistable} if and only if 
%     \end{enumerate}
% \end{topic}

\begin{topic}{kempf-ness-theorem}{Kempf--Ness theorem}
    Let $G$ be a complex \tref{reductive-algebraic-group}{reductive algebraic group} acting linearly on a \tref{non-singular-variety}{smooth} complex \tref{projective-variety}{projective variety} $X \subset \PP^n_\CC$.
    As $G$ is complex reductive, $G$ is equal to the complexification of its maximal compact subgroup $K$. The restriction of the \tref{DG:fubini-study-form}{Fubini--Study form} on $\PP^n_\CC$ to $X$ gives a \tref{DG:symplectic-manifold}{symplectic form} $\omega$ on $X$.
    After possibly rechoosing coordinates on $\PP^n_\CC$, $K$ acts unitarily on $\PP^n_\CC$, inducing a \tref{DG:symplectic-action}{symplectic action} of $K$ on $X$ admitting a \tref{DG:moment-map}{moment map} $\mu \colon X \to \mathfrak{k}^*$. % TODO: why does such a moment map exist?
    The \textbf{Kempf--Ness theorem} states that
    $\mu^{-1}(0) \subset X^\textup{ss}$ inducing a \tref{TO:homeomorphism}{homeomorphism} between the \tref{DG:marsden-weinstein-quotient}{Marsden--Weinstein quotient} and the \tref{projective-git-quotient}{projective GIT quotient}
    \[ \mu^{-1}(0) / K \cong X \sslash G . \]
\end{topic}

\begin{example}{kempf-ness-theorem}
    Let $G = \GG_m$ act on $X = \PP^n_\CC$ via
    \[ t \cdot (x_0 : x_1 : \cdots : x_n) = (t^{-1} x_0 : t x_1 : \cdots : t x_n) . \]
    The GIT quotient is seen to be $\pi \colon (\PP^n_\CC)^\textup{ss} \cong \AA^n \setminus \{ 0 \} \to \PP^n_\CC \sslash \GG_m \cong \PP^{n - 1}_\CC$.
    
    The action of the maximal compact subgroup $K = U(1)$ acts symplectically on $\PP^n_\CC$ equipped with the Fubini--Study form, and admits a moment map $\mu \colon \PP^n_\CC \to \mathfrak{u}(1)^* \cong (i \RR)^*$ given by
    \[ \mu(x_0 : \cdots : x_n)(i \alpha) = \frac{-|x_0|^2 + |x_1|^2 + \cdots |x_n|^2}{\sum_{i = 0}^{n} |x_i|^2} \alpha . \]
    Therefore,
    \[ \mu^{-1}(0) = \left\{ (x_0 : \cdots : x_n) \mid |x_0|^2 = \sum_{i = 1}^{n} |x_i|^2 \right\} \cong S^{2n - 1} \]
    and thus the Marsden--Weinstein quotient is given by $\mu^{-1}(0) / K \cong S^{2n - 1} / S^1 \cong \PP^{n - 1}_\CC$. In particular, the inclusion $\mu^{-1}(0) \subset (\PP^n_\CC)^\textup{ss}$ induces a homeomorphism from the Marsden--Weinstein quotient to the GIT quotient.
\end{example}

\begin{topic}{hilbert-mumford-weight}{Hilbert--Mumford weight}
    Let $G$ be a \tref{reductive-algebraic-group}{reductive algebraic group} over a field $k$ acting on a \tref{scheme}{scheme} $X$ \tref{proper-morphism}{proper} over $k$, and let $\mathcal{L}$ be an \tref{ample-invertible-sheaf}{ample} \tref{equivariant-sheaf}{$G$-linearized} \tref{invertible-sheaf}{line bundle} on $X$.

    For any closed point $x \in X$ and one-parameter subgroup $\lambda \colon \GG_m \to G$, the limit $y = \lim_{t \to 0} \lambda(t) \cdot x$ exists, as $X$ is proper over $k$, and is fixed under the action of $\GG_m$. Denoting this limit by $y \colon \Spec(k) \to X$, the group $\GG_m$ acts on $y^* \mathcal{L}$ for via a character $\chi \colon \GG_m \to \operatorname{Aut}(y^* \mathcal{L}) \cong \GG_m$ given by $t \mapsto t^r$ for some $r \in \ZZ$. The \textbf{Hilbert--Mumford weight} of $\lambda$ at $x$ with respect to $\mathcal{L}$ is defined as
    \[ \mu^\mathcal{L}(x, \lambda) = r. \]
\end{topic}

\begin{example}{hilbert-mumford-weight}
    Consider $G = \textup{GL}_2$ naturally acting on $X = \PP^1_k$, and let $\mathcal{L} = \mathcal{O}_X(1)$ which is $G$-linearized by the dual action $\hat{\sigma} \colon \mathcal{O}_X(X) \to \mathcal{O}_G(G) \otimes \mathcal{O}_X(X)$ given by $x_i \mapsto \sum_{j = 0}^{1} a_{ij} \otimes x_j$.
    Let $\lambda \colon \GG_m \to G$ be the one-parameter subgroup given by $\lambda(t) = \left(\begin{smallmatrix} t^2 & 0 \\ 0 & t^{-3} \end{smallmatrix}\right)$ and pick a point $x = (x_0 : x_1) \in X$. Then
    \[ y = \lim_{t \to 0} \lambda(t) \cdot x = \lim_{t \to 0} (t^2 x_0 : t^{-3} x_1) = \left\{ \begin{array}{cl} (0 : 1) & \textup{ if } x_1 \ne 0, \\ (1 : 0) & \textup{ if } x_1 = 0 . \end{array} \right. \]
    Hence, the vector space $y^* \mathcal{L}$ is generated by $x_1$ if $x_1 \ne 0$ and by $x_0$ if $x_1 = 0$. In particular,
    \[ \mu^\mathcal{L}(x, \lambda) = \left\{ \begin{array}{cl} -3 & \textup{ if } x_1 \ne 0, \\ 2 & \textup{ if } x_1 = 0 .
    \end{array} \right. \]
\end{example}

\begin{topic}{hilbert-mumford-criterion}{Hilbert--Mumford criterion}
    Let $G$ be a \tref{reductive-algebraic-group}{reductive algebraic group} over a field $k$ acting on a \tref{scheme}{scheme} $X$ \tref{proper-morphism}{proper} over $k$, and let $\mathcal{L}$ be an \tref{ample-invertible-sheaf}{ample} \tref{equivariant-sheaf}{$G$-linearized} \tref{invertible-sheaf}{line bundle} on $X$.
    
    The \textbf{Hilbert--Mumford Criterion} states that for any closed point $x \in X$,
    \begin{enumerate}[label=(\roman*)]
        \item $x$ is \tref{projective-git-quotient}{semistable} if and only if $\mu^\mathcal{L}(x, \lambda) \le 0$ for all one-parameter subgroups $\lambda$,
        \item $x$ is \tref{projective-git-quotient}{stable} if and only if $\mu^\mathcal{L}(x, \lambda) < 0$ for all one-parameter subgroups $\lambda$,
    \end{enumerate}
    where $\mu^\mathcal{L}(x, \lambda)$ denotes the \tref{hilbert-mumford-weight}{Hilbert--Mumford weight}.
\end{topic}

\begin{example}{hilbert-mumford-criterion}
    Consider $X = \PP^n_k$ with a linear action $\sigma \colon G \times X \to X$, that is, $\mathcal{L} = \mathcal{O}_X(1)$ is $G$-linearized. Note that there is a dual action
    \[ \begin{aligned}
        \hat{\sigma} : H^0(X, \mathcal{O}_X(1)) &\to \mathcal{O}_G(G) \otimes H^0(X, \mathcal{O}_X(1)) \\
        x_i &\mapsto \sum_{j = 0}^{n} a_{ij} \otimes x_j
    \end{aligned} \]
    given by some $a_{ij} \in \mathcal{O}_G(G)$. This induces a linear action on $\AA^{n + 1}_k$
    \[ \tilde{\sigma} \colon G \times \AA^{n + 1}_k \to \AA^{n + 1}_k \]
    which is compatible with $\sigma$ in the sense that
    \[ \svg \begin{tikzcd}
        G \times \left(\AA^{n + 1}_k \setminus \{ 0 \} \right) \arrow{d} \arrow{r}{\tilde{\sigma}} & \AA^{n + 1}_k \setminus \{ 0 \} \arrow{d} \\
        G \times X \arrow{r}{\sigma} & X
    \end{tikzcd} \]
    commutes.
    
    Now, for any one-parameter subgroup $\lambda$, choose coordinates on $\AA^{n + 1}_k$ such that $\lambda$ acts diagonally, that is, $\lambda(t) = \text{diag}(t^{r_0}, \ldots, t^{r_n})$ with $r_i \in \ZZ$. For any closed point $x \in X$ with lift $\tilde{x} \in \AA^{n + 1}_k$, one computes the weights
    \[ \mu^{\mathcal{L}}(x, \lambda) = \min \{ r_i : \tilde{x}_i \ne 0 \} . \]
    Moreover, $\lim_{t \to 0} \lambda(t) \cdot \tilde{x}$ does not exist (resp. does exist, resp. is zero) if and only if $\mu(x, \lambda) < 0$ (resp. $\mu(x, \lambda) \ge 0$, resp. $\mu(x, \lambda) > 0$). 
    
    Note that this analysis provides a nice interpretation of $\mu$ as
    \[ \mu^{\mathcal{L}}(x, \lambda) = \max \{ \alpha \in \ZZ \mid \lim_{t \to 0} t^{-\alpha} \lambda(t) \cdot \tilde{x} \text{ exists} \} . \]
\end{example}

\begin{topic}{HKKN-stratification}{HKKN stratification}
    Let $G$ be a \tref{reductive-algebraic-group}{reductive} \tref{algebraic-group}{algebraic group} over a field $k$ acting on a \tref{variety}{variety} $X$. For any one-parameter subgroup $\lambda \colon \GG_m \to G$, let
    \[ P_\lambda = \left\{ g \in G \mid \lim_{t \to 0} \lambda(t) \cdot g \cdot \lambda(t)^{-1} \textup{ exists} \right\} \]
    and write $X^\lambda$ for the fixed-point locus of $X$. For any connected component $Z_0^\lambda$ of $X^\lambda$, let
    \[ Z_\lambda = \left\{ x \in X \mid \lim_{t \to 0} \lambda(t) \cdot x \in Z_\lambda^0 \right\} \quad \textup{ and } \quad S_\lambda = G \cdot Z_\lambda . \]
    An \textbf{HKKN stratification} of $X$ is a sequence of open subvarieties
    \[ X = X_0 \supset X_1 \supset X_1 \supset \cdots \supset X_n \]
    together with one-parameter subgroups $\lambda_i \colon \GG_m \to G$ and choices of connected components $Z_{\lambda_i}^0 \subset X^{\lambda_i}$ for $1 \le i \le n$, such that
    \begin{itemize}
        \item $X_i = X_{i - 1} \setminus S_{\lambda_i}$,
        \item the morphism $G \times_{P_{\lambda_i}} Z_{\lambda_i} \to S_{\lambda_i}$ given by $(g, z) \mapsto g \cdot z$ is an isomorphism,
        \item $S_{\lambda_i}$ is closed in $X_{i - 1}$,
    \end{itemize}
    for all $1 \le i \le n$.
\end{topic}

\begin{example}{HKKN-stratification}
    Consider $G = \GG_m^2$ acting naturally on $X = \AA^2_k$, and let $\lambda \colon \GG_m \to G$ be the one-parameter subgroup given by $\lambda(t) = (t, t^{-1})$. One can only choose $Z_\lambda^0$ to be equal to the fixed-point locus $X^\lambda = \{ (0, 0) \}$, which yields $Z_\lambda = S_\lambda = \{ (x, 0) \in \AA^2_k \}$. Note that $P_\lambda = G$ and indeed the morphism $G \times_{P_\lambda} Z_\lambda \to S_\lambda$ is an isomorphism.
    Therefore, we have constructed an HKKN stratification with $n = 1$ and $X_0 = X$ and $X_1 = X \setminus S_\lambda$.
\end{example}
